\section{The \name Indexed Type System}
\label{sec:typesys}
The \name type system is designed to provide sufficient information to safely eliminate dynamic checks (\ie to ensure that the required preconditions are met to \prechk-tag an instruction).
It is based on the type system from \dtal, which is capable of ensuring similar properties to those we desire to prove.
An indexed type language uses an index language in the type system to encode information within types.
As in \dtal, we use the index language to encode linear constraints on program variables within types, giving us sufficient information to prove the desired properties of instructions.

\subsection{Tracking Program Variables}
\todo{Get rid of this section, certain parts may be useful to use in background about \wasm's type system}
To understand how \name index variables are associated with \name program values, we must first explain more about how the \wasm type system works.

\paragraph{The \wasm type system}
\wasm is a stack-based language, so the type of an instruction consists of a precondition and postcondition on the shape of the stack.
This can be viewed as though instructions \emph{consume} certain values from the stack and then \emph{produce} values to be pushed on the stack.
For example, a binary operation of some type $t$ consumes two values of the given type $t$ on the stack and produces a value of type $t$:

\[
    \inferrule{ }{C \vdash t.binop : t\ t \rightarrow t}
\]

The above example shows what a typical \wasm typing rule looks like.
$C$ is the module type context, which keeps track of various module-level types: functions, globals, the table, memory, locals, the label stack (\ie the expected types for branching instructions), and the return stack (\ie the expected type of the return instruction).
The type associated with the instruction $t.binop$ is a \wasm function type, which is just the precondition (on the left of the $\rightarrow$) and postcondition (on the right of the $\rightarrow$) on the stack.

Then, \name preconditions and postconditions look like $ti^{*}; l; \phi$:

\[
    \inferrule{ }{C \vdash t.binop : \ti{t}{a_1}\; \ti{t}{a_2};l;\phi \rightarrow \ti{t}{a_3};l;\phi,\ti{t}{a_3},(=\; a_3\; (binop\; a_1\; a_2))}
\]

\todo{Move to background}
\subsection{Stack Polymorphism}
To compose together the types of many instructions, it is necessary to carry around extra type information about the rest of the stack while type-checking instructions.
We follow \wasm's approach to solving this issue, \emph{stack polymorphism}.
Stack polymorphism allows extending the precondition and postcondition on the shape of the stack with the same arbitrary data to thread unmodified parts of the stack through a list of instructions.
Intuitively, this allows you to ``forget'' the rest of the stack and focus only on the part being manipulated by the current instruction being checked, after which point the ``forgotten'' part can be re-added.

\todo{Use concrete values to walk through this example}
For example, if the stack has the shape $\ti{t_0}{a_0}\; \ti{t_1}{a_1}\; \ti{t_1}{a_2}$, then stack polymorphism allows us to ignore $\ti{t_0}{a_0}$ and typecheck $t.binop$ with $\ti{t_1}{a_1}\; \ti{t_1}{a_2}$ on the stack.
Then the stack would look like $\ti{t_1}{a_3}$, at which point we add $t_2$ back to the postcondtion to get $\ti{t_0}{a_0}\; \ti{t_1}{a_3}$ after executing $t.binop$.

In practice, stack polymorphism is supported by the $stack-poly$ typing rule.
The $stack-poly$ rule allows an arbitrary sequence of indexed types to be prepended to both the precondition and postcondition of the stack:

\[
    \inferrule{C \vdash e^{*} :ti_1^{*};l_1;\phi_1 \rightarrow ti_2^{*};l_2;\phi_2}{C \vdash e^{*} : ti_0^{*}\;ti_1^{*};l_1;\phi_1 \rightarrow ti_0^{*}\;ti_2^{*};l_2;\phi_2}
\]

\subsection{Subtyping, Implication, and Constraint Satisfaction}
One issue with adding the index type context $\phi$ to preconditions and postconditions is that the postcondition of one instruction and the precondition of the next instruction might not match up exactly.
For example, one instruction may ensure a value is greater than ten, but the next just wants the value to be greater than zero.
Intuitively, if a value, ``x'', is greater than ten it must also be greater than zero, and we want the \name type system to be able to figure this out as well.
However, computers as of yet are unable to use intuition, so we must instead formalize this intuition such that a computer can reason about it.

Our formalization of this problem is based on Hoare logic \todo{find proper citation}.
Hoare logic similarly has preconditions and postconditions on statements, and allows those preconditions to be \emph{strengthened} and postconditions to be \emph{weakened}.
Strengthening and weakening is based on implication ($\implies$).
We say that $\phi_1 \implies \phi_2$ when the following holds: if $\phi_1$ is satisfied, then $\phi_2$ must also be satisfied.
If $\phi_1 \implies \phi_2$, then we consider $\phi_1$ to be stronger than $\phi_2$, and $\phi_2$ to be weaker than $\phi_1$.
This solves the aforementioned problem because we can weaken ``x is greater than 10'' to ``x is greater than 0'' (or equivalently strengthen ``x is greater than 0'' to ``x is greater than 10'').

In practice, we reason about implication using the Z3 theorem prover \todo{citation needed}.
To test whether the satisfiability of one index type context $\phi_1$ implies that some other index type context $\phi_2$ is satisfiable, we first generate Z3 constraints based on the propositions in both contexts.
Then, we assert that the constraints generated for the first context must hold.
Finally, we ask Z3 to find an assignment to the variables declared in the type index contexts where the constraints from the second context do not hold (a counterexample).
If a counterexample cannot be found then the implication must hold, otherwise it does not hold.

\todo{Need to format the digression to set it apart}
\thought{Maybe this is better moved to the discussion section to discuss as part of the implementation?}
\paragraph{Digression about impact of using Z3.}
Our choice of using Z3 has impacted \name in several ways.
\begin{itemize}
    \item The biggest impact is that we currently do not supporting floating point values and certain unary and binary operators because Z3 is unable to reason about them.
    \item Because of how we test implication, we require that the index type contexts use the same names to refer to the same index variables.
    \item The requirement of adding explicit type annotations for index variables and constants that appear in type indices comes from needing to know what width the variable will be when we convert it to a Z3 bitvector.
\end{itemize}

\todo{End of digression}

To fit strengthening and weakening into the type system, we define a subtyping judgment based on implication.
The subtyping judgement says that if an indexed function type $tfi_1$ has a stronger precondition and weaker postcondition than some other indexed function type $tfi_2$, and is otherwise equivalent, then $tfi_1$ is a subtype of $tfi_2$:

\[
    \inferrule{
        \phi_0 \implies \phi_1 \and
        \phi_2 \implies \phi_3
    }{
        ti_1^{*};l_1;\phi_0 \rightarrow ti_2^{*};l_2;\phi_3 <: ti_1^{*};l_1;\phi_1 \rightarrow ti_2^{*};l_2;\phi_2
    }
\]

We then use this in the \name type system by adding a typing rule that allows the indexed function type for a list of instructions to be replaced by a subtype of that indexed function type:

\[
    \inferrule{
        ti_1^{*};l_1;\phi_0 \rightarrow ti_2^{*};l_2;\phi_3 <: ti_1^{*};l_1;\phi_1 \rightarrow ti_2^{*};l_2;\phi_2 \\
        C \vdash e^{*} \rightarrow ti_1^{*};l_1;\phi_1 \rightarrow ti_2^{*};l_2;\phi_2
    }{
        C \vdash e^{*} \rightarrow ti_1^{*};l_1;\phi_0 \rightarrow ti_2^{*};l_2;\phi_3
    }
\]

\subsection{The Complete \name Type System}

\todo{Find a place for this}
At compile time, \name can \emph{pre-check} instructions that normally require dynamic safety checks.
If \name can prove that an instruction will not trap, then it ``tags'' the instruction with \prechk.
A compiler can then know not to generate run-time checks for \prechk-tagged instructions.

\name can currently eliminate checks for four types of instructions: memory loads and stores (similar to \dtal), integer division, and indirect function calls.
Tagging memory loads and stores with \prechk requires ensuring that the memory index is valid.
The integer division instruction can be \prechk tagged if the second argument is provably non-zero.
Indirect function calls are a particularly interesting case that is specific to \wasm: they are used to safely handle function pointers when compiling from C/C++ and require a dynamic check to ensure that the called function has a suitable type.
Proving the safety of an indirect function call involves showing that every possible function that could be called will not cause a run-time type error.

Integer division simply requires that the second argument is non-zero.
For example, the second rule in Figure \todo{figure} ensures that pre-checked division instructions can be safely executed.
The premise $\phi \implies (neq\ b\ 0)$ requires that the index constraints satisfy the proposition $b \neq 0$ for the pre-checked instruction to be safe.
Therefore, since a divide-by-zero is provably absent, it is safe to replace the division operator instruction with the \prechk-tagged version.

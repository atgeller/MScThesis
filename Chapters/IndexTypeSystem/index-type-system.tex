\section{The \name Indexed Type System}
\label{sec:typesys}
The \name type system was designed to provide sufficient information to safely eliminate dynamic checks (\ie to ensure that the required preconditions are met to \prechk-tag an instruction).
It is based on the type system from \dtal, which is capable of proving similar properties.
An indexed type language uses an index language in the type system to encode information within types.
As in \dtal, we use the index language to encode linear constraints on program variables within types, giving us sufficient information to prove the desired properties of instructions.

\subsection{The \name Index Type Language}
\begin{figure}[ht]
    \begin{math}
        \begin{array}{rcl}
            t &:: & \<ithreetwo> \mid \<isixfour> \\
            a &::= & Var \\
            x\;y &::=& a \mid (t\;c) \mid (binop\;x\;y) \\
            P &::=& (testop\;x) \mid (relop\;x\;y) \mid \neg P \mid P \land P \mid P \lor P \\
            \phi &::=& \circ \mid \phi, (t\;a) \mid \phi, P \\
        \end{array}
    \end{math}
    \caption{Syntax of the \name index type language}
    \label{fig:itsyntax}
\end{figure}

Figure~\ref{fig:itsyntax} shows the syntax for the index type language.
Syntax written in a \tbbf{blue, bold font} denotes a \wasm keyword.
Below is a quick overview of each of the terms.

\begin{itemize}
    \item $t$ represents a primitive \wasm type.
    We do not reason about floating points, so it is either a 32-bit integer ($i32$) or a 64-bit integer ($i64$).
    \item $a$ is a type index variable which is associated with a program variable.
    \item $x$ and $y$ are type indices, they can be an index type variable, a constant with an explicit type, or an operation on a type index.
    \item $P$ is a proposition (constraint) on type indices.
    \item $\phi$ is the index type context which stores index type variable declarations and propositions (constraints).
\end{itemize}

\subsection{Extending The \wasm Type System}
\thought{The overall hope is that the exposition suffices to make most of the typing rules appear straightforward once one understands all of the magic symbols. The section aims to provide a little more insight into how exactly the type system was magically conceived.}

In \name, we extend \wasm's technique for handling the stack to include an index context $\phi$ (which contains type index variable declarations and the constraints on indices) and the index types of local and global variables (similar to the Register file in \dtal) in the precondition and postcondition.
Now, a precondition on the stack would look like $(t\ a); \phi; l; l$, where $(t\ a)$ is an index type, and $l$ and $g$ map local and global variables, respectively, to index type variables.

\todo{Examples!}

\subsection{The Complete \name Type System}

\todo{Don't use $s$ as stack type}

\paragraph{Instructions}
\begin{figure}[h]
    \begin{mathpar}
        \inferrule[]{ }{ %% const
            \typerule{t.\<const>c} {
                \insttype{\type{\epsilon}{l}{g}{\phi}}
                    {\type{\ti{t}{a}}{l}{g}{\phi,\ti{t}{a},(\<eq>a\;(t\;c))}}}
        } \and
        %% unop unsupported
        \inferrule[]{ }{ %% binop
            \typerule{t.binop} {
                \insttype{\type{\ti{t}{a_1}\ti{t}{a_2}}{l}{g}{\phi}}
                      {\type{\ti{t}{a_3}}{l}{g}{\phi,\ti{t}{a_3},(\<eq>a_3\;(binop\;a_1\;a_2))}
                }
            }
        } \and
        \inferrule[]{ }{ %% testop
            \typerule{t.testop} {
                \insttype{\type{\ti{t}{a_1}}{l}{g}{\phi}}
                      {\type{\ti{t}{a_2}}{l}{g}{\phi,\ti{t}{a_2},(\<eq>a_2\;(testop\;a_1))}
                }
            }
        } \and
        \inferrule[]{ }{ %% relop
            \typerule{t.relop} {
                \insttype{\type{\ti{t}{a_1}\ti{t}{a_2}}{l}{g}{\phi}}
                      {\type{\ti{t}{a_3}}{l}{g}{\phi,\ti{t}{c},(\<eq>a_3\;(relop\;a_1\;a_2))}
                }
            }
        } \and
        %% convert, reinterpret unsupported
        \inferrule[]{ }{ %% unreachable
            \typerule{\<unreachable>} {
                \insttype{\type{s_1}{l_1}{g_1}{\phi_1}}
                      {\type{s_2}{l_2}{g_2}{\phi_2}}
            }
        } \and 
        \inferrule[]{ }{ %% nop
            \typerule{\<nop>} {
                \insttype{\type{\epsilon}{l}{g}{\phi}}
                      {\type{\epsilon}{l}{g}{\phi}}
            }
        } \and 
        \inferrule[]{ }{ %% drop
            \typerule{\<drop>} {
                \insttype{\type{\ti{t}{a}}{l}{g}{\phi}}
                      {\type{\epsilon}{l}{g}{\phi}}
            }
        } \and
        \inferrule[]{ }{ %% select
            C \vdash \<select> : {\begin{stackTL}
                \ti{t}{a_1}\;\ti{t}{a_2}\;\ti{i32}{a};l;g;\phi
                \\ \rightarrow \ti{t}{a_3};l;g;\phi,\ti{t}{a_3}, 
                {\begin{stackTL}
                    ((\<eqz> a) \land (\<eq> a_3\;a_2)) 
                    \\ \lor (\neg(\<eqz> a) \land (\<eq> a_3\;a_1)))
                \end{stackTL}} \\
            \end{stackTL}}
        } \and
        \inferrule[]{ %% block
            tfi = s_1;l_1;g_1;\phi_1 \rightarrow s_2;l_2;g_2;\phi_2 \\
            C_2,\text{label } (s_2;l_2;g_2;\phi_2) \vdash e^{*} : tfi \\
        }
        {
            C \vdash \<block> tfi\; e^{*} \<end> : tfi
        } \and
        \inferrule[]{ %% loop
            tfi = s_1;l_1;g_1;\phi_1 \rightarrow s_2;l_2;g_2;\phi_2 \\
            C_2,\text{label } (s_1;l_1;g_1;\phi_1) \vdash e^{*} : tfi \\
        }
        {
            C \vdash \<loop> tfi\; e^{*} \<end> : tfi
        } \and

        \todo{Important (and kinda cool) note: if we have, for example, that $(\<eqz> a)$ in $\phi_1$, then when type-checking $e_1^{*}$ we assume false (in the form of $(\<eqz> a) \land \neg(<eqz> a)$), and therefore any postcondition can be satisfied. However, this requires that we set up $\implies$ correctly with z3.}
        
        \todo{I tested it and $\implies$ totally works that way and I am an awesome and smart person for setting it up that way!}
        
        \inferrule[]{ %% if
            tfi = s_1;l_1;g_1;\phi_1 \rightarrow s_2;l_2;g_2;\phi_2 \\
            C_2,\text{label } (s_2;l_2;g_2;\phi_2) \vdash e_1^{*} : s_1;l_1;g_1;\phi_1, \neg(\<eqz> a) \rightarrow s_2;l_2;g_2;\phi_2 \\
            C_2,\text{label } (s_2;l_2;g_2;\phi_2) \vdash e_2^{*} : s_1;l_1;g_1;\phi_1, (\<eqz> a) \rightarrow s_2;l_2;g_2;\phi_2 \\
        }
        {
            C \vdash \<if> tfi\; e_1^{*} \<else> e_2^{*} \<end> : tfi
        } \and
        \inferrule[]{ %% br
            C_{\text{label}}(i) = ti^{*};l_1;g_1;\phi_1
        }
        {
            C \vdash \<br> i : s_1\;ti^{*};l_1;g_1;\phi_1 \rightarrow s_2;l_2;g_2;\phi_2
        } \and
        \inferrule[]{ %% br_if
            C_{\text{label}}(i) = s_1;l_1;g_1;\phi_1 
        }
        {
            C \vdash \<brif> i : s_1\;\index{i32}{a};l_1;g_1;\phi_1 \rightarrow s_1\;;l_1;g_1;\phi_1,\neg(\<eqz> a)
        } \and
        \inferrule[]{ %% br_table
            (C_{\text{label}}(i) = ti^{*};l_1;g_1;\phi_1)^{+}
        }
        {
            C \vdash \<brtable> i^{+} : s_1\;ti^{*}\;\index{i32}{a};l_1;g_1;\phi_1 \rightarrow s_2;l_2;g_2;\phi_2
        } \and
        \inferrule[]{ %% return
            C_{\text{return}} = ti^{*};l_1;g_1;\phi \and
            \phi_1 \implies \phi
        }
        {
            C \vdash \<return> : s_1\;ti^{*};l_1;g_1;\phi_1 \rightarrow s_2;l_2;g_2;\phi_2
        }
    \end{mathpar}
    \label{fig:typerules}
\end{figure}

\begin{figure}[h]
    \ContinuedFloat
    \begin{mathpar}
        \inferrule[]{ %% call
            C_{func}(i) = ti_1^{*};l_1;g_1;\phi_2 \rightarrow ti_2^{*};l_2;g_2;\phi_3 \and
            \phi_4 = \phi_1,\phi_3 \and
            \phi_1 \implies \phi_2
        }
        {
            C \vdash \<call> i : ti_1^{*};l;g_1;\phi_1 \rightarrow ti_2^{*};l;g_2;\phi_4
        } \and
        \inferrule[]{ %% call_indirect
            C_{table}(i) = (j_1, j_2^{*}) \and
            tfi = ti_1^{*};l_1;g_1;\phi_2 \rightarrow ti_2^{*};l_2;g_2;\phi_3 \and
            \phi_1 \implies \phi_2
        }
        {
            C \vdash \<callindirect> tfi : ti_1^{*}\;\ti{i32}{a};l;g_1;\phi_1 \rightarrow ti_2^{*};l;g_2;\phi_3
        } \and
        \inferrule[]{ %% get_local
            C_{\text{local}}(i) = t \and
            l(i) = \ti{t}{a}
        }
        {
            C \vdash \<getlocal> i : \epsilon;l;g;\phi \rightarrow \ti{t}{a_2};l;g;\phi, \ti{t}{a_2}, (\<eq> a_2\; a)
        } \and
        \inferrule[]{ %% set_local
            C_{\text{local}}(i) = t \and
            l_2 = l_1 \text{ except } l_2(i) = \ti{t}{a_2}
        }
        {
            C \vdash \<setlocal> i : \ti{t}{a};l_1;g;\phi \rightarrow \epsilon;l_2;g;\phi, \ti{t}{a_2}, (\<eq> a_2\; a)
        } \and
        \inferrule[]{ %% tee_local
            C_{\text{local}}(i) = t \and
            l_2 = l_1 \text{ except } l_2(i) = \ti{t}{a_2}
        }
        {
            C \vdash \<teelocal> i : \ti{t}{a};l_1;g;\phi \rightarrow \ti{t}{a};l_2;g;\phi, \ti{t}{a_2}, (\<eq> a_2\; a)
        } \and
        %% TODO: need typing rules for safe load and store
        \inferrule[]{ %% empty
        }
        {
            C \vdash \epsilon : \epsilon;l;g;\phi \rightarrow \epsilon;l;g;\phi
        } \and
        \inferrule[]{ %% extra vars
            C \vdash e^{*} : s_1;l_1;g_1;\phi_1 \rightarrow s_2;l_2;g_2;\phi_2
        }
        {
            C \vdash e^{*} : s\;s_1;l_1;g_1;\phi_1 \rightarrow s\;s_2;l_2;g_2;\phi_2
        } \and
        \inferrule[]{ %% combine
            C \vdash e_1^{*} : s_1;l_1;g_1;\phi_1 \rightarrow s_2;l_2;g_2;\phi_2 \\
            C \vdash e_2 : s_2;l_2;g_2;\phi_2 \rightarrow s_3;l_3;g_3;\phi_3
        }
        {
            C \vdash e_1^{*}\;e_2 : s_1;l_1;g_1;\phi_1 \rightarrow s_3;l_3;g_3;\phi_3
        } \and
        \inferrule[]{ %% strengthen precondition, weaken postcondition
            \phi_1 \implies \phi_2 \and
            \phi_3 \implies \phi_4 \\
            C \vdash e^{*} : s_2;l_2;g_2;\phi_2 \rightarrow s_3;l_3;g_3;\phi_3
        }
        {
            C \vdash e^{*} : s_2;l_2;g_2;\phi_1 \rightarrow s_3;l_3;g_3;\phi_4
        } \and
    \end{mathpar}
    \label{fig:typerules}
    \caption{Complete instruction type rules for the \name type system}
\end{figure}

\paragraph{Modules}
\begin{figure}[h]
    \ContinuedFloat
    \begin{mathpar}
        \inferrule[]{ %% local function
            tfi = \ti{t_1}{a_1}^{*};\epsilon ;g_1;\phi_1 \rightarrow ti_2^{*};\epsilon ;g_2;\phi_1 \\
            C_2 = C,\text{local } t_1^{*}\;t^{*},\text{label } (ti_2^{*}),\text{return } (ti_2^{*}) \\
            C_2 \vdash e^{*} : \epsilon ;\ti{t_1}{a_1}^{*}\;\ti{t}{a_2}^{*};g_1;\phi_1 \rightarrow ti_2^{*};l_2;g_2;\phi_2
        } {
            C \vdash ex^{*}\; \<func> tfi\; \<local> t^{*}\; e^{*} : ex^{*}\; tfi
        } \and

        \inferrule[]{ %% imported function, can't change globals
            tfi = ti_1^{*};\epsilon ;g;\phi_1 \rightarrow ti_2^{*};\epsilon ;g;\phi_1
        } {
            C \vdash ex^{*}\; \<func> tfi\; im : ex^{*}\; tfi
        } \\

        \inferrule[]{ %% local global
            tg = mut^{?}\;t \and
            ex^{*} = \epsilon \lor tg = t \and
            C \vdash e^{*} : \epsilon; \epsilon; g \; \phi_1 \rightarrow \ti{t}{a}; \epsilon; g \; \phi_2 \and
        } {
            C \vdash ex^{*}\; \<global> tg\; e^{*} : ex^{*}\; tg
        } \and

        \inferrule[]{ %% imported global
            tg = t \and
        } {
            C \vdash ex^{*}\; \<global> tg\; im : ex^{*}\; tg
        } \and

        \inferrule[]{ %% local table
            (C_{\text{func}}(i) = tfi)^n
        } {
            C \vdash ex^{*}\; \<table> n\; i^n : ex^{*}\; (n,tfi^n)
        } \and

        \inferrule[]{ %% imported table
        } {
            C \vdash ex^{*}\; \<table> (n,tfi^n) im : ex^{*}\; (n,tfi^n)
        } \and

        \inferrule[]{ %% local memory
        } {
            C \vdash ex^{*}\; \<memory> n : ex^{*}\; n
        } \and

        \inferrule[]{ %% imported memory
        } {
            C \vdash ex^{*}\; \<memory> n im : ex^{*}\; n
        } \and

        \inferrule[]{ %% imported memory
        } {
            C \vdash ex^{*}\; \<memory> n im : ex^{*}\; n
        } \and
    \end{mathpar}
    \label{fig:modulerules}
\end{figure}

\paragraph{Administrative Typing Rules}
\begin{figure}[h]
    \ContinuedFloat
    \begin{mathpar}
        \inferrule{ %% admin program
            \vdash s : S \and
            S;\epsilon \vdash_i s;v^{*};e^{*} : ti^{*};l;\phi
        } {
            \vdash_i s;v^{*};e^{*} : ti^{*};l;\phi
        } \and
        \inferrule[]{ %% admin code
            (\vdash v : \ti{t}{a};\phi_v)^{*}\\
            C = S_{\text{inst}}(i),\text{local} \; t^{*}, \text{return} \; (ti^n;l;\phi)^{?}\\
            S;C \vdash s;v^{*};e^{*} : \epsilon:\ti{t}{a}^{*};\phi_v^{*} \rightarrow ti^n;l;\phi
        } {
            S;(ti^n;l;\phi)^{?} \vdash_i s;v^{*};e^{*} : ti^n;l;\phi
        } \and
        \inferrule[]{ %% admin const
        } {
            \vdash t.\<const> c : \ti{t}{a};\circ,\ti{t}{a},(\<eq> a \; \ti{t}{c})
        } \and
        \inferrule[]{ %% trap
        } {
            C \vdash \<trap> : tfi
        } \and
        \inferrule[]{ %% local
            S;(ti^n;l_2;\phi_2) \vdash_i s;v_l^{*};e^{*} : ti^n;l_2;\phi_2
        } {
            S;C \vdash s;v^{*};\<local> \{ i;v_l^{*} \} \; e^{*} \<end> : \epsilon;l_1;\phi_1 \rightarrow ti^n;l_1;\phi_1,\phi_2
        } \and
    \end{mathpar}
    \label{fig:adminrules}
\end{figure}


\subsection{Constraint Satisfaction}
\todo{The exact details and positioning of this section are TBD}

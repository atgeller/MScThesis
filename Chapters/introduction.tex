\chapter{Introduction}
\label{chp:intro}

Types systems are useful for reasoning about programs.
They can be used to reason about the correctness of programs, usually in the form of safety guarantees.
For example, type safety is the property that a well-typed program will never become \emph{stuck}, that is, it will always be able to reduce the current expression. %% TODO: This sentence is super weak
The safety guarantees provided by type systems provide a degree of trust in programs as a well typed program implicitly contains a checkable proof that it will not exhibit behavior disallowed by the type system.

Expressive type systems can allow near complete trust in a program by verifying that it will never exhibit unsafe behavior.
Generally, such type systems are attached to high level language, where the language abstractions make reasoning about programs in that language easier.
However, prior work has attached expressive type systems that permit complex correctness guarantees to simple low-level languages.

Using such an expressive type system for a low-level language, we can alleviate overhead otherwise required to ensure safety of untrusted low-level programs.
Typically, executing code in an untrusted context requires dynamic safety checks which introduce potentially unnecessary instructions, slowing down execution.
However, with the safety guarantees provided by the type system, we can determine when these checks are unnecessary and remove them.
This would allow low-level programs to be downloaded, checked, and executed in a safe and efficient manner.

Browsers and Internet-of-Things (IoT) are established and widely used technologies.
Both potentially require running untrusted code, usually code that has been downloaded from somewhere on the internet.
Therefore, it is crucial to ensure the safety of the code being executed.
Typically, this is accomplished using dynamic safety checks and sandboxing.
However, dynamic safety checks introduce extra potentially unnecessary instructions, slowing down code.

%% This transition is a bit rushed

Type systems can be used to alleviate the need for safety checks, reducing overhead, by providing safety guarantees about programs before they are run.
The idea of using type systems to ensure the safety of low-level code is not a new one.
Several projects have attached expressive type systems to low-level languages to attach proofs of correctness to low-level programs.
However, the focus in these cases is on correctness, not on performance.
Further, the low-level languages in question were toy languages not used in practice.

%% Meh, too blunt
The eventual goal is to use types to maximize performance while ensuring safety in real-world low-level programs.
A necessary intermediate step is to create a type system which guarantees safety and has potential to improve performance.
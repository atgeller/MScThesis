\chapter{The \name Type System}
\label{chp:typesys}

The goal of \name is to be used to eliminate unnecessary dynamic checks.
To accomplish this, it must be able to 1) Statically identify instructions which may require dynamic checks, and 2) Prove that certain instructions do not require dynamic checks.
With these challenges in mind, we designed an indexed type system based on the type system presented in \dtal, modified to fit \wasm.
To understand the modifications we made to the \name type system to work with \wasm, we must first discuss how the \wasm type system works.

\todo{I'm decently happy with the high-high-level path here: Design constraints $\rightarrow$ WASM basics $\rightarrow$ building on WASM basics to fulfill design, but I'm worried the digression in WASM kinda ruins the flow}

\todo{Also, I'm unsure where the best place to introduce and discuss what an indexed type system is: the intro? here? related work about \dtal? Some combination of those with differing amounts of detail?}

\todo{I think the way I'm going to handle appealing to z3 for constraint satisfaction is to basically treat it as a black box and include a writeup in the appendix.}

\section{The \wasm Type System}
\wasm is a stack-based language, so the type of an instruction in \wasm consists of a precondition and postcondition on the shape of the stack.
This can be viewed as though instructions \emph{consume} certain values from the stack and then \emph{produce} values to be pushed on the stack.
For example, a binary operation of some type $t$ consumes two values of the given type $t$ on the stack and produces a value of type $t$:

\[
    \inferrule{ }{C \vdash t.binop : t\ t \rightarrow t}
\]

In the above rule, $C$ is the module type context, which keeps track of various module-level types.

Because most instructions only manipulate the top of the stack, it is necessary to carry around extra type information about the rest of the stack while type-checking instructions.
\emph{Stack polymorphism} allows extending the precondition and postcondition with the same data to thread unmodified parts of the stack through a list of instructions.
Intuitively, this allows you to ``forget'' the rest of the stack and focus only on the part being manipulated by the instruction being checked, after which point the ``forgotten'' part can be re-added.

\todo{The explanation of this example seems really clunky}
For example, if the stack has the shape $t_2\; t\; t$, then stack polymorphism allows us to ignore $t_2$ and typecheck $t.binop$ with $t\;t$ on the stack.
Then the stack would look like $t$, at which point we add $t_2$ back to the postcondtion to get $t_2\; t$ after executing $t.binop$.

\section{Extending The \wasm Type System With Indexed Types}
\todo{I put this figure here cause why not, but honestly I think it should just be exposition getting us from WASM to indexed types and briefly setting up the avalanche of typing rules to follow.}
\begin{figure}[ht]
    \begin{math}
        \begin{array}{rcl}
            t &:: & \<ithreetwo> | \<isixfour> \\
            a &::= & Var \\
            x\;y &::=& a | (t\;c) | (binop\;x\;y) \\
            P &::=& (testop\;x) | (relop\;x\;y) | \neg P | P \land P | P \lor P \\
            \phi &::=& \circ | \phi, (t\;a) | \phi, P \\
        \end{array}
    \end{math}
    \caption{Syntax of the \name index type language}
    \label{fig:itsyntax}
\end{figure}

\todo{Details on the index type language}
An indexed type system uses an index language in the type system to track constraints on program variables.
Index variables are used to 

In \name, we extend \wasm's technique for handling the stack to include an index context $\phi$ (which contains type index variable declarations and the constraints on indices) and the index types of local and global variables (similar to the Register file in DTAL) in the precondition and postcondition.
Now, a precondition on the stack would look like $(t\ a); \phi; L; G$, where $(t\ a)$ is an index type, and $L$ and $G$ map local and global variables, respectively, to index type variables.

\section{The Complete \name Type System}
Figure~\ref{fig:itsyntax} shows the syntax for the index type language.
Syntax written in a \tbbf{blue, bold font} denotes a \wasm keyword.
Below is a quick overview of each of the terms.

\begin{itemize}
    \item $t$ represents a primitive \wasm type.
    We do not reason about floating points, so it is either a 32-bit integer ($i32$) or a 64-bit integer ($i64$).
    \item $a$ Is a type index variable which is associated with a program variable.
\end{itemize}

\todo{Don't use $s$ as stack type}

\paragraph{Instructions}
\begin{figure}[h]
    \begin{mathpar}
        \inferrule[]{ }{ %% const
            \typerule{t.\<const>c} {
                \insttype{\type{\epsilon}{l}{g}{\phi}}
                    {\type{\ti{t}{a}}{l}{g}{\phi,\ti{t}{a},(\<eq>a\;(t\;c))}}}
        } \and
        %% unop unsupported
        \inferrule[]{ }{ %% binop
            \typerule{t.binop} {
                \insttype{\type{\ti{t}{a_1}\ti{t}{a_2}}{l}{g}{\phi}}
                      {\type{\ti{t}{a_3}}{l}{g}{\phi,\ti{t}{a_3},(\<eq>a_3\;(binop\;a_1\;a_2))}
                }
            }
        } \and
        \inferrule[]{ }{ %% testop
            \typerule{t.testop} {
                \insttype{\type{\ti{t}{a_1}}{l}{g}{\phi}}
                      {\type{\ti{t}{a_2}}{l}{g}{\phi,\ti{t}{a_2},(\<eq>a_2\;(testop\;a_1))}
                }
            }
        } \and
        \inferrule[]{ }{ %% relop
            \typerule{t.relop} {
                \insttype{\type{\ti{t}{a_1}\ti{t}{a_2}}{l}{g}{\phi}}
                      {\type{\ti{t}{a_3}}{l}{g}{\phi,\ti{t}{c},(\<eq>a_3\;(relop\;a_1\;a_2))}
                }
            }
        } \and
        %% convert, reinterpret unsupported
        \inferrule[]{ }{ %% unreachable
            \typerule{\<unreachable>} {
                \insttype{\type{s_1}{l_1}{g_1}{\phi_1}}
                      {\type{s_2}{l_2}{g_2}{\phi_2}}
            }
        } \and 
        \inferrule[]{ }{ %% nop
            \typerule{\<nop>} {
                \insttype{\type{\epsilon}{l}{g}{\phi}}
                      {\type{\epsilon}{l}{g}{\phi}}
            }
        } \and 
        \inferrule[]{ }{ %% drop
            \typerule{\<drop>} {
                \insttype{\type{\ti{t}{a}}{l}{g}{\phi}}
                      {\type{\epsilon}{l}{g}{\phi}}
            }
        } \and
        \inferrule[]{ }{ %% select
            C \vdash \<select> : {\begin{stackTL}
                \ti{t}{a_1}\;\ti{t}{a_2}\;\ti{i32}{a};l;g;\phi
                \\ \rightarrow \ti{t}{a_3};l;g;\phi,\ti{t}{a_3}, 
                {\begin{stackTL}
                    ((\<eqz> a) \land (\<eq> a_3\;a_2)) 
                    \\ \lor (\neg(\<eqz> a) \land (\<eq> a_3\;a_1)))
                \end{stackTL}} \\
            \end{stackTL}}
        } \and
        \inferrule[]{ %% block
            tfi = s_1;l_1;g_1;\phi_1 \rightarrow s_2;l_2;g_2;\phi_2 \\
            C_2,\text{label } (s_2;l_2;g_2;\phi_2) \vdash e^{*} : tfi \\
        }
        {
            C \vdash \<block> tfi\; e^{*} \<end> : tfi
        } \and
        \inferrule[]{ %% loop
            tfi = s_1;l_1;g_1;\phi_1 \rightarrow s_2;l_2;g_2;\phi_2 \\
            C_2,\text{label } (s_1;l_1;g_1;\phi_1) \vdash e^{*} : tfi \\
        }
        {
            C \vdash \<loop> tfi\; e^{*} \<end> : tfi
        } \and

        \todo{Important (and kinda cool) note: if we have, for example, that $(\<eqz> a)$ in $\phi_1$, then when type-checking $e_1^{*}$ we assume false (in the form of $(\<eqz> a) \land \neg(<eqz> a)$), and therefore any postcondition can be satisfied. However, this requires that we set up $\implies$ correctly with z3.}
        
        \todo{I tested it and $\implies$ totally works that way and I am an awesome and smart person for setting it up that way!}
        
        \inferrule[]{ %% if
            tfi = s_1;l_1;g_1;\phi_1 \rightarrow s_2;l_2;g_2;\phi_2 \\
            C_2,\text{label } (s_2;l_2;g_2;\phi_2) \vdash e_1^{*} : s_1;l_1;g_1;\phi_1, \neg(\<eqz> a) \rightarrow s_2;l_2;g_2;\phi_2 \\
            C_2,\text{label } (s_2;l_2;g_2;\phi_2) \vdash e_2^{*} : s_1;l_1;g_1;\phi_1, (\<eqz> a) \rightarrow s_2;l_2;g_2;\phi_2 \\
        }
        {
            C \vdash \<if> tfi\; e_1^{*} \<else> e_2^{*} \<end> : tfi
        } \and
        \inferrule[]{ %% br
            C_{\text{label}}(i) = ti^{*};l_1;g_1;\phi_1
        }
        {
            C \vdash \<br> i : s_1\;ti^{*};l_1;g_1;\phi_1 \rightarrow s_2;l_2;g_2;\phi_2
        } \and
        \inferrule[]{ %% br_if
            C_{\text{label}}(i) = s_1;l_1;g_1;\phi_1 
        }
        {
            C \vdash \<brif> i : s_1\;\index{i32}{a};l_1;g_1;\phi_1 \rightarrow s_1\;;l_1;g_1;\phi_1,\neg(\<eqz> a)
        } \and
        \inferrule[]{ %% br_table
            (C_{\text{label}}(i) = ti^{*};l_1;g_1;\phi_1)^{+}
        }
        {
            C \vdash \<brtable> i^{+} : s_1\;ti^{*}\;\index{i32}{a};l_1;g_1;\phi_1 \rightarrow s_2;l_2;g_2;\phi_2
        } \and
        \inferrule[]{ %% return
            C_{\text{return}} = ti^{*};l_1;g_1;\phi \and
            \phi_1 \implies \phi
        }
        {
            C \vdash \<return> : s_1\;ti^{*};l_1;g_1;\phi_1 \rightarrow s_2;l_2;g_2;\phi_2
        }
    \end{mathpar}
    \label{fig:typerules}
\end{figure}

\begin{figure}[h]
    \ContinuedFloat
    \begin{mathpar}
        \inferrule[]{ %% call
            C_{func}(i) = ti_1^{*};l_1;g_1;\phi_2 \rightarrow ti_2^{*};l_2;g_2;\phi_3 \and
            \phi_4 = \phi_1,\phi_3 \and
            \phi_1 \implies \phi_2
        }
        {
            C \vdash \<call> i : ti_1^{*};l;g_1;\phi_1 \rightarrow ti_2^{*};l;g_2;\phi_4
        } \and
        \inferrule[]{ %% call_indirect
            C_{table}(i) = (j_1, j_2^{*}) \and
            tfi = ti_1^{*};l_1;g_1;\phi_2 \rightarrow ti_2^{*};l_2;g_2;\phi_3 \and
            \phi_1 \implies \phi_2
        }
        {
            C \vdash \<callindirect> tfi : ti_1^{*}\;\ti{i32}{a};l;g_1;\phi_1 \rightarrow ti_2^{*};l;g_2;\phi_3
        } \and
        \inferrule[]{ %% get_local
            C_{\text{local}}(i) = t \and
            l(i) = \ti{t}{a}
        }
        {
            C \vdash \<getlocal> i : \epsilon;l;g;\phi \rightarrow \ti{t}{a_2};l;g;\phi, \ti{t}{a_2}, (\<eq> a_2\; a)
        } \and
        \inferrule[]{ %% set_local
            C_{\text{local}}(i) = t \and
            l_2 = l_1 \text{ except } l_2(i) = \ti{t}{a_2}
        }
        {
            C \vdash \<setlocal> i : \ti{t}{a};l_1;g;\phi \rightarrow \epsilon;l_2;g;\phi, \ti{t}{a_2}, (\<eq> a_2\; a)
        } \and
        \inferrule[]{ %% tee_local
            C_{\text{local}}(i) = t \and
            l_2 = l_1 \text{ except } l_2(i) = \ti{t}{a_2}
        }
        {
            C \vdash \<teelocal> i : \ti{t}{a};l_1;g;\phi \rightarrow \ti{t}{a};l_2;g;\phi, \ti{t}{a_2}, (\<eq> a_2\; a)
        } \and
        %% TODO: need typing rules for safe load and store
        \inferrule[]{ %% empty
        }
        {
            C \vdash \epsilon : \epsilon;l;g;\phi \rightarrow \epsilon;l;g;\phi
        } \and
        \inferrule[]{ %% extra vars
            C \vdash e^{*} : s_1;l_1;g_1;\phi_1 \rightarrow s_2;l_2;g_2;\phi_2
        }
        {
            C \vdash e^{*} : s\;s_1;l_1;g_1;\phi_1 \rightarrow s\;s_2;l_2;g_2;\phi_2
        } \and
        \inferrule[]{ %% combine
            C \vdash e_1^{*} : s_1;l_1;g_1;\phi_1 \rightarrow s_2;l_2;g_2;\phi_2 \\
            C \vdash e_2 : s_2;l_2;g_2;\phi_2 \rightarrow s_3;l_3;g_3;\phi_3
        }
        {
            C \vdash e_1^{*}\;e_2 : s_1;l_1;g_1;\phi_1 \rightarrow s_3;l_3;g_3;\phi_3
        } \and
        \inferrule[]{ %% strengthen precondition, weaken postcondition
            \phi_1 \implies \phi_2 \and
            \phi_3 \implies \phi_4 \\
            C \vdash e^{*} : s_2;l_2;g_2;\phi_2 \rightarrow s_3;l_3;g_3;\phi_3
        }
        {
            C \vdash e^{*} : s_2;l_2;g_2;\phi_1 \rightarrow s_3;l_3;g_3;\phi_4
        } \and
    \end{mathpar}
    \label{fig:typerules}
    \caption{Complete instruction type rules for the \name type system}
\end{figure}

\paragraph{Modules}
\begin{figure}[h]
    \ContinuedFloat
    \begin{mathpar}
        \inferrule[]{ %% local function
            tfi = \ti{t_1}{a_1}^{*};\epsilon ;g_1;\phi_1 \rightarrow ti_2^{*};\epsilon ;g_2;\phi_1 \\
            C_2 = C,\text{local } t_1^{*}\;t^{*},\text{label } (ti_2^{*}),\text{return } (ti_2^{*}) \\
            C_2 \vdash e^{*} : \epsilon ;\ti{t_1}{a_1}^{*}\;\ti{t}{a_2}^{*};g_1;\phi_1 \rightarrow ti_2^{*};l_2;g_2;\phi_2
        } {
            C \vdash ex^{*}\; \<func> tfi\; \<local> t^{*}\; e^{*} : ex^{*}\; tfi
        } \and

        \inferrule[]{ %% imported function, can't change globals
            tfi = ti_1^{*};\epsilon ;g;\phi_1 \rightarrow ti_2^{*};\epsilon ;g;\phi_1
        } {
            C \vdash ex^{*}\; \<func> tfi\; im : ex^{*}\; tfi
        } \\

        \inferrule[]{ %% local global
            tg = mut^{?}\;t \and
            ex^{*} = \epsilon \lor tg = t \and
            C \vdash e^{*} : \epsilon; \epsilon; g \; \phi_1 \rightarrow \ti{t}{a}; \epsilon; g \; \phi_2 \and
        } {
            C \vdash ex^{*}\; \<global> tg\; e^{*} : ex^{*}\; tg
        } \and

        \inferrule[]{ %% imported global
            tg = t \and
        } {
            C \vdash ex^{*}\; \<global> tg\; im : ex^{*}\; tg
        } \and

        \inferrule[]{ %% local table
            (C_{\text{func}}(i) = tfi)^n
        } {
            C \vdash ex^{*}\; \<table> n\; i^n : ex^{*}\; (n,tfi^n)
        } \and

        \inferrule[]{ %% imported table
        } {
            C \vdash ex^{*}\; \<table> (n,tfi^n) im : ex^{*}\; (n,tfi^n)
        } \and

        \inferrule[]{ %% local memory
        } {
            C \vdash ex^{*}\; \<memory> n : ex^{*}\; n
        } \and

        \inferrule[]{ %% imported memory
        } {
            C \vdash ex^{*}\; \<memory> n im : ex^{*}\; n
        } \and

        \inferrule[]{ %% imported memory
        } {
            C \vdash ex^{*}\; \<memory> n im : ex^{*}\; n
        } \and
    \end{mathpar}
    \label{fig:modulerules}
\end{figure}

\paragraph{Administrative Typing Rules}
\begin{figure}[h]
    \ContinuedFloat
    \begin{mathpar}
        \inferrule{ %% admin program
            \vdash s : S \and
            S;\epsilon \vdash_i s;v^{*};e^{*} : ti^{*};l;\phi
        } {
            \vdash_i s;v^{*};e^{*} : ti^{*};l;\phi
        } \and
        \inferrule[]{ %% admin code
            (\vdash v : \ti{t}{a};\phi_v)^{*}\\
            C = S_{\text{inst}}(i),\text{local} \; t^{*}, \text{return} \; (ti^n;l;\phi)^{?}\\
            S;C \vdash s;v^{*};e^{*} : \epsilon:\ti{t}{a}^{*};\phi_v^{*} \rightarrow ti^n;l;\phi
        } {
            S;(ti^n;l;\phi)^{?} \vdash_i s;v^{*};e^{*} : ti^n;l;\phi
        } \and
        \inferrule[]{ %% admin const
        } {
            \vdash t.\<const> c : \ti{t}{a};\circ,\ti{t}{a},(\<eq> a \; \ti{t}{c})
        } \and
        \inferrule[]{ %% trap
        } {
            C \vdash \<trap> : tfi
        } \and
        \inferrule[]{ %% local
            S;(ti^n;l_2;\phi_2) \vdash_i s;v_l^{*};e^{*} : ti^n;l_2;\phi_2
        } {
            S;C \vdash s;v^{*};\<local> \{ i;v_l^{*} \} \; e^{*} \<end> : \epsilon;l_1;\phi_1 \rightarrow ti^n;l_1;\phi_1,\phi_2
        } \and
    \end{mathpar}
    \label{fig:adminrules}
\end{figure}


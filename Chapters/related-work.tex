\chapter{Related Work}
\label{chp:relwork}

%% TIL
Using type information to improve compiler optimizations is not a new idea.
In 1996, a paper by Tarditi et al. used strongly typed intermediate languages (TIL) to improve optimizations of SML code ~\cite{TIL}.
Compiling SML involves many translations to intermediate languages, and by preserving type information through those translations and in the intermediate language Tarditi et al. were able to safely perform additional compiler operations.
Using TIL in the compilation of programs led to significantly faster programs.
TIL focuses on compiler optimizations and eventually translates into untyped languages and finally runnable assembly, so the guarantees of the type system are lost along the way.

%% FToTAL
\todo{Is TAL actually the first typed assembly language? From what I remember of 539B this is true, but I may be forgetting something}
The idea of a typed assembly language was first introduced in 1999 by Morrisett et al. with the strongly typed assembly language (TAL) ~\cite{FToTAL}.
Morrisett et al. showed how types could provably be preserved during five different compilation passes to get from System F to TAL.
The purpose of TAL was much more focused on safety than on optimizations.
Although Morrisett et al. argued that the type-preserving compilation passes would permit similar optimizations to TIL, they didn't include further optimizations based on TAL.
However, Morrisett et al. did argue that the guarantees of TAL were sufficient to allow untrusted code to be safely executed.

%% LTAL
\todo{LTAL}

%% PC
\todo{Blah}
Proof carrying code (PCC) was an idea introduced in 1997 by Necula et al. ~\cite{PCC}.
While typed assembly languages carry implicit proofs in their types, PCC attached explicit proofs that low-level code satisfies some safety properties.
The proof can then be quickly checked to ensure the safety of the code before it is run.
This provides a way to verify that untrusted code will not violate correctness or security invariants of a program.
The author provides a detailed example of invariants for extensions to TIL to ensure type safety of compiled code.
The example uses the Edinburgh Logical Framework (LF) to encode the proof.
A type safety proof of a LF program is a proof of correctness.
However, PCC puts a burden on developers to formally specify safety and correctness properties.

%% DTAL
Xi and Harper created a much more expressive type system for an assembly language which had the potential to allow more compiler optimizations ~\cite{dtal}.
Their language, a dependently typed assembly language (\dtal), used a limited dependent type system to safely remove some run-time checks, including array bounds checks.
The goal of \dtal, similar to TAL, was to support type-preserving compilation from a high-level language for both optimizations and safety.
\dtal intended to support type-preserving compilation from Dependent ML as well as SML.
However, these goals were never realized in a practical system.

%% Wasm
After almost two decades of JavaScript being the dominant language in browsers, it was decided that an alternative was necessary for performance-critical code.
The alternative that was jointly created by the major browser developers was WebAssembly (\wasm) ~\cite{Wasm}.
\wasm is a stack-based assembly language with structured control flow.
It is designed to be safe as well as performant.
The \wasm type system is simple, only encoding primitive types, but strong enough to ensure type safety.
Memory safety in \wasm is enforced using run-time checks.
\wasm is supported by most major browsers, and is increasingly used in IoT devices due to its portability and safety.
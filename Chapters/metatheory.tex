\chapter{Metatheory}
\label{chp:metatheory}

\section{Converting Between \wasm and \name}
We want to show two properties about the relationship between \wasm and \name.
First, we want to show that well-typed \name programs are backwards compatible with \wasm programs.
This is accomplished in \autoref{subsec:erasure} using an erasure function that turns \name programs and types into \wasm programs and types.
Second, it should be possible to convert well-typed \wasm programs into well-typed \name programs with no additional developer effort.
We demonstrate a simple yet naive way of embedding \wasm programs into \name in \autoref{subsec:embedding}.

\subsection{Erasing \name to \wasm}
\label{subsec:erasure}
We provide an erasure function for \name that transforms \name programs into \wasm programs by discarding the extra information from the \name type system and replacing \prechk-tagged instructions with their non-tagged counterparts.
Erasure is useful in the type safety proof because it lets us reuse much of the proof of progress from \wasm (see \autoref{subsec:progress}).
Therefore, we define erasure not just for the surface syntax, like we did for embedding, but also for typing constructs (such as the module type context), administrative instructions, and runtime data structures (such as the store).
We show that erasing a well-typed \name program produces a well-typed \wasm program.

As with the presentation of the embedding, we typeset \name instructions in a \tbsf{blue sans serif font} and \wasm instruction in a \trbf{bold red font}.

Erasing an indexed type function keeps only the primitive \wasm types ($t_1^{*}$ and $t_2^{*}$) from the indexed types representing the stack ($\ti{t_1}{a_1}^{*}$ and $\ti{t_2}{a_2}^{*}$), and discards everything else.

\begin{definition}{$\erase{tfi} = \mathredbold{tf}$}

    $\erase[tfi]{\ti{t_1}{a_1}^{*};l_1;\phi_1 \rightarrow \ti{t_2}{a_2}^{*};l_2;\phi_2} = t_1^{*} \rightarrow t_2^{*}$
\end{definition}

Erasing instructions involves erasing the indexed function types for every instruction that includes it as part of their syntax (blocks and indirect function calls).
We must also remove the \prechk tag from \prechk-tagged instructions to turn them into instructions that exist in \wasm.

\begin{definition}{$\erase[e]{e} = \mathredbold{e}$}
    \begin{mathpar}
        \begin{array}{rcl}
            \erase[e]{\<block>\; tfi\; e^{*} \<end>} &=& \<wblock> \erase[tfi]{tfi}\; \erase[e]{e^{*}} \<wend> \\

            \erase[e]{\<loop>\; tfi\; e^{*} \<end>} &=& \<wloop> \erase[tfi]{tfi}\; \erase[e]{e^{*}} \<wend> \\

            \erase[e]{\<if>\; tfi\; e_1^{*}\; e_2^{*} \<end>} &=& {\begin{stackTL}\<wif> {\begin{stackTL}\erase[tfi]{tfi} \\ \erase[e]{e_1^{*}} \\ \erase[e]{e_2^{*}} \end{stackTL}} \\ \<wend> \end{stackTL}} \\

            \erase[e]{\<callindirect> tfi} &=& \<wcallindirect> \erase[tfi]{tfi} \\

            \erase[e]{t.\<divpc>} &=& t.\<wdiv> \\

            \erase[e]{t.\<callindirectpc>} &=& t.\<wcallindirect> \\

            \erase[e]{t.\<storepc> tp^{?}\; align\; o} &=& t.\<wstore> tp^{?}\; align\; o \\

            \erase[e]{t.\<loadpc> (tp\_sx)^{?}\; align\; o} &=& t.\<wload> (tp\_sx)^{?}\; align\; o \\

            \erase[e]{e} &=& e \text{, otherwise} \\
        \end{array}
    \end{mathpar}
\end{definition}

To erase a module type context, we must erase all of the function types $tfi^{*}$, the table type $(n,tfi_2^{*})$ if one is present, and the postconditions in the label stack $(\ti{t_1}{a_1}^{*};l_1;\phi_1)^{*}$ and the return stack $(\ti{t_2}{a_2}^{*};l_2;\phi_2)^{?}$.
We erasing postconditions the same way we erase the postconditions of indexed function types.
Erasing a table type means discarding the type information about the functions in the table.

\begin{definition}{$\erase[c]{C} = \mathredbold{C}$}
    \begin{mathpar}
        \begin{array}{rcl}
            {\begin{stackTL} erase_C(\{
                {\begin{stackTL}
                    \text{func } tfi^{*}, \text{ global } tg^{*},
                    \\ \text{table } (n,tfi_2^{*})^{?},
                    \\ \text{memory } m^{?}, \text{ local } t^{*},
                    \\ \text{label } (\ti{t_1}{a_1}^{*};l_1;\phi_1)^{*},
                    \\ \text{return } (\ti{t_2}{a_2}^{*};l_2;\phi_2)^{?}\})
                \end{stackTL}}
            \end{stackTL}}
            &=&
            \{{\begin{stackTL}
                \text{func } \erase[tfi]{tfi^{*}},
                \\ \text{global } tg^{*}, \text{ table } n^{?},
                \\ \text{memory }\; m^{?}, \text{local } t^{*},
                \\ \text{label } (t_1^{*})^{*}, \text{ return } (t_2^{*})^{?}\}
            \end{stackTL}}
        \end{array}
    \end{mathpar}
\end{definition}

We show that erasing a \name instruction sequence $e^{*}$, that is well-typed under a module type context $C$, produces a \wasm instruction sequence $e'^{*}=\erase[e]{e^{*}}$ that is well-typed under the erased module type context $C'=\erase[C]{C}$.

\todo{erased-well-typed-typesys}
\todo{Do this one for the typing rules explained in \autoref{sec:wasmtyping}}

To erase a function definition $f$, we simply erase the type declaration $tfi$ and the body $e^{*}$.
We can also erase an imported function by erasing the declared type $tfi$.

\begin{definition}{$\erase[f]{f} = \mathredbold{f}$}
    \begin{mathpar}
        \begin{array}{rcl}
            \erase[f]{\<func> tfi\;\<local>\; t^{*}\; e^{*})}
            &=&
            \<wfunc> \erase[tfi]{tfi}\; \<wlocal>\; t^{*}\; \erase[e]{e^{*}} \\

            \erase[f]{\<func> tfi\; im}
            &=&
            \<wfunc> \erase[tfi]{tfi}\; im \\
        \end{array}
    \end{mathpar}
\end{definition}

We show that erasing a \name function $f$, that is well-typed under a module type context $C$, produces a \wasm function $\erase[f]{f}$ that is well-typed under the erased module type context $\erase[C]{C}$.
This is useful not just for erasing the surface syntax, but also because functions are a part of closures which are used at run-time (as part of module instances and tables).

\todo{erased-well-typed-func lemma}

Erasing a module instance erases all of the functions $f$ in the closures (which we have expanded inline to $\{\text{inst } i, \text{ func } f\}$) within the module instance.

\begin{definition}{$\erase[inst]{inst} = \mathredbold{inst}$}
    \begin{mathpar}
        \begin{array}{rcl}
            {\begin{stackTL} erase_inst(\{
                {\begin{stackTL}
                    \text{func } \{\text{inst } i, \text{ func } f\}^{*},
                    \\ \text{global } v^{*}, \text{ table } i^{?},
                    \\ \text{memory } j^{?}\})
                \end{stackTL}}
            \end{stackTL}}
            &=&
            \{{\begin{stackTL}
                \text{func } \{\text{inst } i, \text{ func } \erase{f}\}^{*},
                \\ \text{global } v^{*}, \text{ table } i^{?},
                \\ \text{memory } j^{?}\}
            \end{stackTL}}
        \end{array}
    \end{mathpar}
\end{definition}

We erase store contexts by erasing all of the module type instances $C^{*}$ and table types $(n,tfi^{*})^{*}$ within.

\begin{definition}{$\erase[S]{S} = \mathredbold{S}$}
    \begin{mathpar}
        \begin{array}{rcl}
            erase_S(\{ {\begin{stackTL}
                    \text{inst } C^{*},
                    \\ \text{tab } (n,tfi^{*})^{*}, \text{mem } m^{*}\}
                \end{stackTL}}
            &=& \{ {\begin{stackTL}
                \text{inst } \erase[c]{C}^{*},
                \\ \text{tab } n^{*}, \text{mem } m^{*}\} \end{stackTL}}
        \end{array}
    \end{mathpar}
\end{definition}

We now prove that if a \name module instance $inst$ has type $C$ under the store context $S$, then the erased \wasm instance $\erase{inst}$ will have the erased type $\erase[c]{C}$ under the erased store context $\erase[S]{S}$.
This will be useful for proving that a well-typed \name store $s$ erases to a well-typed \wasm store $\erase{s}{s}$ since stores contain many instances.

\begin{lemma}{(Erased-Well-Typed-Context)}
    If $S \vdash inst : C$, then $\erase{S} \vdash \erase{inst} : \erase{C}$
\end{lemma}
\begin{proof}
    We want to prove that erasing index information a \name runtime module instance $inst$ will result in a well-typed \wasm runtime module instance $\mathredbold{inst}$.
    To do this, we rely on the above lemmas to safely erase index information from function declarations and table declarations (globals and memory have the same type information in both \name and \wasm).

    \todo{Finish using erased-well-typed-func}
\end{proof}

\todo{erased-well-typed-store}

Before we can proof that erasing a well-typed \name program in reduction form $s;v^{*};e^{*}$ produces a well-typed \wasm program (so we can use erasure in progress), we must prove this property for the \name administrative instruction typing judgement $S;C\vdash e^{*}:tfi$.
Then, we can prove this of \refrule{Admin-Code} and finally of \refrule{Admin-Program}, which gives us the property we want.

\begin{lemma}{(Erased-Well-Typed-Admin)}

    If $S;C \vdash e^{*} : ti_1^{*};l_1;\phi_1 \rightarrow ti_2^{*};l_2;\phi_2$,
    \\ then $\erase{S};\erase{C} \vdash \erase{e^{*}} : \erase{\epsilon;l_1;\phi_1 \rightarrow \ti{t}{a}^{*};l_2;\phi_2}$
\end{lemma}
\begin{proof}
    \todo{Enumerate local, label, and call cl}

    We proceed by induction over typing rules. Most proof cases are omitted as they are simple, but we provide a few to give an idea of what the proofs look like.

    \begin{itemize}
        \item $S;C \vdash t.binop : \ti{t}{a_1}\;\ti{t}{a_2};l_1;\phi_1 \rightarrow \ti{t}{a_3};l_1;\phi_1,\ti{t}{a_3},(= a_3\; (binop\;a_1\;a_2))$

        $\erase{S};\erase{C} \vdash \erase{t.binop} : \erase{\ti{t}{a_1}\;\ti{t}{a_2};l_1;\phi_1 \rightarrow \ti{t}{a_3};l_1;\phi_1,\ti{t}{a_3},(= a_3\; (binop\;a_1\;a_2))}$ = $\erase{S};\erase{C} \vdash t.binop : t\;t \rightarrow t$, which holds under \wasm's type system.
        \item $S;C \vdash \<unreachable> : ti_1^{*};l_1;\phi_1 \rightarrow ti_2^{*};l_2;\phi_2$

        $\erase{S};\erase{C} \vdash \erase{\<unreachable>} : \erase{ti_1^{*};l_1;\phi_1 \rightarrow ti_2^{*};l_2;\phi_2}$ = $\erase{S};\erase{C} \vdash \<wunreachable> : t_1^{*} \rightarrow t_2$, which holds under \wasm's type system.

        \item $S;C \vdash \<drop> : \epsilon;l_1;\phi_1 \rightarrow \epsilon;l_1;\phi_1$

        $\erase{S};\erase{C} \vdash \erase{\<nop>} : \erase{\epsilon;l_1;\phi_1 \rightarrow \epsilon;l_1;\phi_1}$ = $\erase{S};\erase{C} \vdash \<wnop> : \epsilon \rightarrow \epsilon$, which holds under \wasm's type system.
    \end{itemize}
\end{proof}

\todo{erased well typed admin code}

\todo{erased well typed admin program}
\subsection{Embedding \wasm in \name}
\label{subsec:embedding}
We present a way to embed \wasm programs in \name.
The embedding function takes a \wasm program and replaces all of the type annotations with indexed function types that have no constraints on the variables.
Intuitively, this is the only part of the surface syntax of \wasm that isn't in \name, so we must figure out a way to bring it over.
While this embedding requires no additional developer effort, it provides no information to the indexed type system beyond what can be inferred from the instructions in the program.
We conjecture that a well-typed \wasm program embedded in \name is also well-typed, but we have not proved it.

The embedding replaces all function types used within the \wasm syntax with \name indexed function types, and adds the function types for all of the functions in a table to the table's type declaration.
This occurs within blocks and indirect function calls, as shown in \autoref{def:embed-e}.
The indexed types simply have fresh index variables which are different in the precondition and postcondition, and the primitive types for the stack are known from the \wasm type $t_1^{*} \rightarrow t_2^{*}$
To know what the local variables are, we parameterize the embedding over the types of local variables ($t^{*}$).

We typeset \name instructions in a \tbsf{blue sans serif font} and \wasm instruction in a \trbf{bold red font} to set them apart.

\begin{definition}{$\embed[e]{e}^{t^{*}}=\mathbluesf{e}$}
    \label{def:embed-e}
    \begin{mathpar}
        %% SPACE HACKS
        \arraycolsep=2pt
        \begin{array}{rcl}
            embed_{e^{*}}({\begin{stackTL}
                \<wblock>
                {\begin{stackTL}
                    (t_1^{*}\rightarrow t_2^{*})\;
                    \\e^{*}
                \end{stackTL}}\\
            \<wend>)^{t^{*}}
            \end{stackTL}}
            &=& {\begin{stackTL}
                    \<block>
                    \\ \quad (\ti{t_1}{a_1}^{*};\ti{t}{a_3}^{*};(\circ,\ti{t_1}{a_1}^{*},\ti{t}{a_3}^{*})
                    \\ \quad\; \rightarrow \ti{t_2}{a_2}^{*};\ti{t}{a_4}^{*};(\circ,\ti{t_2}{a_2}^{*},\ti{t}{a_4}^{*}))
                    \\ \quad \embed[e^{*}]{e^{*}}^{t^{*}}
            \end{stackTL}} \\
            && \<nsend>\\

            embed_{e^{*}}({\begin{stackTL}
                \<wloop>
                {\begin{stackTL}
                    (t_1^{*}\rightarrow t_2^{*})\;
                    \\e^{*}
                \end{stackTL}}\\
            \<wend>)^{t^{*}}
            \end{stackTL}}
            &=& {\begin{stackTL}
                    \<loop>
                    \\ \quad (\ti{t_1}{a_1}^{*};\ti{t}{a_3}^{*};(\circ,\ti{t_1}{a_1}^{*},\ti{t}{a_3}^{*})
                    \\ \quad\; \rightarrow \ti{t_2}{a_2}^{*};\ti{t}{a_4}^{*};(\circ,\ti{t_2}{a_2}^{*},\ti{t}{a_4}^{*}))
                    \\ \quad \embed[e^{*}]{e^{*}}^{t^{*}}
            \end{stackTL}} \\
            && \<nsend>\\

            embed_{e^{*}}({\begin{stackTL}
                \<wif>
                {\begin{stackTL}
                    (t_1^{*}\rightarrow t_2^{*})\;
                    \\e^{*}
                \end{stackTL}}\\
            \<wend>)^{t^{*}}
            \end{stackTL}}
            &=& {\begin{stackTL}
                    \<if>
                    \\ \quad (\ti{t_1}{a_1}^{*};\ti{t}{a_3}^{*};(\circ,\ti{t_1}{a_1}^{*},\ti{t}{a_3}^{*})
                    \\ \quad\; \rightarrow \ti{t_2}{a_2}^{*};\ti{t}{a_4}^{*};(\circ,\ti{t_2}{a_2}^{*},\ti{t}{a_4}^{*}))
                    \\ \quad \embed[e]{e_1^{*}}^{t^{*}}\; \embed[e]{e_2^{*}}^{t^{*}}
                \end{stackTL}} \\
            && \<nsend>\\

            embed_{e^{*}}(
                {\begin{stackTL}
                    \<wcallindirect>
                    \\\quad (t_1^{*}\rightarrow t_2^{*}))^{t^{*}}
                \end{stackTL}}
            &=& {\begin{stackTL}
                \<callindirect>
                \\ \quad (\ti{t_1}{a_1}^{*};\ti{t}{a_3}^{*};(\circ,\ti{t_1}{a_1}^{*},\ti{t}{a_3}^{*})
                \\ \quad\; \rightarrow \ti{t_2}{a_2}^{*};\ti{t}{a_4}^{*};(\circ,\ti{t_2}{a_2}^{*},\ti{t}{a_4}^{*}))
            \end{stackTL}} \\

            \embed[e^{*}]{e}^{t^{*}} &=& e \text{, otherwise} \\
            \embed[e^{*}]{e^{*}}^{t^{*}} &=& (\embed[e^{*}]{e}^{t^{*}})^{*} \\
        \end{array}
    \end{mathpar}
\end{definition}

The embedding of functions, \autoref{def:embed-f}, both must construct an indexed function type for itself and embed its body.
Function bodies have their local variables defined by the function that they are enclosed in.
Thus, when the function body is embedded we pass the local types ($t_1^{*}\;t^{*}$) so the body knows how to constrain local variables.
The indexed function type that gets constructed has the precondition of the expected values on the stack turned into indexed types using fresh index variables and the types $t_1^{*}$ from the \wasm type.
The postcondition does the same for the stack with $t_2^{*}$.
We cannot embed imported functions because we have no way of accessing the types of the local variables of the function.

\begin{definition}{$\embed[f]{f}=\mathbluesf{f}$}
    \label{def:embed-f}
    \begin{mathpar}
        \begin{array}{rcl}
            embed_f(\<wfunc> {\begin{stackTL}
                (t_1^{*}\rightarrow t_2^{*})
                \\\<wlocal>\; t^{*}\; e^{*})
            \end{stackTL}}
            &=& \<func>\;
                {\begin{stackTL}
                    (\ti{t_1}{a_1}^{*};\epsilon;(\circ,\ti{t_1}{a_1}^{*})
                    \\ \rightarrow
                    \begin{stackTL}
                        \ti{t_2}{a_2}^{*};\ti{t_1}{a_3}^{*}\;\ti{t}{a_4}^{*};
                        \\ \quad (\circ,\ti{t_2}{a_2}^{*},\ti{t_1}{a_3}^{*}\ti{t}{a_4}^{*}))
                    \end{stackTL}
                    \\ t^{*}\; \embed[e^{*}]{e^{*}}^{(t_1^{*}\;t^{*})}
                \end{stackTL}} \\
            && \<nsend>\\
        \end{array}
    \end{mathpar}
\end{definition}

Tables in \name must also provide the indexed function types of all the functions they contain.
We do this by parameterizing the embedding of the table $tab$ with all of the declared functions $f^{*}$.
Then, we retrieve the indexed function type $tfi$ of the function pointed to by the function index $i$ in $f^{*}$ for every function index $i$ in the table.
We cannot embed imported tables because we have no way of accessing the types of the functions included in the table.

\begin{definition}{$\embed[tab]{tab}^{f^{*}}=\mathbluesf{tab}$}
    \label{def:embed-t}
    \begin{mathpar}
        \begin{array}{rcl}
            embed_{tab}(\<wtab> n\; i^{n})
            &=& \<tab> n\; tfi^{n} \\
            && \text{where } \forall i. f^{*}(i) = \<func> tfi\; \<local>\; t^{*}\; e^{*} \\
        \end{array}
    \end{mathpar}
\end{definition}

Finally, since tables and functions live in modules, which are the pinnacle syntactic object of the \wasm surface syntax hierarchy, we must embed modules.
Embedding a module $module$ means embedding all of the functions $f^{*}$ in the module, and embedding the table $tab$ parameterized with all of the function definitions $f^{*}$.

\begin{definition}{$\embed[module]{module}^{C}=\mathbluesf{module}$}
    \label{def:embed-t}
    \begin{mathpar}
        \begin{array}{rcl}
            embed_{module}(\<module> f^{*}\; glob^{*}\; tab^{?}\; mem^{?})
            &=& \<module>
            \begin{stackTL}
                \embed[f]{f^{*}}
                \\ glob^{*}
                \\ \embed[tab]{tab^{?}}^{f^{*}}
                \\ mem^{?}
            \end{stackTL} \\
        \end{array}
    \end{mathpar}
\end{definition}

These are not the only differences in the surface syntax between \wasm and \name: we also introduced four new instructions (the \prechk-tagged instructions).
The definition of embedding we have introduced has been entirely syntactic, but that will not work for replacing non-\prechk-tagged instructions with \prechk-tagged versions during embedding since we must be able to ensure that stronger guarantees are met.
Instead, one could, for example, check at every $\<div>$, $\<callindirect>$, $\<load>$, and $\<store>$ whether the \prechk-tagged version of the instruction is well-typed, and only if it is well-typed replace the instruction with the \prechk-tagged version.

\section{Type Soundness}
\label{sec:typesoundness}
\emph{Type soundness} is the property that a well-typed program either reduces to another well-typed program, is an irreducible expression (in the case of \name, a sequence of values), or throws an error (trap, in the case of \name).
Thus, type soundness assures us that the behavior of a well-typed program is always well-defined.
The type soundness of \wasm guarantees a number of important properties, including type safety and memory safety.
Proving the type soundness of \name gives us a high degree of assurance that it has the same level of safety properties as \wasm.

\paragraph{Run Time Typing Rules}
\begin{figure}
    \begin{mathpar}
        \inferrule*[right=\defrule{Closure}]{ %% closure
            S_\text{inst}(i) \vdash f : tfi
        } {
            S \vdash \{ \text{inst} \; i, \text{code} \; f \} : tfi
        }

        \inferrule*[right=\defrule{Admin-Const}]{ %% admin const
        } {
            \vdash t.\<const> c : \ti{t}{a};\circ,\ti{t}{a},(\<eq> a \; \ti{t}{c})
        }

        \inferrule*[right=\defrule{Instance}]{ %% instance
            (S \vdash cl : tfi)^{*} \and
            (\vdash v : \ti{t}{a},\phi_v)^{*} \\
            (S_\text{tab}(i) = n)^{?} \and
            (S_\text{mem}(j) = m)^{?}
        } {
            S \vdash
            {\begin{stackTL}
                \{ \text{func} \; cl^{*}, \text{glob} \; v^{*}, \text{tab} \; i^{?}, \text{mem} \; j^{?} \} 
                \\ : \{ \text{func} \; tfi^{*}, \text{global} \; (\text{mut}^{?} \; t)^{*}, \text{table} \; n^{?}, \text{memory} \; m^{?} \}
            \end{stackTL}}
        }

        \inferrule*[right=\defrule{Store}]{ %% store
            S = \{ \text{inst} \; C^{*}, \text{tab} \; n^{*}, \text{mem} \; m^{*} \}\\
            (S \vdash inst : C)^{*} \and
            ((S \vdash cl : tfi)^{*})^{*} \and
            (n \leq |cl^{*}|)^{*} \and
            (m \leq |b^{*}|)^{*}
        } {
            \vdash \{ \text{inst} \; inst^{*}, \text{tab} \; (cl^{*})^{*}, \text{mem} \; (b^{*})^{*} \} : S
        }

        \inferrule*[right=\defrule{Code}]{ %% admin code
            (\vdash v : \ti{t}{a};\phi_v)^{*}\\
            C = S_{\text{inst}}(i),\text{local} \; t^{*}, \text{return} \; (ti^n;l;\phi)^{?}\\
            S;C \vdash e^{*} : \epsilon:\ti{t}{a}^{*};\phi_v^{*} \rightarrow ti^n;l;\phi
        } {
            S;(ti^n;l;\phi)^{?} \vdash_i v^{*};e^{*} : ti^n;l;\phi
        }

        \inferrule*[right=\defrule{Program}]{ %% admin program
            \vdash s : S \and
            S;\epsilon \vdash_i v^{*};e^{*} : ti^{*};l;\phi
        } {
            \vdash_i s;v^{*};e^{*} : ti^{*};l;\phi
        }
    \end{mathpar}
    \caption{\name Program and Store Typing Rules}
    \label{fig:programrules}
\end{figure}

\autoref{fig:programrules} shows the \name typing rules for module instances, the run time store, and \name programs.
Closures are type checked by \refrule{Closure}, which falls back on the module typing rules \autoref{fig:modulerules} to type check the function definition inside of the closure.
\refrule{Admin-Const} gets the postcondition indexed types and constraints on values, it is used to type check local and global variables.

\refrule{Instance} checks that the module instance $inst$ has the type of the index module context $C$.
It checks all of the closures in $inst$ against their expected types in $C$, and similarly for all of the globals and the size of the table and memory.
\refrule{Store} uses \refrule{Instance} to check that a run time store, $s$ is well typed by the store context $S$ by ensuring that every module instance in $s$ has the type of the index module context in $S$
Further, \refrule{Store} ensures that all of the closures in all of the tables in $s$ are well typed, and the the sizes of all the tables and memory chunks in $S$ do not exceed the actual size of their implementations.

\refrule{Code} checks that a sequence of instructions is well typed with an empty stack and the indexed types and constraints for the given local variables in the precondition.
Since local variables are values, we know that each one of them is equal to some constant, so \refrule{Code} is really just checking that the sequence of instructions has some postcondition reachable from the given local variables.
There is an optional return postcondition for \refrule{Code} because \refrule{Local} has as a premise a judgment of the exactly same form, except with a return postcondition.
\refrule{Program} uses \refrule{Code} without using the optional return postcondition, as well as \refrule{Store}, to ensure that a reducible \name program is well-typed.

\begin{figure}
    \begin{mathpar}
        \inferrule*[right=\defrule{Local}]{ %% local
            S;(ti^n;l_2;\phi_2) \vdash_i v_l^{*};e^{*} : ti^n;l_2;\phi_2
        } {
            S;C \vdash \<local> \{ i;v_l^{*} \} \; e^{*} \<end> : \epsilon;l_1;\phi_1 \rightarrow ti^n;l_1;\phi_1,\phi_2
        }

        \inferrule*[right=\defrule{Call-Cl}]{ %% call closure
            S \vdash cl : tfi
        } {
            S;C \vdash \<call> cl : tfi
        }

        \inferrule*[right=\defrule{Trap}]{ %% trap
        } {
            C \vdash \<trap> : tfi
        }

        \inferrule*[right=\defrule{Label}]{ %% label
            C\vdash e_0^{*} : ti_3^{*};l_3;\phi_3 \rightarrow ti_2^{*};l_2;\phi_2 \\
            C,\text{label } (ti_3^{*};l_3;\phi_3) \vdash e^{*} : \epsilon;l_1;\phi_1 \rightarrow ti_2^{*};l_2;\phi_2
        } {
            C \vdash \<label> \{ e_0^{*} \} \; e^{*} \<end> : \epsilon;l_1;\phi_1 \rightarrow ti_2^{*};l_2;\phi_2
        }
    \end{mathpar}
    \caption{\name Administrative Instruction Rules}
    \label{fig:adminrules}
\end{figure}

\autoref{fig:adminrules} extends the \name typing rules for instructions to include administrative instructions.
\refrule{Local} typechecks the local block using \refrule{Code} to ensure that the body is well typed with the indexed types and constraints for local variables provided by the local block as the precondition and any postcondition.
Since local blocks are inline expansions of function calls, we use the optional return postcondition functionality of \refrule{Code} to ensure that returning from inside the local block will be well-typed.
\refrule{Call-Cl} typechecks calling a closure by ensuring that the closure being called has the same type as the call instruction.
\refrule{Trap} is always well-typed under any precondition and postcondition.
Finally, \refrule{Label} typechecks the body of the label block with the precondition of the saved instructions pushed onto the label stack.
This was, if the label was generated by a loop, then the precondition of the saved values is the precondition of the loop, and we know the loop is well-typed.
Otherwise, the saved instructions will be an empty sequence and will be well typed from the precondition.

Given these additional typing judgments and rules, we can now prove the two theorems necessary to show type soundness: subject-reduction (\autoref{subsec:subject-reduction}) and progress (\autoref{subsec:progress}).

\subsection{Subject Reduction}
\label{subsec:subject-reduction}
\emph{Subject reduction}, also sometimes referred to as ``type preservation'', ensures that if a program has a specific type, then the program will have the same type after a reduction step.
Before we present the subject reduction proof, we first introduce a number of useful lemmas.

\reflemma{Inversion} tells us what typing rules can apply to a given \name instruction sequence, and therefore lets us reason about what the type of that sequence looks like.
For example, if we have a typing derivation, $D$ for $S;C \vdash t.\<const> c : ti_1^{*};l_1;\phi_1 \rightarrow ti_2^{*};l_2;\phi_2$, then we know that $D$ must have at its base \refrule{Const}, because that is the only way we have of typing constant instructions.
$D$ can also include any number of applications of \refrule{Subtyping} and \refrule{Stack-Poly}, because they can be applied to any well-typed sequence of instructions.
Thus, we do not know the exact types, since the typing rules are non-deterministic, but we can reason about the general shape given the base type on top of which \refrule{Subtyping} and \refrule{Stack-Poly} get applied.
Additionally, \refrule{Composition} can be used with the empty sequence and any well-typed single instruction.
However, the addition of \refrule{Composition} with the empty sequence is trivial because the postcondition of an empty instruction sequence must be immediately reachable from the precondition, and therefore the stack and local index store must be the same in both (and the postcondition index type context being reachable from the precondition index type context).

Most cases of \reflemma{Inversion} are omitted.
The complete definition can be found in the appendix (\autoref{sec:subreduxproof}).

\begin{lemma}{\deflemma{Inversion}}

    \begin{itemize}
        %% const
        \item If $S;C \vdash t.\<const> c : ti_1^{*};l_1;\phi_1 \rightarrow ti_2^{*};l_2;\phi_2$,
        then $ti_2^{*} = ti_1^{*}\;\ti{t}{a}$, $l_1 = l_2$,
        and $\phi_1,\ti{t}{a},(= a \; \ti{t}{c}) \implies \phi_2$.

        %% binop
        \item If $S;C \vdash t.binop : ti_1^{*};l_1;\phi_1 \rightarrow ti_2^{*};l_2;\phi_2$,
        then $ti_1^{*} = ti^{*} \; \ti{t}{a_1} \; \ti{t}{a_2}$, $ti_2^{*} = ti^{*} \; \ti{t}{a_3}$, $l_1 = l_2$,
        and $\phi_1,\ti{t}{a_3},(= a_3\;(binop\;a_1\;a_2)) \implies \phi_2$.

        %% block
        \item If $S;C \vdash \<block>\; {\begin{stackTL}(ti_3^{*};l_3;\phi_3 \rightarrow ti_4^m;l_4;\phi_4)\\ e^{*} \<end> : ti_1^{*};l_1;\phi_1 \rightarrow ti_2^{*};l_2;\phi_2\end{stackTL}}$
        \\ then $ti_1^{*} = ti_0^{*}\; ti_3^{*}$, $ti_2^{*}=ti_0^{*}\; ti_4^m$, $l_1=l_3$, $l_2=l_4$, $\phi_1 \implies \phi_3$, $\phi_4 \implies \phi_2$, and $S;C,\text{label}(ti_4^m;l_4;\phi_4) \vdash e^{*} : ti_3^{*};l_3;\phi_3 \rightarrow ti_4^m;l_4;\phi_4$.

        %% br
        \item If $S;C \vdash \<br> i : ti_1^{*};l_1;\phi_1 \rightarrow ti_2^{*};l_2;\phi_2$,
        then $ti_1^{*} = ti_3^{*}\;ti^{*}$, $C_\text{label}(i) = ti^{*};l_1;\phi_3$,
        and $\phi_1 \implies \phi_3$.

        %% call_indirect
        \item If $S;C \vdash \<callindirect> ti_3^{*};l_3;\phi_3 \rightarrow ti_4^{*};l_4;\phi_4 : ti_1^{*};l_1;\phi_1 \rightarrow ti_2^{*};l_2;\phi_2$,
        then $ti_1^{*} = ti_0^{*} \; ti_3^{*}$, $ti_2^{*} = ti_0^{*} \; ti_4^{*}$, $l_2=l_1$, $\phi_1 \implies \phi_3$, and $\phi_3,\phi_4 \implies \phi_2$.

        %% composition
        \item If $S;C \vdash e_1^{*} \; e_2 : ti_1^{*};l_1;\phi_1 \rightarrow ti_3^{*};l_3;\phi_3$,
        then $S;C \vdash e_1^{*} : ti_1^{*};l_1;\phi_1 \rightarrow ti_2^{*};l_2;\phi_2$,
        and $S;C \vdash e_2 : ti_2^{*};l_2;\phi_2 \rightarrow ti_3^{*};l_3;\phi_3$.

        \thought{This relies on a bit of a non-obvious idea that even if the precondition is changed through subtyping, the subtyping can be deferred until a later composition.}

        \thought{This also relies on the idea that even if subtyping appears in the derivation tree, it can equivalently be applied on the premises of the composition.}
    \end{itemize}
\end{lemma}
\begin{proof}
    Proof omitted, but follows from induction over typing derivations.
\end{proof}

The next lemma, \reflemma{Lift-Consts}, shows that if a sequence of constants, $v^n$, has a certain postcondition within a nested context, $L^j$, then it has the same postcondition outside of that context with the precondition of the context.
We use this rule for branching and returning when we have some values $v^n$ inside a reduction context $L^j$.

The intuition for the proof is that the nature of nested contexts are such that all of the instructions preceding $v^n$ are values and therefore only add fresh index variables which are constrained to be equal to constants.
Thus, we can pull $v^n$ outside of the nested context and know that we can still get to the postcondition because we can add back in, using implication, all of the fresh index variables that we would have added from the values preceding.

\begin{lemma}{\deflemma{Lift-Consts}}

    If $S;C \vdash v^n : \epsilon;l_3;\phi_3 \rightarrow ti^n;l_3;\phi_4$ is a subderivation of $S;C \vdash  L^j [v^n] : s_1;l_1;\phi_1 \rightarrow s_2;l_2;\phi_2$,
    \\then $S;C \vdash v^n : \epsilon;l_1;\phi_1 \rightarrow ti^n;l_3;\phi_4$ after reduction
\end{lemma}
\begin{proof}
    By induction on $j$.
    \begin{itemize}
        \item Base case: $j=0$

            We want to show that $S;C \vdash v^n : \epsilon;l_1;\phi_1 \rightarrow ti^n;l_3;\phi_4$ after reduction.

            We have $S;C \vdash v_0^{*} \; v^n \; e^{*} \<end> : s_1;l_1;\phi_1 \rightarrow s_2;l_2;\phi_2$ for some $v_0^{*}$ and $e^{*}$ by expanding $L^0$.

            Then, $S;C \vdash (t.\<const> c)^{*} : \epsilon;l_1;\phi_0 \rightarrow \ti{t}{a}^{*};l_1;\phi_0,\ti{t}{a}^{*},(\<eq> a \; \ti{t}{c})$ where $v_0^{*}=(t.\<const> c)^{*}$ and $\phi_1 \implies \phi_0$ by \reflemma{Inversion} on \refrule{Const}.

            Further, $S;C \vdash v^n : \epsilon;l_3;\phi_0,\ti{t}{a}^{*},(\<eq> a \; \ti{t}{c}) \rightarrow \ti{t}{a}^{*}\;ti^n;l_3;\phi_4$, by \reflemma{Inversion} on \refrule{Const}.

            We now have all the information we need to show what we want to show.

            We know $\phi_0,\ti{t}{a}^{*},(\<eq> a \; \ti{t}{c}) \implies \phi_3$.

            Recall that $S;C \vdash v^n : \epsilon;l_3;\phi_0,\ti{t}{a}^{*},(\<eq> a \; \ti{t}{c}) \rightarrow \ti{t}{a}^{*}\;ti^n;l_3;\phi_4$, then $$S;C \vdash v^n : \ti{t}{a}^{*};l_3;\phi_0,\ti{t}{a}^{*},(\<eq> a \; \ti{t}{c}) \rightarrow \ti{t}{a}^{*},\ti{t}{a}^{*}\;ti^n;l_3;\phi_4$$ by \refrule{Subtyping}.

            If $v_0^{*}$ are not executed (\ie they are not part of the reduced expression), then $a^{*}$ are fresh, so $\phi_0 \implies \phi_0,\ti{t}{a}^{*},(\<eq> a \; \ti{t}{c})$, and therefore $S;C \vdash v^n : \epsilon;l_1;\phi_0 \rightarrow ti^n;l_1;\phi_4$ by \refrule{Subtyping} and since $l_1=l_3$.

            Then, $S;C \vdash v^n : \epsilon;l_1;\phi_1 \rightarrow ti^n;l_1;\phi_4$ by $subtyping$.

        \item Induction case: $j=k+1$

            We want to show that $S;C \vdash v^n : \epsilon;l_1;\phi_1 \rightarrow ti^n;l_3;\phi_4$ after reduction.

            We have $S;C \vdash \<label>_n \{ e_0^{*} \} \; v_0^{*} \; L^k[v^n] \; e_1^{*} \<end> : s_1;l_1;\phi_1 \rightarrow s_2;l_2;\phi_2$ for some $v_0^{*}$, $e_0^{*}$, and $e_1^{*}$ by expanding $L^j$.

            Then, $S;C \vdash (t.\<const> c)^{*} : \epsilon;l_1;\phi_0 \rightarrow \ti{t}{a}^{*};l_1;\phi_0,\ti{t}{a}^{*},(\<eq> a \; \ti{t}{c})$ where $v_0^{*}=(t.\<const> c)^{*}$ and $\phi_1 \implies \phi_0$ by \reflemma{Inversion} on \refrule{Const}.

            Further, $S;C \vdash L^k[v^n] : \ti{t}{a}^{*};l_1;\phi_0,\ti{t}{a}^{*},(\<eq> a \; \ti{t}{c}) \rightarrow s_5;l_5;\phi_5$ for some $s_5;l_5;\phi_5$ by \reflemma{Inversion} on \refrule{Label}.

            Now we can prove want we wanted to show.

            We know $S;C \vdash v^n : \epsilon;l_1;\phi_0,\ti{t}{a}^{*},(\<eq> a \; \ti{t}{c}) \rightarrow ti^n;l_1;\phi_4$ by the inductive hypothesis.

            If $v_0^{*}$ are not executed (\ie after one reduction step), $a^{*}$ are fresh, so $\phi_0 \implies \ti{t}{a}^{*},(\<eq> a \; \ti{t}{c})$, and therefore $S;C \vdash v^n : \epsilon;l_1;\phi_0 \rightarrow \ti{t}{a}^{*}\;ti^n;l_3;\phi_5$ by \refrule{Subtyping} and since $l_1=l_3$.

            Then, $S;C \vdash v^n : \epsilon;l_1;\phi_1 \rightarrow \ti{t}{a}^{*}\;ti^n;l_3;\phi_3$ by \refrule{Subtyping}.

    \end{itemize}
\end{proof}


In many reduction cases, there are values on the stack that get consumed by reducing an instruction.
This creates a bit of a problem because those values represent intermediate state, and as such will introduce new index variables to the index type context in their postcondition.
After reduction, the intermediate state is no longer present, so we lose those index variables from the postconditions.

For example, $(t.\<const> c)\; \<drop>$ could be typed as $\epsilon;l;\phi \rightarrow \epsilon;l;\phi,\ti{t}{a},(= a\; \ti{t}{c})$ where $a$ represent the value on the stack $t.\<const> c$.
This would reduce to $\epsilon$, and then we lose the information about $a$ in the postcondition index type system.
However, this can be solved using implication, as we know $a$ is fresh from the const rule, and therefore we allow saying $\phi \implies \phi,\ti{t}{a},(= a\; \ti{t}{c})$ after reduction.
This pattern will appear in any case of the proof that consumes values.

\begin{theorem}{Subject Reduction}
  If $\vdash_i s;v^{*};e^{*} : ti^{*};l;\phi$ and $s;v^{*};e^{*} \hookrightarrow_i s';v'^{*};e'^{*}$ then $\vdash_i s';v'^{*};e'^{*} : ti^{*};l;\phi$.
\end{theorem}
\begin{proof}
By case analysis on the reduction rules.

\begin{itemize}
    %% Binop -> const
    \item $C\vdash (t.\<const> c_1)\; (t.\<const> c_2)\; t.binop : ti_1^{*};l_1;\phi_1 \rightarrow ti_2^{*};l_2;\phi_2$
    \\ $\land$ $(t.\<const> c_1)\; (t.\<const> c_2)\; t.binop \hookrightarrow t.\<const> c$ where $c=binop(c_1,c_2)$

        By $inversion$ on $const$ and $binop$, we know that $ti_2^{*} = ti_1^{*} \ti{t}{a_3}$, $l_2=l_1$, and that
        \begin{align*}
            \phi_1&,
            \begin{stackTL}
                \ti{t}{a_1}, (= a_1\; \ti{t}{c_1}), \\
                \ti{t}{a_2}, (= a_2\; \ti{t}{c_2}), \\
                \ti{t}{a_3}, (= a_3\; (binop\; a_1\; a_2))
            \end{stackTL} \\
            &\implies \phi_2
        \end{align*}

        By $const$, $C \vdash t.\<const> c :
            \begin{stackTL}
                \epsilon;l_1;\phi_1 \\
                \rightarrow \ti{t}{a_3};l_1;g_1;\phi_1,\ti{t}{a_3},(\<eq> a_3\;\ti{t}{c})
            \end{stackTL}$.

        Because $c=binop_t(c_1,c_2)$, then by $\implies$,
        \begin{align*}
            \phi_1,\ti{t}{a},(= a\; \ti{t}{c}) &\implies \phi_1,
            \begin{stackTL}
                \ti{t}{a_1}, (= a_1\; \ti{t}{c_1}), \\
                \ti{t}{a_2}, (= a_2\; \ti{t}{c_2}), \\
                \ti{t}{a_3}, (= a_3\; (binop\; a_1 a_2))
            \end{stackTL}
        \end{align*}

        Therefore, $C \vdash (t.\<const> c) : ti_1^{*};l_1;\phi_1 \rightarrow ti_1^{*}\; \ti{t}{a_3};l_1;\phi_2$, by $stack-poly$ and $sub-typing$

    %% Binop -> trap
    \item  $C\vdash (t.\<const> c_1)\; (t.\<const> c_2)\; t.binop : ti_1^{*};l_1;\phi_1 \rightarrow ti_2^{*};l_2;\phi_2$
    \\ $\land$ $(t.\<const> c_1)\; (t.\<const> c_2)\; t.binop \hookrightarrow \<trap>$

        Trivially, $C\vdash \<trap> : ti_1^{*};l_1;\phi_1 \rightarrow ti_2^{*};l_2;\phi_2$ by $trap$.

    %% Relop
    \item $C\vdash (t.\<const> c_1)\; (t.\<const> c_2)\; t.relop : ti_1^{*};l_1;\phi_1 \rightarrow ti_2^{*};l_2;\phi_2$
    \\ $\land$ $(t.\<const> c_1)\; (t.\<const> c_2)\; t.relop \hookrightarrow t.\<const> c$ where $c=relop(c_1,c_2)$

        Similar to $binop$.

    %% Testop
    \item $C\vdash (t.\<const> c)\; t.testop : ti_1^{*};l_1;\phi_1 \rightarrow ti_2^{*};l_2;\phi_2$
    \\ $\land$ $(t.\<const> c)\; t.testop \hookrightarrow \<ithreetwo>.\<const> c_2$ where $c_2=testop(c)$

        \todo{This is wonky}

        By $inversion$ on $const$ and $testop$, we know that $ti_2^{*}=ti_1^{*}\; \ti{t}{a_2}$, $l_2=l_1$, and that
        \begin{align*}
            \phi_1&,
            \begin{stackTL}
                \ti{t}{a_1}, (= a_1\;\ti{t}{c}), \\
                \ti{\<ithreetwo>}{a_2}, (= a_2\;(testop\;a_1))
            \end{stackTL} \\
            &\implies \phi_2
        \end{align*}

        By $const$, $C \vdash t.\<const> c :
            \begin{stackTL}
                \epsilon;l_1;g_1;\phi_1 \\
                \rightarrow \ti{\<ithreetwo>}{a_2};l_1;\phi_1,\ti{\<ithreetwo>}{a_2},(= a_2\;\ti{t}{c_2})
            \end{stackTL}$.

        Because $c_2=testop_t(c)$, then by $\implies$,
        \begin{align*}
            \phi_1,\ti{t}{a},(= a\;\ti{t}{c_2}) &\implies \phi_1,
            \begin{stackTL}
                \ti{t}{a_1}, (= a_1\;\ti{t}{c}), \\
                \ti{\<ithreetwo>}{a_2}, (= a_2\;(testop\;a_1))
            \end{stackTL}
        \end{align*}

    %% Unreachable
    \item $C\vdash \<unreachable> : ti_1^{*};l_1;\phi_1 \rightarrow ti_2^{*};l_2;\phi_2$
    \\ $\land$ $\<unreachable> \hookrightarrow \<trap>$

        Trivially, $C\vdash \<trap> : ti_1^{*};l_1;\phi_1 \rightarrow ti_2^{*};l_2;\phi_2$ by $trap$.

    %% Nop
    \item $C\vdash \<nop> : ti_1^{*};l_1;\phi_1 \rightarrow ti_2^{*};l_2;\phi_2$
    \\ $\land$ $\<nop> \hookrightarrow \epsilon$

        By $inversion$ on $nop$, we know that $ti_2^{*} = ti_1^{*}$, $l_2 = l_1$, and $\phi_1 \implies \phi_0$ and $\phi_0 \implies \phi_2$ for some $\phi_0$.

        $C\vdash \epsilon : \epsilon;l;g;\phi_0 \rightarrow \epsilon;l;g;\phi_0$ by $empty$.

        Then, $C \vdash \epsilon ti_1^{*};l;g;\phi_1 \rightarrow ti_1^{*};l;g;\phi_2$ by $stack-poly$ and $sub-typing$.

    %% Drop
    \item $C\vdash (t.\<const> c)\; \<drop> : ti_1^{*};l_1;\phi_1 \rightarrow ti_2^{*};l_2;\phi_2$
    \\ $\land$ $(t.\<const> c)\; \<drop> \hookrightarrow \epsilon$

        By $inversion$ on $compostion$, $const$, and $drop$, we know that $ti_2^{*} = ti_1^{*}$, $l_2 = l_1$, and $\phi_1 \implies \phi_0$ and $\phi_0 \implies \phi_2$ for some $\phi_0$.

        By $empty$, $C\vdash \epsilon : \epsilon;l_1;\phi_0 \rightarrow \epsilon;l_1;\phi_0$.

        Then, $C\vdash \epsilon : ti_1^{*};l_1;\phi_1 \rightarrow ti_1^{*};l_1;\phi_2$ by $stack-poly$ and $sub-typing$.

    %% Select
    \item Case: $C\; {\begin{stackTL}
        \vdash (t.\<const> c_1)\;(t.\<const> c_2)\;(\<ithreetwo>.\<const> 0)\;\<select>
        \\ : ti_1^{*};l_1;\phi_1 \rightarrow ti_2^{*};l_2;\phi_2
    \end{stackTL}}$
    \\ $\land$ $(t.\<const> c_1)\;(t.\<const> c_2)\;(\<ithreetwo>.\<const> 0)\;\<select> \hookrightarrow (t.\<const> c_2)$

        By $const$ and $select$, we know that $ti_2^{*} = ti_1^{*}\;\ti{a_3}$, $l_2 = l_1$, and
        $
        {\begin{stackTL}
            \phi_1, {\begin{stackTL}
                \ti{t}{a_1}, (= a_1\;\ti{t}{c_1}), \\
                \ti{t}{a_2}, (= a_2\;\ti{t}{c_2}), \\
                \ti{\<ithreetwo>}{a}, (= a\;\ti{\<ithreetwo>}{0}), \\
                \ti{t}{a_3},(if\; (= a\; \ti{\<ithreetwo>}{0})\; (= a_3\; a_2)\; (= a_3\; a_1))
            \end{stackTL}} \\
            \implies \phi_2
        \end{stackTL}}
        $

        By $const$, \\
        $ C \vdash (t.\<const> c_2) :
            {\begin{stackTL}
                \epsilon;l_1;\phi_1 \\
                \rightarrow \ti{t}{a_3};l_1;\phi_1,\ti{t}{a_3},(= a_3\; \ti{t}{c_2}) \\
            \end{stackTL}} $

        $C \vdash (t.\<const> c_2) : ti_1^{*};l_1;\phi_1 \rightarrow ti_1^{*}\;\ti{t}{a_3};l_1;\phi_1,\ti{t}{a_3},(= a_3 \; \ti{t}{c_2})$ by $stack-poly$.

        By $\implies$, we have \\
        $\phi_1,\ti{t}{a_3},(\<eq> a_3\; \ti{t}{c_2}) \implies \phi_1, {\begin{stackTL}
            \ti{t}{a_1}, (\<eq> a_1\; \ti{t}{c_1}), \\
            \ti{t}{a_2}, (\<eq> a_2\; \ti{t}{c_2}), \\
            \ti{\<ithreetwo>}{a}, (= a\;\ti{\<ithreetwo>}{0}), \\
            \ti{t}{a_3},(if\; (= a\; \ti{\<ithreetwo>}{0})\; (= a_3\; a_2)\; (= a_3\; a_1))
        \end{stackTL}} \\ $

        Therefore,
        $ C \vdash (t.\<const> c_2) :
        ti_1^{*};l_1;\phi_1
            \rightarrow ti_2^{*}\;\ti{t}{a_3};l_1;\phi_2$ by $sub-typing$

    %% Block
    \item Case: $C \vdash v^n \; \<block> tfi \; e^{*} \<end> : ti_1^{*};l_1;\phi_1 \rightarrow ti_2^{*};l_2;\phi_2$
    \\ $\land$ $v^n \; \<block> tfi \; e^{*} \<end> \hookrightarrow \<label>_m \{ \epsilon \} \; v^n \; e^{*} \<end>$

        Let $ti_3^n;l_3;\phi_3 \rightarrow ti_4^m;l_4;\phi_4=tfi$, $(t.\<const> c)^n=v^n$.

        $C \vdash \<block> tfi \; e^{*} \<end> : ti_1^{*}\; \ti{t}{a}^n;l_1;\phi_1,\ti{t}{a}^n,(\<eq> a \; \ti{t}{c})^n \rightarrow ti_2^{*};l_2;\phi_2$ by $inversion$ on $composition$ and $const$.

        Therefore, by $inversion$ on $block$, $l_1=l_3$ and $l_2=l_4$. We will use $l_1,l_2$ in place of $l_3,l_4$, respectively, for the remainder of the proof case.

        Further, $\ti{t}{a}^n=ti_3^n$, $ti_2^{*}=ti_1^{*}\; ti_4^m$, $\phi_1,\ti{t}{a}^n,(\<eq> a \; \ti{t}{c})^n \implies \phi_3$, and $\phi_4 \implies \phi_2$ by $inversion$ on $block$.

        $C,\text{label}(t_4^{m};l_2;\phi_4) \vdash (t.\<const> c)^n : \epsilon;l_1;\phi_1 \rightarrow \\ \ti{t}{a}^n;l_1;\phi_1,\ti{t}{a}^n,(\<eq> a \; \ti{t}{c})^n$ by $const$.

        $C,\text{label}(t_4^{m};l_2;g_2;\phi_4) \vdash (t.\<const> c)^n : \epsilon;l_1;\phi_1 \rightarrow \\ \ti{t}{a}^n;l_1;\phi_3$ by $sub-typing$.

        $C,\text{label}(t_4^{m};l_2;\phi_4) \vdash e^{*} : \ti{t}{a}^n;l_1;\phi_3 \rightarrow ti_4^m;l_2;\phi_4$ because it is a sub-derivation of $block$ which we have already assumed to hold.

        Then $C,\text{label}(t_4^{m};l_2;\phi_4) \vdash (t.\<const> c)^n\; e^{*} : \epsilon;l_1;\phi_1 \rightarrow \\ ti_4^m;l_2;\phi_4$ by $composition$.

        By $empty$ and $stack-poly$, $C \vdash \epsilon : ti_2^m;l_2;\phi_4 \rightarrow ti_2^m;l_2;\phi_4$.

        Therefore, $C \vdash \<label>_m \{ \epsilon \} \; v^n \; e^{*} \<end> : \epsilon;l_1;\phi_1 \rightarrow ti_2^m;l_2;\phi_4$ by $label$.

        $C \vdash \<label>_m \{ \epsilon \} \; v^n \; e^{*} \<end> : ti_1^{*};l_1;\phi_1 \rightarrow ti_1^{*}\; ti_4^m;l_2;\phi_2$ by $stack-poly$ and $sub-typing$.

    \item Case: $C \vdash v^n \; \<loop> tfi \; e^{*} \<end> : ti_1^{*};l_1;\phi_1 \rightarrow ti_2^{*};l_2;\phi_2$
    \\ $\land$ $v^n \; \<loop> tfi \; e^{*} \<end> \hookrightarrow \<label>_n \{ \<loop> tfi \; e^{*} \<end> \} \; v^n \; e^{*} \<end>$

        Let $ti_3^n;l_3;\phi_3 \rightarrow ti_4^m;l_4;\phi_4=tfi$, $(t.\<const> c)^n=v^n$.

        $C \vdash \<loop> tfi \; e^{*} \<end> : ti_1^{*}\; \ti{t}{a}^n;l_1;\phi_1,\ti{t}{a}^n,(\<eq> a \; \ti{t}{c})^n \rightarrow ti_2^{*};l_2;g_2;\phi_2$ by $inversion$ on $composition$ and $const$.

        Therefore, by $inversion$ on $loop$, $l_1=l_3$ and $l_2=l_4$. We will use $l_1,l_2$ in place of $l_3,l_4$, respectively, for the remainder of the proof case.

        Further, $\ti{t}{a}^n=ti_3^n$, $ti_2^{*}=ti_1^{*}\; ti_4^m$, $\phi_1,\ti{t}{a}^n,(\<eq> a \; \ti{t}{c})^n \implies \phi_3$, and $\phi_4 \implies \phi_2$ by $inversion$ on $loop$.

        $C,\text{label}(t_3^{n};l_1;\phi_3) \vdash (t.\<const> c)^n : \epsilon;l_1;\phi_1 \rightarrow \\ \ti{t}{a}^n;l_1;\phi_1,\ti{t}{a}^n,(\<eq> a \; \ti{t}{c})^n$ by $const$.

        $C,\text{label}(t_3^{n};l_1;\phi_3) \vdash (t.\<const> c)^n : \epsilon;l_1;\phi_1 \rightarrow \\ \ti{t}{a}^n;l_1;\phi_3$ by $sub-typing$.

        $C,\text{label}(t_3^{n};l_1;\phi_3) \vdash e^{*} : ti_1^n;l_1;\phi_3 \rightarrow ti_2^m;l_1;\phi_4$ because it is a sub-derivation of $loop$ which we have already assumed to hold.

        Then $C,\text{label}(t_3^{n};l_1;\phi_3) \vdash (t.\<const> c)^n\; e^{*} : \epsilon;l_1;\phi_1 \rightarrow \\ ti_4^m;l_2;\phi_4$ by $composition$.

        $C \vdash \<loop> tfi \; e^{*} \<end> : \ti{t}{a}^n;l_1;\phi_1,\ti{t}{a}^n,(\<eq> a \; \ti{t}{c})^n \rightarrow ti_4^{m};l_2;g_2;\phi_4$ by $loop$.

        Therefore, $C \vdash \<label>_m \{ \<loop> tfi \; e^{*} \<end> \} \; v^n \; e^{*} \<end> : \epsilon;l_1;\phi_1 \rightarrow ti_4^m;l_2;\phi_4$ by $label$.

        $C \vdash \<label>_m \{ \epsilon \} \; v^n \; e^{*} \<end> : ti_1^{*};l_1;\phi_1 \rightarrow ti_2^{*};l_2;\phi_2$ by $stack-poly$ and $sub-typing$.

    \item Case: $C \vdash (\<ithreetwo>.\<const> 0) \; \<if> tfi \; e_1^{*} \<else> e_2^{*} \<end> : ti_1^{*};l_1;\phi_1 \rightarrow ti_2^{*};l_2;\phi_2$
    \\ $\land$ $(\<ithreetwo>.\<const> 0) \; \<if> tfi \; e_1^{*} \<else> e_2^{*} \<end> \hookrightarrow \<block> tfi \; e_2^{*} \<end>$

        Let $tfi = ti_3^n \; \ti{<ithreetwo>}{a};l_3;\phi_3 \rightarrow ti_4^m;l_4;\phi_4$, \\ $tfi_1 = ti_3^n;l_3;\phi_3,\neg(= a\; \ti{\<ithreetwo>}{0}) \rightarrow ti_4^m;l_4;\phi_4$, \\
        and $tfi_2 = ti_3^n;l_3;\phi_3,(= a\; \ti{\<ithreetwo>}{0}) \rightarrow ti_4^m;l_4;\phi_4$.

        By $inversion$ on $composition$, $const$, and $if$, $ti_1^{*}=ti_0^{*}\; ti_3^{n}$ and $ti_2^{*}=ti_0^{*} \; ti_4^{m}$ for some $ti_0^{*}$, $l_1=l_3$, $l_2=l_4$, $\phi_1,\ti{\<ithreetwo>}{a},(\<eq> a\; 0) \implies \phi_3$, and $\phi_4 \implies \phi_2$.

        $C,\text{label}(ti_4^m;l_4;\phi_4) \vdash e_2^{*} : tfi_2$ because it is a sub-derivation of $if$ which we have assumed to hold.

        Then, $C \vdash \<block> tfi_2 \; e_2^{*} \<end>$ by $block$.

        Since $a$ is fresh after reduction, $\phi_1 \implies \phi_1,\ti{t}{a},(\<eqz> a)$ by $\implies$.

        Therefore, $C \vdash \<block> tfi_2\; e_2^{*} \<end> : \\ ti_0^{*}\; ti_3^n;l_1;\phi_1,\ti{t}{a},(\<eqz> a) \rightarrow s\; ti_0^{*}\;ti_4^m;l_2;\phi_2$ by $extension$ and $sub-typing$.

    \item Case: $C \vdash (\<ithreetwo>.\<const> k+1) \; \<if> tfi \; e_1^{*} \<else> e_2^{*} \<end> : ti_1^{*};l_1;\phi_1 \rightarrow ti_2^{*};l_2;\phi_2$
    \\ $\land$ $(\<ithreetwo>.\<const> k+1) \; \<if> tfi \; e_1^{*} \<else> e_2^{*} \<end> \hookrightarrow \<block> tfi \; e_1^{*} \<end>$

        Similar to above.

    \item Case: $C \vdash \<label>_n \{ e^{*} \} \; v^n \<end> : ti_1^{*};l_1;\phi_1 \rightarrow ti_2^{*};l_2;\phi_2$
    \\ $\land$ $\<label>_n \{ e^{*} \} \; v^n \<end> \hookrightarrow v^n$

        $C \vdash \<label>_n \{ e^{*} \} \; v^n \<end> : \epsilon;l_1;\phi_1 \rightarrow ti_4^{n};l_2;\phi_2$ by $inversion$ on $label$.

        By $inversion$, we know $ti_2^{*}=ti_1^{*}\;ti_4^{n}$.

        $C \vdash v^n : \epsilon;l_1;\phi_1 \rightarrow ti_4^{n};l_2;\phi_2$ because it is a premise of $label$ which we have assumed to hold.

        Therefore, $C \vdash v^n : ti_1^{*};l_1;\phi_1 \rightarrow ti_1^{*}\;ti_4^{n};l_1;\phi_2$ by $stack-poly$.

    \item Case: $C \vdash \<label>_n \{ e^{*} \} \; \<trap> \<end> : ti_1^{*};l_1;\phi_1 \rightarrow ti_2^{*};l_2;\phi_2$
    \\ $\land$ $\<label>_n \{ e^{*} \} \; \<trap> \<end> \hookrightarrow \<trap>$

        Trivially, $C\vdash \<trap> : ti_1^{*};l_1;\phi_1 \rightarrow ti_2^{*};l_2;\phi_2$ by $trap$.

    \item Case: $C \vdash \<label>_n \{ e^{*} \} \; L^j [v^n \; (\<br> j)] \<end> : ti_1^{*};l_1;\phi_1 \rightarrow ti_2^{*};l_2;\phi_2$
    \\ $\land$ $\<label>_n \{ e^{*} \} \; L^j [v^n \; (\<br> j)] \hookrightarrow v^n \; e^{*}$

        By $inversion$, $ti_2^{*}=ti_1^{*}\;ti_4^{*}$.

        Let $(t.\<const> c)^n = v^n$.

        $C,\text{label}(ti_1^n;l_3;\phi_5)^j \vdash v^n\; (\<br> j) : \epsilon;l_3;\phi_3 \rightarrow ti_\emptyset^{*};l_\emptyset;\phi_\emptyset$ for some $l_3$ and $\phi_3$, where $\phi_5=\phi_3,\ti{t}{a}^n,(= a\; \ti{t}{c})^n$, by $inversion$ on $label$ and $br$.

        $C,\text{label}(ti_1^n;l_3;\phi_5)^j \vdash (\<br> j) : ti_1^n;l_3;\phi_5 \rightarrow ti_\emptyset^{*};l_\emptyset;\phi_\emptyset$, by $inversion$ on $composition$ and $const$.

        Then, $C,\text{label}(ti_1^n;l_3;\phi_5)^j \vdash v^n : \epsilon;l_3;\phi_3 \rightarrow ti_1^n;l_3;\phi_5$ since it is a premise of $composition$ which we have assumed to hold.

        $C \vdash e^{*} : ti_1^n;l_3;\phi_5 \rightarrow ti_2^{*};l_2;\phi_4$ since it is a premise of $label$ which we have assumed to hold, and $\phi_4 \implies \phi_2$ by $inversion$ on $label$.

        Then, $C \vdash v^n \; e^{*} : \epsilon;l_1;\phi_1 \rightarrow ti_2^{*};l_2;\phi_4$ by $nested-type-preserved$ and $composition$.

        Finally, $C \vdash v^n \; e^{*} : ti_1^{*};l_1;\phi_1 \rightarrow ti_1^{*}\;ti_4^{*};l_2;\phi_2$ by $stack-poly$ and $sub-typing$.

    \item Case: $C \vdash (\<ithreetwo>.\<const> 0)\;(\<brif> j) : ti_1^{*};l_1;\phi_1 \rightarrow ti_2^{*};l_2;\phi_2$
    \\ $\land$ $(\<ithreetwo>.\<const> 0)\;(\<brif> j) \hookrightarrow \epsilon$

        $ti_1^{*}=ti_2^{*}$, $l_1=l_2$, and $\phi_1,\ti{\<ithreetwo>}{a},(\<eq> a\; \ti{\<ithreetwo>}{0}),(\<eqz> a) \implies \phi_2$ by $inversion$ on $composition$, $const$, and $br \_ if$.

        $C \vdash \epsilon : \epsilon;l_1;\phi_1 \rightarrow \epsilon;l_1;\phi_1$ by $empty$.

        $C \vdash \epsilon : ti_1^{*};l_1;\phi_1 \rightarrow ti_1^{*};l_1;\phi_1$ by $stack-poly$.

        $\phi_1 \implies \phi_1,\ti{\<ithreetwo>}{a},(\<eq> a\; \ti{\<ithreetwo>}{0}),(\<eqz> a)$ because $a$ is fresh after reduction, and therefore $\phi_1 \implies \phi_2$.

        Then, $C \vdash \epsilon : ti_1^{*};l_1;\phi_1 \rightarrow ti_1^{*};l_1;\phi_2$ by $sub-typing$.

    \item Case: $C \vdash (\<ithreetwo>.\<const> k+1)\;(\<brif> j) : ti_1^{*};l_1;\phi_1 \rightarrow ti_2^{*};l_2;\phi_2$
    \\ $\land$ $(\<ithreetwo>.\<const> k+1)\;(\<brif> j) \hookrightarrow \<br> j$

        $C_label(j)=ti_1^{*};l_1;\phi_1,\ti{t}{a},\neg(\<eqz> a)$ because it is a side condition of $br\_if$ which we have assumed to hold.

        $C \vdash \<br> j : ti_1^{*};l_1;\phi_1,\ti{t}{a},\neg(\<eqz> a) \rightarrow ti_2^{*};l_2;\phi_2$ by $br$.

        Because $a$ is fresh after reduction, $\phi_1 \implies \phi_1,\ti{\<ithreetwo>}{a},\neg(\<eqz> a)$.

        Therefore, $C \vdash \<br> j : ti_1^{*};l_1;\phi_1 \rightarrow ti_2^{*};l_2;\phi_2$ by $sub-typing$.

    \item Case: $C \vdash (\<ithreetwo>.\<const> k)\;(\<brtable> j_1^k\; j\; j_2^{*}) : ti_1^{*};l_1;\phi_1 \rightarrow ti_2^{*};l_2;\phi_2$
    \\ $\land$ $(\<ithreetwo>.\<const> k)\;(\<brtable> j_1^k\; j\; j_2^{*}) \hookrightarrow \<br> j$

        By $inversion$, we know that $C_\text{label}(j) = ti^{*};l_1;\phi_3$, $ti_1^{*} = ti_0^{*} \; ti^{*}$ for some $ti_0^{*}$, and $phi_1 \implies \phi_3$.

        $C \vdash \<br> j : ti_1^{*};l_1;\phi_3 \rightarrow ti_2^{*};l_2;\phi_2$ by $br$.

        $C \vdash \<br> j : ti_1^{*};l_1;\phi_1 \rightarrow ti_2^{*};l_2;\phi_2$ by $sub-typing$.

    \item Case: $C \vdash (\<ithreetwo>.\<const> k+n)\;(\<brtable> j_1^k\; j) : ti_1^{*};l_1;\phi_1 \rightarrow ti_2^{*};l_2;\phi_2$
    \\ $\land$ $(\<ithreetwo>.\<const> k+n)\;(\<brtable> j_1^k\; j) \hookrightarrow \<br> j$

        By $inversion$, we know that $C_\text{label}(j) = ti^{*};l_1;\phi_3$, $ti_1^{*} = ti_0^{*} \; ti^{*}$ for some $ti_0^{*}$, and $phi_1 \implies \phi_3$.

        $C \vdash \<br> j : ti_1^{*};l_1;\phi_3 \rightarrow ti_2^{*};l_2;\phi_2$ by $br$.

        $C \vdash \<br> j : ti_1^{*};l_1;\phi_1 \rightarrow ti_2^{*};l_2;\phi_2$ by $sub-typing$.

    \item Case: $S;C \vdash \<call> j : ti_1^{*};l_1;\phi_1 \rightarrow ti_2^{*};l_2;\phi_2$
    \\ $\land$ $s;\<call> j \hookrightarrow_i \<call> s_\text{func}(i,j)$

        By $inversion$, we know that $l_2 = l_1$, $ti_1^{*} = ti^{*} \; ti_3^{*}$, $ti_1^{*} = ti^{*} \; ti_4^{*}$, $\phi_1 \implies \phi_3$, and $\phi_3,\phi_4 \implies \phi_2$, where $ti_3^{*};l_3;\phi_3 \rightarrow ti_4^{*};l_4;\phi_4 = C_\text{func}(j)$.

        We know $S \vdash s_\text{inst}(i) : C$ since it is a premise of $\vdash s : S$ which we have assumed to hold.

        Then we know $S \vdash s_\text{func}(i,j) : ti_3^{*};l_3;\phi_3 \rightarrow ti_4^{*};l_4;\phi_4$ because it is a premise of $S \vdash s_\text{inst}(i) : C$.

        Therefore, $S;C\vdash \<call> s_\text{func}(i,j) : ti_3^{*};l_3;\phi_3 \rightarrow ti_4^{*};l_4;\phi_4$ by $call-cl$.

        $S;C\vdash \<call> s_\text{func}(i,j) : ti_1^{*};l_1;\phi_1 \rightarrow ti_2^{*};l_2;\phi_2$ by $stack-poly$ and $sub-typing$.

    \item Case: $S;C \vdash_i (\<ithreetwo>.\<const> j)\; \<callindirect> tfi : ti_1^{*};l_1;\phi_1 \rightarrow ti_2^{*};l_2;\phi_2$
    \\ $\land$ $s;(\<ithreetwo>.\<const> j)\; \<callindirect> tfi \hookrightarrow_i \<call> s_\text{tab}(i,j)$ where $s_\text{tab}(i,j)_\text{code}=(\<func> tfi_0\; \<local>\; t^{*}\; e^{*})$ and $tfi_0 <: tfi$

        Let $ti_3^{*};l_3;\phi_3 \rightarrow ti_4^{*};l_4;\phi_4 = tfi$

        By $inversion$ on $composition$, $const$, and $call-indirect$, we know that $ti_1^{*}=ti_0^{*}\; ti_3^{*}$ and $ti_2^{*}=ti_0^{*}\; ti_4^{*}$ for some $ti_0^{*}$, $l_1=l_3$, $l_2=l_4$, $\phi_1 \implies \phi_3$, and $\phi_4 \implies \phi_2$.

        $S \vdash s_\text{tab}(i,j) : tfi_0$ since it is a premise of $\vdash s : S$ which we have assumed to hold.

        Then, $S;C \vdash_i \<call> s_\text{tab}(i,j) : tfi_0$ by $call-cl$.

        $S;C \vdash_i \<call> s_\text{tab}(i,j) : tfi$ by $sub-typing$.

        Therefore, $S;C \vdash_i \<call> s_\text{tab}(i,j) : ti_0^{*}\;ti_1^{*};l_1;\phi_1 \rightarrow ti_0^{*}\;ti_1^{*};l_2;\phi_2$ by $stack-poly$.

    \item Case: $S;C \vdash_i (\<ithreetwo>.\<const> j)\; \<callindirect> tfi : ti_1^{*};l_1;\phi_1 \rightarrow ti_2^{*};l_2;\phi_2$
    \\ $\land$ $s;(\<ithreetwo>.\<const> j)\; \<callindirect> tfi \hookrightarrow_i \<trap>$.

        Trivially, $S;C \vdash_i \<trap> : ti_1^{*};l_1;\phi_1 \rightarrow ti_2^{*};l_2;\phi_2$ by $trap$.

    \item Case: $S;C \vdash_i v^n\; \<call> cl : ti_1^{*};l_1;\phi_1 \rightarrow ti_2^{*};l_2;\phi_2$
    \\ $\land$ $s;v^n\; \<call> cl \hookrightarrow_i \<local>_m\{j;v^n \; (t.\<const> 0)^k\} \; \<block> tfi_1\; e^{*} \<end> \<end>$
    \\ where $cl_\text{code} = \<func> tfi_2\; \<local>\; t^k \; e^{*}$ and $cl_\text{inst} = j$

        \todo{This needs an overhaul}

        Let $tfi_0 = ti_1^{*};l_1;\phi_1 \rightarrow ti_2^{*};l_2;\phi_2$, $tfi_1 = \epsilon;l_3;\phi_3 \rightarrow ti_4^{m};l_4;\phi_4$, and $tfi_2 = ti_3^{n};\ti{t_2}{a_2}^{n}\; \ti{t}{a}^k;\phi_3,\ti{t}{a}^k,(= a \;\ti{t}{0})^k \rightarrow ti_4^{m};l_4;\phi_4$ by $inversion$.

        $S;C \vdash (t_2 \<const> c)^n : ti_1^{*};l_1;\phi_1 \rightarrow ti_1^{*}\;ti_5^n;l_1;\phi_1,\ti{t_2}{a_2},(= a_2 \; \ti{t_2}{c})$, where $v^n=(t_2 \<const> c)^n$,
        and $S;C\vdash \<call> cl : ti_1^{*}\;ti_5^n;l_1;\phi_1,\ti{t_2}{a_2}^{n},(= a_2 \; \ti{t_2}{c})^{n} \rightarrow ti^{*}\;ti_2^m;l_2;\phi_2$ because they are premises of $composition$ which we have assumed to hold.

        By inversion, $l_2=l_1$, $ti_2^{*}=ti_1^{*}\;ti_4^m$, $\phi_1,\ti{t_2}{a_2},(\<eq> a_2 \; \ti{t_2}{c}) \implies \phi_3$, $\phi_4 \implies \phi_2$, and $S\vdash cl : tfi_1$.

        Therefore, $C \vdash \<func> tfi_1\; \<local>\; t^k \; e^{*} : tfi_1$ because it is a premise of $S \vdash cl : tfi_1$.

        $S;C,\text{local } t_2^n\; t^k,\text{label }(ti_4^{m};l_4;g_4;\phi_4),\text{return }(ti_4^{m};l_4;g_4;\phi_4) \vdash e^{*}: tfi_2$ because it is a premise of the above derivation.

        $S;C,\text{local } t_2^n\; t^k,\text{return }(ti_4^{m};l_4;g_4;\phi_4) \vdash \<block> tfi_2\; e^{*} \<end> : tfi_2$ by $block$.

        There are now two cases, based on whether or not the called closure was in the current module being called:

        \begin{itemize}
            \item Case: $i=j$
                By $inversion$, $g_1=g_3$ and $g_2=g_4$, so we will use $g_1$ instead of $g_3$ and $g_2$ instead of $g_4$.

                $C \vdash_j v^n \; (t \<const> 0)^k : \epsilon;l_3;g_1;\phi_1 \rightarrow \ti{t_2}{a_2}^n\;\ti{t}{a}^k ;l_3;g_1;\phi_1,\ti{t_2}{a_2},(\<eq> a_2 \; \ti{t_2}{c})^n,\ti{t}{a},(\<eq> a \; \ti{t}{0})^k$ by $const$.

                $\phi_1,\ti{t_2}{a_2},(\<eq> a_2 \; \ti{t_2}{c})^n \implies \phi_3$, and therefore $\phi_1,\ti{t_2}{a_2},(\<eq> a_2 \; \ti{t_2}{c})^n,\ti{t}{a},(\<eq> a \; \ti{t}{0})^k \implies \phi_3,\ti{t}{a}^k,(\<eq> a \;\ti{t}{0})^k$.

                $S;C,\text{local } t_2^n\; t^k,\text{return }(ti_4^{m};l_4;g_2;\phi_4) \vdash \<block> tfi_2\; e^{*} \<end> :  ti_3^{n};\ti{t_2}{a_2}^{n}\; \ti{t}{a}^k;g_1;\phi_1,\ti{t_2}{a_2},(\<eq> a_2 \; \ti{t_2}{c})^n,\ti{t}{a},(\<eq> a \; \ti{t}{0})^k \rightarrow ti_4^{m};l_4;g_2;\phi_4$ by $sub-typing$.

                $S;(ti_4^{m};l_4;g_2;\phi_4) \vdash_j v^n \; (t \<const> 0)^k;\<block> tfi_2\; e^{*} \<end> : \epsilon;l_3;g_1;\phi_1 \rightarrow ti_4^{m};l_4;g_2;\phi_4$ by $with-return$.

                $S;C \vdash_i \<local>_m\{j;v^n \; (t.\<const> 0)^k\} \; \<block> tfi_2\; e^{*} \<end> \<end> : \epsilon;l_1;g_1;\phi_1 \rightarrow \epsilon\;ti_4^m;l_1;g_2;\phi_4$ by $local-same-inst$.

                $S;C \vdash_i \<local>_m\{j;v^n \; (t.\<const> 0)^k\} \; \<block> tfi_2\; e^{*} \<end> \<end> : tfi_0$ by $stack-poly$ and $sub-typing$.

            \item Case: $i \neq j$
                By $inversion$, $g_2=\ti{t_g}{a_3}^{*}$ where $C_\text{global}=(mut?\; t_g)^{*}$ and $a_3^{*}$ are fresh.

                $C \vdash_j v^n \; (t \<const> 0)^k : \epsilon;l_3;g_3;\phi_1 \rightarrow \ti{t_2}{a_2}^n\;\ti{t}{a}^k ;l_3;g_3;\phi_1,\ti{t_2}{a_2},(\<eq> a_2 \; \ti{t_2}{c})^n,\ti{t}{a},(\<eq> a \; \ti{t}{0})^k$ by $const$

                $\phi_1,\ti{t_2}{a_2},(\<eq> a_2 \; \ti{t_2}{c})^n \implies \phi_3$, and therefore $\phi_1,\ti{t_2}{a_2},(\<eq> a_2 \; \ti{t_2}{c})^n,\ti{t}{a},(\<eq> a \; \ti{t}{0})^k \implies \phi_3,\ti{t}{a}^k,(\<eq> a \;\ti{t}{0})^k$.

                $S;C,\text{local } t_2^n\; t^k,\text{return }(ti_4^{m};l_4;g_4;\phi_4) \vdash \<block> tfi_2\; e^{*} \<end> :  ti_3^{n};\ti{t_2}{a_2}^{n}\; \ti{t}{a}^k;g_3;\phi_1,\ti{t_2}{a_2},(\<eq> a_2 \; \ti{t_2}{c})^n,\ti{t}{a},(\<eq> a \; \ti{t}{0})^k \rightarrow ti_4^{m};l_4;g_4;\phi_4$ by $sub-typing$.

                $S;(ti_4^{m};l_4;g_4;\phi_4) \vdash_j v^n \; (t \<const> 0)^k;\<block> tfi_2\; e^{*} \<end> : \epsilon;l_3;g_3;\phi_1 \rightarrow ti_4^{m};l_4;g_4;\phi_4$ by $with-return$.

                $S;C \vdash_i \<local>_m\{j;v^n \; (t.\<const> 0)^k\} \; \<block> tfi_2\; e^{*} \<end> \<end> : \epsilon;l_1;g_1;\phi_1 \rightarrow \epsilon\;ti_4^m;l_1;\ti{t_g}{a_3}^{*};\phi_4$ by $local-diff-inst$.

                $S;C \vdash_i \<local>_m\{j;v^n \; (t.\<const> 0)^k\} \; \<block> tfi_2\; e^{*} \<end> \<end> : tfi_0$ by $stack-poly$ and $sub-typing$.
        \end{itemize}

    \item Case: $S;C \vdash \<local>_n \{ i;v_l^{*} \} \; v^n \<end> : ti_1^{*};l_1;\phi_1 \rightarrow ti_2^{*};l_2;\phi_2$
    \\ $\land$ $\<local>_n \{ i;v^{*}_l \} \; v^n \<end> \hookrightarrow_j \; v^n$

        By \reflemma{Inversion} on \refrule{Local}, $ti_2^{*} = ti_2^{*} \; ti^n$, $l_1 = l_2$,
        $S;(ti^n;l_3;\phi_3) \vdash_i v_l^{*};v^n : ti^n;l_3;\phi_3$,
        and $\phi_1,\phi_3 \implies \phi_2$.

        $(\vdash v_l : ti_l;\phi_l)^{*}$ and $S;C_l \vdash v^n : \epsilon:ti_l^{*};\phi_l^{*} \rightarrow ti^n;l_3;\phi_3$
        because they are premises of \refrule{Code} which we have assumed to hold.

        $\phi_l^{*} = \circ,\ti{t}{a}^{*},(= a\;\ti{t}{c})^{*}$ because it is a premise of \refrule{Admin-Const} which we have assumed to hold.

        By \reflemma{Inversion} on \refrule{Const}, $\phi_l^{*},\phi_v^n \implies \phi_3$.

        Since $a^{*}$ are fresh, $\phi_v^n \implies \phi_3$.

        \thought{Is this enough justification?}

        $S;C \vdash v^n : \epsilon;l_1;\phi_1 \rightarrow ti^n;l_2;\phi_1,\phi_v^n$ by \refrule{Const}.

        $S;C \vdash v^n : \epsilon;l_1;\phi_1 \rightarrow ti^n;l_2;\phi_1,\phi_3$ by \refrule{Implies}.

        $S;C \vdash v^n : \epsilon;l_1;\phi_1 \rightarrow ti^n;l_2;\phi_2$ by \refrule{Implies}.

        Therefore $S;C \vdash v^n : ti_1^{*};l_1;\phi_1 \rightarrow ti_2^{*};l_2;\phi_2$ by \refrule{Stack-Poly}.

    \item Case: $S;C \vdash \<local>_n \{ i;v_l^{*} \} \; \<trap> \<end> : ti_1^{*};l_1;\phi_1 \rightarrow ti_2^{*};l_2;\phi_2$
    \\ $\land$ $\<local>_n \{ i;v_l^{*} \} \; \<trap> \<end> \hookrightarrow \; \<trap>$

        Trivially, $S;C \vdash \<trap> : ti_1^{*};l_1;\phi_1 \rightarrow ti_2^{*};l_2;\phi_2$ by \refrule{Trap}.

    \item Case: $S;C \vdash \<local>_n \{ i;v_l^{*} \} \; L^k[v^n \; \<return>] \<end> : ti_1^{*};l_1;\phi_1 \rightarrow ti_2^{*};l_2;\phi_2$
    \\ $\land$ $\<local>_n \{ i;v_l^{*} \} \; L^k[v^n \; \<return>] \<end> \hookrightarrow_j \; v^n$

        By \reflemma{Inversion} on \refrule{Local}, $ti_2^{*} = ti_1^{*} \; ti^n$, $l_1 = l_2$,
        $S;(ti^n;l_3;\phi_3) \vdash_i v_l^{*};L^k[v^n \; \<return>] : ti^n;l_3;\phi_3$,
        and $\phi_1,\phi_3 \implies \phi_2$.

        $(\vdash v_l : ti_l;\phi_l)^{*}$ and $S;C_l \vdash L^k[v^n \; \<return>] : \epsilon;ti_l^{*};\phi_l^{*} \rightarrow ti^n;l_3;\phi_3$,
        where $C_l = S_\text{inst}(i),\text{local} \; t^{*}, \text{return} \; (ti^n;l_3;\phi_3)$,
        because they are premises of \refrule{Code} that we have assumed to hold.

        $ti_l^{*} = \ti{t_l}{a_l}^{*}$ because it is a premise of \refrule{Admin-Const} which we have assumed to hold.

        By \reflemma{Inversion} on \refrule{Composition},
        $C_l \vdash v^n : ti_4^{*};l_4;\phi_4 \rightarrow ti_5^{*};l_5;\phi_5$,
        and $C_l \vdash \<return> : ti_5^{*};l_5;\phi_5 \rightarrow ti_6^{*};l_6;\phi_6$.

        By \reflemma{Inversion} on \refrule{Return},
        $ti_5^{*} = ti_7^{*}\;ti^n$, $l_5 = l_3$, and $\phi_5 \implies \phi_3$.

        By \reflemma{Inversion} on \refrule{Const},
        $ti_5^{*} = ti_4^{*}\;ti^n$, $l_4 = l_5$,
        and $\phi_4,\phi_v^n \implies \phi_5$.

        $C_l \vdash v^n : \epsilon;l_3;\phi_4 \rightarrow ti^n;l_3;\phi_4,\phi_v^n$ by \refrule{Const}.

        $C_l \vdash v^n : \epsilon;ti_l^{*};\phi_l^{*} \rightarrow ti^n;l_3;\phi_4,\phi_v^n$ by \reflemma{Lift-Consts}.

        By \reflemma{Inversion} on \refrule{Const}, $\phi_l^{*} \implies \phi_4$.

        \thought{This is kinda non-obvious}

        Since $a_l^{*}$ are fresh, $\circ \implies \phi_4$.

        $C_l \vdash v^n : \epsilon;l_1;\phi_1 \rightarrow ti^n;l_2;\phi_1,\phi_v^n$ by \refrule{Const}.

        $C_l \vdash v^n : \epsilon;l_1;\phi_1 \rightarrow ti^n;l_2;\phi_1,\phi_3$ by \refrule{Implies}.

        \thought{This is using like 5 different implication arrows.}

        $C_l \vdash v^n : \epsilon;l_1;\phi_1 \rightarrow ti^n;l_2;\phi_2$ by \refrule{Implies}.

        Therefore, $C_l \vdash v^n : ti_1^{*};l_1;\phi_l^{*} \rightarrow ti_2^{*};l_2;\phi_2$ by \refrule{Stack-Poly}.

        \thought{There has to be a better way to do this proof...}

    \item Case: $S;\epsilon \vdash_i v_1^j\;v\;v_2^k;\<getlocal> j : ti^{*};l;\phi$
    \\ $\land$ $v_1^j\;v\;v_2^k;\<getlocal> j \hookrightarrow v$

        $\vdash v : \ti{t}{a};\phi_v$ and $S;C \vdash \<getlocal> j : \epsilon;l_1;\phi_1^j,\phi_v,\phi_2^k \rightarrow ti^{*};l;\phi$,
        because they are premises of \refrule{Code} that we have assumed to hold.

        $t.\<const> c = v$, and $\phi_v = \circ,\ti{t}{a},(= a\;\ti{t}{c})$, because it is a premise of \refrule{Admin-Const} that we have assumed to hold.

        By \reflemma{Inversion} on \refrule{Get-Local},
        $ti^{*} = \ti{t}{a_2}$, $l_1 = l$, and $\phi_1^j,\phi_v,\phi_2^k,\ti{t}{a_2},(= a_2\;a) \implies \phi$.

        $C \vdash v : \epsilon;l;\phi_1^j,\phi_v,\phi_2^k \rightarrow \ti{t}{a_2};l;\phi_1^j,\phi_v,\phi_2^k,\ti{t}{a_2},(= a_2\;\ti{t}{c})$ by \refrule{Const}.

        $\phi_v,\ti{t}{a_2},(= a_2\;\ti{t}{c}) \iff \phi_v,\ti{t}{a_2},(= a_2\;a)$ trivially.

        $C \vdash v : \epsilon;l;\phi_1^j,\phi_v,\phi_2^k \rightarrow \ti{t}{a_2};l;\phi$ by \refrule{Implies}.

        Therefore, $S;\epsilon \vdash_i v_1^j\;v\;v_2^k;v : ti^{*};l;\phi$ by \refrule{Code}.

    \item Case: $S;\epsilon \vdash_i v_1^j \; v \; v_2^k; v' \; (\<setlocal> j) : ti^{*};l;\phi$
    \\ $\land$ $v_1^j \; v \; v_2^k;v' \; \<setlocal> j \hookrightarrow v_1^j \; v' \; v_2^k;\epsilon$

        $\vdash v : \ti{t}{a};\phi_v$,
        $S;C \vdash v' \; (\<setlocal> j) : \epsilon:l_1;\phi_1^j,\phi_v,\phi_2^k \rightarrow ti^{*};l;\phi$,
        $l_1(j) = \ti{t}{a}$, and $C_\text{local}(j) = t$ because they are premises of \refrule{Code} that we have assumed to hold.

        $t.\<const> c = v$ and $\phi_v = \circ,\ti{t}{a},(= a\;\ti{t}{c})$ because they are premises of \refrule{Admin-Const} that we have assumed to hold.

        By \reflemma{Inversion} on \refrule{Composition},
        $C \vdash v' : \epsilon;l_1;\phi_1^j,\phi_v,\phi_2^k \rightarrow ti_3^{*};l_3;\phi_3$,
        $C \vdash \<setlocal> j : ti_3^{*};l_3;\phi_3 \rightarrow ti^{*};l;\phi$.

        Recalling that $t = C_\text{local}(j)$;
        by \reflemma{Inversion} on \refrule{Set-Local},
        $ti_3^{*} = ti^{*} \; \ti{t}{a'}$,
        $l = l_3[j := \ti{t}{a'}]$,
        and $\phi_3 \implies \phi$.

        By \reflemma{Inversion} on \refrule{Const},
        $t.\<const> c' = v'$, $ti^{*} = \epsilon$, $l_1 = l_3$, and\\
        $\phi_1^j,\phi_v,\phi_2^k,\ti{t}{a'},(= a'\;\ti{t}{c'}) \implies \phi_3$.

        $C \vdash \epsilon : \epsilon;l;\phi \rightarrow \epsilon;l;\phi$ by \refrule{Empty}.

        $C \vdash \epsilon : \epsilon;l;\phi_1^j,\phi_v,\phi_2^k,\ti{t}{a'},(= a'\;\ti{t}{c'}) \rightarrow \epsilon;l;\phi$ by \refrule{Implies}.

        % Since $a$ is fresh in the typing derivation, $C \vdash \epsilon : \epsilon;l;\phi_1^j,\phi_2^k,\ti{t}{a'},(= a'\;\ti{t}{c'}) \rightarrow \epsilon;l;\phi$.

        Since $a$ is fresh, $\phi_1^j,\phi_2^k,\ti{t}{a'},(= a'\;\ti{t}{c'}) \implies \phi_1^j,\phi_v,\phi_2^k,\ti{t}{a'},(= a'\;\ti{t}{c'})$.

        $C \vdash \epsilon : \epsilon;l;\phi_1^j,\phi_2^k,\ti{t}{a'},(= a'\;\ti{t}{c'}) \rightarrow \epsilon;l;\phi$ by \refrule{Implies}.

        $\vdash v' : \ti{t}{a'};\circ,\ti{t}{a'},(= a'\;\ti{t}{c'})$ by \refrule{Admin-Const}.

        Therefore, $S;\epsilon \vdash_i v_1^j\;v'\;v_2^k;\epsilon : ti^n;l;\phi$ by \refrule{Code}.

        \thought{I set out to write a very verbose proof case, but I didn't expect it to be this verbose.}

    \item Case: $C \vdash v \; (\<teelocal> j) : ti_1^{*};l_1;\phi_1 \rightarrow ti_2^{*};l_2;\phi_2$
    \\ $\land$ $v \; (\<teelocal> j) \hookrightarrow v\;v\;(\<setlocal> j)$

        By \reflemma{Inversion} on \refrule{Composition},
        $C \vdash v : ti_1^{*};l_1;\phi_1 \rightarrow ti_3^{*};l_3;\phi_3$,
        and $C \vdash \<teelocal> j : ti_3^{*};l_3;\phi_3 \rightarrow ti_2^{*};l_2;\phi_2$.

        By \reflemma{Inversion} on \refrule{Tee-Local},
        $ti_3^{*} = ti^{*} \; \ti{t}{a}$, $ti_2^{*} = ti^{*} \; \ti{t}{a_2}$, $l_2 = l_3[j := \ti{t}{a}]$,
        and $\phi_3,\ti{t}{a_2},(= a_2\;a) \implies \phi_2$.

        By \reflemma{Inversion} on \refrule{Const},
        $t.\<const> c = v$, $ti_1^{*} = ti^{*}$, $l_3 = l_1$,
        and $\phi_1,\ti{t}{a},(= a\;\ti{t}{c}) \implies \phi_3$.

        $C \vdash v\;v : \epsilon;l_1;\phi_1 \rightarrow \ti{t}{a_2}\;\ti{t}{a};l_1;\phi_1,\ti{t}{a_2},(= a_2\;\ti{t}{c}),\ti{t}{a},(= a\;\ti{t}{c})$ by \refrule{Const}.

        $\ti{t}{a_2},(= a_2\;\ti{t}{c}),\ti{t}{a},(= a\;\ti{t}{c}) \iff \ti{t}{a_2},(= a_2\;a),\ti{t}{a},(= a\;\ti{t}{c})$ trivially.

        $C \vdash v\;v : \epsilon;l_1;\phi_1 \rightarrow \ti{t}{a_2}\;\ti{t}{a};l_1;\phi_2$ by \refrule{Implies}.

        $C \vdash \<setlocal> j : \ti{t}{a};l_1;\phi_2 \rightarrow \epsilon;l_1[j := \ti{t}{a}];\phi_2$ by \refrule{Set-Local}.

        Therefore, $C \vdash v\;v\;(\<setlocal> j) : ti_1^{*};l_1;\phi_1 \rightarrow ti_2^{*};l_2;\phi_2$ by \refrule{Composition} and \refrule{Stack-Poly}.

    \item Case: $C \vdash \<getglobal> j : ti_1^{*};l_1;\phi_1 \rightarrow ti_2^{*};l_2;\phi_2$
    \\ $\land$ $s;\<getglobal> j \hookrightarrow_i s_\text{glob}(i,j)$

        $\vdash s : S$ and $S;\epsilon \vdash_i v^{*};\<getglobal> j : ti^{*};l;\phi$ because they are premises of \refrule{Program} that we have assumed to hold.

        $S;C \vdash \<getglobal> j : \epsilon;l_1;\phi_v^{*} \rightarrow ti^{*};l;\phi$ because it is a premise of \refrule{Code} that we have assumed to hold.

        By \reflemma{Inversion} on \refrule{Get-Global}, $ti^{*} = \ti{t}{a}$, $l = l_1$, $C_\text{global}(j) = \text{mut}^{?} t$,
        and $\phi_v^{*},\ti{t}{a} \implies \phi$.

        $S \vdash s_\text{inst}(i) : C$ because it is a premise of \refrule{Store} that we have assumed to hold.

        Recalling that $C_\text{global}(j) = \text{mut}^{?} t$;
        $\vdash s_\text{glob}(i,j) : \ti{t}{a_1};\phi_1$ because it is a premise of \refrule{Instance} that we have assumed to hold.

        $S;C \vdash t.\<const> c : \epsilon;l;\phi_v^{*} \rightarrow \ti{t}{a};l;\phi_v^{*},\ti{t}{a},(= a \; \ti{t}{c})$ by \refrule{Const},
        where $t.\<const> c = s_\text{glob}(i,j)$.

        $\ti{t}{a},(= a\;\ti{t}{c}) \implies \ti{t}{a}$ trivially.

        $S;C \vdash s_\text{glob}(i,j) : \epsilon;l;\phi_v^{*} \rightarrow \ti{t}{a};l;\phi$ by \refrule{Implies}.

        $S;\epsilon \vdash_i v^{*};s_\text{glob}(i,j) : \ti{t}{a};l;\phi$ by \refrule{Code}, having assumed that the other premises hold.

        \todo{Phrase this better.}

        Therefore, $\vdash_i s;v^{*};s_\text{glob}(i,j) : \ti{t}{a};l;\phi$ by \refrule{Program}.

    \item Case: $\vdash_i s;v_l^{*};v \; (\<setglobal> j) : ti^{*};l;\phi$
    \\ $\land$ $s;v \; (\<setglobal> j) \hookrightarrow_i s';\epsilon$, where $s' = s$ with $\text{glob}(i,j) = v$

        $\vdash s : S$ and $S;\epsilon \vdash_i v_l^{*};v \; (\<setglobal> j) : ti^{*};l;\phi$ because they are premises of \refrule{Program} that we have assumed to hold.

        $S;C \vdash v \; (\<setglobal> j) : \epsilon;l_1;\phi_1 \rightarrow ti^{*};l;\phi$ because it is a premise of \refrule{Code} that we have assumed to hold.

        By \reflemma{Inversion} on \refrule{Composition}, \refrule{Set-Global}, and \refrule{Const},
        $t.\<const> c = v$, $ti^{*} = \epsilon$, $l_1 = l$, $C_\text{global}(j) = \text{mut}\;t$,
        and $\phi_1,\ti{t}{a},(= a\;\ti{t}{c}) \implies \phi$.

        % By \reflemma{Inversion} on \refrule{Composition},
        % $C \vdash v : \epsilon;l_1;\phi_1 \rightarrow ti_2^{*};l_2;\phi_2$,
        % and $C \vdash \<setglobal> j : ti_2^{*};l_2;\phi_2 \rightarrow ti^{*};l;\phi$.

        % By \reflemma{Inversion} on \refrule{Set-Global},
        % $ti_2^{*} = ti^{*} \; \ti{t}{a}$, $l_2 = l$, $C_\text{global}(j) = \text{mut}\;t$,
        % and $\phi_2, \implies \phi$.

        % By \reflemma{Inversion} on \refrule{Const},
        % $t.\<const> c = v$, $ti_2^{*} = \ti{t}{a}$, $l_1 = l_2$,
        % and $\phi_1,\ti{t}{a},(= a\;\ti{t}{c}) \implies \phi_2$.

        $S;C \vdash \epsilon : \epsilon;l;\phi \rightarrow \epsilon;l;\phi$ by \refrule{Empty}.

        Since $a$ is fresh, $\phi_1 \implies \phi_1,\ti{t}{a},(= a\;\ti{t}{c})$.

        $S;C \vdash \epsilon : \epsilon;l;\phi_1 \rightarrow \epsilon;l;\phi$ by \refrule{Implies}.

        $S;\epsilon \vdash_i v_l^{*};\epsilon : ti^{*};l;\phi$ by \refrule{Code}.

        $S \vdash s_\text{inst}(i) : C$ because it is a premise of \refrule{Store} that we have assumed to hold.

        $\vdash s_\text{glob}(i,j) : \ti{t}{a_g};\phi_g$ because it is a premise of \refrule{Instance} that we have assumed to hold.

        $s' : S$ by \refrule{Instance} and \refrule{Store}.

        \thought{This might be skipping too much.}

        Therefore, $\vdash s';\epsilon : ti^{*};l;\phi$ by \refrule{Program}.

    \item Case: $\vdash_i s;v^{*};(\<ithreetwo>.\<const> k)\;(t.\<load> align\;o) : ti^{*};l;\phi$
    \\ $\land$ $s;(\<ithreetwo>.\<const> k)\;(t.\<load> align\;o) \hookrightarrow_i s;t.\<const> \text{const}_t(b^{*})$, where $s_\text{mem}(i,k+o,|t|) = b^{*}$

        $S;\epsilon \vdash_i v^{*};(\<ithreetwo>.\<const> k)\;(t.\<load> align\;o) : ti^{*};l;\phi$ and $\vdash s : S$ because they are premises of \refrule{Program} which we have assumed to hold.

        $S;C \vdash (\<ithreetwo>.\<const> k)\;(t.\<load> align\;o) : \epsilon;l_1;\phi_1 \rightarrow ti^{*};l;\phi$ because it is a premise of \refrule{Code} which we have assumed to hold.

        By \reflemma{Inversion} on \refrule{Composition}, \refrule{Const}, \refrule{Mem-Load},
        $ti^{*} = \ti{t}{a}$, $l_1 = l$, and $\phi_1,\ti{t}{a} \implies \phi$.

        $C \vdash t.\<const> \text{const}_t(b^{*}) : \epsilon;l;\phi_1 \rightarrow \ti{t}{a};l;\phi_1,\ti{t}{a},(= a\;\ti{t}{c})$ by \refrule{Const}.

        $C \vdash t.\<const> \text{const}_t(b^{*}) : \epsilon;l;\phi_1 \rightarrow \ti{t}{a};l;\phi$ by \refrule{Implies}.

        $S;\epsilon \vdash_i v^{*};t.\<const> \text{const}_t(b^{*}) : ti^{*};l;\phi$ by \refrule{Code}.

        Therefore, $s;t.\<const> \text{const}_t(b^{*}) : ti^{*};l;\phi$ by \refrule{Program}.

    \item Case: $\vdash_i s;v^{*};(\<ithreetwo>.\<const> k)\;(t.\<load> tp_sx\;align\;o) : ti^{*};l;\phi$
    \\ $\land$ $s;(\<ithreetwo>.\<const> k)\;(t.\<load> tp_sx\;align\;o) \hookrightarrow_i s;t.\<const> \text{const}_t^{sx}(b^{*})$, where $s_\text{mem}(i,k+o,|tp|) = b^{*}$

        Similar to \refrule{Mem-Load} non-packed.

    \item Case: $C \vdash (\<ithreetwo>.\<const> k)\;(t.\<load> tp\_sx^{?}\;align\;o) : ti_1^{*};l_1;\phi_1 \rightarrow ti_2^{*};l_2;\phi_2$
    \\ $\land$ $s;(\<ithreetwo>.\<const> k)\;(t.\<load> tp\_sx^{?}\;align\;o) \hookrightarrow_i \<trap>$

        Trivially, $C \vdash \<trap> : ti_1^{*};l_1;\phi_1 \rightarrow ti_2^{*};l_2;\phi_2$ by \refrule{Trap}.

    \item Case: $\vdash_i s;v^{*};(\<ithreetwo>.\<const> k)\;(t.\<const> c)\;(t.\<store> align\;o) : ti^{*};l;\phi$
    \\ $\land$ $s;(\<ithreetwo>.\<const> k)\;(t.\<const> c)\;(t.\<store> align\;o) \hookrightarrow_i s';\epsilon$, where $s' = s \text{ with } \text{mem}(i,k+o,|t|) = \text{bits}_t^{|t|}(c)$

        $S;\epsilon \vdash_i v^{*};(\<ithreetwo>.\<const> k)\;(t.\<const> c)\;(t.\<store> align\;o) : ti^{*};l;\phi$ and $\vdash s : S$ because they are premises of \refrule{Program} which we have assumed to hold.

        $S;C \vdash (\<ithreetwo>.\<const> k)\;(t.\<const> c)\;(t.\<store> align\;o) : \epsilon;l_1;\phi_1 \rightarrow ti^{*};l;\phi$ because it is a premise of \refrule{Code} which we have assumed to hold.

        By \reflemma{Inversion} on \refrule{Composition}, \refrule{Const}, and \refrule{Mem-Store},
        $ti^{*} = \epsilon$, $l_1 = l$, and $\phi_1,\ti{\<ithreetwo>}{a_1},(= a_1\;\ti{\<ithreetwo>}{k}),\ti{t}{a_2},(= a_2\;\ti{t}{c}) \implies \phi$.

        Since $a_1$ and $a_2$ are fresh, $\phi_1 \implies \phi$.

        $C \vdash \epsilon : \epsilon;l;\phi_1 \rightarrow \epsilon;l;\phi_1$ by \refrule{Empty}.

        $C \vdash \epsilon : \epsilon;l;\phi_1 \rightarrow \epsilon;l;\phi$ by \refrule{Implies}.

        $s' : S$ trivially.

        \thought{Maybe justify more? Intuitively the contents of the mem function doesn't affect the store typing.}

        $S;\epsilon \vdash_i v^{*};\epsilon : ti^{*};l;\phi$ by \refrule{Code}.

        Therefore, $\vdash_i s';\epsilon : ti^{*};l;\phi$ by \refrule{Program}.

    \item Case: $C \vdash (\<ithreetwo>.\<const> k)\;(t.\<const> c)\;(t.\<store> tp\;align\;o) : ti_1^{*};l_1;\phi_1 \rightarrow ti_2^{*};l_2;\phi_2$
    $\land$ $\vdash s : S$
    \\ $\land$ $s;(\<ithreetwo>.\<const> k)\;(t.\<const> c)\;(t.\<store> tp\;align\;o) \hookrightarrow_i s';\epsilon$, where $s' = s \text{ with } \text{mem}(i,k+o,|tp|)=\text{bits}_t^{|tp|}(c)$

        Similar to \refrule{Mem-Store} non-packed.

    \item Case: $C \vdash (\<ithreetwo>.\<const> k)\;(t.\<const> c)\;(t.\<store> tp^{?}\;align\;o) : ti_1^{*};l_1;\phi_1 \rightarrow ti_2^{*};l_2;\phi_2$
    \\ $\land$ $s;(\<ithreetwo>.\<const> k)\;(t.\<const> c)\;(t.\<store> tp^{?}\;align\;o) \hookrightarrow_i \<trap>$

        Trivially, $C \vdash \<trap> : ti_1^{*};l_1;\phi_1 \rightarrow ti_2^{*};l_2;\phi_2$ by \refrule{Trap}.

    \item Case: $\vdash_i s;v^{*};\<currentmemory> : ti^{*};l;\phi$
    \\ $\land$ $s;\<currentmemory> \hookrightarrow_i \<ithreetwo>.\<const> |s_\text{mem}(i,*)|/64\text{Ki}$

        $S;\epsilon \vdash_i v^{*};\<currentmemory> : ti^{*};l;\phi$, and $\vdash s : S$ because they are premises of \refrule{Program} which we have assumed to hold.

        $S;C \vdash \<currentmemory> : \epsilon;l_1;\phi_1 \rightarrow ti^{*};l;\phi$ because it is a premise of \refrule{Code} which we have assumed to hold.

        By \reflemma{Inversion} on \refrule{Current-Memory}, $ti^{*} = \ti{\<ithreetwo>}{a}$, $l_1 = l$, and $\phi_1,\ti{\<ithreetwo>}{a} \implies \phi$.

        Let $c = |s_\text{mem}(i,*)|/64\text{Ki}$.

        $S;C \vdash \<ithreetwo>.\<const> c : \epsilon;l;\phi_1 \rightarrow \ti{\<ithreetwo>}{a};l;\phi_1,\ti{\<ithreetwo>}{a},(= a\;\ti{\<ithreetwo>}{c})$ by \refrule{Const}.

        $S;C \vdash \<ithreetwo>.\<const> c : \epsilon;l;\phi_1 \rightarrow \ti{\<ithreetwo>}{a};l;\phi$ by \refrule{Implies}.

        $S;\epsilon \vdash_i v^{*};\<ithreetwo>.\<const> c : ti^{*};l;\phi$ by \refrule{Code}.

        Therefore, $\vdash_i s;\<ithreetwo>.\<const> |s_\text{mem}(i,*)|/64\text{Ki} : ti^{*};l;\phi$ by \refrule{Program}.

    \item Case: $\vdash_i s;v^{*};(\<ithreetwo>.\<const> k) \; \<growmemory> : ti^{*};l;\phi$
    \\ $\land$ $s;(\<ithreetwo>.\<const> k) \; \<growmemory> \hookrightarrow_i s';\<ithreetwo>.\<const> |s_\text{mem}(i,*)|/64\text{Ki}$, where $s' = s \text{ with } \text{mem}(i,*) = s_\text{mem}(i,*)(0)^{k \cdot 64\text{Ki}}$

        $S;\epsilon \vdash_i v^{*};(\<ithreetwo>.\<const> k) \; \<growmemory> : ti^{*};l;\phi$, and $\vdash s : S$ because they are premises of \refrule{Program} which we have assumed to hold.

        $S;C \vdash (\<ithreetwo>.\<const> k) \; \<growmemory> : \epsilon;l_1;\phi_1 \rightarrow ti^{*};l;\phi$ because it is a premise of \refrule{Code} which we have assumed to hold.

        By \reflemma{Inversion} on \refrule{Composition}, \refrule{Const}, and \refrule{Grow-Memory},
        $ti^{*} = \ti{\<ithreetwo>}{a_1}$, $l_1 = l$, and $\phi_1,\ti{\<ithreetwo>}{a_2},(= a_2\;\ti{\<ithreetwo>}{k}),\ti{\<ithreetwo>}{a_1} \implies \phi$.

        Since $a_2$ is fresh, $\phi_1,\ti{\<ithreetwo>}{a_1} \implies \phi$.

        $S_\text{mem}(i) \leq |s_\text{mem}(i,*)|$ because it is a premise of \refrule{Store} on $\vdash s : S$ which we have assumed to hold.

        $s' : S$ by \refrule{Store}, using $S_\text{mem}(i) \leq |s_\text{mem}(i,*)(0)^{k \cdot 64\text{Ki}}|$.

        \thought{Maybe explicitly lay out the $\leq$ relationship we're using?}

        Let $c = \<ithreetwo>.\<const> |s_\text{mem}(i,*)|/64\text{Ki}$.

        $C \vdash \<ithreetwo>.\<const> c : \epsilon;l;\phi_1 \rightarrow \ti{\<ithreetwo>}{a_1};l;\phi_1,\ti{\<ithreetwo>}{a_1},(= a_1\;\ti{\<ithreetwo>}{c})$ by \refrule{Const}.

        $C \vdash \<ithreetwo>.\<const> c : \epsilon;l;\phi_1 \rightarrow \ti{\<ithreetwo>}{a_1};l;\phi$ by \refrule{Implies}.

        $S;\epsilon \vdash_i \<ithreetwo>.\<const> c : ti^{*};l;\phi$ by \refrule{Code}.

        Therefore, $\vdash_i s';\<ithreetwo>.\<const> |s_\text{mem}(i,*)|/64\text{Ki} : ti^{*};l;\phi$ by \refrule{Program}.

    \item Case: $\vdash_i s;v^{*};(\<ithreetwo>.\<const> k) \; \<growmemory> : ti^{*};l;\phi$
    \\ $\land$ $s;(\<ithreetwo>.\<const> k) \; \<growmemory> \hookrightarrow_i \<ithreetwo>.\<const> (-1)$

        $S;\epsilon \vdash_i v^{*};(\<ithreetwo>.\<const> k) \; \<growmemory> : ti^{*};l;\phi$, and $\vdash s : S$ because they are premises of \refrule{Program} which we have assumed to hold.

        $S;C \vdash (\<ithreetwo>.\<const> k) \; \<growmemory> : \epsilon;l_1;\phi_1 \rightarrow ti^{*};l;\phi$ because it is a premise of \refrule{Code} which we have assumed to hold.

        By \reflemma{Inversion} on \refrule{Composition}, \refrule{Const}, \refrule{Grow-Memory},
        $ti^{*} = \ti{\<ithreetwo>}{a_1}$, $l_1 = l$, and $\phi_1,\ti{\<ithreetwo>}{a_2},(= a_2\;\ti{\<ithreetwo>}{k}),\ti{\<ithreetwo>}{a_1} \implies \phi$.

        Since $a_2$ is fresh, $\phi_1,\ti{\<ithreetwo>}{a_1} \implies \phi$.

        $C \vdash \<ithreetwo>.\<const> (-1) : \epsilon;l;\phi_1 \rightarrow \ti{\<ithreetwo>}{a_1};l;\phi_1,\ti{\<ithreetwo>}{a_1},(= a_1\;\ti{\<ithreetwo>}{(-1)})$ by \refrule{Const}.

        $C \vdash \<ithreetwo>.\<const> (-1) : \epsilon;l;\phi_1 \rightarrow \ti{\<ithreetwo>}{a_1};l;\phi$ by \refrule{Implies}.

        $S;\epsilon \vdash_i \<ithreetwo>.\<const> (-1) : ti^{*};l;\phi$ by \refrule{Code}.

        Therefore, $\vdash_i s;\<ithreetwo>.\<const> (-1) : ti^{*};l;\phi$ by \refrule{Program}.

\end{itemize}
\end{proof}

\subsection{Progress}
\label{subsec:progress}
\emph{Progress} ensures that if a program is well typed then it either: entirely consists of values, traps, or is reducible (\ie there exists another program that it reduces to).
Proving progress for \name is the key metatheoretic property that ensures that our claim that \name is as safe as \wasm is valid.
This is because it connects the static guarantees of the type system to the dynamic assumptions of \prechk-tagged instructions.
By proving that well-typed \prechk-tagged instructions will always be reducible, we prove that the static guarantees are sufficient to ensure that they will not trap and therefore the dynamic checks are unnecessary.

Since most \name instructions have the same semantics as in \wasm, and every \name type includes all the information of a \wasm type, we can reuse the \wasm proof for those instructions by using the erasure function from Section \ref{subsec:erasure}.
The intuition for this is that the \name indexed type system provides strictly more information than the \wasm type system.
However, for \name instructions that do not have the same semantics as in \wasm, specifically \prechk-tagged instructions, we still must prove those cases.

\begin{theorem}{Progress}
    If $\vdash_i s;v^{*};e^{*} : ti^{*};l;\phi$ then either $e^{*} = v'^{*}$, $e^{*}= \<trap>$, or $s;v^{*};e^{*} \hookrightarrow_i s';v'^{*};e'^{*}$.
\end{theorem}
\begin{proof}
    We proceed by induction on $\vdash_i s;v^{*};e^{*} : ti^{*};l;\phi$.

    Because $\vdash_i s;v^{*};e^{*} : ti^{*};l;\phi$, we know that $\vdash s : S$ for some $S$, and that $S; \epsilon \vdash_i v^{*};e^{*}:ti^{*};l;\phi$ because they are premises of \refrule{Program} which we have assumed to hold.

    Then we know that $(\vdash v: \ti{t_v}{a_v};\phi_v)^{*}$ and $S;S_\text{inst}(i),\text{local } t_v^{*} \vdash e^{*} : \epsilon;\ti{t_v}{a_v}^{*};\phi_v^{*} \rightarrow ti^{*};l;\phi$ because they are premises of \refrule{Code} which we have assumed to hold.

    \begin{itemize}
        \item Case: $\vdash_i s;v^{*};(t.\<const> c_1)\;(t.\<const> c_2)\;t.\<divpc>$

        We must show that $(t.\<const> c_1)\;(t.\<const> c_2)\;t.\<divpc> \hookrightarrow e'^{*}$ for some $e'^{*}$.

        We have $S;() \vdash_i v^{*};(t.\<const> c_1)\;(t.\<const> c_2)\;t.\<divpc> : ti^{*};l;\phi$ for some $ti^{*}$, $l$, and $\phi$ because it is a premise of \refrule{Program} which we have assumed to hold.

        Then, $(\vdash v : \ti{t_v}{a_v};\phi_v)^{*}$ for some $\ti{t_v}{a_v}^{*}$ and $\phi_v^{*}$, since it is a premise of \refrule{Code} which we have assumed to hold.

        It is important to note that $\phi_v^{*}$ cannot contain a contradiction because it contains a single equality constraint per fresh index variable (see \refrule{Admin-Const}).

        Further,
        $$S;S_\text{inst}(i),\text{local } t_v^{*}\;
        {\begin{stackTL}
            \vdash (t.\<const> c_1)\;(t.\<const> c_2)\;t.\<divpc>
            \\: \epsilon;\ti{t_v}{a_v}^{*});\phi_v^{*} \rightarrow ti^{*};l;\phi
        \end{stackTL}}$$
        because it too is a premise of \refrule{Code}.

        Then,
        $$S_\text{inst}(i)
        {\begin{stackTL}
            \vdash (t.\<const> c_1)\;(t.\<const> c_2)
            \\ : \epsilon;\ti{t_v}{a_v}^{*});\phi_v^{*}
            \\ \;\; \rightarrow \ti{t}{a_1}\;\ti{t}{a_2};\ti{t_v}{a_v}^{*});\phi_v^{*},
            {\begin{stackTL}
                \ti{t}{a_1},(= a_1\; \ti{t}{c_1}),
                \\ \ti{t}{a_2},(= a_2\; \ti{t}{c_2})
            \end{stackTL}}
        \end{stackTL}}$$
        where $\phi_v^{*},\ti{t}{a_1},(= a_1\; \ti{t}{c_1}),\ti{t}{a_2},(= a_2\; \ti{t}{c_2}) \implies \neg(= a_2\; \ti{t}{0})$ by \reflemma{Inversion} on \refrule{Composition} and \refrule{Div-Prechk}.

        Therefore, it must be the case that $c_2\neq 0$, and therefore there must exist some $c_3$ such that $c_3=div(c_1,c_2)$ since $div(c_1,c_2)$ is well-defined when $c_2$ is non-zero.
        Then, $s;(t.\<const> c_1)\;(t.\<const> c_2)\;t.\<divpc> \hookrightarrow_i (t.\<const> c_3)$.

        \item Case: $\vdash_i s;v^{*};(\<ithreetwo>.\<const> k)\;(t.\<loadpc> (tp\_sx)\; align\;o)$

        We must show that $s;(\<ithreetwo>.\<const> k)\;(t.\<loadpc> (tp\_sx)\; align\;o) \hookrightarrow e'^{*}$ for some $e'^{*}$.

        We have $S;\epsilon \vdash_i v^{*};(\<ithreetwo>.\<const> k)\;(t.\<loadpc> (tp\_sx)\; align\;o) : ti^{*};l;\phi$ for some $ti^{*}$, $l$, and $\phi$ because it is a premise of \refrule{Program} which we have assumed to hold.

        We also have that $\vdash s : S$, and therefore $(n \leq |b^{*}|)^{*}$ where $S_\text{tab}=n^{*}$ and $s_\text{mem}=(b^{*})^{*}$.

        Then, $(\vdash v : \ti{t_v}{a_v};\phi_v)^{*}$ for some $\ti{t_v}{a_v}^{*}$ and $\phi_v^{*}$, since it is a premise of \refrule{Code} which we have assumed to hold.

        It is important to note that $\phi_v^{*}$ cannot contain a contradiction because it contains a single equality constraint per fresh index variable (see \refrule{Admin-Const}).

        Further, we have that
        $$S;S_\text{inst}(i),\text{local } t_v^{*}\;
        {\begin{stackTL}
            \vdash (\<ithreetwo>.\<const> k)\;(t.\<loadpc> (tp\_sx)\; align\;o)
            \\ : \epsilon;\ti{t_v}{a_v}^{*};\phi_v^{*} \rightarrow ti^{*};l;\phi
        \end{stackTL}}$$
        because it too is a premise of \refrule{Code}.

        Then,
        $$S_\text{inst}(i) \vdash (\<ithreetwo>.\<const> k) :
        {\begin{stackTL}
            \epsilon;\ti{t_v}{a_v}^{*};\phi_v^{*}
            \\ \rightarrow \ti{\<ithreetwo>}{a};\ti{t_v}{a_v}^{*};\phi_v^{*}, \ti{\<ithreetwo>}{a},(= a\; \ti{\<ithreetwo>}{k})
        \end{stackTL}}$$
        where
        $$\phi_v^{*},\ti{\<ithreetwo>}{a},(= a\; \ti{\<ithreetwo>}{k}) \implies
        {\begin{stackTL}
            (\<ge> (\<add> a\; \ti{\<ithreetwo>}{o}) \ti{\<ithreetwo>}{0}),
            \\ (\<le>
            {\begin{stackTL}
                (\<add> a\; (\<add> \ti{\<ithreetwo>}{o+width}))
                \\ \ti{\<ithreetwo>}{n_2*64 \text{Ki}})
            \end{stackTL}}
        \end{stackTL}}$$ and $n_2*64 \text{Ki} = S_\text{mem}(i,j)$
        by \reflemma{Inversion} on \refrule{Composition} and \refrule{Store-Prechk}.

        Because we have
        $$\phi_v^{*},\ti{\<ithreetwo>}{a},(= a\; \ti{\<ithreetwo>}{k}) \implies
        {\begin{stackTL}
            (\<ge> (\<add> a\; \ti{\<ithreetwo>}{o}) \ti{\<ithreetwo>}{0}),
            \\ (\<le>
            {\begin{stackTL}
                (\<add> a\; (\<add> \ti{\<ithreetwo>}{o+width}))
                \\ \ti{\<ithreetwo>}{n_2*64 \text{Ki}})
            \end{stackTL}}
        \end{stackTL}}$$, then we must have $k + o \geq 0$ and $k+o+|tp| \leq n_2*64 \text{Ki}$.

        Recall $\vdash s : S$.
        Then, since $n_2*64 \text{Ki} = S_\text{mem}(i,j)$, we have $s_\text{mem}(i,j)=b_2^{*}$ where $n_2*64 \text{Ki} \leq |b_2^{*}|$.

        Therefore, it must be the case that $k+o \geq 0$ and $k+o+|tp|<|b_2^{*}|$, and therefore $s_\text{mem}(i,k+o,|tp|)=b_3^{*}$ for some $b_3^{*}$ that is a subsequence of $b_2^{*}$.
        Then, $s;(\<ithreetwo>.\<const> k)\;(t.\<loadpc> (tp\_sx)\; align\;o) \hookrightarrow_i t.\<const> \text{const}_t^{sx}(b_3^{*})$.

        \item Case: $\vdash_i s;v^{*};(\<ithreetwo>.\<const> k)\;t.\<loadpc> align\;o$

        Same as above, except with $|t|$ replacing $|tp|$ and $\text{const}_t(b_3^{*})$ instead of $\text{const}_t^{sx}(b_3^{*})$.

        \item Case: $\vdash_i s;v^{*};(\<ithreetwo>.\<const> k)\;(t.\<const> c)\;(t.\<storepc> tp\; align\; o)$

        We must show that $s;(\<ithreetwo>.\<const> k)\;(t.\<storepc> tp\; align\;o) \hookrightarrow e'^{*}$ for some $e'^{*}$.

        We have $$S;() \vdash_i v^{*};(\<ithreetwo>.\<const> k)\;(t.\<storepc> (tp\_sx)\; align\;o) : ti^{*};l;\phi$$ for some $ti^{*}$, $l$, and $\phi$ because it is a premise of \refrule{Program} which we have assumed to hold.

        We also have that $\vdash s : S$, and therefore $(n \leq |b^{*}|)^{*}$ where $S_\text{tab}=n^{*}$ and $s_\text{mem}=(b^{*})^{*}$.

        Then, $(\vdash v : \ti{t_v}{a_v};\phi_v)^{*}$ for some $\ti{t_v}{a_v}^{*}$ and $\phi_v^{*}$, since it is a premise of \refrule{Code} which we have assumed to hold.

        It is important to note that $\phi_v^{*}$ cannot contain a contradiction because it contains a single equality constraint per fresh index variable (see \refrule{Admin-Const}).

        Further, we have that
        $$S;S_\text{inst}(i),\text{local } t_v^{*}\;
        {\begin{stackTL}
            \vdash (\<ithreetwo>.\<const> k)\;(t.\<storepc> tp\; align\;o)
            \\ : \epsilon;\ti{t_v}{a_v}^{*};\phi_v^{*} \rightarrow ti^{*};l;\phi
        \end{stackTL}}$$
        because it too is a premise of \refrule{Code}.

        Then,
        $$S_\text{inst}(i)
        {\begin{stackTL}
            \;\vdash (\<ithreetwo>.\<const> k)\;(t.\<const> c) :
            \\ {\begin{stackTL}
                \epsilon ; \ti{t_v}{a_v}^{*};\phi_v^{*}
                \\ \rightarrow \ti{\<ithreetwo>}{a}\;\ti{t}{a_2};\ti{t_v}{a_v}^{*};\phi_v^{*},
                {\begin{stackTL}
                    \ti{\<ithreetwo>}{a},(= a\; \ti{\<ithreetwo>}{k}),
                    \\ \ti{t}{a_2},(= a_2\;\ti{t}{c})
                \end{stackTL}}
        \end{stackTL}}
        \end{stackTL}}$$
        where
        $$\phi_v^{*},{\begin{stackTL}
            \ti{\<ithreetwo>}{a},(= a\; \ti{\<ithreetwo>}{k}),
            \\ \ti{t}{a_2},(= a_2\;\ti{t}{c})
        \end{stackTL}} \implies
        {\begin{stackTL}
            (\<ge> (\<add> a\; \ti{\<ithreetwo>}{o}) \ti{\<ithreetwo>}{0}),
            \\ (\<le>
            {\begin{stackTL}
                (\<add> a\; (\<add> \ti{\<ithreetwo>}{o+width}))
                \\ \ti{\<ithreetwo>}{n_2*64 \text{Ki}})
            \end{stackTL}}
        \end{stackTL}}$$ and $n_2*64 \text{Ki} = S_\text{mem}(i,j)$
        by \reflemma{Inversion} on \refrule{Composition} and \refrule{Load-Prechk}.

        Because we have
        $${\begin{stackTL}
            \ti{\<ithreetwo>}{a},(= a\; \ti{\<ithreetwo>}{k}),
            \\ \ti{t}{a_2},(= a_2\;\ti{t}{c})
        \end{stackTL}} \implies
        {\begin{stackTL}
            (\<ge> (\<add> a\; \ti{\<ithreetwo>}{o}) \ti{\<ithreetwo>}{0}),
            \\ (\<le>
            {\begin{stackTL}
                (\<add> a\; (\<add> \ti{\<ithreetwo>}{o+width}))
                \\ \ti{\<ithreetwo>}{n_2*64 \text{Ki}})
            \end{stackTL}}
        \end{stackTL}}$$, then we must have $k + o \geq 0$ and $k+o+|tp| \leq n_2*64 \text{Ki}$.

        Recall $\vdash s : S$.
        Then, since $n_2*64 \text{Ki} = S_\text{mem}(i,j)$, we have $s_\text{mem}(i,j)=b_2^{*}$ where $n_2*64 \text{Ki} \leq |b_2^{*}|$.

        It must be the case that $k+o \geq 0$ and $k+o+|tp|<|b_2^{*}|$, and therefore $s_\text{mem}(i,k+0,|tp|)=b_3^{*}$ for some $b_3^{*}$ that is a subsequence of $b_2^{*}$
        Then, we can construct $s'= s$ with $s'_\text{mem}(i,k+o,|tp|)=bits_t^{|tp|}(c)$ because $|bits_t^{|tp|}(c)|=|b_3^{*}|$.
        Then, $$s;(\<ithreetwo>.\<const> k)\;(\<ithreetwo>.\<const> c)\;(t.\<storepc> tp\; align\;o) \hookrightarrow_i s';\epsilon$$

        \item Case: $\vdash_i s;v^{*};(\<ithreetwo>.\<const> c)\;(t.\<storepc> align\;o)$

        Same as above, except with $|t|$ replacing $|tp|$.

        \item Case: $\vdash_i (\<ithreetwo>.\<const> c)\;\<callindirect> ti_1^{*};l_1;\phi_1 \rightarrow ti_2^{*};l_2;\phi_2$

        We must show that $(\<ithreetwo>.\<const> c)\;\<callindirect> ti_1^{*};l_1;\phi_1 \rightarrow ti_2^{*};l_2;\phi_2 \hookrightarrow e'^{*}$ for some $e'^{*}$.

        We have $S;() \vdash_i v^{*};(\<ithreetwo>.\<const> c)\;\<callindirect> ti_1^{*};l_1;\phi_1 \rightarrow ti_2^{*};l_2;\phi_2 : ti^{*};l;\phi$ for some $ti^{*}$, $l$, and $\phi$ because it is a premise of \refrule{Program} which we have assumed to hold.

        We also have that $\vdash s : S$, and therefore $S_\text{tab}(i)=(n,tfi^{n})$ and $(S \vdash cl : tfi)^{*}$ where $s_\text{tab}(i)=cl^{*}$ and $n\leq |cl^{*}|$.

        Then, $(\vdash v : \ti{t_v}{a_v};\phi_v)^{*}$ for some $\ti{t_v}{a_v}^{*}$ and $\phi_v^{*}$, since it is a premise of \refrule{Code} which we have assumed to hold.

        It is important to note that $\phi_v^{*}$ cannot contain a contradiction because it contains a single equality constraint per fresh index variable (see \refrule{Admin-Const}).

        Then,
        $$S_\text{inst}(i)
        {\begin{stackTL}
            \;\vdash (\<ithreetwo>.\<const> c) :
            \\ {\begin{stackTL}
                \epsilon ; \ti{t_v}{a_v}^{*};\phi_v^{*}
                \\ \rightarrow \ti{\<ithreetwo>}{a};\ti{t_v}{a_v}^{*};\phi_v^{*},\ti{\<ithreetwo>}{a},(= a\; \ti{\<ithreetwo>}{c})
        \end{stackTL}}
        \end{stackTL}}$$
        where $\phi_v^{*},\ti{\<ithreetwo>}{a},(= a\; \ti{\<ithreetwo>}{c}) \implies (\<gt>\; n\; a) \land (\<le> \ti{\<ithreetwo>}{0}\; a) $
        by \reflemma{Inversion} on \refrule{Composition} and \refrule{Call-Indirect-Prechk}.

        We have
        $$\forall i.\; (\phi \implies \neg (= \ti{\<ithreetwo>}{i}\; a)) \lor\; tfis(i) <: ti_1^{*};l_1;\phi_1 \rightarrow ti_2^{*};l_2;\phi_2$$
        where $tfis=tfi^{n}$, because it is a premise of \refrule{Call-Indirect-Prechk} which we have assumed to hold by \reflemma{Inversion}.
        Since $\ti{\<ithreetwo>}{a};\ti{t_v}{a_v}^{*};\phi_v^{*},\ti{\<ithreetwo>}{a},(= a\; \ti{\<ithreetwo>}{c}) \implies (= \ti{\<ithreetwo>}{c}\; a)$, then it has to be the case that $tfis(c) <: ti_1^{*};l_1;\phi_1 \rightarrow ti_2^{*};l_2;\phi_2$.

        Let, $\{\text{inst } j, \text{ func } f\} = s_\text{tab}(i,c)$.
        Recall from before that $(S\vdash cl: tfi)^{*}$.
        Then, $S \vdash \{\text{inst } j, \text{ func } f\} : tfi_2$ for some $tfi_2$.

        $S_\text{inst}(j) \vdash f : tfi_2$, as it is a premise of $S \vdash \{\text{inst } j, \text{ func } f\} : tfi_2$.

        Then, we know that $f= \<func> tfi_2\; \<local> \dots$ because it is a premise of $S_\text{inst}(j) \vdash f : tfi_2$, and we know that $tfi_2<:ti_1^{*};l_1;\phi_1 \rightarrow ti_2^{*};l_2;\phi_2$.

        Thus, ${\begin{stackTL}s;(\<ithreetwo>.\<const> c)\;\<callindirect> ti_1^{*};l_1;\phi_1 \rightarrow ti_2^{*};l_2;\phi_2
            \\ \hookrightarrow \<call> \{\text{inst } j, \text{ func } f\}\end{stackTL}}$.

    \item Otherwise, we reuse the \wasm proof, which we can do thanks to \autoref{thm:programerasure}.
    \end{itemize}
\end{proof}


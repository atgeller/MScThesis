\subsection{Progress}
\emph{Progress} ensures that if a program is well-typed then it either: entirely consists of values, traps, or is reducible (\ie there exists another program that it reduces to).
Since most \name instructions have the same semantics as in \wasm, and every \name type includes all the information of a \wasm type, we can reuse the \wasm proof for those instructions by using the erasure function from Section \ref{subsec:erasure}.
The intuition for this is that the \name indexed type system provides strictly more information than the \wasm type system.
However, for \name instructions that do not have the same semantics as in \wasm, specifically \prechk-tagged instructions, we still must prove those cases.

\begin{theorem}{Progress}
    If $\vdash_i s;v^{*};e^{*} : ti^{*};l;\phi$ then either $e^{*} = v'^{*}$, $e^{*}= \<trap>$, or $s;v^{*};e^{*} \hookrightarrow_i s';v'^{*};e'^{*}$.
\end{theorem}
\begin{proof}
    We proceed by induction on $\vdash_i s;v^{*};e^{*} : ti^{*};l;\phi$.

    Because $\vdash_i s;v^{*};e^{*} : ti^{*};l;\phi$, we know that $\vdash s : S$ for some $S$, and that $S; \epsilon \vdash_i v^{*};e^{*}:ti^{*};l;\phi$ because they are premises of $admin-program$ which we have assumed to hold.

    Then we know that $(\vdash v: \ti{t_v}{a_v};\phi_v)^{*}$ and $S;S_\text{inst}(i),\text{local } t_v^{*} \vdash e^{*} : \epsilon;\ti{t_v}{a_v}^{*};\phi_v^{*} \rightarrow ti^{*};l;\phi$ because they are premises of $admin-code$ which we have assumed to hold.

    \todo{Part about expressions to the left reducing or not}

    \begin{itemize}
        \item Case: $\vdash_i s;v^{*};v_2^{*}\;t.\<divpc>$

        We have $S;S_\text{inst}(i),\text{local } t_v^{*} \vdash v_2^{*}\;t.\<divpc> : \epsilon;l_1;\phi_1 \rightarrow ti^{*};l;\phi$.

        $v_2^{*} : \epsilon;l_1;\phi_1 \rightarrow \ti{t}{a_1}\;\ti{t}{a_2};l_1;\phi_1,\ti{t}{a_1},(= a_1\; \ti{t}{c_1}),\ti{t}{a_2},(= a_2\; \ti{t}{c_2})$, where $\satisfies{\phi_1,\ti{t}{a_1},(= a_1\; \ti{t}{c_1}),\ti{t}{a_2},(= a_2\; \ti{t}{c_2})}{\ti{t}{a_2}}{\neg(= a_2\; \ti{t}{0})}$ by \reflemma{Inversion} on \refrule{Composition} and \refrule{Div-Prechk}.

        Suppose $c_2=0$, this would mean that $\satisfies{\phi_1,\ti{t}{a_1},(= a_1\; \ti{t}{c_1}),\ti{t}{a_2},(= a_2\; \ti{t}{c_2})}{\ti{t}{a_2}}{\neg(= a_2\; \ti{t}{0})}$ would be a contradiction.

        Therefore, it must be the case that $c_2\neq 0$, and therefore there must exist some $c_3$ such that $c_3=div(c_1,c_2)$.
        Then, $s;v^{*};v_2^{*}\;t.\<divpc> \hookrightarrow_i s;v^{*};(t.\<const> c_3)$.
    \end{itemize}
\end{proof}
\section{Subject Reduction}

\begin{theorem}{Subject Reduction}
  Bla bla bla
\end{theorem}
\begin{proof}
By case analysis on the reduction rules.
Note that the locals and globals are omitted where they are unnecessary and trivially unchanged.

\todo{The formatting is all over the place because I kept running into problems. I'll try to continuously leave a comment by the last good formatting.}

\begin{itemize}
    %% Binop -> const
    \item $C\vdash (t.\<const> c_1)\; (t.\<const> c_2)\; t.\<binop> : s_1;l_1;g_1;\phi_1 \rightarrow s_2;l_2;g_2;\phi_2$ 
    \\ $\land$ $(t.\<const> c_1)\; (t.\<const> c_2)\; t.\<binop> \hookrightarrow t.\<const> c$ if $c=\<binop>(c_1,c_2)$

    By $const$ and $binop$, we know that $s_2$ has the form $s_1 \ti{t}{a_3}$ and that
    \begin{align*}
        \phi_1&, 
        \begin{stackTL}
            \ti{t}{a_1}, (\<eq> a_1\;\ti{t}{c_1}), \\
            \ti{t}{a_2}, (\<eq> a_2 (t c_2)), \\
            \ti{t}{a_3}, (\<eq> a_3\;(\<binop>\; a_1\; a_2))
        \end{stackTL} \\    
        &\implies \phi_2
    \end{align*}

    By $const$, $C \vdash t.\<const> c :
        \begin{stackTL}
            \epsilon;l_1;g_1;\phi_1 \\ 
            \rightarrow \ti{t}{a_3};l_1;g_1;\phi_1,\ti{t}{a_3},(\<eq> a_3\;\ti{t}{c})
        \end{stackTL}$.

    Because $c=binop_t(c_1,c_2)$, then by $\implies$,
    \begin{align*}
        \phi_1,\ti{t}{a},(\<eq> a\;\ti{t}{c}) &\implies \phi_1,
        \begin{stackTL}
            \ti{t}{a_1}, (\<eq> a_1\; \ti{t}{c_1}), \\
            \ti{t}{a_2}, (\<eq> a_2\; \ti{t}{c_2}), \\
            \ti{t}{a_3}, (\<eq> a_3\;(\<binop>\;a_1 a_2))
        \end{stackTL}
    \end{align*}

    Therefore, $C \vdash (t.\<const> c) : s_1;l_1;g_1;\phi_1 \rightarrow s_1\; \ti{t}{a_3};l_1;g_1;\phi_2$, by $extension$ and $weakening$

    %% Binop -> trap
    \item  $C\vdash (t.\<const> c_1)\; (t.\<const> c_2)\; t.\<binop> : s_1;l_1;g_1;\phi_1 \rightarrow s_2;l_2;g_2;\phi_2$
    \\ $\land$ $(t.\<const> c_1)\; (t.\<const> c_2)\; t.\<binop> \hookrightarrow \<trap>$

        Trivially, $C\vdash \<trap> : s_1;l_1;g_1;\phi_1 \rightarrow s_2;l_2;g_2;\phi_2$ by $trap$.

    %% Relop
    \item $C\vdash (t.\<const> c_1)\; (t.\<const> c_2)\; t.\<relop> : s_1;l_1;g_1;\phi_1 \rightarrow s_2;l_2;g_2;\phi_2$
    \\$\land$ $(t.\<const> c_1)\; (t.\<const> c_2)\; t.\<relop> \hookrightarrow t.\<const> c$ if $c=\<relop>(c_1,c_2)$

        Similar to $\<binop>$.

    %% Testop
    \item $C\vdash (t.\<const> c)\; t.\<testop> : s_1;l_1;g_1;\phi_1 \rightarrow s_2;l_2;g_2;\phi_2$ 
    \\ $\land$ $(t.\<const> c)\; (t.\<const> c)\; t.\<testop> \hookrightarrow t.\<const> c_2$ where $c_2=\<testop>(c)$

        By $const$ and $testop$, we know that $s_2$ has the form $s_1 \ti{t}{a_2}$, $l_2=l_1$, $g_2=g_1$, and that
        \begin{align*}
            \phi_1&, 
            \begin{stackTL}
                \ti{t}{a_1}, (\<eq> a_1\;\ti{t}{c}), \\
                \ti{t}{a_2}, (\<eq> a_2\;(\<testop>\;a_1))
            \end{stackTL} \\
            &\implies \phi_2
        \end{align*}

        By $const$, $C \vdash t.\<const> c :
            \begin{stackTL}
                \epsilon;l_1;g_1;\phi_1 \\ 
                \rightarrow \ti{t}{a_2};l_1;g_1;\phi_1,\ti{t}{a_2},(\<eq> a_2\;\ti{t}{c_2})
            \end{stackTL}$.

        Because $c_2=testop_t(c)$, then by $\implies$,
        \begin{align*}
            \phi_1,\ti{t}{a},(\<eq> a\;\ti{t}{c_2}) &\implies \phi_1,
            \begin{stackTL}
                \ti{t}{a_1}, (\<eq> a_1\;\ti{t}{c}), \\
                \ti{t}{a_2}, (\<eq> a_2\;(\<testop>\;a_1))
            \end{stackTL}
        \end{align*}

    %% Unreachable
    \item $C\vdash \<unreachable> : s_1;l_1;g_1;\phi_1 \rightarrow s_2;l_2;g_2;\phi_2$
    \\ $\land$ $\<unreachable> \hookrightarrow \<trap>$
    
    Trivially, $C\vdash \<trap> : s_1;l_1;g_1;\phi_1 \rightarrow s_2;l_2;g_2;\phi_2$ by $trap$.

    %% Nop
    \item $C\vdash \<nop> : \epsilon;l;g;\phi \rightarrow \epsilon;l;g;\phi$ 
    \\ $\land$ $\<nop> \hookrightarrow \epsilon$
    
        Trivially, $C\vdash \epsilon : \epsilon;l;g;\phi \rightarrow \epsilon;l;g;\phi$ by $empty$. \todo{Not quite this trivial: what about extension and weakening?}

    %% Drop
    \item $C\vdash (t.\<const> c)\; \<drop> : s_1;l_1;g_1;\phi_1 \rightarrow s_2;l_2;g_2;\phi_2$
    \\ $\land$ $(t.\<const> c)\; \<drop> \hookrightarrow \epsilon$

        By $const$ and $drop$, we know that $s_2 = s_1$, $l_2 = l_1$, $g_2 = g_1$, and $\phi_1 \implies \phi_2$.

        By $empty$, $C\vdash \epsilon : \epsilon;l_1;g_1;\phi_1 \rightarrow \epsilon;l_1;g_1;\phi_1$

        By $extension$, $C\vdash \epsilon : s_1;l_1;g_1;\phi_1 \rightarrow s_1;l_1;g_1;\phi_1$

        Since $\phi_1 \implies \phi_2$, then $C\vdash \epsilon : s_1;l_1;g_1;\phi_1 \rightarrow s_1;l_1;g_1;\phi_2$ by $weakening$.

    %% Select
    \item Case: $C\; {\begin{stackTL}
        \vdash (t.\<const> c_1)\;(t.\<const> c_2)\;(\<ithreetwo>.\<const> 0)\;\<select> 
        \\ : s_1;l_1;g_1;\phi_1 \rightarrow s_2;l_2;g_2;\phi_2
    \end{stackTL}}$
    \\ $\land$ $(t.\<const> c_1)\;(t.\<const> c_2)\;(\<ithreetwo>.\<const> 0)\;\<select> \hookrightarrow (t.\<const> c_2)$

    \todo{This is an acceptable format}

        By $const$ and $select$, we know that $s_2$ has the form $s_1\;\ti{a_3}$, $l_2 = l_1$, $g_2 = g_1$, and
        $
        {\begin{stackTL}
            \phi_1, {\begin{stackTL}
                \ti{t}{a_1}, (\<eq> a_1\;\ti{t}{c_1}), \\
                \ti{t}{a_2}, (\<eq> a_2\;\ti{t}{c_2}), \\
                \ti{\<ithreetwo>}{a}, (\<eq> a\;\ti{\<ithreetwo>}{0}), \\
                \ti{t}{a_3}, (
                {\begin{stackTL}
                    (\<eqz> a) \land (\<eq> a_3\;a_2)) \\
                    \lor (\neg(\<eqz> a) \land (\<eq> a_3\;a_1)))
                \end{stackTL}} \\
            \end{stackTL}} \\
            \implies \phi_2
        \end{stackTL}}
        $
        
        By $const$, \\
        $ C \vdash (t.\<const> c_2) :
            {\begin{stackTL}
                \epsilon;l_1;g_1;\phi_1 \\
                \rightarrow \ti{t}{a_3};l_1;g_1;\phi_1,\ti{t}{a_3},(\<eq> a_3\; \ti{t}{c_2}) \\
            \end{stackTL}} $

        Then, by $extension$,\\
        $ C \vdash (t.\<const> c_2) : 
            {\begin{stackTL}
                s_1;l_1;g_1;\phi_1 \\
                \rightarrow s_1\;\ti{t}{a_3};l_1;g_1;\phi_1,\ti{t}{a_3},(\<eq> a_3 \; \ti{t}{c_2}) \\
            \end{stackTL}} $

        By $implies$, we have \\
        $\phi_1,\ti{t}{a_3},(\<eq> a_3\; \ti{t}{c_2}) \implies \phi_1, {\begin{stackTL}
            \ti{t}{a_1}, (\<eq> a_1\; \ti{t}{c_1}), \\
            \ti{t}{a_2}, (\<eq> a_2\; \ti{t}{c_2}), \\
            \ti{\<ithreetwo>}{a}, (\<eq> a\;\ti{\<ithreetwo>}{0}), \\
            \ti{t}{a_3}, (
            {\begin{stackTL}
                (\<eqz> a) \land (\<eq> a_3\;a_2)) \\
                \lor (\neg(\<eqz> a) \land (\<eq> a_3\;a_1)))
            \end{stackTL}} \\
        \end{stackTL}} \\$

        Therefore,
        $ C \vdash (t.\<const> c_2) :
            s_1;l_1;g_1;\phi_1 
            \rightarrow s_1\;\ti{t}{a_3};l_1;g_1;\phi_2$, by $weakening$

    %% Block
    \item Case: $C \vdash v^n \; \<block> \; tfi \; e^{*} \<end> : s_1;l_1;g_1;\phi_1 \rightarrow s_2;l_2;g_2;\phi_2$
    \\ $\land$ $v^n \; \<block> \; tfi \; e^{*} \<end> \hookrightarrow \; \<label>_m \{ \epsilon \} \; v^n \; e^{*} \<end>$

    \todo{THIS SEEMS VERY WRONG }
\end{itemize}
\end{proof}

\subsection{Subject Reduction}
\emph{Subject reduction}, also sometimes referred to as ``type preservation'', ensures that if a program has a specific type, then the program will have the same type after a reduction step.
We use a number of lemmas in the subject reduction proof.
\todo{Worst transition sentence ever}

\subsection{Lemmas Used in Subject-Reduction Proof}
\begin{lemma}{(Inversion)}
    \todo{So many cases to write out}

    %% const
    If $C \vdash t.\<const> c : ti_1^{*};l_1;\phi_1 \rightarrow ti_2^{*};l_2;\phi_2$,
    then $ti_1^{*};l_1;\phi_1 \rightarrow ti_1^{*}\;\ti{t}{a};l_1;\phi_1,\ti{t}{a},(= a \; \ti{t}{c}) <: ti_1^{*};l_1;\phi_1 \rightarrow ti_2^{*};l_2;\phi_2$.

    %% binop
    If $C \vdash t.binop : ti_1^{*};l_1;\phi_1 \rightarrow ti_2^{*};l_2;\phi_2$,
    then $ti_1^{*} = ti^{*} \; \ti{t}{a_1} \; \ti{t}{a_2}$,
    and $ti_1^{*};l_1;\phi_1 \rightarrow ti^{*} \; \ti{t}{a_3};l_1;\phi_1,\ti{t}{a_3},(= a_3\;(binop\;a_1\;a_2)) <: ti_1^{*};l_1;\phi_1 \rightarrow ti_2^{*};l_2;\phi_2$.

    %% testop
    If $C \vdash t.testop : ti_1^{*};l_1;\phi_1 \rightarrow ti_2^{*};l_2;\phi_2$,
    then $ti_1^{*} = ti^{*} \; \ti{t}{a_1}$,
    and $ti_1^{*};l_1;\phi_1 \rightarrow ti^{*} \; \ti{\<ithreetwo>}{a_2};l_1;\phi_1,\ti{\<ithreetwo>}{a_2},(= a_2\;(testop\;a_1)) <: ti_1^{*};l_1;\phi_1 \rightarrow ti_2^{*};l_2;\phi_2$.

    %% relop
    If $C \vdash t.relop : ti_1^{*};l_1;\phi_1 \rightarrow ti_2^{*};l_2;\phi_2$,
    then $ti_1^{*} = ti^{*} \; \ti{t}{a_1} \; \ti{t}{a_2}$,
    and $ti_1^{*};l_1;\phi_1 \rightarrow ti^{*} \; \ti{\<ithreetwo>}{a_3};l_1;\phi_1,\ti{\<ithreetwo>}{a_3},(= a_3\;(relop\;a_1\;a_2)) <: ti_1^{*};l_1;\phi_1 \rightarrow ti_2^{*};l_2;\phi_2$.

    %% set-local
    If $C \vdash \<setlocal> i : ti_1^{*};l_1;\phi_1 \rightarrow ti_2^{*};l_2;\phi_2$,
    then $ti_1^{*} = ti_2^{*} \; \ti{t}{a}$, $l_2 = l_1 \text{ except } l_2(i) = \ti{t}{a_2}$
    and $ti_1^{*};l_1;\phi_1 \rightarrow ti_2^{*};l_2;\phi_1,\ti{t}{a_2},(= a_2\;a) <: ti_1^{*};l_1;\phi_1 \rightarrow ti_2^{*};l_2;\phi_2$,
    where $t = C_\text{local}(i)$.

    %% composition
    If $C \vdash e_1^{*} \; e_2 : ti_1^{*};l_1;\phi_1 \rightarrow ti_2^{*};l_2;\phi_2$,
    then $C \vdash e_1^{*} : ti_3^{*};l_3;\phi_3 \rightarrow ti_4^{*};l_4;\phi_4$,
    $C \vdash e_2 : ti_4^{*};l_4;\phi_4 \rightarrow ti_5^{*};l_5;\phi_5$,
    and $ti_1^{*};l_1;\phi_1 \rightarrow ti_2^{*};l_2;\phi_2 <: ti_3^{*};l_3;\phi_3 \rightarrow ti_5^{*};l_5;\phi_5$.

    %% admin-program
    If $\vdash_i s;v^{*};e^{*} : ti^{*};l;\phi$,
    then $\vdash s : S$,
    and $S;\epsilon \vdash_i v^{*};e^{*} : ti^{*};l;\phi$.

    %% admin-code
    If $S;(ti^n;l;\phi)^{?} \vdash_i s;v^{*};e^{*} : ti^n;l;\phi$,
    then $(\vdash v : \ti{t}{a};\phi_v)^{*}$,
    and $S;C \vdash e^{*} : \epsilon:\ti{t}{a}^{*};\phi_v^{*} \rightarrow ti^n;l;\phi$,
    where $C = S_{\text{inst}}(i),\text{local} \; t^{*}, \text{return} \; (ti^n;l;\phi)^{?}$.

    %% admin-const
    If $\vdash v : ti;\phi_v$,
    then $ti = \ti{t}{a}$,
    and $\phi_v = \circ,\ti{t}{a},(= a \; \ti{t}{c})$,
    where $t.\<const> c = v$.

    %% local
    If $S;C \vdash \<local> \{ i;v_l^{*} \} \; e^{*} \<end> : ti_1^{*};l_1;\phi_1 \rightarrow ti_2^{*};l_2;\phi_2$,
    then $S;(ti^n;l_3;\phi_3) \vdash_i v_l^{*};e^{*} : ti^n;l_3;\phi_3$,
    and $ti_1^{*};l_1;\phi_1 \rightarrow ti_1^{*} \; ti^n;l_1;\phi_1,\phi_3 <: ti_1^{*};l_1;\phi_1 \rightarrow ti_2^{*};l_2;\phi_2$.

\end{lemma}
\begin{proof}
    Proof omitted, but follows from induction over typing derivations.
\end{proof}

\begin{lemma}{(Coherence)}
    \todo{Nested composition can be expressed with only one subtyping relation.}
\end{lemma}
\begin{proof}
\end{proof}

\begin{lemma}{(Nested-Type-Preserved)}

    If $C \vdash v^n : \epsilon;l_3;\phi_3 \rightarrow ti^n;l_3;\phi_4$ is a subderivation of $C \vdash  L^j [v^n] : s_1;l_1;\phi_1 \rightarrow s_2;l_2;\phi_2$,
    \\then $C \vdash v^n : \epsilon;l_1;\phi_1 \rightarrow ti^n;l_3;\phi_4$ after reduction
\end{lemma}
\begin{proof}
    By induction on $j$.
    \begin{itemize}
        \item Base case: $j=0$

            $C \vdash v_0^{*} \; v^n \; e^{*} \<end> : s_1;l_1;\phi_1 \rightarrow s_2;l_2;\phi_2$ for some $v_0^{*}$ and $e^{*}$ by expanding $L^0$.

            $C \vdash (t.\<const> c)^{*} : \epsilon;l_1;\phi_0 \rightarrow \ti{t}{a}^{*};l_1;\phi_0,\ti{t}{a}^{*},(\<eq> a \; \ti{t}{c})$ where $v_0^{*}=(t.\<const> c)^{*}$ and $\phi_1 \implies \phi_0$ by $inversion$.

            Then, $\phi_0,\ti{t}{a}^{*},(\<eq> a \; \ti{t}{c}) \implies \phi_3$.

            $C \vdash v^n : \ti{t}{a}^{*};l_3;\phi_0,\ti{t}{a}^{*},(\<eq> a \; \ti{t}{c}) \rightarrow \ti{t}{a}^{*},\ti{t}{a}^{*}\;ti^n;l_3;\phi_4$, by $subtyping$.

            $C \vdash v^n : \epsilon;l_3;\phi_0,\ti{t}{a}^{*},(\<eq> a \; \ti{t}{c}) \rightarrow \ti{t}{a}^{*}\;ti^n;l_3;\phi_4$, by $inversion$.

            If $v_0^{*}$ are not executed (\ie they are not part of the reduced expression), then $a^{*}$ are fresh, so $\phi_0 \implies \phi_0,\ti{t}{a}^{*},(\<eq> a \; \ti{t}{c})$, and therefore $C \vdash v^n : \epsilon;l_1;\phi_0 \rightarrow ti^n;l_1;\phi_4$ by $subtyping$ and since $l_1=l_3$.

            Then, $C \vdash v^n : \epsilon;l_1;\phi_1 \rightarrow ti^n;l_1;\phi_4$ by $subtyping$.

        \item Induction case: $j=k+1$

            $C \vdash \<label>_n \{ e_0^{*} \} \; v_0^{*} \; L^k[v^n] \; e_1^{*} \<end> : s_1;l_1;\phi_1 \rightarrow s_2;l_2;\phi_2$ for some $v_0^{*}$, $e_0^{*}$, and $e_1^{*}$ by expanding $L^j$.

            $C \vdash (t.\<const> c)^{*} : \epsilon;l_1;\phi_0 \rightarrow \ti{t}{a}^{*};l_1;\phi_0,\ti{t}{a}^{*},(\<eq> a \; \ti{t}{c})$ where $v_0^{*}=(t.\<const> c)^{*}$ and $\phi_1 \implies \phi_0$ by $inversion$.

            Then, $C \vdash L^k[v^n] : \ti{t}{a}^{*};l_1;\phi_0,\ti{t}{a}^{*},(\<eq> a \; \ti{t}{c}) \rightarrow s_5;l_5;\phi_5$ for some $s_5;l_5;\phi_5$ by $inversion$.

            $C \vdash v^n : \epsilon;l_1;\phi_0,\ti{t}{a}^{*},(\<eq> a \; \ti{t}{c}) \rightarrow ti^n;l_1;\phi_4$ by the inductive hypothesis.

            If $v_0^{*}$ are not executed (\ie after one reduction step), $a^{*}$ are fresh, so $\phi_0 \implies \ti{t}{a}^{*},(\<eq> a \; \ti{t}{c})$, and therefore $C \vdash v^n : \epsilon;l_1;\phi_0 \rightarrow \ti{t}{a}^{*}\;ti^n;l_3;\phi_5$ by $subtyping$ and since $l_1=l_3$.

            Then, $C \vdash v^n : \epsilon;l_1;\phi_1 \rightarrow \ti{t}{a}^{*}\;ti^n;l_3;\phi_3$ by $subtyping$.

    \end{itemize}
\end{proof}


In many reduction cases, there are values on the stack that get consumed by reducing an instruction.
This creates a bit of a problem because those values represent intermediate state, and as such will introduce new index variables to the index type context in their postcondition.
After reduction, the intermediate state is no longer present, so we lose those index variables from the postconditions.

For example, $(t.\<const> c)\; \<drop>$ could be typed as $\epsilon;l;\phi \rightarrow \epsilon;l;\phi,\ti{t}{a},(= a\; \ti{t}{c})$ where $a$ represent the value on the stack $t.\<const> c$.
This would reduce to $\epsilon$, and then we lose the information about $a$ in the postcondition index type system.
However, this can be solved using implication, as we know $a$ is fresh from the const rule, and therefore $\satisfies{\phi}{l}{\phi,\ti{t}{a},(= a\; \ti{t}{c})}$ after reduction.
This type of pattern will appear in any case of the proof that consumes values.

\begin{theorem}{Subject Reduction}
  If $\vdash_i s;v^{*};e^{*} : ti^{*};l;\phi$ and $s;v^{*};e^{*} \hookrightarrow_i s';v'^{*};e'^{*}$ then $\vdash_i s';v'^{*};e'^{*} : ti^{*};l;\phi$.
\end{theorem}
\begin{proof}
By case analysis on the reduction rules.

\begin{itemize}
    %% Binop -> const
    \item $C\vdash (t.\<const> c_1)\; (t.\<const> c_2)\; t.binop : ti_1^{*};l_1;\phi_1 \rightarrow ti_2^{*};l_2;\phi_2$
    \\ $\land$ $(t.\<const> c_1)\; (t.\<const> c_2)\; t.binop \hookrightarrow t.\<const> c$ where $c=binop(c_1,c_2)$

        By $inversion$ on $const$ and $binop$, we know that $ti_2^{*} = ti_1^{*} \ti{t}{a_3}$, $l_2=l_1$, and that
        \begin{align*}
            \phi_1&,
            \begin{stackTL}
                \ti{t}{a_1}, (= a_1\; \ti{t}{c_1}), \\
                \ti{t}{a_2}, (= a_2\; \ti{t}{c_2}), \\
                \ti{t}{a_3}, (= a_3\; (binop\; a_1\; a_2))
            \end{stackTL} \\
            &\implies \phi_2
        \end{align*}

        By $const$, $C \vdash t.\<const> c :
            \begin{stackTL}
                \epsilon;l_1;\phi_1 \\
                \rightarrow \ti{t}{a_3};l_1;g_1;\phi_1,\ti{t}{a_3},(\<eq> a_3\;\ti{t}{c})
            \end{stackTL}$.

        Because $c=binop_t(c_1,c_2)$, then by $\implies$,
        \begin{align*}
            \phi_1,\ti{t}{a},(= a\; \ti{t}{c}) &\implies \phi_1,
            \begin{stackTL}
                \ti{t}{a_1}, (= a_1\; \ti{t}{c_1}), \\
                \ti{t}{a_2}, (= a_2\; \ti{t}{c_2}), \\
                \ti{t}{a_3}, (= a_3\; (binop\; a_1 a_2))
            \end{stackTL}
        \end{align*}

        Therefore, $C \vdash (t.\<const> c) : ti_1^{*};l_1;\phi_1 \rightarrow ti_1^{*}\; \ti{t}{a_3};l_1;\phi_2$, by $stack-poly$ and $sub-typing$

    %% Binop -> trap
    \item  $C\vdash (t.\<const> c_1)\; (t.\<const> c_2)\; t.binop : ti_1^{*};l_1;\phi_1 \rightarrow ti_2^{*};l_2;\phi_2$
    \\ $\land$ $(t.\<const> c_1)\; (t.\<const> c_2)\; t.binop \hookrightarrow \<trap>$

        Trivially, $C\vdash \<trap> : ti_1^{*};l_1;\phi_1 \rightarrow ti_2^{*};l_2;\phi_2$ by $trap$.

    %% Relop
    \item $C\vdash (t.\<const> c_1)\; (t.\<const> c_2)\; t.relop : ti_1^{*};l_1;\phi_1 \rightarrow ti_2^{*};l_2;\phi_2$
    \\$\land$ $(t.\<const> c_1)\; (t.\<const> c_2)\; t.relop \hookrightarrow t.\<const> c$ where $c=relop(c_1,c_2)$

        Similar to $binop$.

    %% Testop
    \item $C\vdash (t.\<const> c)\; t.testop : ti_1^{*};l_1;\phi_1 \rightarrow ti_2^{*};l_2;\phi_2$
    \\ $\land$ $(t.\<const> c)\; (t.\<const> c)\; t.testop \hookrightarrow t.\<const> c_2$ where $c_2=testop(c)$

        By $inversion$ on $const$ and $testop$, we know that $ti_2^{*}=ti_1^{*}\; \ti{t}{a_2}$, $l_2=l_1$, and that
        \begin{align*}
            \phi_1&,
            \begin{stackTL}
                \ti{t}{a_1}, (= a_1\;\ti{t}{c}), \\
                \ti{t}{a_2}, (= a_2\;(testop\;a_1))
            \end{stackTL} \\
            &\implies \phi_2
        \end{align*}

        By $const$, $C \vdash t.\<const> c :
            \begin{stackTL}
                \epsilon;l_1;g_1;\phi_1 \\
                \rightarrow \ti{t}{a_2};l_1;\phi_1,\ti{t}{a_2},(= a_2\;\ti{t}{c_2})
            \end{stackTL}$.

        Because $c_2=testop_t(c)$, then by $\implies$,
        \begin{align*}
            \phi_1,\ti{t}{a},(= a\;\ti{t}{c_2}) &\implies \phi_1,
            \begin{stackTL}
                \ti{t}{a_1}, (= a_1\;\ti{t}{c}), \\
                \ti{t}{a_2}, (= a_2\;(testop\;a_1))
            \end{stackTL}
        \end{align*}

    %% Unreachable
    \item $C\vdash \<unreachable> : ti_1^{*};l_1;\phi_1 \rightarrow ti_2^{*};l_2;\phi_2$
    \\ $\land$ $\<unreachable> \hookrightarrow \<trap>$

        Trivially, $C\vdash \<trap> : ti_1^{*};l_1;\phi_1 \rightarrow ti_2^{*};l_2;\phi_2$ by $trap$.

    %% Nop
    \item $C\vdash \<nop> : ti_1^{*};l_1;\phi_1 \rightarrow ti_2^{*};l_2;\phi_2$
    \\ $\land$ $\<nop> \hookrightarrow \epsilon$

        By $inversion$ on $nop$, we know that $ti_2^{*} = ti_1^{*}$, $l_2 = l_1$, and $\phi_1 \implies \phi_0$ and $\phi_0 \implies \phi_2$ for some $\phi_0$.

        $C\vdash \epsilon : \epsilon;l;g;\phi_0 \rightarrow \epsilon;l;g;\phi_0$ by $empty$.

        Then, $C \vdash \epsilon ti_1^{*};l;g;\phi_1 \rightarrow ti_1^{*};l;g;\phi_2$ by $stack-poly$ and $sub-typing$.

    %% Drop
    \item $C\vdash (t.\<const> c)\; \<drop> : ti_1^{*};l_1;\phi_1 \rightarrow ti_2^{*};l_2;\phi_2$
    \\ $\land$ $(t.\<const> c)\; \<drop> \hookrightarrow \epsilon$

        By $inversion$ on $compostion$, $const$, and $drop$, we know that $ti_2^{*} = ti_1^{*}$, $l_2 = l_1$, and $\phi_1 \implies \phi_0$ and $\phi_0 \implies \phi_2$ for some $\phi_0$.

        By $empty$, $C\vdash \epsilon : \epsilon;l_1;\phi_0 \rightarrow \epsilon;l_1;\phi_0$.

        Then, $C\vdash \epsilon : ti_1^{*};l_1;\phi_1 \rightarrow ti_1^{*};l_1;\phi_2$ by $stack-poly$ and $sub-typing$.

    %% Select
    \item Case: $C\; {\begin{stackTL}
        \vdash (t.\<const> c_1)\;(t.\<const> c_2)\;(\<ithreetwo>.\<const> 0)\;\<select>
        \\ : ti_1^{*};l_1;\phi_1 \rightarrow ti_2^{*};l_2;\phi_2
    \end{stackTL}}$
    \\ $\land$ $(t.\<const> c_1)\;(t.\<const> c_2)\;(\<ithreetwo>.\<const> 0)\;\<select> \hookrightarrow (t.\<const> c_2)$

        By $const$ and $select$, we know that $ti_2^{*} = ti_1^{*}\;\ti{a_3}$, $l_2 = l_1$, and
        $
        {\begin{stackTL}
            \phi_1, {\begin{stackTL}
                \ti{t}{a_1}, (= a_1\;\ti{t}{c_1}), \\
                \ti{t}{a_2}, (= a_2\;\ti{t}{c_2}), \\
                \ti{\<ithreetwo>}{a}, (= a\;\ti{\<ithreetwo>}{0}), \\
                \ti{t}{a_3},(if\; (= a\; \ti{\<ithreetwo>}{0})\; (= a_3\; a_2)\; (= a_3\; a_1))
            \end{stackTL}} \\
            \implies \phi_2
        \end{stackTL}}
        $

        By $const$, \\
        $ C \vdash (t.\<const> c_2) :
            {\begin{stackTL}
                \epsilon;l_1;\phi_1 \\
                \rightarrow \ti{t}{a_3};l_1;\phi_1,\ti{t}{a_3},(= a_3\; \ti{t}{c_2}) \\
            \end{stackTL}} $

        $C \vdash (t.\<const> c_2) : ti_1^{*};l_1;\phi_1 \rightarrow ti_1^{*}\;\ti{t}{a_3};l_1;\phi_1,\ti{t}{a_3},(= a_3 \; \ti{t}{c_2})$ by $stack-poly$.

        By $\implies$, we have \\
        $\phi_1,\ti{t}{a_3},(\<eq> a_3\; \ti{t}{c_2}) \implies \phi_1, {\begin{stackTL}
            \ti{t}{a_1}, (\<eq> a_1\; \ti{t}{c_1}), \\
            \ti{t}{a_2}, (\<eq> a_2\; \ti{t}{c_2}), \\
            \ti{\<ithreetwo>}{a}, (= a\;\ti{\<ithreetwo>}{0}), \\
            \ti{t}{a_3},(if\; (= a\; \ti{\<ithreetwo>}{0})\; (= a_3\; a_2)\; (= a_3\; a_1))
        \end{stackTL}} \\$

        Therefore,
        $ C \vdash (t.\<const> c_2) :
        ti_1^{*};l_1;\phi_1
            \rightarrow ti_2^{*}\;\ti{t}{a_3};l_1;\phi_2$ by $sub-typing$

    %% Block
    \item Case: $C \vdash v^n \; \<block> tfi \; e^{*} \<end> : ti_1^{*};l_1;\phi_1 \rightarrow ti_2^{*};l_2;\phi_2$
    \\ $\land$ $v^n \; \<block> tfi \; e^{*} \<end> \hookrightarrow \<label>_m \{ \epsilon \} \; v^n \; e^{*} \<end>$

        Let $ti_3^n;l_3;\phi_3 \rightarrow ti_4^m;l_4;\phi_4=tfi$, $(t.\<const> c)^n=v^n$.

        $C \vdash \<block> tfi \; e^{*} \<end> : ti_1^{*}\; \ti{t}{a}^n;l_1;\phi_1,\ti{t}{a}^n,(\<eq> a \; \ti{t}{c})^n \rightarrow ti_2^{*};l_2;\phi_2$ by $inversion$ on $composition$ and $const$.

        Therefore, by $inversion$ on $block$, $l_1=l_3$ and $l_2=l_4$. We will use $l_1,l_2$ in place of $l_3,l_4$, respectively, for the remainder of the proof case.

        Further, $\ti{t}{a}^n=ti_3^n$, $ti_2^{*}=ti_1^{*}\; ti_4^m$, $\phi_1,\ti{t}{a}^n,(\<eq> a \; \ti{t}{c})^n \implies \phi_3$, and $\phi_4 \implies \phi_2$ by $inversion$ on $block$.

        $C,\text{label}(t_4^{m};l_2;\phi_4) \vdash (t.\<const> c)^n : \epsilon;l_1;\phi_1 \rightarrow \\ \ti{t}{a}^n;l_1;\phi_1,\ti{t}{a}^n,(\<eq> a \; \ti{t}{c})^n$ by $const$.

        $C,\text{label}(t_4^{m};l_2;g_2;\phi_4) \vdash (t.\<const> c)^n : \epsilon;l_1;\phi_1 \rightarrow \\ \ti{t}{a}^n;l_1;\phi_3$ by $sub-typing$.

        $C,\text{label}(t_4^{m};l_2;\phi_4) \vdash e^{*} : \ti{t}{a}^n;l_1;\phi_3 \rightarrow ti_4^m;l_2;\phi_4$ because it is a sub-derivation of $block$ which we have already assumed to hold.

        Then $C,\text{label}(t_4^{m};l_2;\phi_4) \vdash (t.\<const> c)^n\; e^{*} : \epsilon;l_1;\phi_1 \rightarrow \\ ti_4^m;l_2;\phi_4$ by $composition$.

        By $empty$ and $stack-poly$, $C \vdash \epsilon : ti_2^m;l_2;\phi_4 \rightarrow ti_2^m;l_2;\phi_4$.

        Therefore, $C \vdash \<label>_m \{ \epsilon \} \; v^n \; e^{*} \<end> : \epsilon;l_1;\phi_1 \rightarrow ti_2^m;l_2;\phi_4$ by $label$.

        $C \vdash \<label>_m \{ \epsilon \} \; v^n \; e^{*} \<end> : ti_1^{*};l_1;\phi_1 \rightarrow ti_1^{*}\; ti_4^m;l_2;\phi_2$ by $stack-poly$ and $sub-typing$.

    \item Case: $C \vdash v^n \; \<loop> tfi \; e^{*} \<end> : ti_1^{*};l_1;\phi_1 \rightarrow ti_2^{*};l_2;\phi_2$
    \\ $\land$ $v^n \; \<loop> tfi \; e^{*} \<end> \hookrightarrow \<label>_n \{ \<loop> tfi \; e^{*} \<end> \} \; v^n \; e^{*} \<end>$

        Let $ti_3^n;l_3;\phi_3 \rightarrow ti_4^m;l_4;\phi_4=tfi$, $(t.\<const> c)^n=v^n$.

        $C \vdash \<loop> tfi \; e^{*} \<end> : ti_1^{*}\; \ti{t}{a}^n;l_1;\phi_1,\ti{t}{a}^n,(\<eq> a \; \ti{t}{c})^n \rightarrow ti_2^{*};l_2;g_2;\phi_2$ by $inversion$ on $composition$ and $const$.

        Therefore, by $inversion$ on $loop$, $l_1=l_3$ and $l_2=l_4$. We will use $l_1,l_2$ in place of $l_3,l_4$, respectively, for the remainder of the proof case.

        Further, $\ti{t}{a}^n=ti_3^n$, $ti_2^{*}=ti_1^{*}\; ti_4^m$, $\phi_1,\ti{t}{a}^n,(\<eq> a \; \ti{t}{c})^n \implies \phi_3$, and $\phi_4 \implies \phi_2$ by $inversion$ on $loop$.

        $C,\text{label}(t_3^{n};l_1;\phi_3) \vdash (t.\<const> c)^n : \epsilon;l_1;\phi_1 \rightarrow \\ \ti{t}{a}^n;l_1;\phi_1,\ti{t}{a}^n,(\<eq> a \; \ti{t}{c})^n$ by $const$.

        $C,\text{label}(t_3^{n};l_1;\phi_3) \vdash (t.\<const> c)^n : \epsilon;l_1;\phi_1 \rightarrow \\ \ti{t}{a}^n;l_1;\phi_3$ by $sub-typing$.

        $C,\text{label}(t_3^{n};l_1;\phi_3) \vdash e^{*} : ti_1^n;l_1;\phi_3 \rightarrow ti_2^m;l_1;\phi_4$ because it is a sub-derivation of $loop$ which we have already assumed to hold.

        Then $C,\text{label}(t_3^{n};l_1;\phi_3) \vdash (t.\<const> c)^n\; e^{*} : \epsilon;l_1;\phi_1 \rightarrow \\ ti_4^m;l_2;\phi_4$ by $composition$.

        $C \vdash \<loop> tfi \; e^{*} \<end> : \ti{t}{a}^n;l_1;\phi_1,\ti{t}{a}^n,(\<eq> a \; \ti{t}{c})^n \rightarrow ti_4^{m};l_2;g_2;\phi_4$ by $loop$.

        Therefore, $C \vdash \<label>_m \{ \<loop> tfi \; e^{*} \<end> \} \; v^n \; e^{*} \<end> : \epsilon;l_1;\phi_1 \rightarrow ti_4^m;l_2;\phi_4$ by $label$.

        $C \vdash \<label>_m \{ \epsilon \} \; v^n \; e^{*} \<end> : ti_1^{*};l_1;\phi_1 \rightarrow ti_2^{*};l_2;\phi_2$ by $stack-poly$ and $sub-typing$.

    \item Case: $C \vdash (\<ithreetwo>.\<const> 0) \; \<if> tfi \; e_1^{*} \<else> e_2^{*} \<end> : ti_1^{*};l_1;\phi_1 \rightarrow ti_2^{*};l_2;\phi_2$
    \\ $\land$ $(\<ithreetwo>.\<const> 0) \; \<if> tfi \; e_1^{*} \<else> e_2^{*} \<end> \hookrightarrow \<block> tfi \; e_2^{*} \<end>$

        Let $tfi = ti_3^n \; \ti{<ithreetwo>}{a};l_3;\phi_3 \rightarrow ti_4^m;l_4;\phi_4$, \\ $tfi_1 = ti_3^n;l_3;\phi_3,\neg(= a\; \ti{\<ithreetwo>}{0}) \rightarrow ti_4^m;l_4;\phi_4$, \\
        and $tfi_2 = ti_3^n;l_3;\phi_3,(= a\; \ti{\<ithreetwo>}{0}) \rightarrow ti_4^m;l_4;\phi_4$.

        By $inversion$ on $composition$, $const$, and $if$, $ti_1^{*}=ti_0^{*}\; ti_3^{n}$ and $ti_2^{*}=ti_0^{*} \; ti_4^{m}$ for some $ti_0^{*}$, $l_1=l_3$, $l_2=l_4$, $\phi_1,\ti{\<ithreetwo>}{a},(\<eq> a\; 0) \implies \phi_3$, and $\phi_4 \implies \phi_2$.

        $C,\text{label}(ti_4^m;l_4;\phi_4) \vdash e_2^{*} : tfi_2$ because it is a sub-derivation of $if$ which we have assumed to hold.

        Then, $C \vdash \<block> tfi_2 \; e_2^{*} \<end>$ by $block$.

        Since $a$ is fresh after reduction, $\phi_1 \implies \phi_1,\ti{t}{a},(\<eqz> a)$ by $\implies$.

        Therefore, $C \vdash \<block> tfi_2\; e_2^{*} \<end> : \\ ti_0^{*}\; ti_3^n;l_1;\phi_1,\ti{t}{a},(\<eqz> a) \rightarrow s\; ti_0^{*}\;ti_4^m;l_2;\phi_2$ by $extension$ and $sub-typing$.

    \item Case: $C \vdash (\<ithreetwo>.\<const> k+1) \; \<if> tfi \; e_1^{*} \<else> e_2^{*} \<end> : ti_1^{*};l_1;\phi_1 \rightarrow ti_2^{*};l_2;\phi_2$
    \\ $\land$ $(\<ithreetwo>.\<const> k+1) \; \<if> tfi \; e_1^{*} \<else> e_2^{*} \<end> \hookrightarrow \<block> tfi \; e_1^{*} \<end>$

        Similar to above.

    \item Case: $C \vdash \<label>_n \{ e^{*} \} \; v^n \<end> : ti_1^{*};l_1;\phi_1 \rightarrow ti_2^{*};l_2;\phi_2$
    \\ $\land$ $\<label>_n \{ e^{*} \} \; v^n \<end> \hookrightarrow v^n$

        $C \vdash \<label>_n \{ e^{*} \} \; v^n \<end> : \epsilon;l_1;\phi_1 \rightarrow ti_4^{n};l_2;\phi_2$ by $inversion$ on $label$.

        By $inversion$, we know $ti_2^{*}=ti_1^{*}\;ti_4^{n}$.

        $C \vdash v^n : \epsilon;l_1;\phi_1 \rightarrow ti_4^{n};l_2;\phi_2$ because it is a premise of $label$ which we have assumed to hold.

        Therefore, $C \vdash v^n : ti_1^{*};l_1;\phi_1 \rightarrow ti_1^{*}\;ti_4^{n};l_1;\phi_2$ by $stack-poly$.

    \item Case: $C \vdash \<label>_n \{ e^{*} \} \; \<trap> \<end> : ti_1^{*};l_1;\phi_1 \rightarrow ti_2^{*};l_2;\phi_2$
    \\ $\land$ $\<label>_n \{ e^{*} \} \; \<trap> \<end> \hookrightarrow \<trap>$

        Trivially, $C\vdash \<trap> : ti_1^{*};l_1;\phi_1 \rightarrow ti_2^{*};l_2;\phi_2$ by $trap$.

    \item Case: $C \vdash \<label>_n \{ e^{*} \} \; L^j [v^n \; (\<br> j)] \<end> : ti_1^{*};l_1;\phi_1 \rightarrow ti_2^{*};l_2;\phi_2$
    \\ $\land$ $\<label>_n \{ e^{*} \} \; L^j [v^n \; (\<br> j)] \hookrightarrow v^n \; e^{*}$

        By $inversion$, $ti_2^{*}=ti_1^{*}\;ti_4^{*}$.

        Let $(t.\<const> c)^n = v^n$.

        $C,\text{label}(ti_1^n;l_3;\phi_5)^j \vdash v^n\; (\<br> j) : \epsilon;l_3;\phi_3 \rightarrow ti_\emptyset^{*};l_\emptyset;g_\emptyset;\phi_\emptyset$ for some $l_3$ and $\phi_3$, where $\phi_5=\phi_3,\ti{t}{a}^n,(= a\; \ti{t}{c})^n$, by $inversion$ on $label$ and $br$.

        $C,\text{label}(ti_1^n;l_3;\phi_5)^j \vdash (\<br> j) : ti_1^n;l_3;\phi_5 \rightarrow ti_\emptyset^{*};l_\emptyset;g_\emptyset;\phi_\emptyset$, by $inversion$ on $composition$ and $const$.

        Then, $C,\text{label}(ti_1^n;l_3;\phi_5)^j \vdash v^n : \epsilon;l_3;\phi_3 \rightarrow ti_1^n;l_3;\phi_5$ since it is a premise of $composition$ which we have assumed to hold.

        $C \vdash e^{*} : ti_1^n;l_3;\phi_5 \rightarrow ti_2^{*};l_2;\phi_4$ since it is a premise of $label$ which we have assumed to hold, and $\phi_4 \implies \phi_2$ by $inversion$.

        Then, $C \vdash v^n \; e^{*} : \epsilon;l_1;\phi_1 \rightarrow ti_2^{*};l_2;\phi_4$ by $nested-type-preserved$ and $composition$.

        Finally, $C \vdash v^n \; e^{*} : ti_1^{*};l_1;\phi_1 \rightarrow ti_1^{*}\;ti_4^{*};l_2;\phi_2$ by $stack-poly$ and $sub-typing$.

    \item Case: $C \vdash (\<ithreetwo>.\<const> 0)\;(\<brif> j) : ti_1^{*};l_1;\phi_1 \rightarrow ti_2^{*};l_2;\phi_2$
    \\ $\land$ $(\<ithreetwo>.\<const> 0)\;(\<brif> j) \hookrightarrow \epsilon$

        $ti_1^{*}=ti_2^{*}$, $l_1=l_2$, and $\phi_1,\ti{\<ithreetwo>}{a},(\<eq> a\; \ti{\<ithreetwo>}{0}),(\<eqz> a) \implies \phi_2$ by $inversion$ on $composition$, $const$, and $br \_ if$.

        $C \vdash \epsilon : \epsilon;l_1;\phi_1 \rightarrow \epsilon;l_1;\phi_1$ by $empty$.

        $C \vdash \epsilon : ti_1^{*};l_1;\phi_1 \rightarrow ti_1^{*};l_1;\phi_1$ by $stack-poly$.

        $\phi_1 \implies \phi_1,\ti{\<ithreetwo>}{a},(\<eq> a\; \ti{\<ithreetwo>}{0}),(\<eqz> a)$ because $a$ is fresh after reduction, and therefore $\phi_1 \implies \phi_2$.

        Then, $C \vdash \epsilon : ti_1^{*};l_1;\phi_1 \rightarrow ti_1^{*};l_1;\phi_2$ by $sub-typing$.

    \item Case: $C \vdash (\<ithreetwo>.\<const> k+1)\;(\<brif> j) : ti_1^{*};l_1;\phi_1 \rightarrow ti_2^{*};l_2;\phi_2$
    \\ $\land$ $(\<ithreetwo>.\<const> k+1)\;(\<brif> j) \hookrightarrow \<br> j$

        $C_label(j)=ti_1^{*};l_1;\phi_1,\ti{t}{a},\neg(\<eqz> a)$ because it is a side condition of $br\_if$ which we have assumed to hold.

        $C \vdash \<br> j : ti_1^{*};l_1;\phi_1,\ti{t}{a},\neg(\<eqz> a) \rightarrow ti_2^{*};l_2;\phi_2$ by $br$.

        Because $a$ is fresh after reduction, $\phi_1 \implies \phi_1,\ti{\<ithreetwo>}{a},\neg(\<eqz> a)$.

        Therefore, $C \vdash \<br> j : ti_1^{*};l_1;\phi_1 \rightarrow ti_2^{*};l_2;\phi_2$ by $sub-typing$.

    \item Case: $C \vdash (\<ithreetwo>.\<const> k)\;(\<brtable> j_1^k\; j\; j_2^{*}) : ti_1^{*};l_1;\phi_1 \rightarrow ti_2^{*};l_2;\phi_2$
    \\ $\land$ $(\<ithreetwo>.\<const> k)\;(\<brtable> j_1^k\; j\; j_2^{*}) \hookrightarrow \<br> j$

        By $inversion$, we know that $C_\text{label}(j) = ti^{*};l_1;\phi_3$, $ti_1^{*} = ti_0^{*} \; ti^{*}$ for some $ti_0^{*}$, and $phi_1 \implies \phi_3$.

        $C \vdash \<br> j : ti_1^{*};l_1;\phi_3 \rightarrow ti_2^{*};l_2;\phi_2$ by $br$.

        $C \vdash \<br> j : ti_1^{*};l_1;\phi_1 \rightarrow ti_2^{*};l_2;\phi_2$ by $sub-typing$.

    \item Case: $C \vdash (\<ithreetwo>.\<const> k+n)\;(\<brtable> j_1^k\; j) : ti_1^{*};l_1;\phi_1 \rightarrow ti_2^{*};l_2;\phi_2$
    \\ $\land$ $(\<ithreetwo>.\<const> k+n)\;(\<brtable> j_1^k\; j) \hookrightarrow \<br> j$

        By $inversion$, we know that $C_\text{label}(j) = ti^{*};l_1;\phi_3$, $ti_1^{*} = ti_0^{*} \; ti^{*}$ for some $ti_0^{*}$, and $phi_1 \implies \phi_3$.

        $C \vdash \<br> j : ti_1^{*};l_1;\phi_3 \rightarrow ti_2^{*};l_2;\phi_2$ by $br$.

        $C \vdash \<br> j : ti_1^{*};l_1;\phi_1 \rightarrow ti_2^{*};l_2;\phi_2$ by $sub-typing$.

    \item Case: $S;C \vdash \<call> j : ti_1^{*};l_1;\phi_1 \rightarrow ti_2^{*};l_2;\phi_2$
    \\ $\land$ $s;\<call> j \hookrightarrow_i \<call> s_\text{func}(i,j)$

        By $inversion$, we know that $l_2 = l_1$, $ti_1^{*} = ti^{*} \; ti_3^{*}$, $ti_1^{*} = ti^{*} \; ti_4^{*}$, $\phi_1 \implies \phi_3$, and $\phi_3,\phi_4 \implies \phi_2$, where $ti_3^{*};l_3;\phi_3 \rightarrow ti_4^{*};l_4;\phi_4 = C_\text{func}(j)$.

        We know $S \vdash s_\text{inst}(i) : C$ since it is a premise of $\vdash s : S$ which we have assumed to hold.

        Then we know $S \vdash s_\text{func}(i,j) : ti_3^{*};l_3;\phi_3 \rightarrow ti_4^{*};l_4;\phi_4$ because it is a premise of $S \vdash s_\text{inst}(i) : C$.

        Therefore, $S;C\vdash \<call> s_\text{func}(i,j) : ti_3^{*};l_3;\phi_3 \rightarrow ti_4^{*};l_4;\phi_4$ by $call-cl$.

        $S;C\vdash \<call> s_\text{func}(i,j) : ti_1^{*};l_1;\phi_1 \rightarrow ti_2^{*};l_2;\phi_2$ by $stack-poly$ and $sub-typing$.

    \item Case: $S;C \vdash_i (\<ithreetwo>.\<const> j)\; \<callindirect> tfi : ti_1^{*};l_1;\phi_1 \rightarrow ti_2^{*};l_2;\phi_2$
    \\ $\land$ $s;(\<ithreetwo>.\<const> j)\; \<callindirect> tfi \hookrightarrow_i \<call> s_\text{tab}(i,j)$ where $s_\text{tab}(i,j)_\text{code}=(\<func> tfi_0\; \<local>\; t^{*}\; e^{*})$ and $tfi_0 <: tfi$

        Let $ti_3^{*};l_3;\phi_3 \rightarrow ti_4^{*};l_4;\phi_4 = tfi$

        By $inversion$ on $composition$, $const$, and $call-indirect$, we know that $ti_1^{*}=ti_0^{*}\; ti_3^{*}$ and $ti_2^{*}=ti_0^{*}\; ti_4^{*}$ for some $ti_0^{*}$, $l_1=l_3$, $l_2=l_4$, $\phi_1 \implies \phi_3$, and $\phi_4 \implies \phi_2$.

        $S \vdash s_\text{tab}(i,j) : tfi_0$ since it is a premise of $\vdash s : S$ which we have assumed to hold.

        Then, $S;C \vdash_i \<call> s_\text{tab}(i,j) : tfi_0$ by $call-cl$.

        $S;C \vdash_i \<call> s_\text{tab}(i,j) : tfi$ by $sub-typing$.

        Therefore, $S;C \vdash_i \<call> s_\text{tab}(i,j) : ti_0^{*}\;ti_1^{*};l_1;\phi_1 \rightarrow ti_0^{*}\;ti_1^{*};l_2;\phi_2$ by $stack-poly$.

    \item Case: $S;C \vdash_i (\<ithreetwo>.\<const> j)\; \<callindirect> tfi : ti_1^{*};l_1;\phi_1 \rightarrow ti_2^{*};l_2;\phi_2$
    \\ $\land$ $s;(\<ithreetwo>.\<const> j)\; \<callindirect> tfi \hookrightarrow_i \<trap>$.

        Trivially, $S;C \vdash_i \<trap> : ti_1^{*};l_1;\phi_1 \rightarrow ti_2^{*};l_2;\phi_2$ by $trap$.

    \item Case: $S;C \vdash_i v^n\; \<call> cl : ti_1^{*};l_1;\phi_1 \rightarrow ti_2^{*};l_2;\phi_2$
    \\ $\land$ $s;v^n\; \<call> cl \hookrightarrow_i \<local>_m\{j;v^n \; (t.\<const> 0)^k\} \; \<block> tfi_1\; e^{*} \<end> \<end>$
    \\ where $cl_\text{code} = \<func> tfi_2\; \<local>\; t^k \; e^{*}$ and $cl_\text{inst} = j$

        \todo{This needs an overhaul}

        Let $tfi_0 = ti_1^{*};l_1;\phi_1 \rightarrow ti_2^{*};l_2;\phi_2$, $tfi_1 = \epsilon;l_3;\phi_3 \rightarrow ti_4^{m};l_4;\phi_4$, and $tfi_2 = ti_3^{n};\ti{t_2}{a_2}^{n}\; \ti{t}{a}^k;\phi_3,\ti{t}{a}^k,(= a \;\ti{t}{0})^k \rightarrow ti_4^{m};l_4;\phi_4$ by $inversion$.

        $S;C \vdash (t_2 \<const> c)^n : ti_1^{*};l_1;\phi_1 \rightarrow ti_1^{*}\;ti_5^n;l_1;\phi_1,\ti{t_2}{a_2},(= a_2 \; \ti{t_2}{c})$, where $v^n=(t_2 \<const> c)^n$,
        and $S;C\vdash \<call> cl : ti_1^{*}\;ti_5^n;l_1;\phi_1,\ti{t_2}{a_2}^{n},(= a_2 \; \ti{t_2}{c})^{n} \rightarrow ti^{*}\;ti_2^m;l_2;\phi_2$ because they are premises of $composition$ which we have assumed to hold.

        By inversion, $l_2=l_1$, $ti_2^{*}=ti_1^{*}\;ti_4^m$, $\phi_1,\ti{t_2}{a_2},(\<eq> a_2 \; \ti{t_2}{c}) \implies \phi_3$, $\phi_4 \implies \phi_2$, and $S\vdash cl : tfi_1$.

        Therefore, $C \vdash \<func> tfi_1\; \<local>\; t^k \; e^{*} : tfi_1$ because it is a premise of $S \vdash cl : tfi_1$.

        $S;C,\text{local } t_2^n\; t^k,\text{label }(ti_4^{m};l_4;g_4;\phi_4),\text{return }(ti_4^{m};l_4;g_4;\phi_4) \vdash e^{*}: tfi_2$ because it is a premise of the above derivation.

        $S;C,\text{local } t_2^n\; t^k,\text{return }(ti_4^{m};l_4;g_4;\phi_4) \vdash \<block> tfi_2\; e^{*} \<end> : tfi_2$ by $block$.

        There are now two cases, based on whether or not the called closure was in the current module being called:

        \begin{itemize}
            \item Case: $i=j$
                By $inversion$, $g_1=g_3$ and $g_2=g_4$, so we will use $g_1$ instead of $g_3$ and $g_2$ instead of $g_4$.

                $C \vdash_j v^n \; (t \<const> 0)^k : \epsilon;l_3;g_1;\phi_1 \rightarrow \ti{t_2}{a_2}^n\;\ti{t}{a}^k ;l_3;g_1;\phi_1,\ti{t_2}{a_2},(\<eq> a_2 \; \ti{t_2}{c})^n,\ti{t}{a},(\<eq> a \; \ti{t}{0})^k$ by $const$.

                $\phi_1,\ti{t_2}{a_2},(\<eq> a_2 \; \ti{t_2}{c})^n \implies \phi_3$, and therefore $\phi_1,\ti{t_2}{a_2},(\<eq> a_2 \; \ti{t_2}{c})^n,\ti{t}{a},(\<eq> a \; \ti{t}{0})^k \implies \phi_3,\ti{t}{a}^k,(\<eq> a \;\ti{t}{0})^k$.

                $S;C,\text{local } t_2^n\; t^k,\text{return }(ti_4^{m};l_4;g_2;\phi_4) \vdash \<block> tfi_2\; e^{*} \<end> :  ti_3^{n};\ti{t_2}{a_2}^{n}\; \ti{t}{a}^k;g_1;\phi_1,\ti{t_2}{a_2},(\<eq> a_2 \; \ti{t_2}{c})^n,\ti{t}{a},(\<eq> a \; \ti{t}{0})^k \rightarrow ti_4^{m};l_4;g_2;\phi_4$ by $sub-typing$.

                $S;(ti_4^{m};l_4;g_2;\phi_4) \vdash_j v^n \; (t \<const> 0)^k;\<block> tfi_2\; e^{*} \<end> : \epsilon;l_3;g_1;\phi_1 \rightarrow ti_4^{m};l_4;g_2;\phi_4$ by $with-return$.

                $S;C \vdash_i \<local>_m\{j;v^n \; (t.\<const> 0)^k\} \; \<block> tfi_2\; e^{*} \<end> \<end> : \epsilon;l_1;g_1;\phi_1 \rightarrow \epsilon\;ti_4^m;l_1;g_2;\phi_4$ by $local-same-inst$.

                $S;C \vdash_i \<local>_m\{j;v^n \; (t.\<const> 0)^k\} \; \<block> tfi_2\; e^{*} \<end> \<end> : tfi_0$ by $stack-poly$ and $sub-typing$.

            \item Case: $i \neq j$
                By $inversion$, $g_2=\ti{t_g}{a_3}^{*}$ where $C_\text{global}=(mut?\; t_g)^{*}$ and $a_3^{*}$ are fresh.

                $C \vdash_j v^n \; (t \<const> 0)^k : \epsilon;l_3;g_3;\phi_1 \rightarrow \ti{t_2}{a_2}^n\;\ti{t}{a}^k ;l_3;g_3;\phi_1,\ti{t_2}{a_2},(\<eq> a_2 \; \ti{t_2}{c})^n,\ti{t}{a},(\<eq> a \; \ti{t}{0})^k$ by $const$

                $\phi_1,\ti{t_2}{a_2},(\<eq> a_2 \; \ti{t_2}{c})^n \implies \phi_3$, and therefore $\phi_1,\ti{t_2}{a_2},(\<eq> a_2 \; \ti{t_2}{c})^n,\ti{t}{a},(\<eq> a \; \ti{t}{0})^k \implies \phi_3,\ti{t}{a}^k,(\<eq> a \;\ti{t}{0})^k$.

                $S;C,\text{local } t_2^n\; t^k,\text{return }(ti_4^{m};l_4;g_4;\phi_4) \vdash \<block> tfi_2\; e^{*} \<end> :  ti_3^{n};\ti{t_2}{a_2}^{n}\; \ti{t}{a}^k;g_3;\phi_1,\ti{t_2}{a_2},(\<eq> a_2 \; \ti{t_2}{c})^n,\ti{t}{a},(\<eq> a \; \ti{t}{0})^k \rightarrow ti_4^{m};l_4;g_4;\phi_4$ by $sub-typing$.

                $S;(ti_4^{m};l_4;g_4;\phi_4) \vdash_j v^n \; (t \<const> 0)^k;\<block> tfi_2\; e^{*} \<end> : \epsilon;l_3;g_3;\phi_1 \rightarrow ti_4^{m};l_4;g_4;\phi_4$ by $with-return$.

                $S;C \vdash_i \<local>_m\{j;v^n \; (t.\<const> 0)^k\} \; \<block> tfi_2\; e^{*} \<end> \<end> : \epsilon;l_1;g_1;\phi_1 \rightarrow \epsilon\;ti_4^m;l_1;\ti{t_g}{a_3}^{*};\phi_4$ by $local-diff-inst$.

                $S;C \vdash_i \<local>_m\{j;v^n \; (t.\<const> 0)^k\} \; \<block> tfi_2\; e^{*} \<end> \<end> : tfi_0$ by $stack-poly$ and $sub-typing$.
        \end{itemize}

    \item Case: $S;C \vdash \<local>_n \{ i;v_l^{*} \} \; v^n \<end> : ti_1^{*};l_1;\phi_1 \rightarrow ti_2^{*};l_2;\phi_2$
    \\ $\land$ $\<local>_n \{ i;v^{*}_l \} \; v^n \<end> \hookrightarrow_j \; v^n$

        By $inversion$ on $local$, $ti_2^{*} = ti_1^{*} \; ti^n$, $l_1 = l_2$,
        $S;(ti^n;l_3;\phi_3) \vdash_i v_l^{*};v^n : ti^n;l_3;\phi_3$,
        and $\satisfies{\phi_1,\phi_3}{ti_2^{*}\;l_2}{\phi_2}$.

        By $inversion$ on $admin-code$, $(\vdash v_l : ti_l;\phi_l)^{*}$,
        and $S;C_l \vdash v^n : \epsilon:ti_l^{*};\phi_l^{*} \rightarrow ti^n;l_3;\phi_3$.

        By $inversion$ on $admin-const$, $\phi_l^{*} = \circ,\ti{t}{a}^{*},(= a \ti{t}{c})^{*}$.

        By $inversion$ on $const$, $\satisfies{\phi_l^{*},\phi_v}{ti^n\;l_3}{\phi_3}$.

        Since $l_2 \not\in \phi_3$; $\satisfies{\phi_l^{*},\phi_v}{ti^n\;l_2}{\phi_3}$.

        Since $a^{*} \not\in \phi_v,ti^n,l_2$; $\satisfies{\phi_v}{ti^n\;l_2}{\phi_l^{*},\phi_v}$.

        $S;C \vdash v^n : \epsilon;l_1;\phi_1 \rightarrow ti^n;l_2;\phi_1,\phi_v$ by $const$.

        $S;C \vdash v^n : \epsilon;l_1;\phi_1 \rightarrow ti^n;l_2;\phi_1,\phi_3$ by $subtyping$.

        $S;C \vdash v^n : \epsilon;l_1;\phi_1 \rightarrow ti^n;l_2;\phi_2$ by $subtyping$.

        Therefore $S;C \vdash v^n : ti_1^{*};l_1;\phi_1 \rightarrow ti_2^{*};l_2;\phi_2$ by $stack-poly$.

    \item Case: $S;C \vdash \<local>_n \{ i;v_l^{*} \} \; \<trap> \<end> : ti_1^{*};l_1;\phi_1 \rightarrow ti_2^{*};l_2;\phi_2$
    \\ $\land$ $\<local>_n \{ i;v_l^{*} \} \; \<trap> \<end> \hookrightarrow \; \<trap>$

        Trivially, $S;C \vdash \<trap> : ti_1^{*};l_1;\phi_1 \rightarrow ti_2^{*};l_2;\phi_2$ by $trap$.

    \item Case: $S;C \vdash \<local>_n \{ i;v_l^{*} \} \; L^k[v^n \; \<return>] \<end> : ti_1^{*};l_1;\phi_1 \rightarrow ti_2^{*};l_2;\phi_2$
    \\ $\land$ $\<local>_n \{ i;v_l^{*} \} \; L^k[v^n \; \<return>] \<end> \hookrightarrow_j \; v^n$

        By $inversion$, $ti_2^{*} = ti_1^{*} \; ti^n$, $\phi_1,\phi_3 \implies \phi_2$, and $l_1 = l_2$.

        $S;(ti^n;l_3;\phi_3) \vdash_i L^k[v^n \; \<return>] : ti^n;l_3;\phi_3$ because it is a premise of $local$ which we have assumed to hold.

        $(\vdash v_l : ti_l;\phi_l)^{*}$, and\\
        $S;C_l \vdash L^k[v^n \; \<return>] : \epsilon;ti_l^{*};\phi_l^{*} \rightarrow ti^n;l_3;\phi_3$, where\\
        $C_l = S_{\text{inst}}(i),\text{local} \; t^{*}, \text{return} \; (ti^n;l_3;\phi_3)$ because they are premises of $admin-code$ which we have assumed to hold.

        $S;C_l \vdash \<return> : ti_3^{*} \; ti^n;l_3;\phi_3 \rightarrow ti_4^{*};l_4;\phi_4$ by $return$.

        $S;C_l \vdash v^n : \epsilon;l_3;\phi_5 \rightarrow ti^n;l_3;\phi_6$, where $\phi_6 \implies \phi_3$ by $inversion$ on $const$, $composition$, and $subtyping$.

        $S;C_l \vdash v^n : \epsilon;l_3;\phi_l^{*} \rightarrow ti^n;l_3;\phi_6$ by (Nested-Type-Preserved).

        $\phi_l^{*} = \circ,\ti{t}{a}^{*},(\<eq> a \ti{t}{c})^{*}$ by $admin-const$.

        By $inversion$ on $const$, $\phi_6 = \phi_l^{*},\phi_v$.

        \thought{Don't have to account for subtyping here because $\phi_6$ is an existential claim.}

        Because $a^{*}$ are fresh, $\phi_v \implies \phi_l^{*},\phi_v$.

        \thought{Uses the same idea from the values case.}

        $S;C \vdash v^n : \epsilon;l_1;\phi_1 \rightarrow ti^n;l_2;\phi_1,\phi_v$ by $const$.

        $S;C \vdash v^n : \epsilon;l_1;\phi_1 \rightarrow ti^n;l_2;\phi_1,\phi_3$ by $subtyping$.

        $S;C \vdash v^n : \epsilon;l_1;\phi_1 \rightarrow ti^n;l_2;\phi_2$ by $subtyping$.

        Therefore $S;C \vdash v^n : ti_1^{*};l_1;\phi_1 \rightarrow ti_2^{*};l_2;\phi_2$ by $stack-poly$.

    \item Case: $C \vdash \<getlocal> j : ti_1^{*};l_1;\phi_1 \rightarrow ti_2^{*};l_2;\phi_2$
    \\ $\land$ $v_1^j \; v \; v_2^k;\<getlocal> j \hookrightarrow v$

        By $inversion$ on $get-local$, $ti_2^{*} = \ti{t}{a_1} \; ti_1^{*}$, $l_2 = l_1$, and $\phi_1,\ti{t}{a_1},(= a_1 \; a) \implies \phi_2$, where $\ti{t}{a} = l_1(j)$.

        By $inversion$ on $admin-code$ and $admin-const$, $\phi_1 = \circ,\phi_{v_1}^j, \ti{t}{a}, (= a \; \ti{t}{c}), \phi_{v_2}^k$, where $v = (t.\<const> c)$.

        $C \vdash v : \epsilon;l_1;\phi_1 \rightarrow \ti{t}{a_1};l_2;\phi_1,\ti{t}{a_1},(= a_1 \; \ti{t}{c})$ by $const$.

        $\phi_1,\ti{t}{a_1},(= a_1 \; \ti{t}{c}) \iff \phi_1,\ti{t}{a_1},(= a_1 \; a)$ because $\phi_1$ contains the constraint $(= a \; \ti{t}{c})$.

        $C \vdash v : \epsilon;l_1;\phi_1 \rightarrow \ti{t}{a_1};l_2;\phi_2$ by $subtyping$.

        Therefore, $C \vdash v : ti_1^{*};l_1;\phi_1 \rightarrow ti_2^{*};l_2;\phi_2$ by $stack-poly$.

    \item Case: $S;\epsilon \vdash_i v_1^j \; v \; v_2^k; v' \; (\<setlocal> j) : ti^{*};l;\phi$
    \\ $\land$ $v_1^j \; v \; v_2^k;v' \; \<setlocal> j \hookrightarrow v_1^j \; v' \; v_2^k;\epsilon$

        By $inversion$ on $admin-code$, $\vdash v : \ti{t}{a};\phi_v$,\\
        $S;C \vdash v' \; (\<setlocal> j) : \epsilon:l_1;\phi_1^j,\phi_v,\phi_2^k \rightarrow ti^{*};l;\phi$,\\
        $l_1(j) = \ti{t}{a}$, and $C_\text{local}(j) = t$.

        By $inversion$ on $admin-const$, $t.\<const> c = v$, and\\
        $\phi_v = \circ,\ti{t}{a},(= a \; \ti{t}{c})$.

        By $inversion$ on $composition$,
        $C \vdash v' : \epsilon;l_1;\phi_1^j,\phi_v,\phi_2^k \rightarrow ti_3^{*};l_3;\phi_3$,
        $C \vdash \<setlocal> j : ti_3^{*};l_3;\phi_3 \rightarrow ti^{*};l;\phi_4$,
        and $\satisfies{\phi_4}{ti^{*}\;l}{\phi}$.

        Recalling that $t = C_\text{local}(j)$;
        by $inversion$ on $set-local$,
        $ti_3^{*} = ti^{*} \; \ti{t}{a'}$,
        $l = l_3 \text{ except } l(j) = \ti{t}{a_l}$,
        and $\satisfies{\phi_3,\ti{t}{a_l},(= a_l\;a')}{ti^{*}\;l}{\phi_4}$.

        By $inversion$ on $const$,
        $t.\<const> c' = v'$, $ti^{*} = \epsilon$, $l_1 = l_3$, and\\
        $\satisfies{\phi_1^j,\phi_v,\phi_2^k,\ti{t}{a'},(= a' \; \ti{t}{c'})}{\ti{t}{a'}\;l_1}{\phi_3}$.

        $C \vdash \epsilon : \epsilon;l;\phi \rightarrow \epsilon;l;\phi$ by $empty$.

        $C \vdash \epsilon : \epsilon;l;\phi_3,\ti{t}{a_l},(= a_l\;a') \rightarrow \epsilon;l;\phi$ by $subtyping$.

        Since $a_l \not\in \phi_3$; $\satisfies{\phi_1^j,\phi_v,\phi_2^k,\ti{t}{a'},(= a' \; \ti{t}{c'})}{l}{\phi_3}$.

        Since $a \not\in \phi_1^j,\phi_2^k,\ti{t}{a'},(= a' \; \ti{t}{c'}),l$;\\
        $\satisfies{\phi_1^j,\phi_2^k,\ti{t}{a'},(= a' \; \ti{t}{c'})}{l}{\phi_1^j,\phi_v,\phi_2^k,\ti{t}{a'},(= a' \; \ti{t}{c'})}$.

        $C \vdash \epsilon : \epsilon;l;\phi_1^j,\phi_2^k,\ti{t}{a'},(= a' \; \ti{t}{c'}),\ti{t}{a_l},(= a_l\;a') \rightarrow \epsilon;l;\phi$ by $subtyping$.

        Trivially, $\ti{t}{a'},(= a' \; \ti{t}{c'}),\ti{t}{a_l},(= a_l\;a') \iff \ti{t}{a'},(= a' \; \ti{t}{c'}),\ti{t}{a_l},(= a_l \; \ti{t}{c'})$.

        Since $a' \not\in \phi_1^j,\phi_2^k,\ti{t}{a_l},(= a_l \; \ti{t}{c'}),l$;\\
        $\satisfies{\phi_1^j,\phi_2^k,\ti{t}{a_l},(= a_l \; \ti{t}{c'})}{l}{\phi_1^j,\phi_2^k,\ti{t}{a'},(= a' \; \ti{t}{c'}),\ti{t}{a_l},(= a_l\;a')}$.

        $C \vdash \epsilon : \epsilon;l;\phi_1^j,\ti{t}{a_l},(= a_l \; \ti{t}{c'}),\phi_2^k \rightarrow \epsilon;l;\phi$ by $subtyping$.

        $\vdash v' : \ti{t}{a_l};\circ,\ti{t}{a_l},(= a_l \; \ti{t}{c'})$ by $admin-const$.

        Therefore, $S;\epsilon \vdash_i v_1^j\;v'\;v_2^k;\epsilon : ti^n;l;\phi$.

        \thought{I set out to write a very verbose proof case, but I didn't expect it to be this verbose.}

    \item Case: $\vdash s;v^{*};\<getglobal> j : ti^{*};l;\phi$
    \\ $\land$ $s;\<getglobal> j \hookrightarrow_i s_\text{glob}(i,j)$

        By $inversion$ on $admin-program$, $\vdash s : S$, and $S;\epsilon \vdash_i v^{*};\<getglobal> j : ti^{*};l;\phi$.

        By $inversion$ on $admin-code$, $S;C \vdash \<getglobal> j : \epsilon;l_1;\phi_v^{*} \rightarrow ti^{*};l;\phi$.

        By $inversion$ on $get-global$, $ti^{*} = \ti{t}{a}$, $l = l_1$, $C_\text{global}(j) = \text{mut}^{?} t$,
        and $\satisfies{\phi_v^{*},\ti{t}{a}}{\ti{t}{a}\;l}{\phi}$.

        By $inversion$ on $store$, $S \vdash s_\text{inst}(i) : C$.

        Recalling that $C_\text{global}(j) = \text{mut}^{?} t$;
        by $inversion$ on $instance$, $\vdash s_\text{glob}(i,j) : \ti{t}{a_1};\phi_1$.

        $S;C \vdash t.\<const> c : \epsilon;l;\phi_v^{*} \rightarrow \ti{t}{a};l;\phi_v^{*},\ti{t}{a},(= a \; \ti{t}{c})$ by $const$,
        where $t.\<const> c = s_\text{glob}(i,j)$.

        $\satisfies{\ti{t}{a},(= a \; \ti{t}{c})}{\ti{t}{a}\;l}{\ti{t}{a}}$ trivially.

        $S;C \vdash s_\text{glob}(i,j) : \epsilon;l;\phi_v^{*} \rightarrow \ti{t}{a};l;\phi$ by $subtyping$.

        $S;\epsilon \vdash_i v^{*};s_\text{glob}(i,j) : \ti{t}{a};l;\phi$ by $admin-code$, using the premises from $inversion$.

        Therefore, $\vdash_i s;v^{*};s_\text{glob}(i,j) : \ti{t}{a};l;\phi$ by $admin-program$.

    \item Case: $C \vdash v \; (\<setglobal> j) : ti_1^{*};l_1;\phi_1 \rightarrow ti_2^{*};l_2;\phi_2$
    $\land$ $\vdash s : S$
    \\ $\land$ $s;v \; (\<setglobal> j) \hookrightarrow_i s';\epsilon$, where $s' = s$ with $\text{glob}(i,j) = v$

        By $inversion$, $ti_2^{*} = ti_1^{*}$, $l_2 = l_1$, and $\phi_1,\ti{t}{a},(= a \; \ti{t}{c}) \implies \phi_2$, where $(t.\<const> c) = v$.

        Because $a$ is fresh in $ti_2^{*}$, $l_2$, and $\phi_1$, $\phi_1 \implies \phi_1,\ti{t}{a},(= a \; \ti{t}{c})$.

        $C \vdash \epsilon : \epsilon;l_1;\phi_1 \rightarrow \epsilon;l_2;\phi_1$ by $empty$.

        $C \vdash \epsilon : \epsilon;l_1;\phi_1 \rightarrow \epsilon;l_2;\phi_2$ by $subtyping$.

        Therefore, $C \vdash \epsilon : ti_1^{*};l_1;\phi_1 \rightarrow ti_2^{*};l_2;\phi_2$ by $stack-poly$.

        $C_\text{global}(j) = \text{mut} \; t$ because it is a premise of $set-global$ which we have assumed to hold.

        $\vdash v_0 : \ti{t}{a_0};\phi_0$, where $v_0 = s_\text{glob}(i,j)$ because it is a premise of $S \vdash s_\text{inst}(i) : C$, which is a sub-derivation of $\vdash s : S$.

        $\vdash v : \ti{t}{a_1};\phi_v$ by $admin-const$.

        $S \vdash s'_\text{inst}(i) : C$ by $instance$, note that the other premises are the same as the premises from $S \vdash s_\text{inst}(i) : C$.

        \thought{Maybe I should explicitly invoke the other premises?}

        Therefore, $\vdash s' : S$ by $store$, note that the other premises are the same as the premises from $\vdash s : S$.

        \thought{Ditto here.}

\end{itemize}
\end{proof}

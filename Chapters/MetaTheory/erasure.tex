\subsection{Erasing \name to \wasm}
\label{subsec:erasure}
We provide an erasure function for \name that transforms \name programs into \wasm programs by discarding the extra information from the \name type system and replacing \prechk-tagged instructions with their non-tagged counterparts.
Erasure is useful in the type safety proof because it lets us reuse much of the proof of progress from \wasm (see \autoref{subsec:progress}).
Therefore, we define erasure not just for the surface syntax, like we did for embedding, but also for typing constructs (such as the module type context), administrative instructions, and runtime data structures (such as the store).
We show that erasing a well-typed \name program produces a well-typed \wasm program.

As with the presentation of the embedding, we typeset \name instructions in a \tbsf{blue sans serif font} and \wasm instruction in a \trbf{bold red font}.

\paragraph{Erasing Surface Syntax}
As with embedding, we start by defining erasure with the pinnacle syntactic object: the module.
Erasing a module erases all of the functions $f^{*}$ and the table $tab^{?}$.
The globals $glob^{*}$ and optional memory $mem^{?}$ both have the same syntax in \name as in \wasm.

\begin{definition}{\fbox{$\erase[module]{module}=\mathredbold{module}$}}

    \begin{mathpar}
        \begin{array}{rcl}
            \erase[module]{\<module> f^{*}\; glob^{*}\; tab^{?}\; mem^{?}}
            &=& \<wmodule>
            \begin{stackTL}
                \erase[f]{f}^{*}
                \\ glob^{*}
                \\ \erase[tab]{tab}^{?}
                \\ mem^{?}
            \end{stackTL} \\
        \end{array}
    \end{mathpar}
\end{definition}

We show that erasing a well-typed \name module yields a well-typed \wasm module.

\begin{theorem}{Sound Module Erasure}

    If $\vdash \<module> f^{*}\; glob^{*}\; (n,tfi^{n})^{?}\; mem^{?}$,
    \\ then $\vdash \<wmodule> \erase[f]{f}^{*} glob^{*} n^{?} mem^{?}$
\end{theorem}
\begin{proof}

    Note that the globals $glob^{*}$ and memory $mem^{?}$ are not affected by erasure, and have the same module typing rules in \wasm as in \name.
    Thus, we only need to reason about the functions $f^{*}$ and table $tab^{?}$.

    Then, by \reflemma{Sound Function Typing Erasure} and \reflemma{Sound Table Erasure}, we have that $\vdash \<wmodule> \erase[f]{f}^{*} glob^{*} n^{?} mem^{?}$.
\end{proof}

Erasing a table definition $\<table> n\;i^{n}$ does nothing, since a table definition has the same syntax in \name and in \wasm.
However, erasing an imported table declaration $\<table> (n,tfi^{n})\; im$ must get rid of the indexed function types $tfi^{n}$.
We do not use or care about the exports, since they are unchanged and only used for linking, so we omit them.

\begin{definition}{\fbox{$\erase[tab]{tab}=\mathredbold{tab}$}}

    \begin{mathpar}
        \begin{array}{rcl}
            \<table> n\;i^{n}
            &=& \<wtable> n\;i^{n} \\
            \<table> (n,tfi^{n})\; im
            &=& \<wtable> n\; im \\
        \end{array}
    \end{mathpar}
\end{definition}

We show that erasure on well typed \name tables $tab$ is sound with respect to \wasm's type system.
This proof relies on the definition of $\erase[C]{C}$: \autoref{def:erase-C}.

\begin{lemma}{\deflemma{Sound Table Erasure}}

    If $C \vdash tab$,
    \\ then $\erase[C]{C} \vdash \erase[tab]{tab}$
\end{lemma}
\begin{proof}

    By case analysis on $\erase[tab]{tab}$.

    \begin{itemize}
        \item Case: $tab=\<table> n\;i^{n}$

            We know $(C_\text{func}(i)=tfi)^{n}$ because it is a premise of \refrule{Table}.

            Then, $(\erase[C]{C}_\text{func}(i)=tfi)^{n}$ by definition of $\erase[C]{C}$.

            Therefore, $\erase[C]{C} \vdash \<wtable> n\;i^{n}$ because \wasm accepts any module type context in that rule.

        \item Case: $tab=\<table> (n,tfi^{n})\; im$

            Trivially $\erase[C]{C} \vdash \<wtable> n\;im$ because \wasm accepts any module type context and imported table.
    \end{itemize}
\end{proof}

To erase a function definition $f$, we erase both the type declaration $ti_1^{*};l_1;\phi_1 \rightarrow ti_2^{*};l_2;\phi_2$ and the body $e^{*}$.
We can also erase an imported function by erasing the declared type $tfi$.

\begin{definition}{$\erase[f]{f} = \mathredbold{f}$}
    \begin{mathpar}
        \begin{array}{rcl}
            erase_f(\<func>
            {\begin{stackTL}
                ti_1^{*};l_1;\phi_1 \rightarrow ti_2^{*};l_2;\phi_2
                \\ \<local>\; t^{*}\; e^{*})
            \end{stackTL}}
            &=&
            \<wfunc>
            {\begin{stackTL}
                \erase[tfi]{ti_1^{*};l_1;\phi_1 \rightarrow ti_2^{*};l_2;\phi_2}\\ \<wlocal>\; t^{*}\; \erase[e^{*}]{e^{*}}
            \end{stackTL}} \\

            \erase[f]{\<func> ti_1^{*};l_1;\phi_1 \rightarrow ti_2^{*};l_2;\phi_2\; im}
            &=&
            \<wfunc> \erase[tfi]{ti_1^{*};l_1;\phi_1 \rightarrow ti_2^{*};l_2;\phi_2}\; im \\
        \end{array}
    \end{mathpar}
\end{definition}

To erase a function definition $f$, we erase both the type declaration $ti_1^{*};l_1;\phi_1 \rightarrow ti_2^{*};l_2;\phi_2$ and the body $e^{*}$.
We can also erase an imported function by erasing the declared type $ti_1^{*};l_1;\phi_1 \rightarrow ti_2^{*};l_2;\phi_2\;$.

\begin{definition}{$\erase[f]{f} = \mathredbold{f}$}
    \begin{mathpar}
        \begin{array}{rcl}
            erase_f(\<func>
            {\begin{stackTL}
                ti_1^{*};l_1;\phi_1 \rightarrow ti_2^{*};l_2;\phi_2
                \\ \<local>\; t^{*}\; e^{*})
            \end{stackTL}}
            &=&
            \<wfunc>
            {\begin{stackTL}
                \erase[tfi]{ti_1^{*};l_1;\phi_1 \rightarrow ti_2^{*};l_2;\phi_2}\\ \<wlocal>\; t^{*}\; \erase[e^{*}]{e^{*}}
            \end{stackTL}} \\

            \erase[f]{\<func> ti_1^{*};l_1;\phi_1 \rightarrow ti_2^{*};l_2;\phi_2\; im}
            &=&
            \<wfunc> \erase[tfi]{ti_1^{*};l_1;\phi_1 \rightarrow ti_2^{*};l_2;\phi_2}\; im \\
        \end{array}
    \end{mathpar}
\end{definition}

We show that erasing a \name function $f$, that is well typed under a module type context $C$, produces a \wasm function $\erase[f]{f}$ that is well typed under the erased module type context $\erase[C]{C}$.
This is useful not just for erasing the surface syntax, but also because functions are a part of closures which are used at run time (as part of module instances and tables).
The proof relies on \reflemma{Sound Static Typing Erasure} to prove that the body is still well typed.
The case of imported functions is trivial because an imported function is well typed under absolutely any context and with any function type, so it is omitted.

\begin{lemma}{\deflemma{Sound Function Typing Erasure}}

    If
    $$C \vdash \<func>
    {\begin{stackTL}
        \ti{t_1}{a_1}^{*};l_1;\phi_1 \rightarrow \ti{t_2}{a_2};l_2;\phi_2
        \\ \<local>\; t^{*}\; e^{*} : ex^{*}\; \ti{t_1}{a_1}^{*};l_1;\phi_1 \rightarrow \ti{t_2}{a_2};l_2;\phi_2
    \end{stackTL}}$$
    then
    $$\erase[C]{C}\;
    {\begin{stackTL}
        \vdash erase_f(\<func>
        {\begin{stackTL}
            \erase[tfi]{\ti{t_1}{a_1}^{*};l_1;\phi_1 \rightarrow \ti{t_2}{a_2};l_2;\phi_2}
            \\ \<local>\; t^{*}\; \erase[e^{*}]{e^{*}})
        \end{stackTL}} \\
        : ex^{*}\; \erase[tfi]{\ti{t_1}{a_1}^{*};l_1;\phi_1 \rightarrow \ti{t_2}{a_2};l_2;\phi_2}
    \end{stackTL}}$$
\end{lemma}
\begin{proof}
    We must show that $$erase_C(C
    {\begin{stackTL}
        ,\text{local}(t_1^{*}\;t^{*}),\text{label}(\ti{t_2}{a_2};l_2;\phi_2),\text{return}(\ti{t_2}{a_2};l_2;\phi_2))
        \\\vdash \erase[e^{*}]{e^{*}}
        \\: \erase[tfi]{\ti{t_1}{a_1}^{*};l_1;\phi_1 \rightarrow \ti{t_2}{a_2};l_2;\phi_2}
    \end{stackTL}}$$
    since it is the only premise of type checking a function definition in \wasm.

    We know the following because it is a premise of \refrule{Func} which we have assumed to hold.
    $$C
    {\begin{stackTL}
        ,\text{local}(t_1^{*}\;t^{*}),\text{label}(\ti{t_2}{a_2};l_2;\phi_2),\text{return}(\ti{t_2}{a_2};l_2;\phi_2)
        \\\vdash e^{*}
        \\: \ti{t_1}{a_1}^{*};l_1;\phi_1 \rightarrow \ti{t_2}{a_2};l_2;\phi_2
    \end{stackTL}}$$

    Then, by \reflemma{Sound Static Typing Erasure}, we have that
    $$erase_C(C
    {\begin{stackTL}
        ,\text{local}(t_1^{*}\;t^{*}),\text{label}(\ti{t_2}{a_2};l_2;\phi_2),\text{return}(\ti{t_2}{a_2};l_2;\phi_2))
        \\\vdash \erase[e^{*}]{e^{*}}
        \\: \erase[tfi]{\ti{t_1}{a_1}^{*};l_1;\phi_1 \rightarrow \ti{t_2}{a_2};l_2;\phi_2}
    \end{stackTL}}$$

\end{proof}

Erasing an indexed type function keeps only the primitive \wasm types ($t_1^{*}$ and $t_2^{*}$) from the indexed types representing the stack ($\ti{t_1}{a_1}^{*}$ and $\ti{t_2}{a_2}^{*}$), and discards everything else.

\begin{definition}{$\erase[tfi]{tfi} = \mathredbold{tf}$}

    $\erase[tfi]{\ti{t_1}{a_1}^{*};l_1;\phi_1 \rightarrow \ti{t_2}{a_2}^{*};l_2;\phi_2} = t_1^{*} \rightarrow t_2^{*}$
\end{definition}

Erasing instructions involves erasing the indexed function types for every instruction that includes it as part of their syntax (blocks and indirect function calls).
We must also remove the \prechk tag from \prechk-tagged instructions to turn them into instructions that exist in \wasm.

\begin{definition}{$\erase[e^{*}]{e} = \mathredbold{e}$}
    \begin{mathpar}
        \begin{array}{rcl}
            \erase[e^{*}]{\<block>\; tfi\; e^{*} \<end>} &=&
            {\begin{stackTL}
                \<wblock>
                {\begin{stackTL}
                    \erase[tfi]{tfi}
                    \\ \erase[e^{*}]{e^{*}}
                \end{stackTL}} \\
                \<wend>
            \end{stackTL}} \\

            \erase[e^{*}]{\<loop>\; tfi\; e^{*} \<end>} &=&
            {\begin{stackTL}
                \<wloop>
                {\begin{stackTL}
                    \erase[tfi]{tfi}
                    \\ \erase[e^{*}]{e^{*}}
                \end{stackTL}} \\
                \<wend>
            \end{stackTL}} \\

            \erase[e^{*}]{\<if>\; tfi\; e_1^{*}\; e_2^{*} \<end>} &=&
            {\begin{stackTL}
                \<wif>
                {\begin{stackTL}
                    \erase[tfi]{tfi}
                    \\ \erase[e^{*}]{e_1^{*}}
                    \\ \erase[e^{*}]{e_2^{*}}
                \end{stackTL}} \\
                \<wend>
            \end{stackTL}} \\

            \erase[e^{*}]{\<label>_n\; \{ e_0^{*} \}\; e^{*} \<end>} &=&
            {\begin{stackTL}
                \<wlabel>_n \;
                {\begin{stackTL}
                    \{ \erase[e^{*}]{e_0^{*}} \}
                    \\ \erase[e^{*}]{e^{*}}
                \end{stackTL}} \\
                \<wend>
            \end{stackTL}} \\

            \erase[e^{*}]{\<local>_n\; \{ i;v^{*} \}\; e^{*} \<end>} &=&
            {\begin{stackTL}
                \<wlocal>_n \;
                {\begin{stackTL}
                    \{ i;v^{*} \}
                    \\ \erase[e^{*}]{e^{*}}
                \end{stackTL}} \\
                \<wend>
            \end{stackTL}} \\

            \erase[e^{*}]{\<callindirect> tfi} &=& \<wcallindirect> \erase[tfi]{tfi} \\

            \erase[e^{*}]{t.\<divpc>} &=& t.\<wdiv> \\

            \erase[e^{*}]{t.\<callindirectpc>} &=& t.\<wcallindirect> \\

            \erase[e^{*}]{t.\<storepc> tp^{?}\; align\; o} &=& t.\<wstore> tp^{?}\; align\; o \\

            \erase[e^{*}]{t.\<loadpc> (tp\_sx)^{?}\; align\; o} &=& t.\<wload> (tp\_sx)^{?}\; align\; o \\

            \erase[e^{*}]{e} &=& e \text{, otherwise} \\
            \erase[e^{*}]{e^{*}} &=& \erase[e^{*}]{e}^{*} \\
        \end{array}
    \end{mathpar}
\end{definition}

\paragraph{Erasing Typing Constructs}
Here, we proof that erasing a \name static typing dervation is sound with respect to \wasm's type system.
This means that erasure on the \name static typing judgment is sound with respect to \wasm's type system.
Specifically, a \name instruction sequence $e^{*}$, that is well typed under a module type context $C$, produces a \wasm instruction sequence $e'^{*}=\erase[e^{*}]{e^{*}}$ that is well typed under the erased module type context $C'=\erase[C]{C}$.

\begin{lemma}{\deflemma{Sound Static Typing Erasure}}

    If $C \vdash e^{*} : ti_1^{*};l_1;\phi_1 \rightarrow ti_2^{*};l_2;\phi_2$,
    \\ then $\erase[C]{C} \vdash \erase[e^{*}]{e^{*}} : \erase[tfi]{ti_1^{*};l_1;\phi_1 \rightarrow ti_2^{*};l_2;\phi_2}$
\end{lemma}
\begin{proof}
    We proceed by induction over typing derivations. Most proof cases are omitted as they are simple, but we provide a few to give an idea of what the proofs look like.
    Intuitively, we want to show that erasing the typing derivation produces a valid \wasm typing derivation.

    For most of the cases, the sequence of instructions $e^{*}$ contains only a single instruction $e_2$, so we elide the step of turning $\erase[e^{*}]{e^{*}}$ into $\erase[e^{*}]{e_2}$.

    \begin{itemize}
        \item Case: $C \vdash t.binop : {\begin{stackTL}\ti{t}{a_1}\;\ti{t}{a_2};l_1;\phi_1 \\ \rightarrow \ti{t}{a_3};l_1;\phi_1,\ti{t}{a_3},(= a_3\; (binop\;a_1\;a_2))\end{stackTL}}$

            We want to show that
            $$\erase[C]{C}\;
            {\begin{stackTL}
                \vdash \erase[e^{*}]{t.binop}
                \\: erase_{tfi}(
                {\begin{stackTL}
                    \ti{t}{a_1}\;\ti{t}{a_2};l_1;\phi_1
                    \\ \rightarrow \ti{t}{a_3};l_1;\phi_1,\ti{t}{a_3},(= a_3\; (binop\;a_1\;a_2)))
                \end{stackTL}}
            \end{stackTL}}$$

            By the definition of $erase_e$, we want to show that $\erase[C]{C}\vdash t.binop : t\;t \rightarrow t$ is valid in \wasm.

            Trivially, we have $\erase[C]{C}\vdash t.binop : t\;t \rightarrow t$ by \refrule{Wasm-Binop}, since \refrule{Wasm-Binop} works under any module type context.

        \item Case: $C \vdash \<block>\;
        {\begin{stackTL}
            \ti{t_1}{a_1}^{*};l_1;\phi_1
            \\ \rightarrow \ti{t_2}{a_2}^{*};l_2;\phi_2
            \\ e^{*} \<end> :
            {\begin{stackTL}
                \ti{t_1}{a_1}^{*};l_1;\phi_1
                \\ \rightarrow \ti{t_2}{a_2}^{*};l_2;\phi_2
            \end{stackTL}}
        \end{stackTL}}$

            We want to show that $$\erase[C]{C} \vdash erase_e(\<block>\;
            {\begin{stackTL}
                \ti{t_1}{a_1}^{*};l_1;\phi_1
                \\ \rightarrow \ti{t_2}{a_2}^{*};l_2;\phi_2
                \\ e^{*} \<end> :
                {\begin{stackTL}
                    \ti{t_1}{a_1}^{*};l_1;\phi_1
                    \\ \rightarrow \ti{t_2}{a_2}^{*};l_2;\phi_2)
                \end{stackTL}}
            \end{stackTL}}$$

            By the definition of $erase_e$ and $erase_{tfi}$ (we also perform the step of erasing indexed function types here to avoid an extra step), we want to show that $\erase[C]{C} \vdash \<wblock>\; t_1^{*} \rightarrow t_2^{*} e^{*} \<wend> : t_1^{*} \rightarrow t_2^{*}$ is a valid \wasm judgment.

            This proof is slightly more involved, since the derivation for this rule includes a premise that $C,\text{label}(\ti{t_2}{a_2}^{*};l_2;\phi_2) \vdash e^{*} : \ti{t_1}{a_1}^{*};l_1;\phi_1 \rightarrow \ti{t_2}{a_2}^{*};l_2;\phi_2$.

            By the inductive hypothesis for the well-typedness of $e^{*}$, we have that $$\erase[C]{C,\text{label}(\ti{t_2}{a_2}^{*};l_2;\phi_2)}\;
            {\begin{stackTL}
                \vdash \erase[e^{*}]{e^{*}}
                \\: erase_{tfi}(
                {\begin{stackTL}
                    \ti{t_1}{a_1}^{*};l_1;\phi_1
                    \\ \rightarrow \ti{t_2}{a_2}^{*};l_2;\phi_2)
                \end{stackTL}}
            \end{stackTL}}$$

            Then we have $\erase[C]{C},\text{label}(t_2^{*}) \vdash \erase[e^{*}]{e^{*}} : t_1^{*}\rightarrow t_2^{*}$ by definition of $erase_C$ and $erase_{tfi}$.

            Now that we have satisfied the premise, $\erase[C]{C} \vdash \<wblock>\; t_1^{*} \rightarrow t_2^{*} e^{*} \<wend> : t_1^{*} \rightarrow t_2^{*}$ by \refrule{Wasm-Block}.

        \item Case: $C \vdash \<br>\; i : \ti{t_1}{a_1}^{*};l_1;\phi_1 \rightarrow \ti{t_2}{a_2}^{*};l_2;\phi_2$

            We want to show that $$\erase[C]{C}\;
            {\begin{stackTL}
                    \vdash erase_e(\<br> i)\;
                    \\ : \erase[tfi]{\ti{t_1}{a_1}^{*};l_1;\phi_1 \rightarrow \ti{t_2}{a_2}^{*};l_2;\phi_2}
            \end{stackTL}}$$

            We have to reason about $\erase[C]{C}$ because the typing judgment relies on the label stack from $C$.

            From $C \vdash \<br>\; i : \ti{t_1}{a_1}^{*};l_1;\phi_1 \rightarrow \ti{t_2}{a_2}^{*};l_2;\phi_2$, we have that $C_\text{label}(i)=\ti{t_1}{a_1}^{*};l_1;\phi_1$, since it is a premise.

            Then, by the definition of $erase_C$, we have that ${\erase[C]{C}}_\text{label}(i) = t_1^{*}$.

            $$\erase[C]{C}\;
            {\begin{stackTL}
                    \vdash erase_e(\<br> i)\;
                    \\ : {\begin{stackTL}
                        \erase[tfi]{\ti{t_1}{a_1}^{*};l_1;\phi_1
                        \\ \rightarrow \ti{t_2}{a_2}^{*};l_2;\phi_2}
                    \end{stackTL}}
            \end{stackTL}} \\
            = \erase[C]{C} \vdash \<br> i : t_1^{*} \rightarrow t_2^{*}$$

            Recall that ${\erase[C]{C}}_\text{label}(i) = t_1^{*} \rightarrow t_2^{*}$, then $\erase[C]{C} \vdash \<br> i : t_1^{*} \rightarrow t_2^{*}$ by \refrule{Wasm-Br}.

        \item Case: $C \vdash \<call>\; i : \ti{t_1}{a_1}^{*};l_1;\phi_1 \rightarrow \ti{t_2}{a_2}^{*};l_2;\phi_2$

            We want to show that $$\erase[C]{C}\;
            {\begin{stackTL}
                    \vdash erase_e(\<call> i)\;
                    \\ : \erase[tfi]{\ti{t_1}{a_1}^{*};l_1;\phi_1 \rightarrow \ti{t_2}{a_2}^{*};l_2;\phi_2}
            \end{stackTL}}$$

            We again have to reason about $\erase[C]{C}$ because the typing judgment relies on the function type from $C$.

            From $C \vdash \<call>\; i : \ti{t_1}{a_1}^{*};l_1;\phi_1 \rightarrow \ti{t_2}{a_2}^{*};l_2;\phi_2$, we have that $C_\text{func}(i)=\ti{t_1}{a_1}^{*};l_1;\phi_1 \rightarrow \ti{t_2}{a_2}^{*};l_2;\phi_2$, since it is a premise.

            Then, by the definition of $erase_C$, we have that ${\erase[C]{C}}_\text{func}(i) = \erase[tfi]{\ti{t_1}{a_1}^{*};l_1;\phi_1 \rightarrow \ti{t_2}{a_2}^{*};l_2;\phi_2} = t_1^{*} \rightarrow t_2^{*}$

            $$\erase[C]{C}\;
            {\begin{stackTL}
                    \vdash erase_e(\<call> i)\;
                    \\ : {\begin{stackTL}
                        \erase[tfi]{\ti{t_1}{a_1}^{*};l_1;\phi_1
                        \\ \rightarrow \ti{t_2}{a_2}^{*};l_2;\phi_2}
                    \end{stackTL}}
            \end{stackTL}} \\
            = \erase[C]{C} \vdash \<call> i : t_1^{*} \rightarrow t_2^{*}$$

            Recall that ${\erase[C]{C}}_\text{func}(i) = t_1^{*} \rightarrow t_2^{*}$.

            Then $\erase[C]{C} \vdash \<call> i : t_1^{*} \rightarrow t_2^{*}$ by \refrule{Wasm-Call}.

        \item Case: $C \vdash \<setlocal>\; i : \ti{t}{a};l_1;\phi_1 \rightarrow \epsilon;l_1[i := a];\phi_1$

            We want to show that $$\erase[C]{C}\;
            {\begin{stackTL}
                    \vdash erase_e(\<setlocal> i)\;
                    \\ : \erase[tfi]{\ti{t}{a};l_1;\phi_1 \rightarrow \epsilon;l_1[i := a];\phi_1}
            \end{stackTL}}$$

            We again have to reason about $\erase[C]{C}$ because the typing judgment relies on the local variable types from $C$.

            From $C \vdash \<setlocal>\; i : \ti{t}{a};l_1;\phi_1 \rightarrow \epsilon;l_1[i := a];\phi_1$, we have that $C_\text{local}(i)=t$, since it is a premise.

            Then, by the definition of $erase_C$, we have that ${\erase[C]{C}}_\text{local}(i) = t$.

            $$\erase[C]{C}\;
            {\begin{stackTL}
                    \vdash erase_e(\<setlocal> i)\;
                    \\ : {\begin{stackTL}
                        \erase[tfi]{\ti{t}{a}^{*};l_1;\phi_1
                        \\ \rightarrow \epsilon;l_1[i := a];\phi_1}
                    \end{stackTL}}
            \end{stackTL}} \\
            = \erase[C]{C} \vdash \<setlocal> i : t \rightarrow \epsilon$$

            Recall that ${\erase[C]{C}}_\text{local}(i) = t \rightarrow \epsilon$, then $\erase[C]{C} \vdash \<setlocal> i : t \rightarrow \epsilon$ by \refrule{Wasm-Set-Local}.

        \item Case: $C \vdash e_1^{*}\;e_2 : \ti{t_1}{a_1}^{*};l_1;\phi_1 \rightarrow \ti{t_2}{a_2};l_2;\phi_2$

            We want to show that $$\erase[C]{C}\;
            {\begin{stackTL}
                    \vdash erase_e(e_1^{*}\;e_2)\;
                    \\ : \erase[tfi]{\ti{t_1}{a_1}^{*};l_1;\phi_1 \rightarrow \ti{t_2}{a_2};l_2;\phi_2}
            \end{stackTL}}$$

            For this typing rule, we must invoke the inductive hypothesis twice: one on the sequence $e_1^{*}$ and once on the instruction $e_2$. Then we must show that we can compose the erased subsequence together to get a well-typed sequence.

            We know that $C \vdash e_1^{*} : \ti{t_1}{a_1}^{*};l_1;\phi_1 \rightarrow \ti{t_3}{a_3};l_3;\phi_3$ and that $C \vdash e_2 : \ti{t_3}{a_3}^{*};l_3;\phi_3 \rightarrow \ti{t_2}{a_2};l_2;\phi_2$ because they are premieses of \refrule{Composition} which we have assumed to hold.

            $\erase[C]{C} \vdash \erase[e^{*}]{e_1^{*}} : \erase[tfi]{\ti{t_1}{a_1}^{*};l_1;\phi_1 \rightarrow \ti{t_3}{a_3};l_3;\phi_3}$ by the inductive hypothesis on $e_1^{*}$.

            $\erase[C]{C} \vdash \erase[e^{*}]{e_1^{*}} : t_1^{*} \rightarrow t_3^{*}$ by definition of $erase_{tfi}$.

            $\erase[C]{C} \vdash \erase[e^{*}]{e_2} : \erase[tfi]{\ti{t_3}{a_3}^{*};l_3;\phi_3 \rightarrow \ti{t_2}{a_2};l_2;\phi_2}$, by the inductive hypothesis on $e_2$.

            $\erase[C]{C} \vdash \erase[e^{*}]{e_2^{*}} : t_3^{*} \rightarrow t_2^{*}$ by definition of $erase_{tfi}$.

            $$\erase[C]{C}\;
                {\begin{stackTL}
                        \vdash erase_{e^{*}}(e_1^{*}\;e_2)\;
                        \\ : {\begin{stackTL}
                            \erase[tfi]{\ti{t_1}{a_1}^{*};l_1;\phi_1
                            \\ \rightarrow \ti{t_2}{a_2};l_2;\phi_2}
                        \end{stackTL}}
                \end{stackTL}} \\
                = \erase[C]{C}
                {\begin{stackTL}
                    \vdash\;
                    {\begin{stackTL}
                        \erase[e^{*}]{e_1^{*}}
                        \\ \erase[e^{*}]{e_2}
                    \end{stackTL}}
                    \\: t_1^{*} \rightarrow t_2^{*}
                \end{stackTL}}$$

            Recall that we have $\erase[C]{C} \vdash \erase[e^{*}]{e_1^{*}} : t_1^{*} \rightarrow t_3^{*}$ and that $\erase[C]{C} \vdash \erase[e^{*}]{e_2^{*}} : t_3^{*} \rightarrow t_2^{*}$ by definition of $erase_{tfi}$.

            Then, $\erase[C]{C} \; \vdash\; \erase[e^{*}]{e_1^{*}}\; \erase[e^{*}]{e_2}: t_1^{*} \rightarrow t_2^{*}$ by \refrule{Wasm-Composition}.

    \end{itemize}
\end{proof}

To erase a module type context, we must erase all of the function types $tfi^{*}$, the table type $(n,tfi_2^{*})$ if one is present, and the postconditions in the label stack $(\ti{t_1}{a_1}^{*};l_1;\phi_1)^{*}$ and the return stack $(\ti{t_2}{a_2}^{*};l_2;\phi_2)^{?}$.
We erasing postconditions the same way we erase the postconditions of indexed function types: by keeping only the primitive \wasm types ($t_1^{*}$ in the case of a label postcondition).
Recall that erasing a table type means discarding the type information about the functions in the table.

\begin{definition}{$\erase[C]{C} = \mathredbold{C}$}
    \label{def:erase-C}
    \begin{mathpar}
        \begin{array}{rcl}
            {\begin{stackTL} erase_C(\{
                {\begin{stackTL}
                    \text{func } tfi^{*}, \text{ global } tg^{*},
                    \\ \text{table } (n,tfi_2^{*})^{?},
                    \\ \text{memory } m^{?}, \text{ local } t^{*},
                    \\ \text{label } (\ti{t_1}{a_1}^{*};l_1;\phi_1)^{*},
                    \\ \text{return } (\ti{t_2}{a_2}^{*};l_2;\phi_2)^{?}\})
                \end{stackTL}}
            \end{stackTL}}
            &=&
            \{{\begin{stackTL}
                \text{func } \erase[tfi]{tfi^{*}},
                \\ \text{global } tg^{*}, \text{ table } n^{?},
                \\ \text{memory }\; m^{?}, \text{local } t^{*},
                \\ \text{label } (t_1^{*})^{*}, \text{ return } (t_2^{*})^{?}\}
            \end{stackTL}}
        \end{array}
    \end{mathpar}
\end{definition}

\paragraph{Erasing Programs}
Now we will show that erasing a well-typed \name program in reduction form ($s;v^{*};e^{*}$) is sound with respect to \wasm's type system.
Intuitively, we acclomplish this by showing that erasing typing derivations of the \refrule{Program} judgment produce valid \wasm typing derivations, like in \reflemma{Sound Static Typing Erasure}.
To do so, we must show sound erasure for \refrule{Code}, as it is a premise of \refrule{Program}; this is done by \reflemma{Sound Code Typing Erasure}.
Erasing programs involves erasing many run-time data structures, including the store $s$ and store context $S$, as well as modules instances $inst^{*}$ in $s$, and closures $cl$ in modules instances and the optional table.
Erasing the store is shown to be safe by \reflemma{Sound Store Erasure}.

\begin{theorem}{Sound Program Typing Erasure}
    \label{thm:programerasure}

    If $\vdash_i s;v^{*}\;e^{*} : \ti{t_2}{a_2}^{*};l_2;\phi_2$,
    \\ then $\vdash_i \erase[s]{s};v^{*};\erase[e^{*}]{e^{*}} : t_2^{*}$
\end{theorem}
\begin{proof}

    We must show that $\vdash \erase[s]{s} : S$ for some \wasm store context $S$, and that $\erase[S]{S};\empty \vdash_i \erase[e^{*}]{e^{*}} : \erase[tfi]{ti_1^{*};l_1;\phi_1 \rightarrow ti_2^{*};l_2;\phi_2}$.

    We have $\vdash s : S'$, where $S'$ is a \name store context, because it is a premise of \refrule{Program} which we have assumed to hold.

    Then, $\vdash \erase[s]{s} : \erase[S]{S'}$ is valid by \reflemma{Sound Store Erasure}.
    Since $\erase[S]{S'}$ is a \wasm store context, we have that $\vdash \erase{s} : \S$ for some \wasm store context $S$ where $S=\erase[S]{S'}$.

    We also have $S;\empty \vdash_i v^{*}\;e^{*} : \epsilon;l_1;\phi_1 \rightarrow \ti{t_2}{a_2}^{*};l_2;\phi_2$ as a premise of \refrule{Program}.

    In which case we have $\erase[S]{S};\empty \vdash_i \erase[e^{*}]{e^{*}} : \erase[tfi]{ti_1^{*};l_1;\phi_1 \rightarrow ti_2^{*};l_2;\phi_2}$ by \reflemma{Sound Code Typing Erasure}.
\end{proof}

The sound erasure of \refrule{Code} is used in the sound erasure of programs.
Thus, we only prove the case when the optional return stack $(\ti{t_2}{a_2}^{*};l_2;\phi_2)^{?}$ is empty because we are only proving this to use later in \refrule{Program}, which never uses the return stack.
\reflemma{Sound Code Typing Erasure} relies on \reflemma{Sound Admin Typing Erasure}, which shows a similar property, but for the administrative typing judgment $S;C \vdash e^{*} : tfi$.

\begin{lemma}{\deflemma{Sound Code Typing Erasure}}

    If $S;\empty \vdash_i v^{*}\;e^{*} : \epsilon;l_1;\phi_1 \rightarrow \ti{t_2}{a_2}^{*};l_2;\phi_2$,
    \\ then $\erase[S]{S};\empty \vdash_i \erase[e^{*}]{e^{*}} : \erase[tfi]{ti_1^{*};l_1;\phi_1 \rightarrow ti_2^{*};l_2;\phi_2}$
\end{lemma}
\begin{proof}

    We must show that $(\vdash v : t_v)^{*}$ and
    $$\erase[S]{S};erase_C(
    {\begin{stackTL}
        S_\text{inst}(i),\text{local } t_v^{*}
        \\ \vdash \erase[e^{*}]{e^{*}}
        \\: \erase[tfi]{\epsilon;l_1;\phi_1 \rightarrow \ti{t_2}{a_2}^{n};l_2;\phi_2}
    \end{stackTL}}$$

    We have $(\vdash v : t_v)^{*}$ trivially since it is a premise of \refrule{Code} which we have assumed to hold.

    We also have $S;S_\text{inst}(i),\text{local } t_v^{*} \vdash e^{*} : ti_1^{*};l_1;\phi_1 \rightarrow ti_2^{*};l_2;\phi_2$ because it too is a premise of \refrule{Code}.

    Then, by \reflemma{Sound Admin Typing Erasure}, we have that
    $$\erase[S]{S};erase_C(
    {\begin{stackTL}
        S_\text{inst}(i),\text{local } t_v^{*}
        \\ \vdash \erase[e^{*}]{e^{*}}
        \\: \erase[tfi]{\epsilon;l_1;\phi_1 \rightarrow \ti{t_2}{a_2}^{n};l_2;\phi_2}
    \end{stackTL}}$$
\end{proof}

\reflemma{Sound Admin Typing Erasure} builds on \reflemma{Sound Static Typing Erasure} by adding the store context $S$ and typing rules for administrative instructions.
It is necessary to add these rules and extra information because they are used for type checking programs.
Note that while we add $S$ to the judgment used in \reflemma{Sound Static Typing Erasure} to get $S;C \vdash e^{*}; tfi$, none of the rules previously proven reference $S$ in any way, they simply match any store context.

\begin{lemma}{\deflemma{Sound Admin Typing Erasure}}

    If $S;C \vdash e^{*} : ti_1^{*};l_1;\phi_1 \rightarrow ti_2^{*};l_2;\phi_2$,
    \\ then $\erase[S]{S};\erase[C]{C} \vdash \erase[e^{*}]{e^{*}} : \erase[tfi]{ti_1^{*};l_1;\phi_1 \rightarrow ti_2^{*};l_2;\phi_2}$
\end{lemma}
\begin{proof}

    We proceed by induction over typing rules.
    In addition to the prior cases from \reflemma{Sound Static Typing Erasure}, which trivially still hold since the value of $S$ does not matter to those rules, we add proofs for a few administrative typing rules, which may refer to $S$.
    Again, several proof cases are omitted as they are simple.

    \begin{itemize}
        \item $S;C \vdash \<label>_n \{ e_0^{*} \}\; e^{*} \<end> : \epsilon;l_1;\phi_1 \rightarrow \ti{t_2}{a_2}^{n};l_2;\phi_2$

        We must show that
        $$\erase[S]{S};\erase[C]{C} {\begin{stackTL}
            \vdash \erase[e^{*}]{e_0^{*}}
            \\: \erase[tfi]{\ti{t_3}{a_3}^{*};l_3;\phi_3 \rightarrow \ti{t_2}{a_2}^{n};l_2;\phi_2}
        \end{stackTL}}$$
        and
        $$\erase[S]{S};erase_C(C{\begin{stackTL}
            ,\text{label}(\ti{t_3}{a_3}^{*};l_3;\phi_3))
            \\ \vdash \erase[e^{*}]{e^{*}}
            \\ : \erase[tfi]{\epsilon;l_1;\phi_1 \rightarrow \ti{t_3}{a_3}^{*};l_3;\phi_3}
        \end{stackTL}}$$
        as they are the premises of type checking a label block in \wasm.

        We have that $S;C \vdash e_0^{*} : \ti{t_3}{a_3}^{*};l_3;\phi_3 \rightarrow \ti{t_2}{a_2}^{n};l_2;\phi_2$ since it is a premise of \refrule{Label} which we have assumed to hold.

        Then, by the inductive hypothesis for the stored instructions $e_0^{*}$ being well typed, we have that

        $$\erase[S]{S};\erase[C]{C} {\begin{stackTL}
            \vdash \erase[e^{*}]{e_0^{*}}
            \\: \erase[tfi]{\ti{t_3}{a_3}^{*};l_3;\phi_3 \rightarrow \ti{t_2}{a_2}^{*};l_2;\phi_2}
        \end{stackTL}}$$

        $S;C,\text{label}(\ti{t_3}{a_3}^{*};l_3;\phi_3)) \vdash e^{*} : \epsilon;l_1;\phi_1 \rightarrow \ti{t_3}{a_3}^{*};l_3;\phi_3$, because it is a premise of \refrule{Label} which we have assumed to hold.

        By the inductive hypothesis for the body $e^{*}$ being well typed, we have that

        $$\erase[S]{S};erase_C(C{\begin{stackTL}
            ,\text{label}(\ti{t_3}{a_3}^{*};l_3;\phi_3))
            \\ \vdash \erase[e^{*}]{e^{*}}
            \\ : \erase[tfi]{\epsilon;l_1;\phi_1 \rightarrow \ti{t_3}{a_3}^{*};l_3;\phi_3}
        \end{stackTL}}$$

        \item $S;C \vdash \<local>_n \{ i;v^{*} \}\; e^{*} \<end> : \epsilon;l_1;\phi_1 \rightarrow \ti{t_2}{a_2}^{n};l_2;\phi_2$

        The premise of this rule relies on \refrule{Code} with a non-empty return postcondition, which we have not yet proved sound erasure for, so instead we must derive \refrule{Code} for the erased program.

        Thus, we must show that $(\vdash v : t_v)^{*}$ and
        $$\erase[S]{S};erase_C(
        {\begin{stackTL}
            S_\text{inst}(i),\text{local } t_v^{*},\text{ return}(\ti{t_2}{a_2}^{n};l_2;\phi_2))
            \\ \vdash \erase[e^{*}]{e^{*}}
            \\: \erase[tfi]{\epsilon;l_1;\phi_1 \rightarrow \ti{t_2}{a_2}^{n};l_2;\phi_2}
        \end{stackTL}}$$

        We have \refrule{Code} as a premise of \refrule{Local}, which we have assumed to hold.

        Therefore, $(\vdash v : t_v)^{*}$ trivially, since neither values nor primitive types are affected by erasure.

        We also know that
        $$S;S_\text{inst}(i),\text{local } t_v^{*},\text{ return}(\ti{t_2}{a_2}^{n};l_2;\phi_2)) {\begin{stackTL}\vdash e^{*} \<end> \\: \epsilon;l_1;\phi_1 \rightarrow \ti{t_2}{a_2}^{n};l_2;\phi_2\end{stackTL}}$$

        Therefore, by the inductive hypothesis of the well-typedness of $e^{*}$, we have that
        $$\erase[S]{S};erase_C(
        {\begin{stackTL}
            S_\text{inst}(i),\text{local } t_v^{*},\text{ return}(\ti{t_2}{a_2}^{n};l_2;\phi_2))
            \\ \vdash \erase[e^{*}]{e^{*}}
            \\: \erase[tfi]{\epsilon;l_1;\phi_1 \rightarrow \ti{t_2}{a_2}^{n};l_2;\phi_2}
        \end{stackTL}}$$
    \end{itemize}
\end{proof}

We must prove safe erasure about the store $s$ for use in \autoref{thm:programerasure}.
First though, we must define erasure for $s$.
Erasing the store erases all of the modules instances and closures in the tables inside the store.
Note that in the definition we have expanded the definition of a table instance to $(\{\text{inst } i, \text{ func } f\}^{*})^{*}$ for extra clarity.

\begin{definition}{$\erase[s]{s} = \mathredbold{s}$}
    \begin{mathpar}
        \begin{array}{rcl}
            erase_s(\{ {\begin{stackTL}
                    \text{inst } inst^{*},
                    \\ \text{tab } (\{\text{inst } i, \text{ func } f\}^{*})^{*},
                    \\ \text{mem } meminst^{*}\})
                \end{stackTL}}
            &=& \{ {\begin{stackTL}
                \text{inst } \erase[inst]{inst}^{*},
                \\ \text{tab } (\{\text{inst } i, \text{ func } \erase[f]{f}\}^{*})^{*},
                \\ \text{mem } meminst^{*}\}
            \end{stackTL}}
        \end{array}
    \end{mathpar}
\end{definition

\reflemma{Sound Store Erasure} proves that erasing a well-typed \name store results in a well-typed \wasm store.

\begin{lemma}{\deflemma{Sound Store Erasure}}
    If $\vdash s : S$, then $\vdash \erase[s]{s} : \erase[S]{S}$
\end{lemma}
\begin{proof}

    Note that

    \begin{math}
        \begin{array}{rcl}
            s &=& \{\text{inst } inst^{*}, \text{tab } (\{\text{inst } i, \text{ func } f\}^{*})^{*}, \text{ mem } meminst^{*}\} \text{ and}\\
            S &=& \{\text{inst } C^{*}, \text{ tab } (n,tfi^{*})^{*}, \text{ mem } m^{*}\}
        \end{array}
    \end{math}

    Then, $\erase[S]{S}=\{\text{func } \erase[tfi]{tfi}^{*}, \text{ global } tg^{*}, \text{ table } n, \text{ memory } n^{?}, ...\}$ by the definition of $erase_C$.

    Then, we must prove the following properties, as they are the premises of $\vdash \erase[s]{s} : \erase[S]{S}$:
    \begin{enumerate}
        \item $(\erase[S]{S} \vdash \erase[inst]{inst} : \erase[C]{C})^{*}$

        We have that $(S \vdash inst : C)^{*}$, because it is a premise of \refrule{Store} that we have assumed to hold.

        Then, we have $\erase[S]{S} \vdash \erase[inst]{inst} : \erase[C]{C})^{*}$ by \reflemma{Sound Instance Typing Erasure}.

        \item $((\erase[S]{S} \vdash \{\text{inst } i, \text{ func } \erase[f]{f}\} : \erase[tfi]{tfi})^{*})^{*}$

        We have $((S \vdash cl :  tfi)^{*})^{*})$, because it is a premise of \refrule{Store} that we have assumed to hold.

        $((\erase[S]{S} \vdash \{\text{inst } i, \text{ func } \erase[f]{f}\} : \erase[tfi]{tfi})^{*})^{*}$ by \reflemma{Sound Closure Typing Erasure}

        Then, we have $\erase[S]{S} \vdash \erase[inst]{inst} : \erase[C]{C})^{*}$ by \reflemma{Sound Instance Typing Erasure}.

        \item $(n \leq |\{\text{inst } i, \text{ func } \erase[f]{f}\}|)^{*}$

        We have that $(n \leq |\{\text{inst } i, \text{ func } f\}|)^{*}$, because it is a premise of \refrule{Store} that we have assumed to hold.

        Because the number of closures is not affected by erasure, we can then say that $(n \leq |\{\text{inst } i, \text{ func } \erase[f]{f}\}|)^{*}$

        \item $(m \leq |b^{*}|)^{*}$

        Trivially, we have that $(m \leq |b^{*}|)^{*}$, because it is a premise of \refrule{Store} that we have assumed to hold.
    \end{enumerate}
\end{proof}

Erasing a module instance erases all of the functions $f$ in the closures (which we have expanded inline to $\{\text{inst } i, \text{ func } f\}$) within the module instance.

\begin{definition}{$\erase[inst]{inst} = \mathredbold{inst}$}
    \begin{mathpar}
        \begin{array}{rcl}
            {\begin{stackTL} erase_{inst}(\{
                {\begin{stackTL}
                    \text{func } \{\text{inst } i, \text{ func } f\}^{*},
                    \\ \text{global } v^{*}, \text{ table } i^{?},
                    \\ \text{memory } j^{?}\})
                \end{stackTL}}
            \end{stackTL}}
            &=&
            \{{\begin{stackTL}
                \text{func } \{\text{inst } i, \text{ func } \erase[f]{f}\}^{*},
                \\ \text{global } v^{*}, \text{ table } i^{?},
                \\ \text{memory } j^{?}\}
            \end{stackTL}}
        \end{array}
    \end{mathpar}
\end{definition}

We now prove that if a \name module instance $inst$ has type $C$ under the store context $S$, then the erased \wasm instance $\erase[inst]{inst}$ will have the erased type $\erase[c]{C}$ under the erased store context $\erase[S]{S}$.
To do this, we rely on the above lemmas to safely erase index information from function declarations and table declarations (globals and memory have the same type information in both \name and \wasm).
This will be useful for proving that a well-typed \name store $s$ erases to a well-typed \wasm store $\erase[s]{s}$ since stores contain many instances.
To do this, we rely on the above lemmas to safely erase index type information about closures and tables (globals and memory have the same type information in both \name and \wasm).

\begin{lemma}{\deflemma{Sound Instance Typing Erasure}}
    If $S \vdash inst : C$, then $\erase[S]{S} \vdash \erase[inst]{inst} : \erase[C]{C}$
\end{lemma}
\begin{proof}
    Note that

    \begin{math}
        \begin{array}{rcl}
            S &=& \{\text{inst } C^{*}, \text{ tab } (n,tfi^{*})^{*}, \text{ mem } m^{*}\} \text{,} \\
            inst &=& \{\text{func } \{\text{inst } i, \text{ func } f\}^{*}, \text{global } v^{*}, \text{ table } i^{?}, \text{memory } j^{?}\} \text{, and} \\
            C &=& \{\text{func } tfi^{*}, \text{ global } tg^{*}, \text{ table } (n,tfi_2)^{?}, \text{ memory } n^{?}, \dots\} \\
        \end{array}
    \end{math}

    $\erase[C]{C}=\{\text{func } \erase[tfi]{tfi}^{*}, \text{ global } tg^{*}, \text{ table } n, \text{ memory } n^{?}, ...\}$ by the definition of $erase_C$.

    Then, we must prove the following properties, as they are the premises of $\erase[S]{S} \vdash \erase[inst]{inst} : \erase[C]{C}$:
    \begin{enumerate}
        \item $\erase[S]{S} \vdash \{\text{inst } i, \text{ func } f\} : \erase[tfi]{tfi})^{*}$

        We have that $S \vdash \{\text{inst } i, \text{ func } f\} : tfi$, because it is a premise of \refrule{Instance} that we have assumed to hold.

        Then, we have $\erase[S]{S} \vdash \{\text{inst } i, \text{ func } f\} : \erase[tfi]{tfi})^{*}$ by \reflemma{Sound Closure Typing Erasure}.

        \item $(\vdash v : tg)^{*}$
        Trivially, this is a premise of $S \vdash inst : C$ and is not affected by erasure, so therefore it holds.

        \item $\erase[S]{S}_tab(i)=n$
        $\erase[S]{S}_tab(i)=n$ by definition of $erase_S$.

        Therefore, $\erase[S]{S}_tab(i)=n$.

        \item $\erase[S]{S}_mem(i)=n^{?}$
        Trivially, this is a premise of $S \vdash inst : C$ and is not affected by erasure, so therefore it holds.
    \end{enumerate}
\end{proof}

We erase store contexts by erasing all of the module type instances $C^{*}$ and table types $(n,tfi^{*})^{*}$ within.

\begin{definition}{$\erase[S]{S} = \mathredbold{S}$}
    \begin{mathpar}
        \begin{array}{rcl}
            erase_S(\{ {\begin{stackTL}
                    \text{inst } C^{*},
                    \\ \text{tab } (n,tfi^{*})^{*}, \text{ mem } m^{*}\})
                \end{stackTL}}
            &=& \{ {\begin{stackTL}
                \text{inst } \erase[c]{C}^{*},
                \\ \text{tab } n^{*}, \text{mem } m^{*}\} \end{stackTL}}
        \end{array}
    \end{mathpar}
\end{definition}

Finally, we prove that if a \name closure is well typed than the erased closure is well typed.

\begin{lemma}{\deflemma{Sound Closure Typing Erasure}}

    If $S \vdash \{\text{inst } i, \text{ func } f\}^{*} : tfi$, \\
    then $\erase[S]{S} \vdash \{\text{inst } i, \text{ func } \erase[f]{f}\}^{*} : \erase[tfi]{tfi}$
\end{lemma}
\begin{proof}

    We must show that $\erase[S]{S}_\text{inst}(i) \vdash \erase[f]{f} : \erase[tfi]{tfi}$.

    We have $S_\text{inst}(i) \vdash f : tfi$ since it is a premise of \refrule{Closure} which we have assumed to hold.

    Then, $\erase[S]{S}_\text{inst}(i) \vdash \erase[f]{f} : \erase[tfi]{tfi}$ by \reflemma{Sound Function Typing Erasure}.
\end{proof}





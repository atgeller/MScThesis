\subsection{Subject Reduction}
\label{subsec:subject-reduction}
\emph{Subject reduction}, also sometimes referred to as ``type preservation'', ensures that if a program has a specific type, then the program will have the same type after a reduction step.
Before we present the subject reduction proof, we first introduce a number of useful lemmas.

\reflemma{Inversion} tells us what typing rules can apply to a given \name instruction sequence, and therefore lets us reason about what the type of that sequence looks like.
For example, if we have a typing derivation, $D$ for $S;C \vdash t.\<const> c : ti_1^{*};l_1;\phi_1 \rightarrow ti_2^{*};l_2;\phi_2$, then we know that $D$ must have at its base \refrule{Const}, because that is the only way we have of typing constant instructions.
$D$ can also include any number of applications of \refrule{Subtyping} and \refrule{Stack-Poly}, because they can be applied to any well-typed sequence of instructions.
Thus, we do not know the exact types, since the typing rules are non-deterministic, but we can reason about the general shape given the base type on top of which \refrule{Subtyping} and \refrule{Stack-Poly} get applied.
Additionally, \refrule{Composition} can be used with the empty sequence and any well-typed single instruction.
However, the addition of \refrule{Composition} with the empty sequence is trivial because the postcondition of an empty instruction sequence must be immediately reachable from the precondition, and therefore the stack and local index store must be the same in both (and the postcondition index type context being reachable from the precondition index type context).

Most cases of \reflemma{Inversion} are omitted.
The complete definition can be found in the appendix (\autoref{sec:subreduxproof}).

\begin{lemma}{\deflemma{Inversion}}

    \begin{itemize}
        %% const
        \item If $S;C \vdash t.\<const> c : ti_1^{*};l_1;\phi_1 \rightarrow ti_2^{*};l_2;\phi_2$,
        then $ti_2^{*} = ti_1^{*}\;\ti{t}{a}$, $l_1 = l_2$,
        and $\phi_1,\ti{t}{a},(= a \; \ti{t}{c}) \implies \phi_2$.

        %% binop
        \item If $S;C \vdash t.binop : ti_1^{*};l_1;\phi_1 \rightarrow ti_2^{*};l_2;\phi_2$,
        then $ti_1^{*} = ti^{*} \; \ti{t}{a_1} \; \ti{t}{a_2}$, $ti_2^{*} = ti^{*} \; \ti{t}{a_3}$, $l_1 = l_2$,
        and $\phi_1,\ti{t}{a_3},(= a_3\;(binop\;a_1\;a_2)) \implies \phi_2$.

        %% block
        \item If $S;C \vdash \<block>\; {\begin{stackTL}(ti_3^{*};l_3;\phi_3 \rightarrow ti_4^m;l_4;\phi_4)\\ e^{*} \<end> : ti_1^{*};l_1;\phi_1 \rightarrow ti_2^{*};l_2;\phi_2\end{stackTL}}$
        \\ then $ti_1^{*} = ti_0^{*}\; ti_3^{*}$, $ti_2^{*}=ti_0^{*}\; ti_4^m$, $l_1=l_3$, $l_2=l_4$, $\phi_1 \implies \phi_3$, $\phi_4 \implies \phi_2$, and $S;C,\text{label}(ti_4^m;l_4;\phi_4) \vdash e^{*} : ti_3^{*};l_3;\phi_3 \rightarrow ti_4^m;l_4;\phi_4$.

        %% br
        \item If $S;C \vdash \<br> i : ti_1^{*};l_1;\phi_1 \rightarrow ti_2^{*};l_2;\phi_2$,
        then $ti_1^{*} = ti_3^{*}\;ti^{*}$, $C_\text{label}(i) = ti^{*};l_1;\phi_3$,
        and $\phi_1 \implies \phi_3$.

        %% call_indirect
        \item If $S;C \vdash \<callindirect> ti_3^{*};l_3;\phi_3 \rightarrow ti_4^{*};l_4;\phi_4 : ti_1^{*};l_1;\phi_1 \rightarrow ti_2^{*};l_2;\phi_2$,
        then $ti_1^{*} = ti_0^{*} \; ti_3^{*}$, $ti_2^{*} = ti_0^{*} \; ti_4^{*}$, $l_2=l_1$, $\phi_1 \implies \phi_3$, and $\phi_3,\phi_4 \implies \phi_2$.

        %% composition
        \item If $S;C \vdash e_1^{*} \; e_2 : ti_1^{*};l_1;\phi_1 \rightarrow ti_3^{*};l_3;\phi_3$,
        then $S;C \vdash e_1^{*} : ti_1^{*};l_1;\phi_1 \rightarrow ti_2^{*};l_2;\phi_2$,
        and $S;C \vdash e_2 : ti_2^{*};l_2;\phi_2 \rightarrow ti_3^{*};l_3;\phi_3$.

        \thought{This relies on a bit of a non-obvious idea that even if the precondition is changed through subtyping, the subtyping can be deferred until a later composition.}

        \thought{This also relies on the idea that even if subtyping appears in the derivation tree, it can equivalently be applied on the premises of the composition.}
    \end{itemize}
\end{lemma}
\begin{proof}
    Proof omitted, but follows from induction over typing derivations.
\end{proof}

The next lemma, \reflemma{Lift-Consts}, shows that if a sequence of constants, $v^n$, has a certain postcondition within a nested context, $L^j$, then it has the same postcondition outside of that context with the precondition of the context.
We use this rule for branching and returning when we have some values $v^n$ inside a reduction context $L^j$.

The intuition for the proof is that the nature of nested contexts are such that all of the instructions preceding $v^n$ are values and therefore only add fresh index variables which are constrained to be equal to constants.
Thus, we can pull $v^n$ outside of the nested context and know that we can still get to the postcondition because we can add back in, using implication, all of the fresh index variables that we would have added from the values preceding.

\begin{lemma}{\deflemma{Lift-Consts}}

    If $S;C \vdash v^n : \epsilon;l_3;\phi_3 \rightarrow ti^n;l_3;\phi_4$ is a subderivation of $S;C \vdash  L^j [v^n] : s_1;l_1;\phi_1 \rightarrow s_2;l_2;\phi_2$,
    \\then $S;C \vdash v^n : \epsilon;l_1;\phi_1 \rightarrow ti^n;l_3;\phi_4$ after reduction
\end{lemma}
\begin{proof}
    By induction on $j$.
    \begin{itemize}
        \item Base case: $j=0$

            We want to show that $S;C \vdash v^n : \epsilon;l_1;\phi_1 \rightarrow ti^n;l_3;\phi_4$ after reduction.

            We have $S;C \vdash v_0^{*} \; v^n \; e^{*} \<end> : s_1;l_1;\phi_1 \rightarrow s_2;l_2;\phi_2$ for some $v_0^{*}$ and $e^{*}$ by expanding $L^0$.

            Then, $S;C \vdash (t.\<const> c)^{*} : \epsilon;l_1;\phi_0 \rightarrow \ti{t}{a}^{*};l_1;\phi_0,\ti{t}{a}^{*},(\<eq> a \; \ti{t}{c})$ where $v_0^{*}=(t.\<const> c)^{*}$ and $\phi_1 \implies \phi_0$ by \reflemma{Inversion} on \refrule{Const}.

            Further, $S;C \vdash v^n : \epsilon;l_3;\phi_0,\ti{t}{a}^{*},(\<eq> a \; \ti{t}{c}) \rightarrow \ti{t}{a}^{*}\;ti^n;l_3;\phi_4$, by \reflemma{Inversion} on \refrule{Const}.

            We now have all the information we need to show what we want to show.

            We know $\phi_0,\ti{t}{a}^{*},(\<eq> a \; \ti{t}{c}) \implies \phi_3$.

            Recall that $S;C \vdash v^n : \epsilon;l_3;\phi_0,\ti{t}{a}^{*},(\<eq> a \; \ti{t}{c}) \rightarrow \ti{t}{a}^{*}\;ti^n;l_3;\phi_4$, then $$S;C \vdash v^n : \ti{t}{a}^{*};l_3;\phi_0,\ti{t}{a}^{*},(\<eq> a \; \ti{t}{c}) \rightarrow \ti{t}{a}^{*},\ti{t}{a}^{*}\;ti^n;l_3;\phi_4$$ by \refrule{Subtyping}.

            If $v_0^{*}$ are not executed (\ie they are not part of the reduced expression), then $a^{*}$ are fresh, so $\phi_0 \implies \phi_0,\ti{t}{a}^{*},(\<eq> a \; \ti{t}{c})$, and therefore $S;C \vdash v^n : \epsilon;l_1;\phi_0 \rightarrow ti^n;l_1;\phi_4$ by \refrule{Subtyping} and since $l_1=l_3$.

            Then, $S;C \vdash v^n : \epsilon;l_1;\phi_1 \rightarrow ti^n;l_1;\phi_4$ by $subtyping$.

        \item Induction case: $j=k+1$

            We want to show that $S;C \vdash v^n : \epsilon;l_1;\phi_1 \rightarrow ti^n;l_3;\phi_4$ after reduction.

            We have $S;C \vdash \<label>_n \{ e_0^{*} \} \; v_0^{*} \; L^k[v^n] \; e_1^{*} \<end> : s_1;l_1;\phi_1 \rightarrow s_2;l_2;\phi_2$ for some $v_0^{*}$, $e_0^{*}$, and $e_1^{*}$ by expanding $L^j$.

            Then, $S;C \vdash (t.\<const> c)^{*} : \epsilon;l_1;\phi_0 \rightarrow \ti{t}{a}^{*};l_1;\phi_0,\ti{t}{a}^{*},(\<eq> a \; \ti{t}{c})$ where $v_0^{*}=(t.\<const> c)^{*}$ and $\phi_1 \implies \phi_0$ by \reflemma{Inversion} on \refrule{Const}.

            Further, $S;C \vdash L^k[v^n] : \ti{t}{a}^{*};l_1;\phi_0,\ti{t}{a}^{*},(\<eq> a \; \ti{t}{c}) \rightarrow s_5;l_5;\phi_5$ for some $s_5;l_5;\phi_5$ by \reflemma{Inversion} on \refrule{Label}.

            Now we can prove want we wanted to show.

            We know $S;C \vdash v^n : \epsilon;l_1;\phi_0,\ti{t}{a}^{*},(\<eq> a \; \ti{t}{c}) \rightarrow ti^n;l_1;\phi_4$ by the inductive hypothesis.

            If $v_0^{*}$ are not executed (\ie after one reduction step), $a^{*}$ are fresh, so $\phi_0 \implies \ti{t}{a}^{*},(\<eq> a \; \ti{t}{c})$, and therefore $S;C \vdash v^n : \epsilon;l_1;\phi_0 \rightarrow \ti{t}{a}^{*}\;ti^n;l_3;\phi_5$ by \refrule{Subtyping} and since $l_1=l_3$.

            Then, $S;C \vdash v^n : \epsilon;l_1;\phi_1 \rightarrow \ti{t}{a}^{*}\;ti^n;l_3;\phi_3$ by \refrule{Subtyping}.

    \end{itemize}
\end{proof}


\begin{theorem}{Subject Reduction}

    If $\vdash_i s;v^{*};e^{*} : ti^{*};l;\phi$ and $s;v^{*};e^{*} \hookrightarrow_i s';v'^{*};e'^{*}$ then $\vdash_i s';v'^{*};e'^{*} : ti^{*};l;\phi$.
\end{theorem}
\begin{proof}

    We have $\vdash s : S$ and $S;\epsilon \vdash_i v^{*};e^{*} : ti^{*};l;\phi$ because they are premises of \refrule{Program}.

    Then, by \reflemma{Subject Reduction for Code}, we have that $\vdash s' : S$ and $S;\epsilon \vdash_i v'^{*};e'^{*} : ti^{*};l;\phi$.

    Thus, $\vdash s';v'^{*};e'^{*}: ti^{*};l;\phi$ by \refrule{Program}.
\end{proof}

\reflemma{Subject Reduction for Code} proves subject reduction of the \refrule{Code} typing rule.
That is, it proves that if a sequence of instructions $e^{*}$ and local variables $v^{*}$ is typed by the \refrule{Code} typing rule, then after a step of reduction the reduced instructions $e'^{*}$ and locals $v'^{*}$ will have the same postcondition $ti^{*};l;\phi$.
Further, if reduction modifies the store $s$, than the modified store $s'$ will have the same type $S$.

In many reduction cases, there are values on the stack that get consumed by reducing an instruction.
This creates a bit of a problem because those values represent intermediate state, and as such will introduce new index variables to the index type context in their postcondition.
After reduction, the intermediate state is no longer present, so we lose those index variables from the postconditions.

For example, $(t.\<const> c)\; \<drop>$ could be typed as $\epsilon;l;\phi \rightarrow \epsilon;l;\phi,\ti{t}{a},(= a\; \ti{t}{c})$ where $a$ represent the value on the stack $t.\<const> c$.
This would reduce to $\epsilon$, and then we lose the information about $a$ in the postcondition index type system.
However, this can be solved using implication, as we know $a$ is fresh from the \refrule{Const}, and therefore we allow saying $\phi \implies \phi,\ti{t}{a},(= a\; \ti{t}{c})$ after reduction.
This pattern will appear in any case of the proof that consumes values.

\begin{lemma}{\deflemma{Subject Reduction for Code}}

    If $S;(ti;l;\phi)^{?} \vdash v^{*}; e^{*} : ti;l;\phi$,
    $\vdash s : S$, (we omit this on rules that do not use the store)
    and $s;v^{*};e^{*} \hookrightarrow s';v'^{*};e'^{*}$,
    then $S;(ti;l;\phi)^{?} \vdash v'^{*}; e'^{*} : ti;l;\phi$,
    and $\vdash s' : S$ (we omit this on rules that do not change the store)
\end{lemma}
\begin{proof}
    By induction on reduction.

    Most proof cases are omitted, the complete proof can be found in the appendix (\autoref{proof:subreduxcode}).

    \begin{itemize}
        \item Case: $S;(ti^{*};l;\phi)^{?}
        \begin{stackTL}
            \vdash_i v_1^j \; (t.\<const> c) \; v_2^k; (t.\<const> c') \; (\<setlocal> j)
            \\ : ti^{*};l;\phi
        \end{stackTL}$
        \\ $\land$ $v_1^j \; (t.\<const> c) \; v_2^k; (t.\<const> c') \; \<setlocal> j \hookrightarrow v_1^j \; (t.\<const> c') \; v_2^k;\epsilon$

            We want to show that $$S;(ti^{*};l;\phi)^{?} \vdash_i v_1^j \; (t.\<const> c') \; v_2^k; (t.\<const> c') \; (\<setlocal> j) : ti^{*};l;\phi$$

            We know $\vdash (t.\<const> c) : \ti{t}{a};(\circ,\ti{t}{a},(= a\;\ti{t}{c}))$,
            $S;S_\text{inst}(i) \vdash (t.\<const> c') \; (\<setlocal> j) : \epsilon:l_1;\phi_1^j,(\circ,\ti{t}{a},(= a\;\ti{t}{c})),\phi_2^k \rightarrow ti^{*};l;\phi$,
            and $l_1(j) = \ti{t}{a}$, and $C_\text{local}(j) = t$ because they are premises of \refrule{Code} that we have assumed to hold.

            By \reflemma{Inversion} on \refrule{Composition},
            $S;S_\text{inst}(i) \vdash (t.\<const> c') : \epsilon;l_1;\phi_1^j,(\circ,\ti{t}{a},(= a\;\ti{t}{c})),\phi_2^k \rightarrow ti_3^{*};l_3;\phi_3$,
            $S;S_\text{inst}(i) \vdash \<setlocal> j : ti_3^{*};l_3;\phi_3 \rightarrow ti^{*};l;\phi$.

            Recall that $t = C_\text{local}(j)$, then by \reflemma{Inversion} on \refrule{Set-Local} we have
            $ti_3^{*} = ti^{*} \; \ti{t}{a'}$,
            $l = l_3[j := \ti{t}{a'}]$,
            and $\phi_3 \implies \phi$.

            Then, by \reflemma{Inversion} on \refrule{Const},
            $ti^{*} = \epsilon$, $l_1 = l_3$, and\\
            $\phi_1^j,(\circ,\ti{t}{a},(= a\;\ti{t}{c})),\phi_2^k,\ti{t}{a'},(= a'\;\ti{t}{c'}) \implies \phi_3$.

            Now we have all the information we need to derive the same type for $(t.\<const> c')$

            We have $S;S_\text{inst}(i) \vdash \epsilon : \epsilon;l;\phi \rightarrow \epsilon;l;\phi$ by \refrule{Empty}.

            Then, $S;S_\text{inst}(i) \vdash \epsilon : \epsilon;l;\phi_1^j,(\circ,\ti{t}{a},(= a\;\ti{t}{c})),\phi_2^k,\ti{t}{a'},(= a'\;\ti{t}{c'}) \rightarrow \epsilon;l;\phi$ by \refrule{Subtyping}.

            % Since $a$ is fresh in the typing derivation, $C \vdash \epsilon : \epsilon;l;\phi_1^j,\phi_2^k,\ti{t}{a'},(= a'\;\ti{t}{c'}) \rightarrow \epsilon;l;\phi$.

            Since $a$ is fresh, $\phi_1^j,\phi_2^k,\ti{t}{a'},(= a'\;\ti{t}{c'}) \implies \phi_1^j,(\circ,\ti{t}{a},(= a\;\ti{t}{c})),\phi_2^k,\ti{t}{a'},(= a'\;\ti{t}{c'})$.

            Then, $S;S_\text{inst}(i) \vdash \epsilon : \epsilon;l;\phi_1^j,\phi_2^k,\ti{t}{a'},(= a'\;\ti{t}{c'}) \rightarrow \epsilon;l;\phi$ by \refrule{Subtyping}.

            Further, $\vdash (t.\<const> c') : \ti{t}{a'};\circ,\ti{t}{a'},(= a'\;\ti{t}{c'})$ by \refrule{Admin-Const}.

            Therefore, $S;(ti^{*};l;\phi)^{?} \vdash_i v_1^j\;(t.\<const> c')\;v_2^k;\epsilon : ti^{*};l;\phi$ by \refrule{Code}.

            \thought{I set out to write a very verbose proof case, but I didn't expect it to be this verbose.}

        \item Case: $S;(ti^{*};l;\phi)^{?} \vdash_i v^{*};(\<ithreetwo>.\<const> k)\;(t.\<load> align\;o) : ti^{*};l;\phi$
        \\ $\land$ $s;(\<ithreetwo>.\<const> k)\;(t.\<load> align\;o) \hookrightarrow_i
        {\begin{stackTL}
            t.\<const> \text{const}_t(b^{*}),
            \\ \text{where }s_\text{mem}(i,k+o,|t|) = b^{*}
        \end{stackTL}}$

            We want to show that $S;(ti^{*};l;\phi)^{?} \vdash_i v^{*}; (t.\<const> \text{const}_t(b^{*})) : ti^{*};l;\phi$

            We know $(\vdash v : ti_v;\phi_v)^{*}$ and $S;C \vdash (\<ithreetwo>.\<const> k)\;(t.\<load> align\;o) : \epsilon;ti_v^{*};\phi_v^{*} \rightarrow ti^{*};l;\phi$ because they are premises of \refrule{Code} which we have assumed to hold.

            Then, by \reflemma{Inversion} on \refrule{Composition}, \refrule{Const}, \refrule{Mem-Load}, we know
            $ti^{*} = \ti{t}{a}$, $ti_v^{*} = l$, and $\phi_v^{*},\ti{t}{a} \implies \phi$.

            We have $$S;S_\text{inst}(i)
            \begin{stackTL}
                \vdash t.\<const> \text{const}_t(b^{*})
                \\ : \epsilon;ti_v^{*};\phi_v^{*} \rightarrow \ti{t}{a};l;\phi_v^{*},\ti{t}{a},(= a\;\ti{t}{c})
            \end{stackTL}$$ by \refrule{Const}.

            Then, $S;S_\text{inst}(i) \vdash (t.\<const> \text{const}_t(b^{*})) : \epsilon;ti_v^{*};\phi_v^{*} \rightarrow \ti{t}{a};l;\phi$ by \refrule{Subtyping}.

            Recall $(\vdash v : ti_v;\phi_v)^{*}$, then
            $S;(ti^{*};l;\phi)^{?} \vdash_i v^{*};t.\<const> \text{const}_t(b^{*}) : ti^{*};l;\phi$ by \refrule{Code}.

        \item Case: $S;(ti^{*};l;\phi)^{?} \vdash_i v^{*};(\<ithreetwo>.\<const> k)\;(t.\<const> c)\;(t.\<store> align\;o) : ti^{*};l;\phi$
        \\ $\land$ $s;(\<ithreetwo>.\<const> k)\;(t.\<const> c)\;(t.\<store> align\;o) \hookrightarrow_i s';\epsilon$, where $s' = s \text{ with } \text{mem}(i,k+o,|t|) = \text{bits}_t^{|t|}(c)$

            We know $(\vdash v : ti_v;\phi_v)^{*}$ and
            $$S;S_\text{inst}(i)
            {\begin{stackTL}
                \vdash (\<ithreetwo>.\<const> k)\;(t.\<const> c)\;(t.\<store> align\;o)
                \\: \epsilon;ti_v^{*};\phi_v^{*} \rightarrow ti^{*};l;\phi
            \end{stackTL}}$$
            because they are premises of \refrule{Code} which we have assumed to hold.

            Then, by \reflemma{Inversion} on \refrule{Composition}, \refrule{Const}, and \refrule{Mem-Store}, we have
            $ti^{*} = \epsilon$, $ti_v = l$, and $\phi_v^{*},\ti{\<ithreetwo>}{a_1},(= a_1\;\ti{\<ithreetwo>}{k}),\ti{t}{a_2},(= a_2\;\ti{t}{c}) \implies \phi$.

            Since $a_1$ and $a_2$ are fresh, $\phi_v^{*} \implies \phi$.

            We have $S;S_\text{inst}(i) \vdash \epsilon : \epsilon;l;\phi_v^{*} \rightarrow \epsilon;l;\phi_v^{*}$ by \refrule{Empty}.

            Then, $S;S_\text{inst}(i) \vdash \epsilon : \epsilon;ti_v;\phi_v^{*} \rightarrow \epsilon;l;\phi$ by \refrule{Subtyping}.

            Recall that $(\vdash v : ti_v;\phi_v)^{*}$.
            Therefore, $S;(ti^{*};l;\phi)^{?} \vdash_i v^{*};\epsilon : ti^{*};l;\phi$ by \refrule{Code}.

            Now we must ensure that the new store $s'$ is well typed: $\vdash s' : S$.

            Recall $\vdash s : S$, then $S_\text{mem}(i)=n$ and $s_\text{mem}(i)=b^{*}$ where $n \leq |b^{*}|$ because it's a premise of \refrule{Store}.

            Since $s' = s \text{ with } \text{mem}(i,k+o,|t|) = \text{bits}_t^{|t|}(c)$, then $|s'_text{mem}(i)|=|s_text{mem}(i)|$, and therefore $n \leq |s'_text{mem}(i)|$, so $s' : S$ by \refrule{Store}.

        \item Otherwise: we have $(\vdash (t.\<const> c) : \ti{t}{a};(\circ,\ti{t}{a},(= a\; \ti{t}{c})))^{*}$, and $S,S_\text{inst}(i) \vdash e^{*} : \epsilon;\ti{t}{a}^{*};(\circ,\ti{t}{a},(= a\; \ti{t}{c}))^{*} \rightarrow ti;l;\phi$

            By \reflemma{Subject Reduction Without Effects}, we have
            \\$S,S_\text{inst}(i) \vdash e'^{*} : \epsilon;\ti{t}{a}^{*};(\circ,\ti{t}{a},(= a\; \ti{t}{c}))^{*} \rightarrow ti;l;\phi$.

            Then, $S;(ti^{*};l;\phi)^{?} \vdash v^{*};e'^{*} : ti;l;\phi$
    \end{itemize}
\end{proof}

\reflemma{Subject Reduction Without Effects} is used in the subject reduction proof to separate out cases that do not modify state since it simplifies the reasoning.
Further, by separating these cases, we can abstract out the common pattern of building back up to \refrule{Program} from the instruction typing judgment $S;C\vdash tfi$.
To avoid needing to do mutual inversion, we do not include the local block case here.
We do not include any instructions here that modify state, such as $\<setlocal>$ or $\<storepc>$, meaning that this is non-exhaustive.
That is on purpose because we are using this lemma to handle simple cases, and more complex cases are handled separately.
We show that if a sequence of instructions $e^{*}$ reduces to another sequence of instructions $e'^{*}$, and the reduction does not modify the program state (the store $s$ or the locals $v^{*}$), then $e'^{*}$ has the same precondition $ti_1^{*};l_1;\phi_1$ and postcondition $ti_2^{*};l_2;\phi_2$ as $e^{*}$.

\begin{lemma}{\deflemma{Subject Reduction Without Effects}}

    If $S;C \vdash e^{*} : ti_1^{*};l_1;\phi_1 \rightarrow ti_2^{*};l_2;\phi_2$,
    \\ $\vdash s : S$ (note: we omit this for cases which do not use s),
    \\ and $s;v^{*};e^{*} \hookrightarrow s;v^{*};e'^{*}$,
    \\then $S;C \vdash e'^{*} : ti_1^{*};l_1;\phi_1 \rightarrow ti_2^{*};l_2;\phi_2$
\end{lemma}
\begin{proof}
    By case analysis on the reduction rules.

    Most proof cases are omitted, the complete proof can be found in the appendix (\autoref{proof:subreduxnoeffects}).

    \begin{itemize}
        \item $S;C \vdash L^0[\<trap>] : ti_1^{*};l_1;\phi_1 \rightarrow ti_2^{*};l_2;\phi_2$
        \\ $\land$ $L^0[\<trap>] \hookrightarrow \<trap>$

            This case is trivial since $\<trap>$ accepts any precondition and postcondition.
            Thus, $S;C\vdash \<trap> : ti_1^{*};l_1;\phi_1 \rightarrow ti_2^{*};l_2;\phi_2$ by \refrule{Trap}.

        %% Binop -> const
        \item $S;C \vdash (t.\<const> c_1)\; (t.\<const> c_2)\; t.binop : ti_1^{*};l_1;\phi_1 \rightarrow ti_2^{*};l_2;\phi_2$
        \\ $\land$ $(t.\<const> c_1)\; (t.\<const> c_2)\; t.binop \hookrightarrow t.\<const> c$ where $c=binop(c_1,c_2)$

            We want to show that $S;C \vdash t.\<const> c : ti_1^{*};l_1;\phi_1 \rightarrow ti_2^{*};l_2;\phi_2$.

            We begin by reasoning about the type of the original instructions $(t.\<const> c_1)\; (t.\<const> c_2)\; t.binop$

            By \reflemma{Inversion} on \refrule{Composition}, \refrule{Const}, and \refrule{Binop}, we know that $ti_2^{*} = ti_1^{*} \ti{t}{a_3}$, $l_2=l_1$, and that
            $$
                \phi_1,
                {\begin{stackTL}
                    \ti{t}{a_1}, (= a_1\; \ti{t}{c_1}), \\
                    \ti{t}{a_2}, (= a_2\; \ti{t}{c_2}), \\
                    \ti{t}{a_3}, (= a_3\; (binop\; a_1\; a_2))
                \end{stackTL}}
                \implies \phi_2
            $$

            Now we will show that $t.\<const> c$ has the appropriate type.

            By $const$, $S;C \vdash t.\<const> c :
                \begin{stackTL}
                    \epsilon;l_1;\phi_1 \\
                    \rightarrow \ti{t}{a_3};l_1;\phi_1,\ti{t}{a_3},(= a_3\;\ti{t}{c})
                \end{stackTL}$.

            Because $c=binop_t(c_1,c_2)$, then by $\implies$,
            $$
                \phi_1,\ti{t}{a},(= a\; \ti{t}{c}) \implies \phi_1,
                {\begin{stackTL}
                    \ti{t}{a_1}, (= a_1\; \ti{t}{c_1}), \\
                    \ti{t}{a_2}, (= a_2\; \ti{t}{c_2}), \\
                    \ti{t}{a_3}, (= a_3\; (binop\; a_1 a_2))
                \end{stackTL}}
            $$

            Therefore, $S;C \vdash (t.\<const> c) : ti_1^{*};l_1;\phi_1 \rightarrow ti_1^{*}\; \ti{t}{a_3};l_1;\phi_2$, by \refrule{Stack-Poly} and \refrule{Subtyping}.

        %% Binop -> trap
        \item  $C\vdash (t.\<const> c_1)\; (t.\<const> c_2)\; t.binop : ti_1^{*};l_1;\phi_1 \rightarrow ti_2^{*};l_2;\phi_2$
        \\ $\land$ $(t.\<const> c_1)\; (t.\<const> c_2)\; t.binop \hookrightarrow \<trap>$

            This case is trivial since $\<trap>$ accepts any precondition and postcondition.
            Thus, $S;C\vdash \<trap> : ti_1^{*};l_1;\phi_1 \rightarrow ti_2^{*};l_2;\phi_2$ by \refrule{Trap}.

        %% Select
        \item Case: $S;C\; {\begin{stackTL}
            \vdash (t.\<const> c_1)\;(t.\<const> c_2)\;(\<ithreetwo>.\<const> 0)\;\<select>
            \\ : ti_1^{*};l_1;\phi_1 \rightarrow ti_2^{*};l_2;\phi_2
        \end{stackTL}}$
        \\ $\land$ $(t.\<const> c_1)\;(t.\<const> c_2)\;(\<ithreetwo>.\<const> 0)\;\<select> \hookrightarrow (t.\<const> c_2)$

            We want to show that $S;C\vdash (t.\<const> c_2) : ti_1^{*};l_1;\phi_1 \rightarrow ti_2^{*};l_2;\phi_2$.

            First, we reason about what the original type must look like.

            By \reflemma{Inversion} on \refrule{Composition}, \refrule{Const}, and \refrule{Select}, we know that $ti_2^{*} = ti_1^{*}\;\ti{a_3}$, $l_2 = l_1$, and
            $$
            {\begin{stackTL}
                \phi_1, {\begin{stackTL}
                    \ti{t}{a_1}, (= a_1\;\ti{t}{c_1}), \\
                    \ti{t}{a_2}, (= a_2\;\ti{t}{c_2}), \\
                    \ti{\<ithreetwo>}{a}, (= a\;\ti{\<ithreetwo>}{0}), \\
                    \ti{t}{a_3},(if\; (= a\; \ti{\<ithreetwo>}{0})\; (= a_3\; a_2)\; (= a_3\; a_1))
                \end{stackTL}} \\
                \implies \phi_2
            \end{stackTL}}
            $$

            Now we show that $(t.\<const> c_2)$ has the appropriate type.

            By \refrule{Const}, \\
            $ C \vdash (t.\<const> c_2) :
                {\begin{stackTL}
                    \epsilon;l_1;\phi_1 \\
                    \rightarrow \ti{t}{a_3};l_1;\phi_1,\ti{t}{a_3},(= a_3\; \ti{t}{c_2}) \\
                \end{stackTL}} $

            Then, $S;C \vdash (t.\<const> c_2) : ti_1^{*};l_1;\phi_1 \rightarrow ti_1^{*}\;\ti{t}{a_3};l_1;\phi_1,\ti{t}{a_3},(= a_3 \; \ti{t}{c_2})$ by \refrule{Stack-Poly}.

            By $\implies$, we have
            $$\phi_1,\ti{t}{a_3},(= a_3\; \ti{t}{c_2}) \implies \phi_1, {\begin{stackTL}
                \ti{t}{a_1}, (= a_1\; \ti{t}{c_1}), \\
                \ti{t}{a_2}, (= a_2\; \ti{t}{c_2}), \\
                \ti{\<ithreetwo>}{a}, (= a\;\ti{\<ithreetwo>}{0}), \\
                \ti{t}{a_3},(\text{if }
                {\begin{stackTL}
                    (= a\; \ti{\<ithreetwo>}{0})
                    \\ (= a_3\; a_2)
                    \\ (= a_3\; a_1))
                \end{stackTL}}
            \end{stackTL}} \\ $$

            Therefore,
            $S;C \vdash (t.\<const> c_2) :
            ti_1^{*};l_1;\phi_1
                \rightarrow ti_2^{*}\;\ti{t}{a_3};l_1;\phi_2$ by $sub-typing$

        %% Block
        \item Case: $S;C \vdash (t.\<const> c)^n \; \<block>\; (ti_3^n;l_3;\phi_3 \rightarrow ti_4^m;l_4;\phi_4) \; e^{*} \<end> : ti_1^{*};l_1;\phi_1 \rightarrow ti_2^{*};l_2;\phi_2$
        \\ $\land$ ${\begin{stackTL}
            (t.\<const> c)^n \; \<block>\; (ti_3^n;l_3;\phi_3 \rightarrow ti_4^m;l_4;\phi_4) \; e^{*} \<end>
            \\ \hookrightarrow \<label>_m \{ \epsilon \} \; (t.\<const> c)^n \; e^{*} \<end>
        \end{stackTL}}$

            We want to show that $\<label>_m \{ \epsilon \} \; (t.\<const> c)^n \; e^{*} \<end> : ti_1^{*};l_1;\phi_1 \rightarrow ti_2^{*};l_2;\phi_2$.

            First, we reason about $ti_1^{*};l_1;\phi_1 \rightarrow ti_2^{*};l_2;\phi_2$.

            We know $S;C
            {\begin{stackTL}
                \vdash \<block>\; (ti_3^n;l_3;\phi_3 \rightarrow ti_4^m;l_4;\phi_4) \; e^{*} \<end>
                \\ : ti_1^{*}\; \ti{t}{a}^n;l_1;\phi_1,\ti{t}{a}^n,(= a \; \ti{t}{c})^n \rightarrow ti_2^{*};l_2;\phi_2
            \end{stackTL}}$\\ by \reflemma{Inversion} on \refrule{Composition} and \refrule{Const}.

            By \reflemma{Inversion} on \refrule{Block}, $l_1=l_3$ and $l_2=l_4$.
            We will use $l_1,l_2$ in place of $l_3,l_4$, respectively, for the remainder of the case.

            Then, $S;C,\text{label}(t_4^{m};l_2;\phi_4) \vdash e^{*} : \ti{t}{a}^n;l_1;\phi_3 \rightarrow ti_4^m;l_2;\phi_4$ because it is a premise of \refrule{Block} which we have already assumed to hold.

            Also, $\ti{t}{a}^n=ti_3^n$, $ti_2^{*}=ti_1^{*}\; ti_4^m$, $\phi_1,\ti{t}{a}^n,(= a \; \ti{t}{c})^n \implies \phi_3$, and $\phi_4 \implies \phi_2$ by \reflemma{Inversion} on \refrule{Block}.

            Now we have all the information we need to show that
            \\$\<label>_m \{ \epsilon \} \; (t.\<const> c)^n \; e^{*} \<end> : ti_1^{*};l_1;\phi_1 \rightarrow ti_2^{*};l_2;\phi_2$.

            Remember that \refrule{Label} uses the types of both the body \\$(t.\<const> c)^n \; e^{*}$ and the stored instructions $\epsilon$.

            First, we show the type of the body.

            We have $$S;C,\text{label}(t_4^{m};l_2;\phi_4) \vdash
            {\begin{stackTL}
                (t.\<const> c)^n : \epsilon;l_1;\phi_1
                \\ \rightarrow \ti{t}{a}^n;l_1;\phi_1,\ti{t}{a}^n,(= a \; \ti{t}{c})^n
            \end{stackTL}}$$ by \refrule{Const}.

            Then, since $\phi_1,\ti{t}{a}^n,(= a \; \ti{t}{c})^n \implies \phi_3$, we have $$S;C,\text{label}(t_3^{n};l_1;\phi_3) \vdash (t.\<const> c)^n : \epsilon;l_1;\phi_1 \rightarrow \ti{t}{a}^n;l_1;\phi_3$$ by \refrule{Subtyping}.

            Recall we have $S;C,\text{label}(t_4^{m};l_2;\phi_4) \vdash e^{*} : \ti{t}{a}^n;l_1;\phi_3 \rightarrow ti_4^m;l_2;\phi_4$.

            Then $S;C,\text{label}(t_4^m;l_2;\phi_4) \vdash (t.\<const> c)^n\; e^{*} : \epsilon;l_1;\phi_1 \rightarrow ti_4^m;l_2;\phi_4$ by \refrule{Composition}.

            We have the type we want from the body.
            Now we get the type we want of the stored instructions.
            We already have the postcondition we want, $t_4^{m};l_2;\phi_4$, in the label stack, so we want the stored instruction to just pass the information through.
            Since the stored instructions is $\epsilon$, this is simple to show: we have $S;C \vdash \epsilon : ti_2^m;l_2;\phi_4 \rightarrow ti_2^m;l_2;\phi_4$ by \refrule{Empty} and \refrule{Stack-Poly}.

            Therefore, $C \vdash \<label>_m \{ \epsilon \} \; (t.\<const> c)^n \; e^{*} \<end> : \epsilon;l_1;\phi_1 \rightarrow ti_2^m;l_2;\phi_4$ by $label$.

            Finally, since $\phi_4 \implies \phi_2$, $S;C \vdash \<label>_m \{ \epsilon \} \; (t.\<const> c)^n \; e^{*} \<end> : ti_1^{*};l_1;\phi_1 \rightarrow ti_1^{*}\; ti_4^m;l_2;\phi_2$ by \refrule{Stack-Poly} and \refrule{Subtyping}.

        \item Case: $S;C \vdash \<label>_n \{ e^{*} \} \; L^j [(t.\<const> c)^n \; (\<br> j)] \<end> : ti_1^{*};l_1;\phi_1 \rightarrow ti_2^{*};l_2;\phi_2$
        \\ $\land$ $\<label>_n \{ e^{*} \} \; L^j [(t.\<const> c)^n \; (\<br> j)] \hookrightarrow (t.\<const> c)^n \; e^{*}$

            We want to show that $v^n \; e^{*} : ti_1^{*};l_1;\phi_1 \rightarrow ti_2^{*};l_2;\phi_2$.

            Intuitively, this proof works because the premise of \refrule{Br} assumes that $C_\text{label}(i)$ is the precondition ($ti_1^n;l_3;\phi_5$, as we will soon see) of the stored instructions $e^{*}$ in the $i+1$th label, and the postcondition of the label block is immediately reachable from the postcondition of $e^{*}$.
            Meanwhile, that assumptions is ensured by \refrule{Label}, which ensures that $e^{*}$ has the same precondition as the $i+1$th branch postcondition on the label stack and the same postcondition as the label block instruction.

            By \reflemma{Inversion} on \refrule{Label}, $ti_2^{*}=ti_1^{*}\;ti_4^{*}$ for some $ti_4^{*}$.

            Also, $S;C,\text{label}(ti_1^n;l_3;\phi_5)^j \vdash (t.\<const> c)^n\; (\<br> j) : \epsilon;l_3;\phi_3 \rightarrow ti_\emptyset^{*};l_\emptyset;\phi_\emptyset$ for some $l_3$ and $\phi_3$, where $\phi_5=\phi_3,\ti{t}{a}^n,(= a\; \ti{t}{c})^n$, by \reflemma{Inversion} on \refrule{Label} and \refrule{Br}.

            Then, $S;C,\text{label}(ti_1^n;l_3;\phi_5)^j \vdash (\<br> j) : ti_1^n;l_3;\phi_5 \rightarrow ti_\emptyset^{*};l_\emptyset;\phi_\emptyset$, by \reflemma{Inversion} on \refrule{Composition} and \refrule{Const}.

            Then, $S;C,\text{label}(ti_1^n;l_3;\phi_5)^j \vdash (t.\<const> c)^n : \epsilon;l_3;\phi_3 \rightarrow ti_1^n;l_3;\phi_5$ since it is a premise of $composition$ which we have assumed to hold.

            Further, $S;C \vdash e^{*} : ti_1^n;l_3;\phi_5 \rightarrow ti_2^{*};l_2;\phi_4$ since it is a premise of \refrule{Label} which we have assumed to hold, and $\phi_4 \implies \phi_2$ by \reflemma{Inversion} on \refrule{Label}.

            Then, $S;C \vdash (t.\<const> c)^n \; e^{*} : \epsilon;l_1;\phi_1 \rightarrow ti_2^{*};l_2;\phi_4$ by \reflemma{Lift-Consts} and \refrule{Composition}.

            Finally, $C \vdash (t.\<const> c)^n \; e^{*} : ti_1^{*};l_1;\phi_1 \rightarrow ti_1^{*}\;ti_4^{*};l_2;\phi_2$ by \refrule{Stack-Poly} and \refrule{Subtyping}.

        \item Case: $S;C\; {\begin{stackTL}\vdash (\<ithreetwo>.\<const> j)\; \<callindirect> ti_3^{*};l_3;\phi_3 \rightarrow ti_4^{*};l_4;\phi_4 \\: ti_1^{*};l_1;\phi_1 \rightarrow ti_2^{*};l_2;\phi_2\end{stackTL}}$
        \\ $\land$ $s;(\<ithreetwo>.\<const> j)\; \<callindirect> ti_3^{*};l_3;\phi_3 \rightarrow ti_4^{*};l_4;\phi_4 \hookrightarrow_i \<call> s_\text{tab}(i,j)$ where $s_\text{tab}(i,j)_\text{code}=(\<func> tfi_0\; \<local>\; t^{*}\; e^{*})$ and $tfi_0 <: ti_3^{*};l_3;\phi_3 \rightarrow ti_4^{*};l_4;\phi_4$

            We want to show that $\<call> s_\text{tab}(i,j) : ti_1^{*};l_1;\phi_1 \rightarrow ti_2^{*};l_2;\phi_2$.

            By \reflemma{Inversion} on \refrule{Composition}, \refrule{Const}, and \refrule{Call-Indirect}, we know that $ti_1^{*}=ti_0^{*}\; ti_3^{*}$ and $ti_2^{*}=ti_0^{*}\; ti_4^{*}$ for some $ti_0^{*}$, $l_1=l)2$, $\phi_1 \implies \phi_3$, and $\phi_4 \implies \phi_2$.

            We know $S \vdash s_\text{tab}(i,j) : tfi_0$ since it is a premise of $\vdash s : S$ which we have assumed to hold.

            Then, $S;C \vdash \<call> s_\text{tab}(i,j) : tfi_0$ by \refrule{Call-Cl}.

            $S;C \vdash \<call> s_\text{tab}(i,j) : ti_3^{*};l_1;\phi_3 \rightarrow ti_4^{*};l_2;\phi_4$ by \refrule{Subtyping}.

            Therefore, $S;C \vdash \<call> s_\text{tab}(i,j) : ti_0^{*}\;ti_1^{*};l_1;\phi_1 \rightarrow ti_0^{*}\;ti_1^{*};l_2;\phi_2$ by \refrule{Stack-Poly}.
    \end{itemize}
\end{proof}

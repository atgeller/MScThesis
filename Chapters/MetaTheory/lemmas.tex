\reflemma{Inversion} tells us what typing rules can apply to a given \name instruction sequence, and therefore lets us reason about what the type of that sequence looks like.
For example, if we have a typing derivation, $D$ for $S;C \vdash t.\<const> c : ti_1^{*};l_1;\phi_1 \rightarrow ti_2^{*};l_2;\phi_2$, then we know that $D$ must have at its base \refrule{Const}, because that is the only way we have of typing constant instructions.
$D$ can also include any number of applications of \refrule{Subtyping} and \refrule{Stack-Poly}, because they can be applied to any well-typed sequence of instructions.

We do not know the exact types of instructions just from them being well typed, since the typing rules are non-deterministic.
However, we can reason about the general shape of the types given the base type on top of which \refrule{Subtyping} and \refrule{Stack-Poly} get applied.
Additionally, \refrule{Composition} can be used with the empty sequence and any well-typed single instruction.
The addition of \refrule{Composition} with the empty sequence is trivial because the postcondition of an empty instruction sequence must be immediately reachable from the precondition.
Therefore the stack and local index store must be the same in both the precondition and postcondition of the empty sequence in the above case, and the postcondition index type context must be reachable from the precondition index type context.

Most cases of \reflemma{Inversion} are omitted.
The complete definition can be found in the appendix (\autoref{sec:subreduxproof}).

\begin{lemma}{\deflemma{Inversion}}

    \begin{itemize}
        %% const
        \item If $S;C \vdash t.\<const> c : ti_1^{*};l_1;\phi_1 \rightarrow ti_2^{*};l_2;\phi_2$,
        then $ti_2^{*} = ti_1^{*}\;\ti{t}{a}$, $l_1 = l_2$,
        and $\phi_1,\ti{t}{a},(= a \; \ti{t}{c}) \implies \phi_2$.

        %% binop
        \item If $S;C \vdash t.binop : ti_1^{*};l_1;\phi_1 \rightarrow ti_2^{*};l_2;\phi_2$,
        then $ti_1^{*} = ti^{*} \; \ti{t}{a_1} \; \ti{t}{a_2}$, $ti_2^{*} = ti^{*} \; \ti{t}{a_3}$, $l_1 = l_2$,
        and $\phi_1,\ti{t}{a_3},(= a_3\;(binop\;a_1\;a_2)) \implies \phi_2$.

        %% block
        \item If $S;C \vdash \<block>\; {\begin{stackTL}(ti_3^{*};l_3;\phi_3 \rightarrow ti_4^m;l_4;\phi_4)\\ e^{*} \<end> : ti_1^{*};l_1;\phi_1 \rightarrow ti_2^{*};l_2;\phi_2\end{stackTL}}$
        \\ then $ti_1^{*} = ti_0^{*}\; ti_3^{*}$, $ti_2^{*}=ti_0^{*}\; ti_4^m$, $l_1=l_3$, $l_2=l_4$, $\phi_1 \implies \phi_3$, $\phi_4 \implies \phi_2$, and $S;C,\text{label}(ti_4^m;l_4;\phi_4) \vdash e^{*} : ti_3^{*};l_3;\phi_3 \rightarrow ti_4^m;l_4;\phi_4$.

        %% br
        \item If $S;C \vdash \<br> i : ti_1^{*};l_1;\phi_1 \rightarrow ti_2^{*};l_2;\phi_2$,
        then $ti_1^{*} = ti_3^{*}\;ti^{*}$, $C_\text{label}(i) = ti^{*};l_1;\phi_3$,
        and $\phi_1 \implies \phi_3$.

        %% call_indirect
        \item If $S;C \vdash \<callindirect> ti_3^{*};l_3;\phi_3 \rightarrow ti_4^{*};l_4;\phi_4 : ti_1^{*};l_1;\phi_1 \rightarrow ti_2^{*};l_2;\phi_2$,
        then $ti_1^{*} = ti_0^{*} \; ti_3^{*}$, $ti_2^{*} = ti_0^{*} \; ti_4^{*}$, $l_2=l_1$, $\phi_1 \implies \phi_3$, and $\phi_3,\phi_4 \implies \phi_2$.

        %% composition
        \item If $S;C \vdash e_1^{*} \; e_2 : ti_1^{*};l_1;\phi_1 \rightarrow ti_3^{*};l_3;\phi_3$,
        then $S;C \vdash e_1^{*} : ti_1^{*};l_1;\phi_1 \rightarrow ti_2^{*};l_2;\phi_2$,
        and $S;C \vdash e_2 : ti_2^{*};l_2;\phi_2 \rightarrow ti_3^{*};l_3;\phi_3$.

        \thought{This relies on a bit of a non-obvious idea that even if the precondition is changed through subtyping, the subtyping can be deferred until a later composition.}

        \thought{This also relies on the idea that even if subtyping appears in the derivation tree, it can equivalently be applied on the premises of the composition.}
    \end{itemize}
\end{lemma}
\begin{proof}
    Proof omitted, but follows from induction over typing derivations.
\end{proof}

The next lemma, \reflemma{Lift-Consts}, shows that if a sequence of constants, $v^n$, has a certain postcondition within a nested context, $L^j$, then it has the same postcondition outside of that context with the precondition of the context.
We use this rule for branching and returning when we have some values $v^n$ inside a reduction context $L^j$.

The intuition for the proof is that the nature of nested contexts are such that all of the instructions preceding $v^n$ are values and therefore only add fresh index variables which are constrained to be equal to constants.
Thus, we can pull $v^n$ outside of the nested context and know that we can still get to the postcondition because we can add back in, using implication, all of the fresh index variables that we would have added from the values preceding.

\begin{lemma}{\deflemma{Lift-Consts}}

    If $S;C \vdash v^n : \epsilon;l_3;\phi_3 \rightarrow ti^n;l_3;\phi_4$ is a subderivation of $S;C \vdash  L^j [v^n] : s_1;l_1;\phi_1 \rightarrow s_2;l_2;\phi_2$,
    \\then $S;C \vdash v^n : \epsilon;l_1;\phi_1 \rightarrow ti^n;l_3;\phi_4$ after reduction
\end{lemma}
\begin{proof}
    By induction on $j$.
    \begin{itemize}
        \item Base case: $j=0$

            We want to show that $S;C \vdash v^n : \epsilon;l_1;\phi_1 \rightarrow ti^n;l_3;\phi_4$ after reduction.

            We have $S;C \vdash v_0^{*} \; v^n \; e^{*} \<end> : s_1;l_1;\phi_1 \rightarrow s_2;l_2;\phi_2$ for some $v_0^{*}$ and $e^{*}$ by expanding $L^0$.

            Then, $S;C \vdash (t.\<const> c)^{*} : \epsilon;l_1;\phi_0 \rightarrow \ti{t}{a}^{*};l_1;\phi_0,\ti{t}{a}^{*},(\<eq> a \; \ti{t}{c})$ where $v_0^{*}=(t.\<const> c)^{*}$ and $\phi_1 \implies \phi_0$ by \reflemma{Inversion} on \refrule{Const}.

            Further, $S;C \vdash v^n : \epsilon;l_3;\phi_0,\ti{t}{a}^{*},(\<eq> a \; \ti{t}{c}) \rightarrow \ti{t}{a}^{*}\;ti^n;l_3;\phi_4$, by \reflemma{Inversion} on \refrule{Const}.

            We now have all the information we need to show what we want to show.

            We know $\phi_0,\ti{t}{a}^{*},(\<eq> a \; \ti{t}{c}) \implies \phi_3$.

            Recall that $S;C \vdash v^n : \epsilon;l_3;\phi_0,\ti{t}{a}^{*},(\<eq> a \; \ti{t}{c}) \rightarrow \ti{t}{a}^{*}\;ti^n;l_3;\phi_4$, then $$S;C \vdash v^n : \ti{t}{a}^{*};l_3;\phi_0,\ti{t}{a}^{*},(\<eq> a \; \ti{t}{c}) \rightarrow \ti{t}{a}^{*},\ti{t}{a}^{*}\;ti^n;l_3;\phi_4$$ by \refrule{Subtyping}.

            If $v_0^{*}$ are not executed (\ie they are not part of the reduced expression), then $a^{*}$ are fresh, so $\phi_0 \implies \phi_0,\ti{t}{a}^{*},(\<eq> a \; \ti{t}{c})$, and therefore $S;C \vdash v^n : \epsilon;l_1;\phi_0 \rightarrow ti^n;l_1;\phi_4$ by \refrule{Subtyping} and since $l_1=l_3$.

            Then, $S;C \vdash v^n : \epsilon;l_1;\phi_1 \rightarrow ti^n;l_1;\phi_4$ by $subtyping$.

        \item Induction case: $j=k+1$

            We want to show that $S;C \vdash v^n : \epsilon;l_1;\phi_1 \rightarrow ti^n;l_3;\phi_4$ after reduction.

            We have $S;C \vdash \<label>_n \{ e_0^{*} \} \; v_0^{*} \; L^k[v^n] \; e_1^{*} \<end> : s_1;l_1;\phi_1 \rightarrow s_2;l_2;\phi_2$ for some $v_0^{*}$, $e_0^{*}$, and $e_1^{*}$ by expanding $L^j$.

            Then, $S;C \vdash (t.\<const> c)^{*} : \epsilon;l_1;\phi_0 \rightarrow \ti{t}{a}^{*};l_1;\phi_0,\ti{t}{a}^{*},(\<eq> a \; \ti{t}{c})$ where $v_0^{*}=(t.\<const> c)^{*}$ and $\phi_1 \implies \phi_0$ by \reflemma{Inversion} on \refrule{Const}.

            Further, $S;C \vdash L^k[v^n] : \ti{t}{a}^{*};l_1;\phi_0,\ti{t}{a}^{*},(\<eq> a \; \ti{t}{c}) \rightarrow s_5;l_5;\phi_5$ for some $s_5;l_5;\phi_5$ by \reflemma{Inversion} on \refrule{Label}.

            Now we can prove want we wanted to show.

            We know $S;C \vdash v^n : \epsilon;l_1;\phi_0,\ti{t}{a}^{*},(\<eq> a \; \ti{t}{c}) \rightarrow ti^n;l_1;\phi_4$ by the inductive hypothesis.

            If $v_0^{*}$ are not executed (\ie after one reduction step), $a^{*}$ are fresh, so $\phi_0 \implies \ti{t}{a}^{*},(\<eq> a \; \ti{t}{c})$, and therefore $S;C \vdash v^n : \epsilon;l_1;\phi_0 \rightarrow \ti{t}{a}^{*}\;ti^n;l_3;\phi_5$ by \refrule{Subtyping} and since $l_1=l_3$.

            Then, $S;C \vdash v^n : \epsilon;l_1;\phi_1 \rightarrow \ti{t}{a}^{*}\;ti^n;l_3;\phi_3$ by \refrule{Subtyping}.

    \end{itemize}
\end{proof}

\chapter{Discussion}
\label{chp:discussion}
 
By creating \name, we have taken the first step towards creating a practical system in which an expressive type system is used with a low-level language for safety and performance.
\name provides concrete ways to use type information for compiler optimization at the assembly language level.
This is a first step in the sense that it provides the scaffolding to build such a system that could be part of the infrastructure of the internet: unlike prior work, it is backwards compatible with a commonly used langauge, \wasm, supported by many major browsers.
However, there are still a number of unanswered questions.
We have a number of future ideas for this work some based on what we think is necessary to realize our eventual goal of making \name practical in the real-world, and others based on problems identified during the course of the project so far.

\paragraph{Empirical Evaluation}
The first step would be to implement a compiler for \name so we are able to perform experiments and measure the real performance benefit provided by \name.
This would allow us to empirically test whether \name actually improves performance.
Our plan is to implement \name in Rust building on the CraneLift compilier.
This will also require constructing an algorithm to build typing derivations as well as checking them.

\paragraph{More Optimizations}
There is also the potential to find other optimizations we can perform with the additional type information.
For example, remember from \autoref{sec:typesys} that an $\<if>$ may have a contradiction in the index type context one one of its branches.
In this case, that branch will never be executed, and therefore the other branch must always be taken, so we can safely replace the $\<if>$ instruction with the other branch.
We can do similar optimizations with $\<brif>$ and $\<select>$.

\paragraph{Types Annotations}
Recall that the embedding of \wasm into \name from \autoref{subsec:embedding} does not take advantage of the possibility of using type annotations on functions and blocks to check stronger guarantees about programs.
Type annotations can be added by the developer, who will then get stronger guarantees of correctness along with the potential for more optimizations.
However, we would prefer for the developer not to have to hand-annotate compiled \wasm.
Instead, we could use static analysis to try to find the weakest preconditions that guarantee the safety of \prechk-tagging instructions.
We could also attempt to have have a compiler from a higher-level language to \wasm add annotations as a form of type preserving compilation similar to Sytem F to Typed Assembly Language \cite{FtoTAL}.

\paragraph{Reasoning About Global Variables}
Reasoning about global variables is made difficult because static type checking is restricted to within the module we are checking.
Thus, it is difficult to reason about global variables imported from another module.
Concretely, imagine, in the $j$th module calling a function $f$ that was imported from the $i$th module.
The call instruction will be reduced to $\<call> \{\text{inst } i, \text{ func } f\}$ where $i$ is the module index for the module instance where $f$ is defined.
Theoretically, $f$ should not change the global variables in the $j$th module.
However, it may call a function in the $j$th module which could change the globals in the $j$th module, and since we do not know what the behavior of $f$ is statically within $j$, we have to assume the worst and can make no assumptions about the global variables after $f$ returns.

\paragraph{Handling the Dynamic Resizing of Memory}
While linear memory chunks are initialized with a static size, \wasm supports dynamically growing memory using the $\<growmemory>$ instruction.
Currently, \name only supports erasing memory bounds checks based on the static size.
However, it should be possible to reason about the size of memory being increased by inserting a dependency on the result of the $\<growmemory>$ instruction.
If the result is -1, we know that the memory will not have grown and remains the same size.
Otherwise, the result will be equal to the new memory size.
This would require more dependency in the type system then we currently have with indexed types, since static type values would depend on dynamic control flow.

\paragraph{Support Streaming Compilation}
The format of \wasm code allows compilation and execution to begin with only part of the program downloaded.
Similar streaming compilation is theoretically possible with \name, but there are unanswered questions about how to work in type checking in such a compilation pipeline.
Here are two examples of problems that we expect to face implementing such a system.
First of all, we must make sure such a system is safe, which is complicated by the fact that we may begin executing code before we have finished type checking.
This should not be too much of an issue as long as we ensure code is type checked before we can execute it, so we only execute well-typed code and if we come across code that is not well typed we halt execution and throw a type error.
Second of all, this will require highly efficient type checking, preferably performed in parallel to type check many functions at one.
We could also try to be clever and prioritize type checking on functions that we expect to be executed sooner.

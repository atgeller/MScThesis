\chapter{Complete \name Typing Judgment Definition}
\label{chp:completejudgment}

\begin{mathpar}
    \boxed{tfi <: tfi} \\

    \inferrule*[right=\rulename{Implies}]{
        \phi_0 \implies \phi_1 \and
        \phi_2 \implies \phi_3
    }{
        ti_1^{*};l_1;\phi_1 \rightarrow ti_2^{*};l_2;\phi_2 <: ti_1^{*};l_1;\phi_0 \rightarrow ti_2^{*};l_2;\phi_3
    }
\end{mathpar}

\begin{mathpar}
    \boxed{C\vdash e^{*} : tfi} \\

    \inferrule*[right=\rulename{Unreachable}]{ }{ %% unreachable
        C \vdash \<unreachable> : ti_1^{*};l_1;\phi_1 \rightarrow ti_2^{*};l_2;\phi_2
    }

    \inferrule*[right=\rulename{Nop}]{ }{ %% nop
        C \vdash \<nop> : \epsilon;l;\phi \rightarrow \epsilon;l;\phi
    }

    \inferrule*[right=\rulename{Drop}]{ }{ %% drop
        C \vdash \<drop> : \ti{t}{a};l;\phi \rightarrow \epsilon;l;\phi
    }

    \inferrule*[right=\rulename{Const}]{ a \not\in \phi }{ %% const
        C \vdash t.\<const> c : \epsilon;l;\phi \rightarrow \ti{t}{a};l;\phi,\ti{t}{a},(= a \ti{t}{c})
    }

    \inferrule*[right=\rulename{Binop}]{ a_3 \not\in \phi }{ %% binop
        C \vdash t.binop :
        {\begin{stackTL}
            \ti{t}{a_1}\;\ti{t}{a_2};l;\phi
            \\ \rightarrow \ti{t}{a_3};l;\phi,\ti{t}{a_3},(= a_3\;(\|binop\|\;a_1\;a_2))
        \end{stackTL}}
    }

    \inferrule*[right=\rulename{Testop}]{ a_3 \not\in \phi }{ %% testop
        C \vdash t.testop :
        {\begin{stackTL}
            \ti{t}{a_1}\;l;\phi
            \\ \rightarrow \ti{\<ithreetwo>}{a_2};l;\phi,\ti{t}{a_2},(= a_2\;(\|testop\|\;a_1))
        \end{stackTL}}
    }

    \inferrule*[right=\rulename{Relop}]{ a_3 \not\in \phi }{ %% relop
        C \vdash t.relop :
        {\begin{stackTL}
            \ti{t}{a_1}\;\ti{t}{a_2};l;\phi
            \\ \rightarrow \ti{t}{a_3};l;\phi,\ti{t}{a_3},(= a_3\;(\|relop\|\;a_1\;a_2))
        \end{stackTL}}
    }

    \inferrule*[right=\rulename{Select}]{a_3 \not\in \phi}{ %% select
        C \vdash \<select> : {\begin{stackTL}
            \ti{t}{a_1}\;\ti{t}{a_2}\;\ti{i32}{a};l;\phi
            \\ \rightarrow \ti{t}{a_3};l;\phi,
            {\begin{stackTL}
                \ti{t}{a_3},
                \\ (\text{if}\; (= a\; \ti{\<ithreetwo>}{0})\; (= a_3\;a_2)\; (= a_3\;a_1))
            \end{stackTL}}
        \end{stackTL}}
    }

    \inferrule*[right=\rulename{Block}]{ %% block
        C_2,\text{label } (ti_2^{*};l_2;\phi_2) \vdash e^{*} : ti_1^{*};l_1;\phi_1 \rightarrow ti_2^{*};l_2;\phi_2 \\
    }
    {
        C \vdash \<block>\; (ti_1^{*};l_1;\phi_1 \rightarrow ti_2^{*};l_2;\phi_2)\; e^{*} \<end> : ti_1^{*};l_1;\phi_1 \rightarrow ti_2^{*};l_2;\phi_2
    }
\end{mathpar}

\begin{mathpar}
    \inferrule*[right=\rulename{Loop}]{ %% loop
        C_2,\text{label } (ti_1^{*};l_1;\phi_1)^{*} \vdash e^{*} : ti_1^{*};l_1;\phi_1 \rightarrow ti_2^{*};l_2;\phi_2 \\
    }
    {
        C \vdash \<loop>\; (ti_1^{*};l_1;\phi_1 \rightarrow ti_2^{*};l_2;\phi_2)\; e^{*} \<end> : ti_1^{*};l_1;\phi_1 \rightarrow ti_2^{*};l_2;\phi_2
    }

    \inferrule*[right=\rulename{If}]{ %% if
        C_2,\text{label } (ti_2^{*};l_2;\phi_2) \vdash e_1^{*} : ti_1^{*};l_1;\phi_1, \neg(= a\; \ti{\<ithreetwo>}{0}) \rightarrow ti_2^{*};l_2;\phi_2 \\
        C_2,\text{label } (ti_2^{*};l_2;\phi_2) \vdash e_2^{*} : ti_1^{*};l_1;\phi_1, (= a\; \ti{\<ithreetwo>}{0})) \rightarrow ti_2^{*};l_2;\phi_2 \\
    }
    {
        C \vdash \<if>\; (ti_1^{*};l_1;\phi_1 \rightarrow ti_2^{*};l_2;\phi_2)\; e_1^{*} \<else> e_2^{*} \<end> : ti_1^{*};l_1;\phi_1 \rightarrow ti_2^{*};l_2;\phi_2
    }

    \inferrule*[right=\rulename{Br}]{ %% br
        C_{\text{label}}(i) = ti^{*};l_1;\phi_1
    }
    {
        C \vdash \<br> i : ti_1^{*}\;ti^{*};l_1;\phi_1 \rightarrow ti_2^{*};l_2;\phi_2
    }

    \inferrule*[right=\rulename{Return}]{ %% return
        C_{\text{return}} = ti^{*};l_1;\phi_1
    }
    {
        C \vdash \<return> : ti_1^{*}\;ti^{*};l_1;\phi_1 \rightarrow ti_2^{*};l_2;\phi_2
    }

    \inferrule*[right=\rulename{Br-If}]{ %% br_if
        C_{\text{label}}(i) = ti^{*};l_1;\phi_1,\neg(= a\; \ti{\<ithreetwo>}{0})
    }
    {
        C \vdash \<brif> i : ti^{*}\;\index{i32}{a};l_1;\phi_1 \rightarrow ti^{*};l_1;\phi_1,(= a\; \ti{\<ithreetwo>}{0})
    }

    \inferrule*[right=\rulename{Br-Table}]{ %% br_table
        (C_{\text{label}}(i) = ti^{*};l_1;\phi_i)^{+} \and
        (\phi_1 \implies \phi_i)^n
    }
    {
        C \vdash \<brtable> i^{+} : ti_1^{*}\;ti^{*}\;\index{i32}{a};l_1;\phi_1 \rightarrow ti_2^{*};l_2;\phi_2
    }

    \inferrule*[right=\rulename{Call}]{ %% call
        C_\text{func}(i) = ti_1^{*};l_1;\phi_1 \rightarrow ti_2^{*};l_2;\phi_2
    }
    {
        C \vdash \<call> i : ti_1^{*};l;\phi_1 \rightarrow ti_2^{*};l;\phi_1,\phi_2
    }

    \inferrule*[right=\rulename{Call-Indirect}]{ %% call_indirect
        C_\text{table}(i) = (j, \tfi_2^{*}) \and
    }
    {
        C\;
        {\begin{stackTL}
            \vdash \<callindirect>  ti_1^{*};l_1;\phi_1 \rightarrow ti_2^{*};l_2;\phi_2
            \\ : ti_1^{*}\;\ti{i32}{a};l;\phi_1 \rightarrow ti_2^{*};l;\phi_1,\phi_2
        \end{stackTL}}
    }

    \inferrule*[right=\rulename{Get-Local}]{ %% get_local
        C_{\text{local}}(i) = t \and
        l(i) = \ti{t}{a} \and
        a_2 \not\in \phi
    }
    {
        C \vdash \<getlocal> i : \epsilon;l;\phi \rightarrow \ti{t}{a_2};l;\phi,\ti{t}{a_2},(= a_2 \; a)
    }

    \inferrule*[right=\rulename{Set-Local}]{ %% set_local
        C_{\text{local}}(i) = t \and
        l_2 = l_1[i := \ti{t}{a}] \and
    }
    {
        C \vdash \<setlocal> i : \ti{t}{a};l_1;\phi \rightarrow \epsilon;l_2;\phi
    }
\end{mathpar}

\begin{mathpar}
    \inferrule*[right=\rulename{Tee-Local}]{ %% tee_local
        C_{\text{local}}(i) = t \and
        l_2 = l_1[i := \ti{t}{a}] \and
        a_2 \not\in \phi
    }
    {
        C \vdash \<teelocal> i : \ti{t}{a};l_1;\phi \rightarrow \ti{t}{a_2};l_2;\phi,\ti{t}{a_2},(= a_2 \; a)
    }

    \inferrule*[right=\rulename{Get-Global}]{ %% get_global
        C_\text{global}(i) = \text{mut}^{?}\; t \and
        a \not\in \phi
    }
    {
        C \vdash \<getglobal> i : \epsilon;l;\phi \rightarrow \ti{t}{a};l;\phi,\ti{t}{a}
    }

    \inferrule*[right=\rulename{Set-Global}]{ %% set_global
        C_\text{global}(i) = \text{mut } t
    }
    {
        C \vdash \<setglobal> i : \ti{t}{a};l;\phi \rightarrow \epsilon;l;\phi
    }

    \inferrule*[right=\rulename{Mem-Load}]{ %% memory load
        C_\text{memory} = n \and
        2^{align} \leq (|tp| <)^{?} |t| \and
        a_2 \not\in \phi
    }
    {
        C \vdash t.\<load> (tp\_sx)^{?}\; align\; o : \ti{\<ithreetwo>}{a_1};l;\phi \rightarrow \ti{t}{a_2};l;\phi,\ti{t}{a_2}
    }

    \inferrule*[right=\rulename{Mem-Store}]{ %% memory store
        C_\text{memory} = n \and
        2^{align} \leq (|tp| <)^{?} |t|
    }
    {
        C \vdash t.\<store> tp^{?}\; align\; o : \ti{\<ithreetwo>}{a_1}\;\ti{t}{a_2};l;\phi \rightarrow \epsilon;l;\phi
    }

    \inferrule*[right=\rulename{Current-Memory}]{ %% current mem
        C_\text{memory} = n \and
        a \not\in \phi
    }
    {
        C \vdash \<currentmemory> : \epsilon;l;\phi \rightarrow \ti{\<ithreetwo>}{a};l;\phi,\ti{\<ithreetwo>}{a}
    }

    \inferrule*[right=\rulename{Grow-Memory}]{ %% grow mem
        C_\text{memory} = n \and
        a_2 \not\in \phi
    }
    {
        C \vdash \<growmemory> : \ti{\<ithreetwo>}{a_1};l;\phi \rightarrow \ti{\<ithreetwo>}{a_2};l;\phi,\ti{\<ithreetwo>}{a_2}
    }

    \inferrule*[right=\rulename{Empty}]{ %% empty
    }
    {
        C \vdash \epsilon : \epsilon;l;\phi \rightarrow \epsilon;l;\phi
    }

    \inferrule*[right=\rulename{Stack-Poly}]{ %% extra vars
        C \vdash e^{*} : ti_1^{*};l_1;\phi_1 \rightarrow ti_2^{*};l_2;\phi_2
    }
    {
        C \vdash e^{*} : ti^{*}\;ti_1^{*};l_1;\phi_1 \rightarrow ti^{*}\;ti_2^{*};l_2;\phi_2
    }
\end{mathpar}

\begin{mathpar}
    \inferrule*[right=\rulename{Composition}]{ %% combine
        C \vdash e_1^{*} : ti_1^{*};l_1;\phi_1 \rightarrow ti_2^{*};l_2;\phi_2 \\
        C \vdash e_2 : ti_2^{*};l_2;\phi_2 \rightarrow ti_3^{*};l_3;\phi_3
    }
    {
        C \vdash e_1^{*}\;e_2 : ti_1^{*};l_1;\phi_1 \rightarrow ti_3^{*};l_3;\phi_3
    }

    \inferrule*[right=\rulename{Subtyping}]{
        ti_1^{*};l_1;\phi_1 \rightarrow ti_2^{*};l_2;\phi_2 <: ti_1^{*};l_1;\phi_0 \rightarrow ti_2^{*};l_2;\phi_3 \\
        C \vdash e^{*} \rightarrow ti_1^{*};l_1;\phi_1 \rightarrow ti_2^{*};l_2;\phi_2
    }{
        C \vdash e^{*} \rightarrow ti_1^{*};l_1;\phi_0 \rightarrow ti_2^{*};l_2;\phi_3
    }

    \inferrule*[right=\rulename{Div-Prechk}]{
        \phi \implies \neg(=\ a_2\ 0) \and
        a_3 \not\in \phi
    }{
        C \vdash t.\<divpc> :
        {\begin{stackTL}
            \ti{t}{a_1}\;\ti{t}{a_2};l;\phi
            \\ \rightarrow \ti{t}{a_3};l;\phi,\ti{t}{a_3},(= a_3\;(\|div\|\;a_1\;a_2))
        \end{stackTL}}
    }

    \inferrule*[right=\rulename{Load-Prechk}]{ %% memory load
        C_\text{memory} = n \and
        2^{align} \leq (|tp| <)^{?} |t| \and
        a_3 \not\in \phi \\
        \phi \implies
        {\begin{stackTL}
            (\<ge> (\<add> a_1\; \ti{\<ithreetwo>}{o}) \ti{\<ithreetwo>}{0}),
            \\ (\<le> (\<add> a_1\; (\<add> \ti{\<ithreetwo>}{o+width}))\; \ti{\<ithreetwo>}{n*64 \text{Ki}})
        \end{stackTL}}
    }
    {
        C \vdash t.\<loadpc> (tp\_sx)^{?}\; align\; o :
        {\begin{stackTL}
            \ti{\<ithreetwo>}{a_1};l;\phi
            \\ \rightarrow \ti{t}{a_2};l;\phi,\ti{t}{a_2}
        \end{stackTL}}
    }

    \inferrule*[right=\rulename{Store-Prechk}]{ %% memory store
        C_\text{memory} = n \and
        2^{align} \leq (|tp| <)^{?} |t| \\
        \phi \implies
        {\begin{stackTL}
            (\<ge> (\<add> a_1\; \ti{\<ithreetwo>}{o})\; \ti{\<ithreetwo>}{0}),
            \\ (\<le> (\<add> a_1\; (\<add> \ti{\<ithreetwo>}{o+width}))\; \ti{\<ithreetwo>}{n*64\text{Ki}})
        \end{stackTL}}
    }
    {
        C \vdash t.\<storepc> tp^{?}\; align\; o : \ti{\<ithreetwo>}{a_1}\;\ti{t}{a_2};l;\phi \rightarrow \epsilon;l;\phi
    }

    \inferrule*[right=\rulename{Call-Indirect-Prechk}]{ %% call_indirect
        C_{table}(i) = (n, (\tfi_2 ...)) \\
        \phi \implies (\<gt>\; n\; a) \land (\<le> \ti{\<ithreetwo>}{0}\; a) \\
        \mathit{tfis}=(\tfi_2 ...) \\
        \tfi = ti_1^{*};l_1;\phi_1 \rightarrow ti_2^{*};l_2;\phi_2 \\
        \forall 0 < i \leq n.\; (\phi \implies \neg (= \ti{\<ithreetwo>}{i}\; a)) \lor\; \mathit{tfis}(i) <: \tfi
    }
    {
        C \vdash \<callindirectpc> \tfi :
        {\begin{stackTL}
            ti_1^{*}\;\ti{i32}{a};l;\phi_1
            \\ \rightarrow ti_2^{*};l;\phi_1,\phi_2
        \end{stackTL}}
    }
\end{mathpar}
\documentclass[draft,msc]{ubcthesis}
\usepackage[usenames]{color}
\usepackage{savesym}
\usepackage{amsmath}
\usepackage{amsthm}
\usepackage{amssymb}
\savesymbol{program}
\savesymbol{@program}
\usepackage{semantic}   % Tools for typesetting PL semantics
\usepackage{braket}     % Easy angle-bracket notation
\restoresymbol{}{program}
\restoresymbol{}{@program}
\usepackage{syntax}
\usepackage{mathpartir}
\usepackage{xspace}
\usepackage{dblfloatfix}
\usepackage{multicol}
\usepackage{subcaption}
\usepackage{tikz}
\usepackage{stmaryrd}
\usepackage{changepage}
\usepackage{multirow}
\usepackage{mathtools}
\usepackage{calc}
\usepackage{natbib}

% Always Use these
\usepackage{hyperref}
\usepackage{microtype}
\usepackage[utf8]{inputenc}
\usepackage[T1]{fontenc}

\newcommand{\shrug}[1][]{%
\begin{tikzpicture}[baseline,x=0.8\ht\strutbox,y=0.8\ht\strutbox,line width=0.125ex,#1]
\def\arm{(-2.5,0.95) to (-2,0.95) (-1.9,1) to (-1.5,0) (-1.35,0) to (-0.8,0)};
\draw \arm;
\draw[xscale=-1] \arm;
\def\headpart{(0.6,0) arc[start angle=-40, end angle=40,x radius=0.6,y radius=0.8]};
\draw \headpart;
\draw[xscale=-1] \headpart;
\def\eye{(-0.075,0.15) .. controls (0.02,0) .. (0.075,-0.15)};
\draw[shift={(-0.3,0.8)}] \eye;
\draw[shift={(0,0.85)}] \eye;
% draw mouth
\draw (-0.1,0.2) to [out=15,in=-100] (0.4,0.95); 
\end{tikzpicture}}
\newcommand{\thought}[1]{\textcolor{red}{\textit{(Thought: #1 \shrug)}}}
\newcommand{\todo}[1]{\textcolor{red}{\textit{(TODO: #1)}}}
\newcommand{\feedback}[2]{#2}

\newcommand{\prechk}[0]{$prechk$\xspace}
\newcommand{\name}[0]{Wasm-prechk\xspace}
\newcommand{\wasm}[0]{Wasm\xspace}
\newcommand{\dtal}[0]{DTAL\xspace}
\newcommand{\ie}[0]{\emph{i.e.,}\xspace}
\newcommand{\eg}[0]{\emph{e.g.,}\xspace}

\newcommand{\erase}[1]{erase(#1)}
\newcommand{\embed}[1]{embed(#1)}

\theoremstyle{definition}
\newtheorem{definition}{\theoremstyle{definition}Definition}
\newtheorem{theorem}{Theorem}
\theoremstyle{plain}
\newtheorem{lemma}{Lemma}
%\Crefname{lemma}{Lemma}{Lemmas}

% Stack formatting
\newenvironment{stackTL}{
    \setlength{\arraycolsep}{0pt}
    \begin{array}[t]{l}\ignorespacesafterend
} {
    \end{array}\ignorespacesafterend
}

% Some custom notations for object language stuff
% Typeset object language notation in blue with sans serif font
\newcommand{\tbbf}[1]{\textbf{\color{blue}#1}}
\reservestyle{\keywords}{\tbbf}
\keywords{ithreetwo[i32], isixfour[i64], binop[binop], testop[testop], relop[relop], const[const\;], eq[eq\;], ne[ne\;], eqz[eqz\;], le[le\;], add[add\;], ge[ge\;], unreachable[unreachable], nop[nop], drop[drop], select[select], block[block], end[\;end], loop[loop], if[if], else[\;else\;], br[br\;], brif[br\rule{1ex}{.4pt}if\;], brtable[br\rule{1ex}{.4pt}table\;], return[return], call[call\;], callindirect[call\rule{1ex}{.4pt}indirect\;], getlocal[get\rule{1ex}{.4pt}local\;], setlocal[set\rule{1ex}{.4pt}local\;], teelocal[tee\rule{1ex}{.4pt}local\;], getglobal[get\rule{1ex}{.4pt}global\;], setglobal[set\rule{1ex}{.4pt}global\;], trap[trap], func[func\;], local[local], global[global\;], table[table\;], memory[memory\;], label[label], div[div], load[load\;], store[store\;], currentmemory[current\rule{1ex}{.4pt}memory\;], growmemory[grow\rule{1ex}{.4pt}memory\;], divpc[div\textsubscript{prechk}], callindirectpc[call\rule{1ex}{.4pt}indirect\textsubscript{prechk}\;], loadpc[load\textsubscript{prechk}\;], storepc[store\textsubscript{prechk}\;], module[module\;], fthreetwo[f32], fsixfour[f64], ieight[i8], isixteen[i16], clz[clz], ctz[ctz], popcnt[popcnt],
sub[sub], shl[shl], or[or], gt[gt], import[import\;], export[export\;], convert[convert], reinterpret[reinterpret]}

\mathlig{;}{\tbbf{;\;}}
\mathlig{:}{:}

\newcommand{\trsf}[1]{\textcolor{red}{\mathsf{#1}}}
\reservestyle{\wkeywords}{\trsf}
\wkeywords{withreetwo[i32], wisixfour[i64], wbinop[binop], wtestop[testop], wrelop[relop], wconst[const\;], weq[eq\;], wneq[neq\;], weqz[eqz\;], wle[le\;], wadd[add\;], wge[ge\;], wunreachable[unreachable], wnop[nop], wdrop[drop], wselect[select], wblock[block\;], wend[\;end], wloop[loop\;], wif[if\;], welse[\;else\;], wbr[br\;], wbrif[br\rule{1ex}{.4pt}if\;], wbrtable[br\rule{1ex}{.4pt}table\;], wreturn[return], wcall[call\;], wcallindirect[call\rule{1ex}{.4pt}indirect\;], wgetlocal[get\rule{1ex}{.4pt}local\;], wsetlocal[set\rule{1ex}{.4pt}local\;], wteelocal[tee\rule{1ex}{.4pt}local\;], wgetglobal[get\rule{1ex}{.4pt}global\;], wsetglobal[set\rule{1ex}{.4pt}global\;], wtrap[trap], wfunc[func\;], wlocal[local], wglobal[global\;], wtable[table\;], wmemory[memory\;], wlabel[label], wdiv[div], wload[load\;], wstore[store\;], wcurrentmemory[current\rule{1ex}{.4pt}memory\;], wgrowmemory[grow\rule{1ex}{.4pt}memory\;], wdivpc[div\textsubscript{prechk}], wcallindirectpc[call\rule{1ex}{.4pt}indirect\textsubscript{prechk}\;], wloadpc[load\textsubscript{prechk}\;], wstorepc[store\textsubscript{prechk}\;], wmodule[module\;]}

\hyphenation{Web-Assembly}

\newcommand{\typerule}[2]{C \vdash #1:#2}
%% stack ; locals ; globals ; index context
\newcommand{\ti}[2]{(#1\;#2)}
\newcommand{\type}[4]{#1;#2;#3;#4}
\newcommand{\insttype}[2]{#1 \rightarrow #2}

\makeatletter

\newcommand{\techprefix}{}
\newcommand{\rulename}[1]{{\scshape #1}}

\LetLtxMacro{\rulelabel}{\label}
% For rules in particular
\newcommand{\@defruleStar}[3][\techprefix]{\phantomsection{\rulename{#3}}\expandafter\rulelabel{rule:#1:#2}}
\newcommand{\@defruleNoStar}[2][\techprefix]{\@defruleStar[#1]{#2}{#2}}
\newcommand{\defrule}{\@ifstar\@defruleStar\@defruleNoStar}

\newcommand{\@refruleStar}[3][\techprefix]{\hyperref[rule:#1:#2]{Rule \rulename{#3}}}
\newcommand{\@refruleNoStar}[2][\techprefix]{\@refruleStar[#1]{#2}{#2}}
\newcommand{\refrule}{\@ifstar\@refruleStar\@refruleNoStar}

\newcommand{\lemmaname}[1]{{\scshape #1}}

\LetLtxMacro{\lemmalabel}{\label}
% For lemmas in particular
\newcommand{\@deflemmaStar}[3][\techprefix]{\phantomsection{\lemmaname{#3}}\expandafter\lemmalabel{lemma:#1:#2}}
\newcommand{\@deflemmaNoStar}[2][\techprefix]{\@deflemmaStar[#1]{#2}{#2}}
\newcommand{\deflemma}{\@ifstar\@deflemmaStar\@deflemmaNoStar}

\newcommand{\@reflemmaStar}[3][\techprefix]{\hyperref[lemma:#1:#2]{Lemma \rulename{#3}}}
\newcommand{\@reflemmaNoStar}[2][\techprefix]{\@reflemmaStar[#1]{#2}{#2}}
\newcommand{\reflemma}{\@ifstar\@reflemmaStar\@reflemmaNoStar}

\makeatother

\newcommand{\satisfies}[3]{#1 \xRightarrow{#2} #3}

\includeonly{
    Preamble/acknowledgements,
    Preamble/dedication,
    Preamble/abstract,
    Chapters/introduction,
    Chapters/background,
    Chapters/wasm-prechk,
    Chapters/metatheory,
    Chapters/discussion,
    Chapters/conclusion,
}

%%%%%%%%%%%%%%%%%%%%%%%%%%%%%%%%%%%%%%%%%%%%%%%%%%%%%%%%%%%%%%%

\author{Adam T. Geller}
\title{An Indexed Type System for Faster and Safer WebAssembly}
%% \subtitle{With a Subtitle}

\institution{The University Of British Columbia}
\faculty{The Faculty of Graduate Studies}
\institutionaddress{Vancouver}
\program{Computer Science}
\department{Computer Science}

%%\previousdegree{AAS-DTA (With High Distinction), Bellevue College, 2016}
%%\previousdegree{B.Sc., The University of Washington, 2018}

\copyrightyear{2020}
\submitdate{\monthname\ \number\year} % The "\ " is required after
                                      % \monthname to prevent the
                                      % command from eating the space.
%%%%%%%%%%%%%%%%%%%%%%%%%%%%%%%%%%%%%%%%%%%%%%%%%%%%%%%%%%%%%%%

\begin{document}
\frontmatter

\maketitle

\begin{abstract}
    Downloading and executing untrusted code is inherently unsafe, but also something that happens often on the internet.
    Therefore, untrusted code often requires run-time checks to ensure safety during execution.
    These checks compromise performance and may be unnecessary.
    We present the \name language, an assembly language based on WebAssembly that is intended to ensure safety while justifying the static elimination of run-time checks.
\end{abstract}

\maketitle

\chapter{Acknowledgements}
\todo{This is pretty out of date, also quite long}
This section proceeds through my life in reverse chronological order.

\section*{University of British Columbia}
First and foremost I would like to thank my wonderful advisors, who made this whole process much easier and significantly contributed to my edification.
\begin{itemize}
    \item William Bowman - Despite being a new professor, William has been a wonderful advisor and provided a large amount of guidance when I was struggling.
    William has been a fantastic resource for all of the work here.
    \item Ivan Beschastnikh - Ivan helped me navigate through the system when I was a wide-eyed first-year and learn from various mistakes I made along the way.
    Ivan was a great source of wisdom and gave me the freedom to pursue the topics I found most interesting.
\end{itemize}

I would also like to thank various professors for providing guidance and advice, some of whom are listed below.
\begin{itemize}
    \item Ron Garcia - Ron is definitely one of the nicest people I have met and has forgotten way more about PL than I have ever learned. He also provided great feedback on this thesis (part of which led to much clearer definitions).
    \item Margo Seltzer - Margo is a kickass professor who is insanely smart and cares a lot about students.
    \item Alan Hu - Alan has provided me with many nuggets of wisdom.
    \item Alex Summers - Alex spent time to give me helpful feedback that greatly improved this thesis.
\end{itemize}

Thank you to Justin Frank, for helping to prove type safety and develop the reference implementation.

Finally, all of my fellow graduate students and lab mates who have become my friends and helped me in various ways. There are way too many awesome people to thank them all, but I include some notable names below.
\begin{itemize}
    \item Anny Gakhokidze, Puneet Mehrotra, Nico Ritschel, Chris Chen, Felipe Ba\~{n}ados Schwerter, Paulette Koronkevich, Aarti Kashyap, Clement Fung, Vaastav Anand, and countless others.
\end{itemize}

%%\section*{University of Washington}
%%During my undergrad, I spent a little over a year working on the Cassius project.
%%This was my first research experience and I think I learned just as much from that experience as the rest of my undergrad combined.
%%\begin{itemize}
%%    \item Pavel Panchekha - Pavel was very patient and provided a lot of help getting me off the ground and learning about what doing research is like.
%%    \item Michael D. Ernst - As well as advising me throughout my undergrad research, Mike guided me through the process of applying to grad school.
%%    \item Zach Tatlock - Zach is super nice and always made sure I wasn't lost.
%%    \item Shoaib Kamil - For all his help in research and helping me find my way at my first conference alone.
%%    \item James Wilcox - For introducing me to PL and helping me find and get started on the Cassius project.
%%    \item All the other profs and graduate students of UWPLSE, who were extremely friendly and welcoming.
%%\end{itemize}

%%\section*{Various Others}
%%Many different people have helped me on my path at various times. Below I list a few notable ones.
%%\begin{itemize}
%%    \item B.J. Unti - For first giving me the idea to go to grad school.
%%    \item Xander Veerhoff - For first teaching me how to program, which was not easy.
%%\end{itemize}

\section*{My Parents}
Last but certainly not least, I'd like to thank my awesome parents for everything they have done and continue to do for me.
I'm not sure one could ask for better parents than I have been given (even if I did not ask for them).
They have both always been there for me, especially my mom who spent a huge amount of effort and time taking care of every issue that came up in my early years, as well as homeschooling me.
My Dad has always been my role model and is the smartest person I know.
He tried to introduce me to set theory when I was around 4 and calculus when I was about 13 (the first one stuck pretty well, the second one not so much).


\chapter{Dedications}
This is dedicated to Tabby, Koko, and Socrates.
These have been my best (feline) friends without whom I most likely would not have stayed sane.

\tableofcontents                %% Mandatory
\listoftables                   %% Mandatory if thesis has tables
\listoffigures                  %% Mandatory if thesis has figures

\mainmatter

\chapter{Introduction}
\label{chp:intro}

Types systems are useful for reasoning about programs.
They can be used to reason about the correctness of programs, usually in the form of safety guarantees.
For example, type safety is the property that a well-typed program will never become \emph{stuck}, that is, it will always be able to reduce the current expression or the current expression is a well-formed irreducible value.
The safety guarantees of type systems provide a degree of trust in programs as a well typed program implicitly contains a checkable proof that it will not exhibit behavior disallowed by the type system.

More expressive type systems that can encode richer invariants, can allow near complete trust in a program by verifying that it will never exhibit unsafe behavior.
Generally, such type systems are attached to high-level languages, where explicit abstractions make it easy to reason about programs.
Conversely, using expressive type systems in low-level languages often requires reasoning about program state, which introduces more complexity into the type system.
However, prior work has attached expressive type systems, that permit complex correctness guarantees, to simple low-level languages.

Using such an expressive type system for a low-level language, we can alleviate overhead otherwise required to ensure safety of untrusted low-level programs.
Typically, executing code in an untrusted context requires dynamic safety checks which introduce potentially unnecessary instructions, slowing down execution.
However, with the safety guarantees provided by the type system, we can determine when these checks are unnecessary and remove them.
This would allow low-level programs to be downloaded, checked, and executed safely and efficiently.

Browsers and Internet-of-Things (IoT) require running untrusted code, that may have been downloaded from anywhere.
It is crucial to ensure the safety of the code being executed.
Typically this is accomplished using dynamic safety checks and sandboxing, but dynamic safety checks introduce potentially unnecessary instructions, slowing down code.
\todo{Introduce WebAssembly}

Type systems can be used to alleviate the need for safety checks, reducing overhead, by providing safety guarantees about programs before they are run.
The idea of using type systems to ensure the safety of low-level code is not a new one.
Several projects have attached expressive type systems to low-level languages to attach proofs of correctness to low-level programs.
However, the focus in these cases is on correctness, not on performance.
Using a more expressive type system, we can ensure safety \emph{and} improve the performance of low-level code in untrusted environments.

\section{Contributions}
We want to use types to improve performance while ensuring safety in real-world low-level programs.
Towards that goal, we have created a indexed type system, \name, for WebAssembly (\wasm) which guarantees safety and has the potential to improve performance.
\thought{Only one sentence of justification? It's a good sentence but it doesn't capture the entire argument behind \wasm. Also, this really doesn't belong in the contributions part.}
We chose \wasm because it is used in browsers and IoT devices, so both performance and safety are critical concerns.

\todo{This is weak}
\name is an index type system which is able to encode linear constraints on program variables.
These constraints provide sufficient information to tell if dynamic checks are unnecessary for a given operation.

\section{Roadmap}
\todo{I don't think this is needed with combined technical sections}
Chapter 2 discusses related work to provide background and highlight the contributions of our work.
In chapter 3, we present the \name index type system and explain how it works.
Chapter 4 explains how we facilitate dynamic check elimination using \name.
To ensure safety, we prove the type safety of \name in Chapter 5.
Chapter 6 offers a discussion of our results.
Finally, Chapter 7 concludes the thesis.

\chapter{Background}
\section{\wasm}
\label{sec:wasm}
Here we present an overview of \wasm so reader's have some familiarity with it for when we present \name.
We do not cover the entirety of the \wasm language as presented in the 2017 paper, but rather selected important points about the syntax, semantics, and type system.
It is recommended that the reader first skim this section to understand the basics and then refer back while reading \autoref{chp:prechk} and \autoref{chp:metatheory}.

\subsection{\wasm Syntax}
\autoref{fig:wasmsyntaxtypes} shows the types of \wasm.
Primitive \wasm types and values, represented as $t$, include 32 and 64-bit floats and integers.
Packed types, $tp$, include 8, 16, and 32 bit integers, they are used in memory operations.
Function types, $tf$, map from one sequence of types to another, but they are just syntax, not function types in the traditional sense, as explained in \autoref{subsec:wasmtyping}.

\begin{figure}
\begin{math}
\begin{array}{rcl}
    t & ::= & \<ithreetwo> \mid \<isixfour> \mid \<fthreetwo> \mid \<fsixfour> \\
    tp & ::= & \<ieight> \mid \<isixteen> \mid \<ithreetwo> \\
    tf & ::= & t^{*} \rightarrow t^{*} \\
    tg & ::= & \text{mut}^{?}\; t
\end{array}
\end{math}
\caption{\wasm Types}
\label{fig:wasmsyntaxtypes}
\end{figure}

The \wasm syntax uses the Kleene star within its BNF to refer to possibly empty sequences.
Kleene stars are commonly used in regular expressions to denote any number of a certain pattern, for example, $Nat^{*}= 1\; 2\; 3$, or $Nat^{*} = 1\; 1$.
For example, instructions, represented by the metavariable $e$, are usually grouped into sequences $e^{*}$, which are possibly empty $\epsilon$.
$e_1$ and $e_2$ refer to different instructions (although the actual instruction they represent may be the same).
Similarly, $e_1^{*}$ and $e_2^{*}$ refer to different sequences of instructions (again, the sequences may be the same).

There is no requirement that a sequence of non-terminals, $e_1^{*}$, be made up of entirely the same pattern, unless it is explicitly written out as in $(t.const c)^{*}$.
For example, $e_1^{*}$ matches $(t.\<const> c_1)\; (t.\<const> c_2)\; (t.binop)$.
Further, we may separate out subsequences: from $(t.\<const> c)^{*}$ we may separate out $t^{*}$ and $c^{*}$ to refer to the sequences of types and constant values respectively.

We can use different annotations in place of the Kleene star to add additional information.
The Kleene star may be replaced with an exact value $n$ when we know that the sequence has length $n$ (\eg a sequence of 3 types be phrased as $t^{3}$).
We can also use a question mark to represent either an empty sequence ($\epsilon$), or a sequence with exactly one item (\eg, $v^{?}=v' \lor v^{?}=\epsilon$).

With this notation in mind, we can now look over the \wasm instructions in \autoref{fig:wasminstructions}.
As a convention, syntax written in a \tbbf{blue, bold font} denotes a keyword, while text written in $italics$ represents a metavariable.
Throughout the \wasm syntax there are many metavariables used to represent natural numbers: $n$ and $m$ are usually used for the table and memory sizes, $i$ and $j$ are often used as indexes (\eg to reference a local variable), $o$ and $align$ are used within memory operations (we replace $a$ with $align$ for clarity and since we use $a$ elsewhere), and lastly $c$ is used as a constant variable (which could also be a float).
Confusingly enough, $iN$ is used to annotate operations that support integers, and $fN$ is used to annotate operations that support floats.

Some instructions, such as $\<loop> tf\; e^{*} \<end>$ include a sequence of instructions $e^{*}$.
We refer to such instructions as block instructions, since they define control flow blocks for the instructions inside (not to be confused with the $\<block>$ insturction, which is a block instruction).
Similarly, we refer to $e^{*}$ as the body.

\begin{figure}
    \begin{math}
    \begin{array}{rcl}
        unop_{iN} & ::= & \<clz> \mid \<ctz> \mid \<popcnt> \\
        testop_{iN} & ::= & \<eqz> \\
        binop_{iN} & ::= & \<add> \mid \<sub> \mid \<shl> \mid \<or> \mid ... \\
        relop_{iN} & ::= & \<eq> \mid \<ne> \mid \<gt> \mid \<ge> \mid ... \\
        cvtop & ::= & \<convert> \mid \<reinterpret> \\
    \end{array}
    \end{math}

    \begin{math}
    \begin{array}{rcl}
        e & ::= & \<unreachable> \mid \<nop> \mid \<drop> \mid \<select> \mid \\
        && \<block>\; tf\; e^{*} \<end> \mid \<loop>\; tf\; e^{*} \<end> \mid \<if>\; tf\; e^{*} \<else> e^{*} \<end> \mid \\
        && \<br> i \mid \<brif> i \mid \<brtable> i^{+} \mid \<return> \mid \<call> i \mid \<callindirect> tf \mid \\
        && \<getlocal> i \mid \<setlocal> i \mid \<teelocal> i \mid \<getglobal> i \mid \\
        && \<setglobal> i \mid t.\<load> (tp\_sx)^{?}\; align\; o \mid t.\<store> tp^{?}\; align\; o \mid \\
        && \<currentmemory> \mid \<growmemory> \mid t.\<const> c \mid \\
        && t.unop_t \mid t.binop_t \mid t.testop_t \mid t.relop_t \mid t.cvtop\; t\_sx^{?} \\
    \end{array}
    \end{math}
    \caption{\wasm Instructions}
    \label{fig:wasminstructions}
\end{figure}

\wasm has modules that include functions ($f$), global variables ($glob$), an optional function table ($tab$), and an optional linear memory chunk ($mem$), as seen in \autoref{fig:wasmmodules}.
Functions, globals, the table, and memory can be imported, using $\<import> "name_1"\; "name_2"$, which imports $name_2$ from the file $name_1$.
Similarly, they can also be exported under any number of names using $\<export> "name"$.

Functions include a list of local variable declarations to use within the body (a sequence of instructions).
Additionally, function arguments are accessible as local variables within the body of functions.
Global variables may be mutable (although, exported global variables cannot be mutable, as we will see later), and are initialized via a sequence of instructions.
Function tables store references to functions that can be called using indirect function calls, they are used to more safely represent function pointers.
Indirect function calls must supply a function type that gets checked against the table function that ends up being called at run-time.
Linear memory is simply a very large continuous chunk of memory.
Memory load and store operations require run-time bounds checks to ensure that they operate within the chunk of memory.

\begin{figure}
    \begin{math}
    \begin{array}{rcl}
        im &::=& \<import> "name_1"\; "name_2" \\
        ex &::=& \<export> "name" \\
        f &::=& ex^{*}\; \<func> tf\; \<local>\; t^{*}\; e^{*} \mid ex^{*}\; \<func> tf\; im \\
        glob &::=& ex^{*}\; \<global> tg\; e^{*} \mid ex^{*}\; \<global> tg\; im \\
        tab &::=& ex^{*}\; \<table> n\; i^{*} \mid ex^{*}\; \<table> n\; im \\
        mem &::=& ex^{*}\; \<memory> n\; \mid ex^{*}\; \<memory> n\; im \\
        module &::=& \<module> f^{*}\; glob^{*}\; tab^{?}\; mem^{?}
    \end{array}
    \end{math}
    \caption{\wasm Module Definitions}
    \label{fig:wasmmodules}
\end{figure}

\subsection{\wasm Dynamic Semantics}
\label{subsec:wasmsemantics}
\wasm is a stack-based assembly language that uses reduction semantics.
Before we introduce the \wasm semantics, we first must introduce some administrative instructions that are used in the reduction relation.
\autoref{fig:wasmadmin} shows the new administrative instructions.

\begin{figure}
    \begin{math}
    \begin{array}{rcl}
        cl &::=& \{\text{inst } i, \text{func } f\} \\
        b &::=& 0x00, 0x01, ..., 0xff \\
        tabinst &::=& cl^{*} \\
        meminst &::=& b^{*} \\
        inst &::=& \{\text{func } cl^{*}, \text{glob } v^{*}, \text{tab } i^{?},\text{mem } i^{?}\} \\
        s &::=& \{\text{inst } inst^{*}, \text{tab } tabinst^{*}, \text{mem } meminst^{*}\} \\
        v &::=& t.\<const> c \\
        e &::=& ... \mid \<trap> \mid \<call> cl \mid \<label>_n\{ e^{*}\}\; e^{*}\<end> \mid \\
        && \<local>_n\{ i;v^{*}\}\; e^{*}\<end>\\
        L^0 &::=& v^{*}\; \square \; e^{*} \\
        L^{k+1} &::=& v^{*}\; \<label>_n\{ e^{*}\}\; L^{k}\<end> \; e^{*} \\
    \end{array}
    \end{math}
    \caption{\wasm Administrative instructions}
    \label{fig:wasmadmin}
\end{figure}

$cl$ represents a \wasm closures.
Closures include the module instance that the function is defined in, as well as the function definition (which cannot be an import) with any exports erased.
Intuitively, a closure represents a function closed under linking.
$b$ represents a byte.
The runtime store, $s$, includes runtime instances for every module ($inst^{*}$), as well as all of the tables ($tabinst^{*}$), and memory chunks ($meminst^{*}$).
In other words, $s$ includes full definitions for every module.
Module runtime instances, $inst$, refer to their table/memory (if they have one), by indexing into the list of runtime instances of tables/memory chunks.

A representation of values, $v$, is added as a metavariable to refer to constant instructions $t.\<const> c$.
A \emph{trap} ($\<trap>$) is the \wasm term for a run time error.
$\<call> cl$ is a function call on a closure.
As we will see, it is an intermediate step for performing both direct and indirect function calls.

We also have two types of block instructions introduced.
The first, label blocks, are used in handling control flow.
Specifically, they are used to hand branching.
All block instructions (\<block>, \<loop>, and \<if>) reduce to label blocks.
Local blocks can store instructions (inside the curly braces), and the annotation $n$ is equal to the expected number of inputs to those instructions.
This is explained more when we explain how branching works.

The second new block instruction is the local block.
Locals blocks are how closure calls get expanded.
They introduce an environment consisting of the module instance and local variables inside which their body is reduced \todo{Executed? Reduced? Can we use them interchangeably? I think this comes up in other places as well, something to watch out for}.

Finally, we introduce reduction contexts, $L^{k}$, where $k$ is the nesting depth.
Reduction contexts are based on label blocks, so $L^{k}$ essentially contains $k$ nested local blocks.

There are a few final notational digressions we must make before describing the reduction relation.
First, objects such as $s=\{\text{inst } inst^{*}, \text{tab } tabinst^{*}, \text{mem } meminst^{*}\}$ can be deferenced using their keywords (\eg inst).
For example, $s_\text{inst}=inst^{*}$ given the above definition of $s$.
Second, we can index into a sequence to get a specific element (\eg $inst^{*}(i)$ returns the $i$th $inst$ in $inst^{*}$).
Lastly, \wasm uses several shorthands to get information out of module instances in $s$: $s_\text{func}(i)(j)=s_\text{inst}(i)_\text{func}(j)$.
Essentially, this allows us to implicity deference the $i$th module instance to get the $j$th function inside of the instance.
This shorthand is used similarly for glob, tab, and mem.

\paragraph{The \wasm Reduction Relation} works on \emph{configurations}, that include the store $s$, local variables (represented as a sequence of values $v^{*}$), and the instruction sequence $e^{*}$.
The store, local variables, and module index are omitted when not used.
We present all the different flavors of reduction rules below.

\begin{mathpar}
    \boxed{s;v^{*};e^{*} \rightarrow s';v'^{*};e'^{*}}
\end{mathpar}

Some rules also include the current module index $i$, which is used when dereferencing the store to know which module instance to look at.
Instructions are reduced in place by decomposing the program using reduction contexts.
Intuitively, we pull out the next instruction to execute, reduce it, and plug the result back in.
The ``stack'' is just the sequence of values preceding the first reducible instruction.
When an instruction reduces to a value, that value becomes part of the stack and the next instruction is reduced.
This way of decomposing ensures that all of the instructions preceding an instruction getting reduced are values.

Binary and relation operations consume two values and either return the specified operation applied to those values, or trap if it is not defined (in the case of dividing by zero).
Test operators only consume one value, and do not trap, but are otherwise similar.
\<unreachable> causes a trap, \<nop> reduces to the empty sequence, and \<drop> consumes one value and reduces to the empty sequence.
We have shown the true case of select, which returns the second value consumes ($k+1$ is a common shorthand for a non-zero value).

\begin{mathpar}
    \begin{array}{rcl}
        (t.\<const> c_1)\; (t.\<const> c_2)\; t.binop &\hookrightarrow& t.\<const> c \\
        \text{if } c=binop(c_1,c_2) && \\ %% binop

        (t.\<const> c_1)\; (t.\<const> c_2)\; t.binop &\hookrightarrow& \<trap> \\ %% binop to trap
        otherwise && \\

        (t.\<const> c)\; t.testop &\hookrightarrow& \<ithreetwo>.testop_t(c) \\

        \<unreachable> &\hookrightarrow& \<trap> \\

        \<nop> &\hookrightarrow& \epsilon \\

        v\;\<drop> &\hookrightarrow& \epsilon \\

        v_1\;v_2\;(\<ithreetwo>.\<const> k+1)\;\<select> &\hookrightarrow& v_2 \\

        v_1\;v_2\;(\<ithreetwo>.\<const> k+1)\;\<select> &\hookrightarrow& v_1 \\
    \end{array}
\end{mathpar}

Blocks and loops reduce to labels.
Stored instructions are only added by reduced a \<loop>, in which case it stores the loop code so it can run the loop again.
We show the true case of an $\<if>$ block, which reduces to the first body inside of a block, the false case does the same but with the second body.
If the body of a label block is a \<trap> or a sequence of values then the \<trap>/values replace the block.
Since decomposition happens on label blocks, we have included the inductive reduction rule, which intuitively pulls instructions out of the context, reduces them outside the context, and then plugs them right back in.

\begin{mathpar}
    \inferrule[]{
        s;v^{*};e^{*} \hookrightarrow s';v'^{*};e'^{*}
    } {
        s;v^{*};L^k[e^{*}] \hookrightarrow s';v'^{*};L^k[e'^{*}]
    } \\

    \begin{array}{rcl}
        L^{0}[\<trap>] &\hookrightarrow& \<trap> \\

        v^n\;\<block>\; (t_1^{n}\rightarrow t_2^{m})\; e^{*} \<end> &\hookrightarrow& \<label>_m \{\} v^n\;e^{*} \<end> \\

        v^n\;\<loop>\; (t_1^{n}\rightarrow t_2^{m})\; e^{*} \<end> &
        \hookrightarrow&
        {\begin{stackTL}
            \<label>_n
            {\begin{stackTL}
                \{ \<loop>\; (t_1^{n}\rightarrow t_2^{m})\; e^{*}
                \\ \<end> \}
                \\ v^n\;e^{*}
            \end{stackTL}} \\
            \<end>
        \end{stackTL}} \\

        (\<ithreetwo>.\<const> 0)\; \<if>\; tf\; e_1^{*} \<else> e_2^{*} \<end> &\hookrightarrow& \<block>\; tf\; e_1 \<end> \\

        (\<ithreetwo>.\<const> k+1)\; \<if>\; tf\; e_1^{*} \<else> e_2^{*} \<end> &\hookrightarrow& \<block>\; tf\; e_1 \<end> \\

        \<label>_n\; \{ e_0^{*} \}\; v^{*} \<end> &\hookrightarrow& v^{*} \\

        \<label>_n\; \{ e_0^{*} \}\; \<trap> \<end> &\hookrightarrow& \<trap> \\
    \end{array}
\end{mathpar}

$\<brif>$ consumes a value and reduces to a branch if the value is non-zero, otherwise it reduces to the empty sequence.
$\<brtable>$ has a list of one or more numbers that it may branch to.
It consumes a $\<ithreetwo>\; k$ and branches based on the $k$th number or the last one if there is no $k$th number.
Branching from inside a label block essentially peels back the specified number of label blocks, and continues executing with the stored instructions $e_0^{*}$ of the containing label block.
\autoref{fig:branching} has several examples of branching in action.

\begin{mathpar}
    \begin{array}{rcl}
        \<label>_n\; \{ e_0^{*} \}\; L^j[v^{n}\; \<br> j] \<end> &\hookrightarrow& v^n\; e_0^{*} \\

        (\<ithreetwo>.\<const> 0)\; \<brif> j &\hookrightarrow& \epsilon \\

        (\<ithreetwo>.\<const> k+1)\; \<brif> j &\hookrightarrow& \<br> j \\

        (\<ithreetwo>.\<const> k)\; \<brtable> j_1^{k}\;j\;j_2^{*} &\hookrightarrow& \<br> j \\

        (\<ithreetwo>.\<const> k+n)\; \<brtable> j_1^{k} j &\hookrightarrow& \<br> j \\
    \end{array}
\end{mathpar}

\begin{figure}
\begin{math}
\begin{array}{l}
    {\begin{stackTL}
        \<label>_0
        {\begin{stackTL}
            \{\} \\
            \<label>_0 {\begin{stackTL}
                \{\} \\
                \<label>_0 \{\}\; \<br> 1 \<end>
            \end{stackTL}} \\
            \<nsend>
        \end{stackTL}} \\
        \<nsend>
    \end{stackTL}} \\
    \hookrightarrow \<label>_0 \{\}\; \<end> \\\\
    \<label>_0 \{ \<loop> \dots \<end> \}\; \<br> 0 \<end> \\
    \hookrightarrow \<loop> \dots \<end> \\\\
    \<label>_0
    \begin{stackTL}
        \{ \}\; \\
        \<label>_0 \{ \<loop> \dots \<end> \}\; \<br> 1 \<end>
    \end{stackTL} \\
    \<nsend> \\
    \hookrightarrow \epsilon
\end{array}
\end{math}
\caption{Branching Examples}
\label{fig:branching}
\end{figure}

Direct and indirect function calls are expanded in two steps.
First, the associated closure is fetched either from the current module instance (for direct calls) or from the table (for indirect calls, which traps if the type of the fetched closure doesn't match the expected type).
This step reduces a direct or indirect call to a $\<call> cl$.
Then, the closure body is placed into a local block with the arguments from the stack and locals declared by the function, which are zero-initialized, being used as the local variables.

\begin{mathpar}
    \begin{array}{rcl}
        s;\<call> j &\hookrightarrow_i& s_\text{func}(i,j) \\

        s;\<callindirect> j &\hookrightarrow_i& s_\text{tab}(i,j) \\
        \text{if } s_\text{tab}(i,j)_\text{code}=(\<func> tf\; \<local>\; t^{*}\; e^{*}) && \\

        s;\<callindirect> j &\hookrightarrow_i& \<trap> \\
        \text{otherwise} && \\

        v^{n}\;(\<call> cl) &\hookrightarrow_i&
        {\begin{stackTL}
            \<local>_m
            {\begin{stackTL}
                \{
                    {\begin{stackTL}
                        cl_\text{inst}; \\
                        v^{n}\;(t.\<const> 0)^{k}\}
                    \end{stackTL}} \\
                \<block>\;(\epsilon \rightarrow t_2^{m})\; e^{*} \\
                \<nsend>
            \end{stackTL}} \\
            \<nsend> \\
        \end{stackTL}} \\
    \end{array}
\end{mathpar}

The local block has the same module index, $i$, as the closure, so the body of the local block is reduced within the module that the closure is defined it, so it uses the global variables, table, and memory of the module instance referenced by $i$.
This is handled by the inductive reduction rule (which has much more of a small-step flavor).
Inside the local block is a label block, so at the top level of a function $\<br> 0$ is essentially equivalent to $\<return>$, except with an additional reduction step (in general, $\<return>$ can be thought of as $\<br> k$, where $k$ is the context depth).
\todo{Was that note confusing? Was that sentence confusing?}
Returning, somewhat similarly to branching, simply replaces the local block with the arguments to the return instruction, except that it skips over any label blocks.
If the body of a local block is a \<trap> or sequence of values, then that is what the local block reduces to, similar to branching.

\begin{mathpar}
    \inferrule[]{
            s;v^{*};e^{*} \hookrightarrow_i s';v'^{*};e'^{*}
        } {
            s;v_0^{*};\<local>_n \{ i;v^{*} \} e^{*} \<end> \hookrightarrow_j s';v_0^{*};\<local>_n \{i;v'^{*}\} e'^{*} \<end>
        } \\

    \begin{array}{rcl}
        \<local>_n \{ i;v_l^{*} \} v^{n} &\hookrightarrow& v^{n} \\
        \<local>_n \{ i;v_l^{*} \} \<trap> &\hookrightarrow& \<trap> \\
        \<local>_n \{ i;v_l^{*} \} L^{k}[v^n \<return>] &\hookrightarrow& v^{n} \\
    \end{array}
\end{mathpar}

Local variables are just a list of values at run time!
They are get/set by indexing into them, similar to, well, everything else in \wasm.
The same is true of global variables, except there is an extra step since they are stored in the current module instance inside the store $s$.

\begin{mathpar}
    \begin{array}{rcl}
        v_1^{j}\;v\;v_2^{};\<getlocal> j &\hookrightarrow& v \\

        v_1^{j}\;v\;v_2^{};v'\; (\<setlocal> j) &\hookrightarrow& v_1^{j}\;v'\;v_2^{};\epsilon \\

        v_1^{j}\;v\;v_2^{};v'\; (\<teelocal> j) &\hookrightarrow& v_1^{j}\;v'\;v_2^{};v' \\

        s;\<getglobal> j &\hookrightarrow_i& s_\text{glob}(i,j) \\

        s\; (\<setglobal> j) &\hookrightarrow_i& s';\epsilon \\

        \text{where} s' = s \text{ with } glob(i,j)=v' \\
    \end{array}
\end{mathpar}

Finally, there are the memory instructions.
One can load or store a value from or to memory, get the current memory size, or try to grow the memory.
There is a lot of minutia detail, but none of it is particularly important.
The key high level takeaway is that load and store will trap if the supplied index plus the static offset is out of bounds.
$|t|$ is used to represent the size of the type (\eg $|t| = 8$ bytes).
We omit two rules, one each for store and load, that include the ability to use packed types to load/store smaller values and to load signed/unsigned.

\begin{mathpar}
    \begin{array}{rcl}
        s;(\<ithreetwo>.\<const> k) &&\\
        (t.\<load> tp\_sx\; align\; o) &\hookrightarrow_i& s;(t.\<const> const_t(b^{*})) \\
        \text{if } s_\text{mem}(i,k+o,|t|)=b^{*} && \\

        &&\\

        s;(\<ithreetwo>.\<const> k) &&\\
        (t.\<load> tp\_sx\; align\; o) &\hookrightarrow_i& \<trap> \\
        \text{otherwise} && \\

        &&\\

        s;(\<ithreetwo>.\<const> k)\; (t.\<const> c) && \\
        (t.\<store> tp\_sx\; align\; o) &\hookrightarrow_i& s';\epsilon \\
        \text{if } s'=s &\text{ with }& \text{mem}(i,k+o,|t|)=bits_t(c) \\

        &&\\

        s;(\<ithreetwo>.\<const> k)\; (t.\<const> c) && \\
        (t.\<store> tp\_sx\; align\; o) &\hookrightarrow_i& \<trap> \\
        \text{otherwise} && \\

    \end{array}
\end{mathpar}

\begin{mathpar}
    \begin{array}{rcl}
        s;\<currentmemory> &\hookrightarrow_i& \<ithreetwo>.\<const> |s_\text{mem}(i,*) | / 64\text{Ki} \\

        &&\\

        s;(\<ithreetwo>.\<const> k)&&\\
        \<growmemory> &\hookrightarrow_i& s';\<ithreetwo>.\<const> | s'_\text{mem}(i,*) | / 64\text{Ki} \\
        \text{if } s'=s &\text{ with }& \text{mem}(i,*)=s_\text{mem}(i,*)(0)^{k*64\text{Ki}} \\

        &&\\

        s;(\<ithreetwo>.\<const> k)&&\\
        \<growmemory> &\hookrightarrow_i& \<ithreetwo>.\<const> (-1) \\
        \text{otherwise} && \\
    \end{array}
\end{mathpar}

\subsection{The \wasm Type System}
\label{subsec:wasmtyping}
Instructions in \wasm are typed under a module type context $C$.
$C$ keeps track of various module-level types: functions, globals, the table, memory, locals, the label stack (\ie the expected types for branching instructions), and the return stack (\ie the expected type of the return instruction).

$$ C::= \{ {\begin{stackTL}
    \text{func } tf^{*}, \text{ global } tg^{*}, \text{ table } n^{?}, \text{ memory } m^{?},
    \\ \text{local } t^{*}, \text{ label } (t^{*})^{*}, \text{ return } (t^{*})^{?} \}
\end{stackTL}} $$

\wasm is a stack-based language, so the type of an instruction in \wasm consists of a precondition and postcondition on the shape of the stack.
This can be viewed as though instructions \emph{consume} certain values from the stack and then \emph{produce} values to be pushed on the stack.
For example, a binary operation of some type $t$ consumes two values of the given type $t$ on the stack and produces a value of type $t$:

\[
    \inferrule{ }{C \vdash t.binop : t\; t \rightarrow t}
\]

The above example shows what a typical \wasm typing rule looks like.
The type associated with the instruction $t.binop$ is a \wasm function type, which is just the precondition (on the left of the $\rightarrow$) and postcondition (on the right of the $\rightarrow$) on the stack.
Most typing rules are for a single instruction and there are a few rules which can combine rules.
Thus, the main \wasm typing judgement is as follows:

$$\boxed{C \vdash e^{*} : tf}$$

We reproduce and explain a few selected typing rules from \wasm.
The rule for typing a block typechecks the body $e^{*}$ under the module type context with the postcondition $t_2^{m}$ appended to the label stack.
This is yet another common notational atrocity where $x,y$ means $x$ extended with $y$.
The branch rule accepts any precondition, extended with the $i$th postcondition on the label stack (counting backwards), and returns to any postcondition.
A branch will return the $n$ values before it, so it is ok if there are more values on the stack, as they will be discarded.
Execution does not proceed after branching, so the postcondition can be anything.
For unction calls we lookup the type of the function in the context.
Finally, setting a local variable consumes a value of the type that is given by looking up the type of the local in the context.

\begin{mathpar}
    \inferrule{ }{C \vdash t.binop : t\; t \rightarrow t}

    \inferrule{
        tf = t_1^{n} \rightarrow t_2^{m} \and
        C,\text{label}(t_2^{m})\vdash e^{*} : tf
    }{
        C \vdash \<block>\; tf\; e^{*} \<end> : tf
    }

    \inferrule{
        C,\text{label}(i) = t^{n}
    }{
        C \vdash \<br> i \<end> : t_1^{*}\;t^{n} \rightarrow t_2^{*}
    }

    \inferrule{
        C_\text{func}(i) = tf
    }{
        C \vdash \<call> i \<end> : tf
    }

    \inferrule{
        C_\text{local}(i) = t
    }{
        C \vdash \<setlocal> i : t \rightarrow \epsilon
    }
\end{mathpar}

The empty instruction sequence has an empty precondition and postcondition.
An instruction $e_2$ can be appended to a sequence of instructions $e_1^{*}$ if the precondition of $e_2$ is the same as the postcondition of $e_1^{*}$.
Then, the precondition of the full sequence $e_1^{*}\;e_2$ is the precondition of $e_1^{*}$ and the postcondition of $e_1^{*}\;e_2$ is the postcondition of $e_2$.

\begin{mathpar}
    \inferrule{ }{C \vdash \epsilon : \epsilon \rightarrow \epsilon}

    \inferrule{
        C \vdash e_1^{*} : t_1^{*} \rightarrow t_2^{*}
        C \vdash e_2 : t_2^{m} \rightarrow t_3^{*}
    }{
        C \vdash e_1^{*}\;e_2 : t_1^{*} \rightarrow t_3^{*}
    }
\end{mathpar}

\paragraph{Stack Polymorphism.}
To compose together the types of many instructions, it is necessary to carry around extra type information about the rest of the stack while type-checking instructions.
\emph{Stack polymorphism} allows extending the precondition and postcondition with the same data to thread unmodified parts of the stack through a list of instructions.
Intuitively, this allows you to ``forget'' the rest of the stack and focus only on the part being manipulated by the instruction being checked, after which point the ``forgotten'' part can be re-added.

For example, if the stack has the shape $\<isixfour>\; \<ithreetwo>\; \<ithreetwo>$, then stack polymorphism allows us to ignore $\<isixfour>$ and typecheck $\<ithreetwo>.binop$ with $\<ithreetwo>\;\<ithreetwo>$.
Then the stack would look like $\<ithreetwo>$, at which point we add $\<isixfour>$ back to the postcondtion to get $\<isixfour>\; \<ithreetwo>$ after executing $\<ithreetwo>.binop$.

\chapter{\name}
\label{chp:prechk}
The goal of \name is to eliminate unnecessary dynamic checks.
To accomplish this, it must (1) have instructions that do not require dynamic checks and (2) statically prove that the assumptions of those instructions are met.
\name extends \wasm with new instructions that explicitly do not require dynamic checks, and an indexed type system to reason about the safety of omitting checks.
Intuitively, we are replacing dynamic checks with static checks whenever possible.

\begin{figure}[b]
    $${\begin{stackTL}
        \<block>
        {\begin{stackTL}
            \;(\<ithreetwo>\;\<ithreetwo> \rightarrow \<ithreetwo>)
            \\ (\<teelocal> 0)
            \\ (\<ithreetwo>.\<const> 1)
            \\ (\<getlocal> 0)
            \\ (\<select>)
            \\ (\<ithreetwo>.\<div>)
        \end{stackTL}}
        \\ \<nsend>
    \end{stackTL}}$$

    \caption{An Example of an Unnecessary \wasm check}
    \label{fig:unnecessarycheck}
\end{figure}

Consider the example of a \wasm program with an unnecessary dynamic check in \autoref{fig:unnecessarycheck}.
The program consists of a $\<block>$ instruction that consumes two arguments from the stack and produces onto the stack their quotient if the second argument is non-zero, and the first argument otherwise.
In \wasm, this program would have a dynamic check for division-by-zero inserted for the division instruction ($\<ithreetwo>.\<div>$).
However, this check would be unnecessary since the instructions preceding the division instruction ensures that the second argument is non-zero.
This is guaranteed because the $\<select>$ instruction selects the first value (the local) if the local is non-zero, and otherwise selects the second value ($\<ithreetwo>.\<const> 1$).

In the above example, it is possible to check the necessary precondition of the division instruction (that the second argument is non-zero) statically, rather than dynamically.
\name performs such static checks using an indexed type system.
An indexed type language uses an index language in the type system to encode information within types.
\name's index language must be capable of capturing enough information about a \name program to statically verify the preconditions of \prechk-tagged instructions.

The \name index language is designed to encode linear constraints on program values (the details of they are encoded is discussed in \autoref{subsec:indexlang}).
To do this, we ``shadow'' \name program values using what we call \emph{index variables} to track constraints on and relationships between program values.
Many of these constraints/relationships are written using \wasm operators (\eg $binop$), since they are the predominant way that \wasm values end up being related to each other.
Index variables are associated with program values using \emph{indexed types}, which combine the type information from \wasm: the primitive type $t$ of the value, with the indexed type variable that represents the value in the index language.
Finally, we also use index variables to track local variables (specifically the current value of local variables, since they are mutable and may change) via an \emph{index local store}.

\section{\name Syntax}
The syntax of \name has the same structure as \wasm, but different instructions and richer types.
First, \name introduces four additional instructions, which are referred to as ``\prechk-tagged'' instructions.
Second, \name does not support floating point values or unary operators on integers since they are difficult to reason about (this is explained in more detail in \autoref{chp:implementation}).
While it would be possible to support them, we would have no more type information about them than \wasm, and the focus of this work is on the type information.
\name uses a different representation of types within instructions and functions, as we see in \autoref{subsec:indexlang}.

Recall from \autoref{sec:wasmsemantics} that four \wasm instructions require run-time checks: integer division, indirect function calls, and memory loads and stores.
``\prechk-tagged'' instructions refer to four \name instructions, listed in \autoref{fig:newinstructions}, that are counterparts to these four \wasm instructions.
Intuitively, we add a tag to the instruction to show that it doesn't require run-time checks.
Formally, however, different instructions have different semantics and typing rules, as explained below.

\begin{figure}
    \begin{math}
    \begin{array}{rcl}
        testop_{iN} & ::= & \<eqz> \\
        binop_{iN} & ::= & \<add> \mid \<sub> \mid \<shl> \mid \<or> \mid ... \\
        relop_{iN} & ::= & \<eq> \mid \<ne> \mid \<gt> \mid \<ge> \mid ... \\
    \end{array}
    \end{math}

    \begin{math}
    \begin{array}{rcl}
        e & ::= & \<unreachable> \mid \<nop> \mid \<drop> \mid \<select> \mid \\
        && \<block>\; \hl{\tfi}\; e^{*} \<end> \mid \<loop>\; \hl{\tfi}\; e^{*} \<end> \mid \<if>\; \hl{\tfi}\; e^{*} \<else> e^{*} \<end> \mid \\
        && \<br> i \mid \<brif> i \mid \<brtable> i^{+} \mid \<return> \mid \<call> i \mid \<callindirect> \hl{\tfi} \mid \\
        && \<getlocal> i \mid \<setlocal> i \mid \<teelocal> i \mid \<getglobal> i \mid \\
        && \<setglobal> i \mid t.\<load> (tp\_sx)^{?}\; align\; o \mid t.\<store> tp^{?}\; align\; o \mid \\
        && \<currentmemory> \mid \<growmemory> \mid t.\<const> c \mid \\
        && t.binop_t \mid t.testop_t \mid t.relop_t \mid \\
        \hline
        && \hl{t.\<divpc>} \mid \hl{t.\<callindirectpc>} \mid \\
        && \hl{t.\<loadpc> (tp\_sx)^{?}\; align\;o} \mid \hl{t.\<storepc> tp^{?\;} align\;o} \\
    \end{array}
    \end{math}
    \caption{\name syntax including the four \prechk-tagged instructions}
    \label{fig:newinstructions}
\end{figure}

\subsection{The \name Index Language}
\label{subsec:indexlang}
\name uses an indexed type system.
We use the \name index language to encode constraints on program values within types.
\autoref{fig:itsyntax} shows the syntax for the index type language.
Remember, syntax written in a \tbsf{blue sans serif font} denotes a \wasm keyword.
Below is a quick overview of each metavariable.

\begin{figure}
    \begin{math}
        \begin{array}{rcl}
            t &::= & \<ithreetwo> \mid \<isixfour> \\
            a &::= & IndexVariable \\
            x\;y &::=& a \mid \ti{t}{c} \mid (\<binop>\;x\;y) \mid (\<testop>\;x) \mid (\<relop>\;x\;y) \\
            P &::=& (=\; x \; y) \mid (\text{if}\; P\; P\; P) \mid \neg P \mid P \land P \mid P \lor P \\
            \phi &::=& \circ \mid \phi, \ti{t}{a} \mid \phi, P \\
        \end{array}
    \end{math}
    \caption{Syntax of the \name index type language}
    \label{fig:itsyntax}
\end{figure}

\begin{itemize}
    \item $t$ represents a primitive \wasm type.
    We do not reason about floating point values, so it is either a 32-bit integer (\<ithreetwo>) or a 64-bit integer (\<isixfour>).
    \item $a$ is a type index variable, which is used to track constraints on program values.
    \item $x$ and $y$ are type indices; they can be an index type variable, a constant with an explicit type, or a \wasm operation on a type index.
    \item $P$ is a proposition about type indices which can encode equality constraints on type indices, or combine propositions using common first-order logic operators.
    \item $\phi$ is the type index context which stores index type variable declarations and propositions.
    Essentially, it contains all of the knowledge we have about all of the index variables.
\end{itemize}

\begin{figure}[t]
    \begin{math}
        \begin{array}{rcl}
            ti &::=& \ti{t}{a} \\
            l &::=& ti^{*} \\
            \tfi &::=& ti^{*};l;\phi \rightarrow ti^{*};l;\phi \\
            C &::=& \{
                {\begin{stackTL}
                    \text{func } \; \tfi^{*}, \text{ global } \; (\text{mut}^{?} \; t)^{*}, \text{ table} \; n^{?}, \text{ memory }  m^{?}, \\
                    \text{ local } \; t^{*}, \text{ label}(ti^{*};l;\phi)^{*}, \text{ return}\;(ti^{*};l;\phi)^{? }\}
                \end{stackTL}}
        \end{array}
    \end{math}
    \caption{\name indexed function types}
    \label{fig:tfisyntax}
\end{figure}

Indexed types are used to associate index variables $a$ with values in the program.
\autoref{fig:tfisyntax} shows the form of an indexed type, $ti$, which includes both the type $t$ and an index variable $a$.
In \name, we represent the shape of the stack as a sequence of indexed types $ti^{*}$.

The index local store associates index variables with local variables.
It has an identical form to the stack: a sequence of indexed types to associate index variables with local variables.
We use the shorthand $l$ to refer to the index local store.
The index type context $\phi$ is used to reason about the possible values of computations.
It stores constraints on and between program values tracked by indexed types representing the stack and index local store.

\name uses indexed ``function'' types $\tfi$, which, similar to \wasm's function types, are just a precondition and postcondition.
However, indexed function types include much more information in their precondition and postcondition!
They represent the stack using a sequence of indexed types and track local variables using the index local store, and include $\phi$ which contains constraints about those values.
We see how this information is used in \autoref{subsec:checkelim}.

We retain $C$ to refer to the module type context in \name, although the representation of module types is slightly different.
\wasm function types are replaced with \name indexed function types.
Further, the postconditions in the label stack and return stack are replaced with \name indexed postconditions including indexed types, the local index store, and the index type context.

We can now introduce the \name typing judgement for instructions.
It is similar to the \wasm typing judgment, but uses indexed function types which include much more information by tracking constraints about program values.

\begin{mathpar}
    \boxed{C \vdash e^{*} : \tfi}
\end{mathpar}

Recall that certain \wasm instructions (such as $\<block>$ and $\<callindirect>$) include \wasm function types to declare the expected types of their bodies.
In \name, we replace those function types with indexed function types.

\section{\name Dynamic Semantics}
\name uses the same reduction relation with the same structure as \wasm (explained in detail in \autoref{sec:wasmsemantics}).
All the reduction rules for all of the \name instructions are the same as they are for \wasm, as presented in \autoref{sec:wasmsemantics}, except that indexed function types are used instead of \wasm function types.
We also have four new instructions, for which we introduce new reduction rules.

\label{sec:newinstructions}
\begin{figure}[t]
    \begin{mathpar}
        \boxed{s;v^{*};e^{*} \rightarrow s';v'^{*};e'^{*}}
    \end{mathpar}

    \begin{math}
        \arraycolsep=1.4pt
        \begin{array}{rcl}
            (t.\<const> c_1)\; (t.\<const> c_2)\; t.binop &\hookrightarrow& t.\<const> c \\
            && \text{if } c=binop(c_1,c_2) \\ %% binop

            (t.\<const> c_1)\; (t.\<const> c_2)\; t.binop &\hookrightarrow& \<trap> \\ %% binop to trap
            && otherwise \\

            s;\<callindirect> j &\hookrightarrow_i& s_\text{tab}(i,j) \\
            && \text{if } s_\text{tab}(i,j)_\text{code}=(\<func> tf\; \<local>\; t^{*}\; e^{*}) \\

            s;\<callindirect> j &\hookrightarrow_i& \<trap> \\
            && \text{otherwise} \\

            s;(\<ithreetwo>.\<const> k) &&\\
            (t.\<load> tp\_sx\; align\; o) &\hookrightarrow_i& s;(t.\<const> const_t(b^{*})) \\
            && \text{if } s_\text{mem}(i,k+o,|t|)=b^{*} \\

            s;(\<ithreetwo>.\<const> k) &&\\
            (t.\<load> tp\_sx\; align\; o) &\hookrightarrow_i& \<trap> \\
            && \text{otherwise} \\

            s;(\<ithreetwo>.\<const> k)\; (t.\<const> c) && \\
            (t.\<store> tp\_sx\; align\; o) &\hookrightarrow_i& s';\epsilon \\
            && \text{if }
            {\begin{stackTL}
                s'=s \text{ with}
                \\ \text{mem}(i,k+o,|t|)=bits_t(c)
            \end{stackTL}} \\

            s;(\<ithreetwo>.\<const> k)\; (t.\<const> c) && \\
            (t.\<store> tp\_sx\; align\; o) &\hookrightarrow_i& \<trap> \\
            && \text{otherwise} \\
        \end{array}
    \end{math}
    \caption{\wasm instructions that have preconditions for reduction}
    \label{fig:checked}
\end{figure}

\emph{The formal reason why certain \wasm instructions require run-time checks is because they have preconditions as part of their semantics.}
If the preconditions are not met then those instructions trap to avoid undefined behavior (we've reproduced the reduction rules for those instructions in \autoref{fig:checked}).
The \wasm type system is not expressive enough to ensure these preconditions statically, so they instead must be checked at run-time.
However, the \name type system is capable of statically checking these preconditions.

In \name, ``\prechk-tagged'' instructions can assume that the preconditions on their behavior hold because it is enforced by the \name type system.
This can be seen in the reduction rules for the ``\prechk-tagged'' instructions in Figure \autoref{fig:prechkredux}, where they do not have rules to trap when their preconditions do not hold.
For example, in the $\<divpc>$ rule, the second argument $c_2$ is guaranteed to be non-zero, so there will be no trap on division-by-zero.
\<callindirectpc> can assume that the function that gets pulled from the table $\tfi_2$ has a subtype of the expected type $\tfi$, so it is a valid type for the indirect call (we will go over subtyping in more detail in \autoref{subsec:subtyping}).
The \prechk-tagged memory operations $\<loadpc>$ and $\<storepc>$ can assume that the memory operation takes place inside the memory bounds.

\begin{figure}[t]
    \begin{mathpar}
        \boxed{s;v^{*};e^{*} \rightarrow s';v'^{*};e'^{*}}
    \end{mathpar}

    \begin{math}
        \arraycolsep=1.4pt
        \begin{array}{rcl}
            (t.\<const> c_1)\;(t.\<const> c_2) && \\
            t.\<divpc> & \hookrightarrow & c \\
            && \text{where } c_2 \neq 0 \land c=c_1/c_2 \\
            s;(t.\<const> j) && \\
            t.\<callindirectpc> \tfi & \hookrightarrow_i & \<call> s_{tab}(i,j) \\
            &&\text{where } s_{tab}(i,j) = \\
            && \<func> \tfi_2\; \<local>\; t^{*}\;e^{*} \\
            && \text{and } \tfi_2<:\tfi \\
            s;(\<ithreetwo>.\<const> k) && \\
            (t.\<loadpc> (tp\_sx)^{?}\; a\;o) & \hookrightarrow_i & t.\<const> const_t(b^{*}) \\
            && \text{where } s_{mem}(i,k+o,|t|)=b^{*} \\
            s;(\<ithreetwo>.\<const> k)\;(t.\<const> c) && \\
            (t.\<storepc> tp^{?}\; a\;o) & \hookrightarrow_i & s';\epsilon \\
            && \text{where } s'=s \\
            && \text{with mem}(i,k+o,|t|)=bits_t^{|t|}(c) \\
        \end{array}
    \end{math}
    \caption{Behavior of new \prechk-tagged instructions}
    \label{fig:prechkredux}
\end{figure}

Now, we can rewrite the example from \autoref{fig:unnecessarycheck} to use the \prechk-tagged division instruction, as seen in \autoref{fig:prechkexample}.
Remember that we know the second argument to $\<divpc>$ is guaranteed to be non-zero, so we know the assumption of the $\<divpc>$ instruction ($c_2 \neq 0$) holds.
Since $\<divpc>$ does not require a dynamic check, this program will presumably be faster than the version with the dynamic check.
We still have not shown how we statically ensure that this assumption holds, which we will do in \autoref{subsec:checkelim}.

\begin{figure}[b]
    $${\begin{stackTL}
        \<block>
        {\begin{stackTL}
            \;(\<ithreetwo>\;\<ithreetwo> \rightarrow \<ithreetwo>)
            \\ (\<teelocal> 0)
            \\ (\<ithreetwo>.\<const> 1)
            \\ (\<getlocal> 0)
            \\ (\<select>)
            \\ (\<ithreetwo>.\<divpc>)
        \end{stackTL}}
        \\ \<nsend>
    \end{stackTL}}$$

    \caption{An Example of Using a \name \prechk-tagged Instruction}
    \label{fig:prechkexample}
\end{figure}

\section{The \name Indexed Type System}
\label{sec:typesys}
The \name type system is designed to provide sufficient information to safely eliminate dynamic checks (\ie to ensure that the required preconditions are met to \prechk-tag an instruction).
As explained in ~\ref{subsec:indexlang}, the \name type system can encode constraints on program values in the preconditions and postconditions of instructions.
We will now show how these constraints are added and used.

Recall the form of the \name typing judgement for instructions.
\begin{mathpar}
    \boxed{C \vdash e^{*} : ti_1^{*};l_1;\phi_1 \rightarrow ti_2^{*};l_2;\phi_2}
\end{mathpar}

Under $C$, the module type context, $e^{*}$ has the precondition $ti_1^{*};l_1;\phi_1$ and postcondition $ti_2^{*};l_2;\phi_2$.
We sometimes use the metavariable abbreviation $\tfi ::= ti_1^{*};l_1;\phi_1 \rightarrow ti_2^{*};l_2;\phi_2$ as shorthand for the precondition and postcondition of an instruction.

As in \wasm, \name generally has two kinds of typing rules.
Most rules are for inferring or checking the types of instructions (in which case $e^{*}$ will be a single instruction).
There are also a few rules to compose together instruction sequences.
We present the typing rules mixed with discussion of those rules.
The typing judgment definition in its entirety is reproduced in the appendix \autoref{chp:completejudgment}.

Here are some of the simpler rules.
These rules don't use or modify index type information.
\refrule{Unreachable} accepts any precondition and guarantees any postcondition since it just causes a trap.
In \refrule{Nop}, no changes are made from the precondition to the post condition because the instruction does nothing.
\refrule{Drop} consumes the top value from the stack (without caring about its type) and does not change the local index store or index type context.
\begin{mathpar}
    \inferrule*[right=\defrule{Unreachable}]{ }{ %% unreachable
        C \vdash \<unreachable> : ti_1^{*};l_1;\phi_1 \rightarrow ti_2^{*};l_2;\phi_2
    }

    \inferrule*[right=\defrule{Nop}]{ }{ %% nop
        C \vdash \<nop> : \epsilon;l;\phi \rightarrow \epsilon;l;\phi
    }

    \inferrule*[right=\defrule{Drop}]{ }{ %% drop
        C \vdash \<drop> : \ti{t}{a};l;\phi \rightarrow \epsilon;l;\phi
    }
\end{mathpar}

The constant instruction is a simple example of how indexed types work.
\refrule{Const} adds a new indexed type onto the stack to track the new program variable $\ti{t}{a}$, declares the new indexed type in the index type context $\phi$ (the $\ti{t}{a}$ part of $\phi,\ti{t}{a},(= a\; \ti{t}{c})$), and constrains that indexed type to be equal to the constant in $\phi$ (the $(= a\; \ti{t}{c})$ part of $\phi,\ti{t}{a},(= a\; \ti{t}{c})$).
We require $a$ to be fresh using the premise $a_3 \not\in \phi$, so that we know $a$ is not constrained/referenced anywhere in the precondition.
This is a common pattern to see in rules which introduce new index variables.
Since \<const> does not change or reference the local variables, the local index store $l$ is unchanged between the precondition and postcondition.
\begin{mathpar}
    \inferrule*[right=\defrule{Const}]{ a \not\in \phi }{ %% const
        C \vdash t.\<const> c : \epsilon;l;\phi \rightarrow \ti{t}{a};l;\phi,\ti{t}{a},(= a \ti{t}{c})
    }
\end{mathpar}

There are several different kinds of operations, but they all work similarly.
The binary operator instruction adds constraints between new and old program values, since the result of the instruction is a new program value, while the consumed values may already be constrained.
A binary operation consumes two values from the stack, which have associated indexed types $\ti{t}{a_1}$ and $\ti{t}{a_2}$, and produces a value which is associated with the fresh indexed type $\ti{t}{a_3}$.
The index type declaration $\ti{t}{a_3}$ is added to the index type context $\phi$ and $a_3$ is constrained to be equal to the binary operator applied to the index variables that correspond to the input $(= a_3\;(\|binop\|\;a_1\;a_2)$.
As a side note, we use $\|binop\|$ to indicate that we are moving the $binop$ (or relop or testop) from \name to the index language, where it will be interpreted by the semantics of the index language.
Binary operators do not affect or use local variables, so the local index store. $l$, is the same in the precondition and postcondition.
\begin{mathpar}
    \inferrule*[right=\defrule{Binop}]{ a_3 \not\in \phi }{ %% binop
        C \vdash t.binop :
        {\begin{stackTL}
            \ti{t}{a_1}\;\ti{t}{a_2};l;\phi
            \\ \rightarrow \ti{t}{a_3};l;\phi,\ti{t}{a_3},(= a_3\;(\|binop\|\;a_1\;a_2))
        \end{stackTL}}
    }

    \inferrule*[right=\defrule{Testop}]{ a_3 \not\in \phi }{ %% testop
        C \vdash t.testop :
        {\begin{stackTL}
            \ti{t}{a_1}\;l;\phi
            \\ \rightarrow \ti{\<ithreetwo>}{a_2};l;\phi,\ti{t}{a_2},(= a_2\;(\|testop\|\;a_1))
        \end{stackTL}}
    }

    \inferrule*[right=\defrule{Relop}]{ a_3 \not\in \phi }{ %% relop
        C \vdash t.relop :
        {\begin{stackTL}
            \ti{t}{a_1}\;\ti{t}{a_2};l;\phi
            \\ \rightarrow \ti{t}{a_3};l;\phi,\ti{t}{a_3},(= a_3\;(\|relop\|\;a_1\;a_2))
        \end{stackTL}}
    }
\end{mathpar}

\refrule{Select} constrains indexed types in a rather complex way.
Select consumes three values from the stack, it returns the second value if the third value is zero, and otherwise returns the first value (similar to C's ternary operator).
We use the type-level ``if'' to allow the constraint on the result to depend on the third value consumed: $(\text{if}\; (= a\; \ti{\<ithreetwo>}{0})\; (= a_3\;a_2)\; (= a_3\;a_1))$.

\begin{mathpar}
    \inferrule*[right=\defrule{Select}]{a_3 \not\in \phi}{ %% select
        C \vdash \<select> : {\begin{stackTL}
            \ti{t}{a_1}\;\ti{t}{a_2}\;\ti{i32}{a};l;\phi
            \\ \rightarrow \ti{t}{a_3};l;\phi,
            {\begin{stackTL}
                \ti{t}{a_3},
                \\ (\text{if}\; (= a\; \ti{\<ithreetwo>}{0})\; (= a_3\;a_2)\; (= a_3\;a_1))
            \end{stackTL}}
        \end{stackTL}}
    }
\end{mathpar}

The rules for the three different kinds of blocks (\<block>, \<loop>, and \<if>) are similar to \wasm.
They simply ensure that the interior instruction sequence has the expected type under the context with the expected postcondition (or precondition in the case of \<loop>) appended to the local stack.
In \name, \<if> blocks make extra assumptions about the consumed value in the subsequences (that it is non-zero in the first sequence and zero in the second), because those constraints must be true for that sequence to be executed.
While \<if> and \<block> append their postcondition to the label stack for typechecking branching instructions within the block, \<loop> appends its precondition because branching to a loop means running the loop again.

\begin{mathpar}
    \inferrule*[right=\defrule{Block}]{ %% block
        C_2,\text{label } (ti_2^{*};l_2;\phi_2) \vdash e^{*} : ti_1^{*};l_1;\phi_1 \rightarrow ti_2^{*};l_2;\phi_2 \\
    }
    {
        C \vdash \<block>\; (ti_1^{*};l_1;\phi_1 \rightarrow ti_2^{*};l_2;\phi_2)\; e^{*} \<end> : ti_1^{*};l_1;\phi_1 \rightarrow ti_2^{*};l_2;\phi_2
    }

    \inferrule*[right=\defrule{Loop}]{ %% loop
        C_2,\text{label } (ti_1^{*};l_1;\phi_1)^{*} \vdash e^{*} : ti_1^{*};l_1;\phi_1 \rightarrow ti_2^{*};l_2;\phi_2 \\
    }
    {
        C \vdash \<loop>\; (ti_1^{*};l_1;\phi_1 \rightarrow ti_2^{*};l_2;\phi_2)\; e^{*} \<end> : ti_1^{*};l_1;\phi_1 \rightarrow ti_2^{*};l_2;\phi_2
    }

    \inferrule*[right=\defrule{If}]{ %% if
        C_2,\text{label } (ti_2^{*};l_2;\phi_2) \vdash e_1^{*} : ti_1^{*};l_1;\phi_1, \neg(= a\; \ti{\<ithreetwo>}{0}) \rightarrow ti_2^{*};l_2;\phi_2 \\
        C_2,\text{label } (ti_2^{*};l_2;\phi_2) \vdash e_2^{*} : ti_1^{*};l_1;\phi_1, (= a\; \ti{\<ithreetwo>}{0})) \rightarrow ti_2^{*};l_2;\phi_2 \\
    }
    {
        C \vdash \<if>\; (ti_1^{*};l_1;\phi_1 \rightarrow ti_2^{*};l_2;\phi_2)\; e_1^{*} \<else> e_2^{*} \<end> : ti_1^{*};l_1;\phi_1 \rightarrow ti_2^{*};l_2;\phi_2
    }
\end{mathpar}

One thing to note is that all three of these rules include their expected preconditions and postconditions as part of their syntax.
We consider the index variables in these indexed function types to be unification variables rather than literals, allowing them to match any literal as long as the types unify.
Intuitively, this is very similar to alpha equivalence, where the precondition matches any preceding postcondition with the same structure as long as the variable can be renamed to match.
The postcondition appended to the label stack also has unification variables instead of the supplied literals.

The rules for branching instructions and return are similar to \wasm.
However, $\<brif>$ adds the assumption that the consumed value is zero to its postcondition.
This assumption can be safely added because the value must be zero for execution to continue without a branch occurring.
If the consumed value is constrained to be non-zero in the indexed type system, then this will cause a contradiction in the constraints of the index type context $\phi$.
However, that is fine since this means that no instructions following the $\<brif>$ will be executed.
Also remember the above note that the postconditions on the label stack contain unification variables, not literals.

Recall from \autoref{sec:wasmsemantics} that $\<brtable>$ branches to one of many different labels.
Thus, we must ensure that every possible branching postcondition to which it might branch is implied by the precondition.

\begin{mathpar}
    \inferrule*[right=\defrule{Br}]{ %% br
        C_{\text{label}}(i) = ti^{*};l_1;\phi_1
    }
    {
        C \vdash \<br> i : ti_1^{*}\;ti^{*};l_1;\phi_1 \rightarrow ti_2^{*};l_2;\phi_2
    }

    \inferrule*[right=\defrule{Return}]{ %% return
        C_{\text{return}} = ti^{*};l_1;\phi_1
    }
    {
        C \vdash \<return> : ti_1^{*}\;ti^{*};l_1;\phi_1 \rightarrow ti_2^{*};l_2;\phi_2
    }

    \inferrule*[right=\defrule{Br-If}]{ %% br_if
        C_{\text{label}}(i) = ti^{*};l_1;\phi_1,\neg(= a\; \ti{\<ithreetwo>}{0})
    }
    {
        C \vdash \<brif> i : ti^{*}\;\index{i32}{a};l_1;\phi_1 \rightarrow ti^{*};l_1;\phi_1,(= a\; \ti{\<ithreetwo>}{0})
    }

    \inferrule*[right=\defrule{Br-Table}]{ %% br_table
        (C_{\text{label}}(i) = ti^{*};l_1;\phi_i)^{+} \and
        (\phi_1 \implies \phi_i)^n
    }
    {
        C \vdash \<brtable> i^{+} : ti_1^{*}\;ti^{*}\;\index{i32}{a};l_1;\phi_1 \rightarrow ti_2^{*};l_2;\phi_2
    }
\end{mathpar}

Recall that functions are declared within the module with a specific indexed function type $\tfi$, that is a precondition and postcondition.
These declared indexed function types are placed inside the module type context $C$.
Direct function calls $\<call> i$ have the same type as the declared indexed function type of the function they are calling with two differences.
First, the local index store is unchanged, since the called function will have been turned into a closure that operates on separate local variables in a local block.
Second, the index type context in the postcondition of the call is extended with the declarations and constraints from the precondition of the call.
The precondition and postcondition of a function can only contain constraints about the arguments supplied to that function, so simply copying the postcondition of the function would result in the loss of information about all other index variables.

Indirect function calls $\<callindirect> ti_1^{*};l_1;\phi_1 \rightarrow ti_2^{*};l_2;\phi_2$ include the expected indexed function type $ti_1^{*};l_1;\phi_1 \rightarrow ti_2^{*};l_2;\phi_2$ provided as part of their syntax (the same note about index variables being unification variables from above holds).
Remember that indirect function calls perform a run time typecheck against the closure that they end up calling, so we assume statically that the check will proceed because if it does not the program will trap (\<trap> satisfies any $\tfi$, including the one for the indirect call) and not be able to do any harm.
The same two differences described above between the expected indexed function type $\tfi$ and the type of the $\<callindirect> \tfi$ instruction also hold.

\begin{mathpar}
    \inferrule*[right=\defrule{Call}]{ %% call
        C_\text{func}(i) = ti_1^{*};l_1;\phi_1 \rightarrow ti_2^{*};l_2;\phi_2
    }
    {
        C \vdash \<call> i : ti_1^{*};l;\phi_1 \rightarrow ti_2^{*};l;\phi_1,\phi_2
    }

    \inferrule*[right=\defrule{Call-Indirect}]{ %% call_indirect
        C_\text{table}(i) = (j, \tfi_2^{*}) \and
    }
    {
        C\;
        {\begin{stackTL}
            \vdash \<callindirect>  ti_1^{*};l_1;\phi_1 \rightarrow ti_2^{*};l_2;\phi_2
            \\ : ti_1^{*}\;\ti{i32}{a};l;\phi_1 \rightarrow ti_2^{*};l;\phi_1,\phi_2
        \end{stackTL}}
    }
\end{mathpar}

The only instructions that actually mutate the local index store are those that operate on local variables.
$\<getlocal>$ produces a fresh indexed type $\ti{t}{a_2}$ that is constrained to be equal to the index variable associated with the local being retrieved.
$\<setlocal>$ works in the reverse direction, replacing the index variable associated with the local being set.
Because $\<setlocal>$ reasons about local variables, which are not part of the instruction sequence (unlike values on the stack), we can copy the index variable instead of creating a fresh one and constraining it to be equal like for $\<getlocal>$
Finally, $\<teelocal>$ is effectively a combined $\<setlocal>$ and $\<getlocal>$ that consumes and immediately regurgitates a value, like the Unix tool ``tee''.
Thus, the typing rule is similarly a combination of the $\<setlocal>$ and $\<getlocal>$ rules, where the indexed type from the stack replaces the local variable indexed type, and a fresh indexed type is produced that is constrained to be equal to the consumed index variable.
\begin{mathpar}
    \inferrule*[right=\defrule{Get-Local}]{ %% get_local
        C_{\text{local}}(i) = t \and
        l(i) = \ti{t}{a} \and
        a_2 \not\in \phi
    }
    {
        C \vdash \<getlocal> i : \epsilon;l;\phi \rightarrow \ti{t}{a_2};l;\phi,\ti{t}{a_2},(= a_2 \; a)
    }

    \inferrule*[right=\defrule{Set-Local}]{ %% set_local
        C_{\text{local}}(i) = t \and
        l_2 = l_1[i := \ti{t}{a}] \and
    }
    {
        C \vdash \<setlocal> i : \ti{t}{a};l_1;\phi \rightarrow \epsilon;l_2;\phi
    }

    \inferrule*[right=\defrule{Tee-Local}]{ %% tee_local
        C_{\text{local}}(i) = t \and
        l_2 = l_1[i := \ti{t}{a}] \and
        a_2 \not\in \phi
    }
    {
        C \vdash \<teelocal> i : \ti{t}{a};l_1;\phi \rightarrow \ti{t}{a_2};l_2;\phi,\ti{t}{a_2},(= a_2 \; a)
    }
\end{mathpar}

Instructions for getting and setting globals produce and consume unconstrained values respectively.
Global variables are difficult to reason about in the type system since they are different between modules.
At compile-time, before linking, a module has no information about globals from another module which would be necessary for reasoning about the types of functions imported from the other module.
Therefore, we do not track index variables for globals (we just treat them as unconstrained values when they are introduced onto the stack by \<getglobal>).
We do still statically ensure the same properties as \wasm: that the value is of the correct \wasm type and in the case of setting a global variable that the global variable is mutable (has the $mut$ flag in its type).
\begin{mathpar}
    \inferrule*[right=\defrule{Get-Global}]{ %% get_global
        C_\text{global}(i) = \text{mut}^{?}\; t \and
        a \not\in \phi
    }
    {
        C \vdash \<getglobal> i : \epsilon;l;\phi \rightarrow \ti{t}{a};l;\phi,\ti{t}{a}
    }

    \inferrule*[right=\defrule{Set-Global}]{ %% set_global
        C_\text{global}(i) = \text{mut } t
    }
    {
        C \vdash \<setglobal> i : \ti{t}{a};l;\phi \rightarrow \epsilon;l;\phi
    }
\end{mathpar}

The typing rules for memory instructions are very similar to \wasm as we do not reason about the contents of memory or how its size can change throughout a program.
As in \wasm, there are many small details related to how exactly values are loaded and stored that are not particularly important to the understanding of the type system, but they are explained with the reduction rules for these values in \autoref{sec:wasmsemantics}.
One thing that does not appear in the \wasm reduction rules but mysteriously appears in the typing rules without much explanation is $align$.
It is checked against the size of the type of the value being stored/loaded $|t|$ (or optionally $|tp|$, which should be less than $|t|$) in the premise $2^{align} \leq (|tp| <)^{?} |t|$.

\begin{mathpar}
    \inferrule*[right=\defrule{Mem-Load}]{ %% memory load
        C_\text{memory} = n \and
        2^{align} \leq (|tp| <)^{?} |t| \and
        a_2 \not\in \phi
    }
    {
        C \vdash t.\<load> (tp\_sx)^{?}\; align\; o : \ti{\<ithreetwo>}{a_1};l;\phi \rightarrow \ti{t}{a_2};l;\phi,\ti{t}{a_2}
    }

    \inferrule*[right=\defrule{Mem-Store}]{ %% memory store
        C_\text{memory} = n \and
        2^{align} \leq (|tp| <)^{?} |t|
    }
    {
        C \vdash t.\<store> tp^{?}\; align\; o : \ti{\<ithreetwo>}{a_1}\;\ti{t}{a_2};l;\phi \rightarrow \epsilon;l;\phi
    }

    \inferrule*[right=\defrule{Current-Memory}]{ %% current mem
        C_\text{memory} = n \and
        a \not\in \phi
    }
    {
        C \vdash \<currentmemory> : \epsilon;l;\phi \rightarrow \ti{\<ithreetwo>}{a};l;\phi,\ti{\<ithreetwo>}{a}
    }

    \inferrule*[right=\defrule{Grow-Memory}]{ %% grow mem
        C_\text{memory} = n \and
        a_2 \not\in \phi
    }
    {
        C \vdash \<growmemory> : \ti{\<ithreetwo>}{a_1};l;\phi \rightarrow \ti{\<ithreetwo>}{a_2};l;\phi,\ti{\<ithreetwo>}{a_2}
    }
\end{mathpar}

The last rules are the ones that can be used to compose sequences of instructions.
The first rule is for the empty instruction sequence $\epsilon$, which, similar to \wasm, simply has the same precondition and postcondition $\epsilon;l;\phi$.
Second, we have \refrule{Stack-Poly} to add stack polymorphism (see \autoref{subsec:stackpoly}).
Third, there is a rule to compose a sequence of instructions $e_1^{*}$ with another instruction $e_2$.
\begin{mathpar}
    \inferrule*[right=\defrule{Empty}]{ %% empty
    }
    {
        C \vdash \epsilon : \epsilon;l;\phi \rightarrow \epsilon;l;\phi
    }

    \inferrule*[right=\defrule{Stack-Poly}]{ %% extra vars
        C \vdash e^{*} : ti_1^{*};l_1;\phi_1 \rightarrow ti_2^{*};l_2;\phi_2
    }
    {
        C \vdash e^{*} : ti^{*}\;ti_1^{*};l_1;\phi_1 \rightarrow ti^{*}\;ti_2^{*};l_2;\phi_2
    }

    \inferrule*[right=\defrule{Composition}]{ %% combine
        C \vdash e_1^{*} : ti_1^{*};l_1;\phi_1 \rightarrow ti_2^{*};l_2;\phi_2 \\
        C \vdash e_2 : ti_2^{*};l_2;\phi_2 \rightarrow ti_3^{*};l_3;\phi_3
    }
    {
        C \vdash e_1^{*}\;e_2 : ti_1^{*};l_1;\phi_1 \rightarrow ti_3^{*};l_3;\phi_3
    }
\end{mathpar}

\subsection{Subtyping, Implication, and Constraint Satisfaction}
\label{subsec:subtyping}
One issue with adding the index type context $\phi$ to preconditions and postconditions is that the postcondition of one instruction and the precondition of the next instruction might not match up exactly.
For example, one instruction may ensure a value is greater than ten, but the next just wants the value to be greater than zero.
Intuitively, if a value, ``x'', is greater than ten it must also be greater than zero, and we want the \name type system to be able to figure this out as well.
However, computers as of yet are unable to use intuition, so we must instead formalize this.

Our formalization of this problem is to allow \emph{strengthening} preconditions and \emph{weakening} postconditions.
Strengthening and weakening is based on implication ($\implies$).
We say that $\phi_1 \implies \phi_2$ when the following holds: if $\phi_1$ is satisfied, then $\phi_2$ must also be satisfied.
If $\phi_1 \implies \phi_2$, then we consider $\phi_1$ to be stronger than $\phi_2$, and $\phi_2$ to be weaker than $\phi_1$.
This solves the aforementioned problem because we can weaken ``x is greater than 10'' to ``x is greater than 0'' (or equivalently strengthen ``x is greater than 0'' to ``x is greater than 10'').

To fit strengthening and weakening into the type system, we define a subtyping judgment based on implication.
We are essentially parameterizing our typing judgment with the implication relation.
As an aside, we show that this is a practical thing to do and that such a relation exists by implementing such an implication relation using Z3 (see \autoref{subsec:z3}).
The \refrule{Implies} says that if an indexed function type $\tfi_1$ has a stronger precondition and weaker postcondition than some other indexed function type $\tfi_2$, and is otherwise equivalent, then $\tfi_1$ is a subtype of $\tfi_2$ since it can safely be used in place of $\tfi_2$.

\begin{mathpar}
    \inferrule*[right=\defrule{Implies}]{
        \phi_0 \implies \phi_1 \and
        \phi_2 \implies \phi_3
    }{
        ti_1^{*};l_1;\phi_1 \rightarrow ti_2^{*};l_2;\phi_2 <: ti_1^{*};l_1;\phi_0 \rightarrow ti_2^{*};l_2;\phi_3
    }
\end{mathpar}

We then use this in the \name type system by adding a typing rule that allows the indexed function type for a list of instructions to be replaced by a subtype of that indexed function type.

\[
    \inferrule*[right=\defrule{Subtyping}]{
        ti_1^{*};l_1;\phi_1 \rightarrow ti_2^{*};l_2;\phi_2 <: ti_1^{*};l_1;\phi_0 \rightarrow ti_2^{*};l_2;\phi_3 \\
        C \vdash e^{*} \rightarrow ti_1^{*};l_1;\phi_1 \rightarrow ti_2^{*};l_2;\phi_2
    }{
        C \vdash e^{*} \rightarrow ti_1^{*};l_1;\phi_0 \rightarrow ti_2^{*};l_2;\phi_3
    }
\]

\subsection{Using Types for Check Elimination}
\label{subsec:checkelim}
In Section \ref{sec:newinstructions} we explained that \prechk-tagged instructions do not need dynamic checks because of the static guarantees of the \name type system.
Here, we see how the \name type system provides those guarantees by looking at the typing rules for each of the \prechk-tagged instructions.

Integer division simply requires that the second argument is non-zero.
The premise $\phi \implies \neg(=\ a_2\ 0)$ requires that the index constraints satisfy the proposition $a_2 \neq 0$ for the pre-checked instruction to be safe.
Therefore, since a divide-by-zero is provably absent, it is safe to use the \prechk-tagged division instruction.
As an aside, recall that $a_3 \not\in \phi$ ensures that $a_3$ is fresh.
\begin{mathpar}
    \inferrule*[right=\defrule{Div-Prechk}]{
        \phi \implies \neg(=\ a_2\ 0) \and
        a_3 \not\in \phi
    }{
        C \vdash t.\<divpc> :
        {\begin{stackTL}
            \ti{t}{a_1}\;\ti{t}{a_2};l;\phi
            \\ \rightarrow \ti{t}{a_3};l;\phi,\ti{t}{a_3},(= a_3\;(\|div\|\;a_1\;a_2))
        \end{stackTL}}
    }
\end{mathpar}

Tagging memory loads and stores with \prechk requires ensuring that the memory index is valid.
Since \wasm and \name use linear memory, which is a contiguous block of memory, we simply have to ensure that the index is within those bounds.
The initial memory size is the number of $64$ Ki pages ($65,536$ bytes), so we check that the constraints in the index type context ensure that the memory index plus the static offset is between $0$ and $65,536-width$.
We use $width$ as a shorthand to denote the number of bytes that is being stored/loaded, it is equal to $|t|/8$ if $tp^{?}=\epsilon$, and otherwise equal to $|tp|/8$.

Unfortunately, while the size of memory may be grown during program execution, we are currently unable to reason about changing memory size.
Therefore, we just use the initial memory size.
\begin{mathpar}
    \inferrule*[right=\defrule{Load-Prechk}]{ %% memory load
        C_\text{memory} = n \and
        2^{align} \leq (|tp| <)^{?} |t| \and
        a_3 \not\in \phi \\
        \phi \implies
        {\begin{stackTL}
            (\<ge> (\<add> a_1\; \ti{\<ithreetwo>}{o}) \ti{\<ithreetwo>}{0}),
            \\ (\<le> (\<add> a_1\; (\<add> \ti{\<ithreetwo>}{o+width}))\; \ti{\<ithreetwo>}{n*64 \text{Ki}})
        \end{stackTL}}
    }
    {
        C \vdash t.\<loadpc> (tp\_sx)^{?}\; align\; o :
        {\begin{stackTL}
            \ti{\<ithreetwo>}{a_1};l;\phi
            \\ \rightarrow \ti{t}{a_2};l;\phi,\ti{t}{a_2}
        \end{stackTL}}
    }

    \inferrule*[right=\defrule{Store-Prechk}]{ %% memory store
        C_\text{memory} = n \and
        2^{align} \leq (|tp| <)^{?} |t| \\
        \phi \implies
        {\begin{stackTL}
            (\<ge> (\<add> a_1\; \ti{\<ithreetwo>}{o})\; \ti{\<ithreetwo>}{0}),
            \\ (\<le> (\<add> a_1\; (\<add> \ti{\<ithreetwo>}{o+width}))\; \ti{\<ithreetwo>}{n*64\text{Ki}})
        \end{stackTL}}
    }
    {
        C \vdash t.\<storepc> tp^{?}\; align\; o : \ti{\<ithreetwo>}{a_1}\;\ti{t}{a_2};l;\phi \rightarrow \epsilon;l;\phi
    }
\end{mathpar}

Indirect function calls in \wasm require a dynamic check to ensure that the index into the table points to a function of a suitable type (recall the explanation of tables and \<callindirect> from \autoref{sec:wasmsemantics}).
Proving the safety of an indirect function call involves showing that every possible function that could be called will not cause a run-time type error.
We ensure this by requiring that the type of every function at every possible index value has a subtype of the expected type: $\forall 0 < i \leq n. (\phi \implies \neg (= \ti{\<ithreetwo>}{i}\; a)) \lor \tfi s(i) <: \tfi$ where $\tfi s=(\tfi_2 ...)$.
The $\forall$ and $\lor$ (as in $(\phi \implies \neg (= \ti{\<ithreetwo>}{i}\; a)) \lor\; \tfi s(i) <: \tfi$) in this case are at the meta level and not within the index language.
Further, we must show that the provided table index is within the table boundaries: $\phi \implies (\<gt>\; n\; a) \land (\<le> \ti{\<ithreetwo>}{0}\; a)$.
\begin{mathpar}
    \inferrule*[right=\defrule{Call-Indirect-Prechk}]{ %% call_indirect
        C_{table}(i) = (n, (\tfi_2 ...)) \\
        \phi \implies (\<gt>\; n\; a) \land (\<le> \ti{\<ithreetwo>}{0}\; a) \\
        \mathit{tfis}=(\tfi_2 ...) \\
        \tfi = ti_1^{*};l_1;\phi_1 \rightarrow ti_2^{*};l_2;\phi_2 \\
        \forall 0 < i \leq n.\; (\phi \implies \neg (= \ti{\<ithreetwo>}{i}\; a)) \lor\; \mathit{tfis}(i) <: \tfi
    }
    {
        C \vdash \<callindirectpc> \tfi :
        {\begin{stackTL}
            ti_1^{*}\;\ti{i32}{a};l;\phi_1
            \\ \rightarrow ti_2^{*};l;\phi_1,\phi_2
        \end{stackTL}}
    }
\end{mathpar}

\paragraph{An Example of Using Types for Check Elimination}
\label{elimexample}
Here we present a short contrived example of using types for check elimination, based on the example from \autoref{fig:prechkexample}.
We are typechecking a $\<divpc>$.
First, we give the module type context $C_1$, which contains one local variable, which is an $\<ithreetwo>$, and the instruction sequence we are typing, which is a safe division happening inside of a block (\autoref{fig:prechkexample}).
We also use the shorthand $C_2$ for $C_1$ extended with the label postcondition for the instructions inside the block.

\begin{math}
    \begin{array}{rcl}
        C_1 &=& \{ {\begin{stackTL}
        \text{func } \epsilon, \text{ global } \epsilon, \text{ table } \epsilon, \text{ memory } \epsilon,
        \\ \text{local } \<ithreetwo>, \text{ label } \epsilon, \text{ return } \epsilon\}
        \end{stackTL}} \\

        C_2 &=& \{ {\begin{stackTL}
        \text{func } \epsilon, \text{ global } \epsilon, \text{ table } \epsilon, \text{ memory } \epsilon,
        \\ \text{local } \<ithreetwo>, \text{ label } (\ti{\<ithreetwo>}{a_0};\ti{\<ithreetwo>}{a_2};\circ), \text{ return } \epsilon\}
        \end{stackTL}} \\
    \end{array}
\end{math}

The block takes two integers, and, using the local as temporary storage, either divides the first by the second, or the first by 1 if the second is 0.
In the example we build up type derivations to reach a typing derivation for the whole block.
We first state the rule that we will use, than give the derivation (in many cases we apply \refrule{Stack-Poly} inline for brevity).
Typechecking the example relies on $\<select>$, where the second argument $(\<ithreetwo>.\<const> 1)$ is chosen if the first argument (and the conditional) $\<getlocal> 0$ is equal to zero, so the result must be non-zero.

\begin{enumerate}
    \item \refrule{Tee-Local} and \refrule{Stack-Poly}

        Note that $\phi_1 = \circ,\ti{\<ithreetwo>}{a_1},\ti{\<ithreetwo>}{a_2}$, and $\phi_2 = \phi_1,(\<ithreetwo>\; a_4),(= a_4\; a_2)$.
        $$\infer{
            C_{2\text{local}}(0)=\<ithreetwo> \\
            \ti{\<ithreetwo>}{a_2} = \ti{\<ithreetwo>}{a_3}[0 := \ti{\<ithreetwo>}{a_2}] \\
            a_4 \not\in \phi_1 \\
        }{
            C_2 \vdash \<teelocal> 0 :
            {\begin{stackTL}
                \ti{\<ithreetwo>}{a_1}\;\ti{\<ithreetwo>}{a_2};\ti{\<ithreetwo>}{a_3};\phi_1
                \\ \rightarrow \ti{\<ithreetwo>}{a_1}\;\ti{\<ithreetwo>}{a_4};\ti{\<ithreetwo>}{a_2};\phi_2
            \end{stackTL}}
        }$$

    \item \refrule{Const} and \refrule{Stack-Poly}

        Note that $\phi_3 = \phi_2,(= a_4\; a_2),(\<ithreetwo>\; a_5),(= a_5\;\ti{\<ithreetwo>}{1})$.
        $$\infer{
            a_5 \not\in \phi_2 \\
        }{
            C_2 \vdash \<ithreetwo>.\<const> 1 :
            {\begin{stackTL}
                \ti{\<ithreetwo>}{a_1}\;\ti{\<ithreetwo>}{a_4};\ti{\<ithreetwo>}{a_2};\phi_2
                \\ \rightarrow \ti{\<ithreetwo>}{a_1}\;\ti{\<ithreetwo>}{a_4}\;\ti{\<ithreetwo>}{a_5};\ti{\<ithreetwo>}{a_2};\phi_3
            \end{stackTL}}
        }$$

    \item \refrule{Composition}
        $$\infer{
            1. \\ 2.
        }{
            C_2\;
            {\begin{stackTL}
                \vdash (\<teelocal> 0)\; (\<ithreetwo>.\<const> 1)
                \\ :
                {\begin{stackTL}
                    \ti{\<ithreetwo>}{a_1}\;\ti{\<ithreetwo>}{a_2};\ti{\<ithreetwo>}{a_3};\phi_1
                    \\ \rightarrow \ti{\<ithreetwo>}{a_1}\;\ti{\<ithreetwo>}{a_4}\;\ti{\<ithreetwo>}{a_5};\ti{\<ithreetwo>}{a_2};\phi_3
                \end{stackTL}}
            \end{stackTL}}
        }$$

    \item \refrule{Get-Local} and \refrule{Stack-Poly}

        Note that $\phi_4 = \phi_3,(\<ithreetwo>\; a_6),(= a_6\;a_2)$.
        $$\infer{
            a_6 \not\in \phi_3 \\
        }{
            C_2 \vdash \<getlocal> 0 :
            {\begin{stackTL}
                \ti{\<ithreetwo>}{a_1}\;\ti{\<ithreetwo>}{a_4}\;\ti{\<ithreetwo>}{a_5};\ti{\<ithreetwo>}{a_2};\phi_3
                \\ \rightarrow \ti{\<ithreetwo>}{a_1}\;\ti{\<ithreetwo>}{a_4}\;\ti{\<ithreetwo>}{a_5}\;\ti{\<ithreetwo>}{a_6};\ti{\<ithreetwo>}{a_2};\phi_4
            \end{stackTL}}
        }$$

    \item \refrule{Composition}

        Note that $\phi_4 = \phi_3,(\<ithreetwo>\; a_6),(= a_6\;a_2)$.
        $$\infer{
            3. \\ 4.
        }{
            C_2\;
            {\begin{stackTL}
                \vdash (\<teelocal> 0)\; (\<ithreetwo>.\<const> 1)\; (\<getlocal> 0)
                \\ :
                {\begin{stackTL}
                    \ti{\<ithreetwo>}{a_1}\;\ti{\<ithreetwo>}{a_2};\ti{\<ithreetwo>}{a_3};\phi_1
                    \\ \rightarrow \ti{\<ithreetwo>}{a_1}\;\ti{\<ithreetwo>}{a_4}\;\ti{\<ithreetwo>}{a_5}\;\ti{\<ithreetwo>}{a_6};\ti{\<ithreetwo>}{a_2};\phi_4
                \end{stackTL}}
            \end{stackTL}}
        }$$

    \item \refrule{Select} and \refrule{Stack-Poly}

        Note that $\phi_5 = \phi_4,(\<ithreetwo>\; a_7),(\text{if }(= a_6\; \ti{\<ithreetwo>}{0})\;(= a_7\;a_5) (= a_7\;a_4))$.
        $$\infer{
            a_7 \not\in \phi_5
        }{
            C_2\;
            {\begin{stackTL}
                \vdash \<select>
                \\ :
                {\begin{stackTL}
                    \rightarrow \ti{\<ithreetwo>}{a_1}\;\ti{\<ithreetwo>}{a_4}\;\ti{\<ithreetwo>}{a_5}\;\ti{\<ithreetwo>}{a_6};\ti{\<ithreetwo>}{a_2};\phi_4
                    \\ \rightarrow (\ti{\<ithreetwo>}{a_1}\;\ti{\<ithreetwo>}{a_7};\ti{\<ithreetwo>}{a_2};\phi_5
                \end{stackTL}}
            \end{stackTL}}
        }$$

    \item \refrule{Composition}

        $$\infer{
            5. \\ 6.
        }{
            C_2\;
            {\begin{stackTL}
                \vdash (\<teelocal> 0)\; (\<ithreetwo>.\<const> 1)\; (\<getlocal> 0)\; (\<select>)
                \\ :
                {\begin{stackTL}
                    \ti{\<ithreetwo>}{a_1}\;\ti{\<ithreetwo>}{a_2};\ti{\<ithreetwo>}{a_3};\phi_1
                    \\ \rightarrow (\ti{\<ithreetwo>}{a_1}\;\ti{\<ithreetwo>}{a_7};\ti{\<ithreetwo>}{a_2};\phi_5
                \end{stackTL}}
            \end{stackTL}}
        }$$

    \item \refrule{Div-Prechk}

        Note that $\phi_5 = \phi_5,(\<ithreetwo>\; a_8),(= a_8 (\|\<div>\|\; a_1\; a_7))$.
        $$\infer{
            \phi_5 \implies \neg(= a_7\; 0) \\ a_8 \not\in \phi_5
        }{
            C_2\;
            {\begin{stackTL}
                \vdash (\<ithreetwo>.\<divpc>)
                \\ :
                {\begin{stackTL}
                    (\ti{\<ithreetwo>}{a_1}\;\ti{\<ithreetwo>}{a_7};\ti{\<ithreetwo>}{a_2};\phi_5
                    \\ \rightarrow \ti{\<ithreetwo>}{a_8};\ti{\<ithreetwo>}{a_2};\phi_6
                \end{stackTL}}
            \end{stackTL}}
        }$$

    \item \refrule{Composition}

        $$\infer{
            7. \\ 8.
        }{
            C_2\;
            {\begin{stackTL}
                \vdash (\<teelocal> 0)\; (\<ithreetwo>.\<const> 1)\; (\<getlocal> 0)\; (\<select>)\; (\<ithreetwo>.\<divpc>)
                \\ :
                {\begin{stackTL}
                    \ti{\<ithreetwo>}{a_1}\;\ti{\<ithreetwo>}{a_2};\ti{\<ithreetwo>}{a_3};\phi_1
                    \\ \rightarrow \ti{\<ithreetwo>}{a_8};\ti{\<ithreetwo>}{a_2};\phi_6
                \end{stackTL}}
            \end{stackTL}}
        }$$

    \item \refrule{Block}

    $$\infer{9.}{
        C_1 \vdash\;
        {\begin{stackTL}
            \<block>\;
            {\begin{stackTL}\ti{\<ithreetwo>}{a_1}\;\ti{\<ithreetwo>}{a_2};\ti{\<ithreetwo>}{a_3};\phi_1
                \\ \rightarrow \ti{\<ithreetwo>}{a_8};\ti{\<ithreetwo>}{a_2};\phi_6
            \end{stackTL}} \\
            \quad{\begin{stackTL}
                (\<teelocal> 0)
                \\ (\<ithreetwo>.\<const> 1)
                \\ (\<getlocal> 0)
                \\ (\<select>)
                \\ (\<ithreetwo>.\<div>)
            \end{stackTL}}
            \\ \<nsend> : \ti{\<ithreetwo>}{a_1}\;\ti{\<ithreetwo>}{a_2};\ti{\<ithreetwo>}{a_3};\phi_1
            \\ \rightarrow \ti{\<ithreetwo>}{a_8};\ti{\<ithreetwo>}{a_2};\phi_6
        \end{stackTL}}}$$
\end{enumerate}

\subsection{Module Types}
The complete module typing rules are in Figure \ref{fig:modulerules} (note that $im$ is an import and $ex$ is an export).
Functions $f$, typecheck their body $e^{*}$ under the module type context $C$ with the expected postcondition $ti_2^{*};l_2;\phi_2$ in the label stack and return position, and with the local index store $\ti{t_1}{a_1}^{*}\;\ti{t}{a_2}^{*}$ constructed from the function's arguments $\ti{t_1}{a_1}^{*}$ and declared locals $\ti{t}{a_2}^{*}$.
Global variables $glob$ must ensure that their initialization instructions $e^{*}$ produce a value of the proper type $t$.
Exported global variables cannot be mutable, if there are any exports defined, the global cannot have the mutable tag $mut$: $ex^{*}=\epsilon \lor tg=t$.
Tables $tab$ ensure that the indices $i^{n}$ refer to well-typed functions and there are exactly as many indices as the expected size $n$.
Memory $mem$ simply has its declared initial size $n$ from which it can only grow bigger.
All imported functions, globals, tables, and memories are expected to have their declared type.
They are typechecked during linking.

Typechecking a module involves typechecking every component of the module.
Functions, $f$, are typechecked under the module type context, $C$, containing the entirety of the module.
This means that functions can refer to themselves, other functions, all globals, the table, and memory.
This may seem to be a circular definition, but the type of the module is declared statically (as the combined declared types of all the module components), so it is just checking against the expected module index type context.
Globals, $glob$, are typechecked under the module index context containing only the global variable declarations preceding the current declaration.

\begin{figure}
    \begin{mathpar}
        \inferrule*[right=\defrule{Func}]{ %% local function
            \tfi = \ti{t_1}{a_1}^{*};\epsilon;\phi_1 \rightarrow ti_2^{*};l_2;\phi_2 \\
            C_2 = C,\text{local } t_1^{*}\;t^{*},\text{label } (ti_2^{*};l_2;\phi_2),\text{return } (ti_2^{*},l_2,\phi_2) \\
            C_2 \vdash e^{*} : \epsilon ;\ti{t_1}{a_1}^{*}\;\ti{t}{a_2}^{*};\phi_1 \rightarrow ti_2^{*};l_2;\phi_2
        } {
            C \vdash ex^{*}\; \<func> \tfi\; \<local> t^{*}\; e^{*} : ex^{*}\; \tfi
        } \and

        \inferrule*[]{ %% imported function
        } {
            C \vdash ex^{*}\; \<func> \tfi\; im : ex^{*}\; \tfi
        } \\

        \inferrule*[]{ %% local global
            tg = mut^{?}\;t \and
            ex^{*} = \epsilon \lor tg = t \and
            C \vdash e^{*} : \epsilon; \epsilon; \phi_1 \rightarrow \ti{t}{a};\epsilon; \phi_2
        } {
            C \vdash ex^{*}\; \<global> tg\; e^{*} : ex^{*}\; tg
        } \and

        \inferrule*[]{ %% imported global
            tg = t \and
        } {
            C \vdash ex^{*}\; \<global> tg\; im : ex^{*}\; tg
        } \and

        \inferrule*[right=\defrule{Table}]{ %% local table
            (C_{\text{func}}(i) = \tfi)^n
        } {
            C \vdash ex^{*}\; \<table> n\; i^n : ex^{*}\; (n,\tfi^n)
        } \and

        \inferrule*[]{ %% imported table
        } {
            C \vdash ex^{*}\; \<table> (n,\tfi^n)\; im : ex^{*}\; (n,\tfi^n)
        } \and

        \inferrule*[]{ %% local memory
        } {
            C \vdash ex^{*}\; \<memory> n : ex^{*}\; n
        }

        \inferrule*[]{ %% imported memory
        } {
            C \vdash ex^{*}\; \<memory> n\; im : ex^{*}\; n
        }

        \inferrule*[]{ %% module
            (C\vdash f : ex_f^{*}\; \tfi)^{*} \and
            (C_i \vdash glob_i : ex_g^{*}\; tg_i)^{*} \\
            (C \vdash tab : ex_t^{*}\; (n,\tfi^n))^{?} \and
            (C \vdash mem : ex_m^{*}\; n)^{?} \\
            (C_i=\{\text{global } tg^{i-1}\})_i^{*} \and
            ex_f^{*\;*}\; ex_g^{*\;*}\; ex_t^{*\;?}\; ex_m^{*\;?} \text{ distinct} \\
            C = \{\text{func } \tfi^{*}, \text{global } tg^{*}, \text{table } (n,\tfi^n)^{?}, \text{memory } n^{?}\}
        } {
            \vdash \<module> f^{*}\; glob^{*}\; tab^{?}\; mem^{?}
        }
    \end{mathpar}
    \caption{Indexed Module Typing Rules}
    \label{fig:modulerules}
\end{figure}

\chapter{Metatheory}
\label{chp:metatheory}

Now that we have introduced \name and shown how it can be used for reasoning, it is time to reason about \name itself.
First, we will take a look at the relationship between \wasm and \name, by showing methods to convert \wasm programs to \name programs and vice versa.
Then, we will prove the type safety of \name, to ensure that our claim that \name is as safe as \wasm is valid.
However, before we can do any of that, we must ``complete'' our reasoning ability by creating a way to connect the reduction relation form with the type system.

\section{Administrative Typing Rules}
While we have shown the \name typing rules for instructions within a static context, we still need typing rules for administrative instructions and the store used in reduction.
\autoref{fig:programrules} shows the \name typing rules for module instances $inst$, the run time store $s$, and various data structures contained within $s$.
$S$ is the store context, it is to $s$ as $C$ is to $inst$.
That is, it contains the type information for everything in $s$.
There are many different judgments being introduced, so we explicitly state the form of the judgment before stating the rule for that judgment.

Closures are type checked by \refrule{Closure}, which falls back on the module typing rules \autoref{fig:modulerules} to type check the function definition inside of the closure.
\refrule{Admin-Const} gets the postcondition indexed types and constraints on values, it is used to type check local and global variables.

\refrule{Instance} checks that a module instance is well-typed by the index module context under the store context $S$.
It checks all of the closures $cl^{*}$ against their expected types $tfi^{*}$ in $C$, and similarly for all of the globals ($v^{*}$ and $()\text{mut}^{?} t)^{*}$).
The table and memory indices ($i$ and $j$, respectively) are used to lookup the the relevant types ($(n,tfi^{*})$ and $m$, respectively) in the store context $S$.

\refrule{Store} uses \refrule{Instance} to check that a run time store, $s$ is well typed by the store context $S$ by ensuring that every module instance $inst$ in $s$ has the type of the index module context $C$ in $S$
Further, \refrule{Store} ensures that all of the closures in all of the tables in $s$ are well typed, and the the sizes of all the tables and memory chunks in $S$ do not exceed the actual size of their implementations.

\begin{figure}
    \begin{mathpar}
        \boxed{S \vdash cl : tfi} \\

        \inferrule*[right=\defrule{Closure}]{ %% closure
            S_\text{inst}(i) \vdash f : tfi
        } {
            S \vdash \{ \text{inst} \; i, \text{code} \; f \} : tfi
        }

        \\ \boxed{\vdash v : ti;\phi} \\

        \inferrule*[right=\defrule{Admin-Const}]{ %% admin const
        } {
            \vdash t.\<const> c : \ti{t}{a};\circ,\ti{t}{a},(\<eq> a \; \ti{t}{c})
        }

        \\ \boxed{S \vdash inst : C} \\

        \inferrule*[right=\defrule{Instance}]{ %% instance
            (S \vdash cl : tfi)^{*} \and
            (\vdash v : \ti{t}{a},\phi_v)^{*} \\
            (S_\text{tab}(i) = n)^{?} \and
            (S_\text{mem}(j) = m)^{?}
        } {
            S \vdash
            {\begin{stackTL}
                \{ \text{func} \; cl^{*}, \text{glob} \; v^{*}, \text{tab} \; i^{?}, \text{mem} \; j^{?} \}
                \\ : \{ \text{func} \; tfi^{*}, \text{global} \; (\text{mut}^{?} \; t)^{*}, \text{table} \; n^{?}, \text{memory} \; m^{?} \}
            \end{stackTL}}
        }

        \\ \boxed{\vdash s : S} \\

        \inferrule*[right=\defrule{Store}]{ %% store
            S = \{ \text{inst} \; C^{*}, \text{tab} \; n^{*}, \text{mem} \; m^{*} \}\\
            (S \vdash inst : C)^{*} \and
            ((S \vdash cl : tfi)^{*})^{*} \and
            (n \leq |cl^{*}|)^{*} \and
            (m \leq |b^{*}|)^{*}
        } {
            \vdash \{ \text{inst} \; inst^{*}, \text{tab} \; (cl^{*})^{*}, \text{mem} \; (b^{*})^{*} \} : S
        }

        \\ \boxed{S;(ti^{*};l;\phi)^{?} \vdash_i v^{*};e^{*} : ti^{*};l;\phi} \\

        \inferrule*[right=\defrule{Code}]{ %% admin code
            (\vdash v : \ti{t}{a};\phi_v)^{*}\\
            C = S_{\text{inst}}(i),\text{local} \; t^{*}, \text{return} \; (ti^n;l;\phi)^{?}\\
            S;C \vdash e^{*} : \epsilon:\ti{t}{a}^{*};\phi_v^{*} \rightarrow ti^n;l;\phi
        } {
            S;(ti^n;l;\phi)^{?} \vdash_i v^{*};e^{*} : ti^n;l;\phi
        }

        \boxed{\vdash s;v^{*};e^{*}} \\

        \inferrule*[right=\defrule{Program}]{ %% admin program
            \vdash s : S \and
            S;\epsilon \vdash_i v^{*};e^{*} : ti^{*};l;\phi
        } {
            \vdash_i s;v^{*};e^{*} : ti^{*};l;\phi
        }
    \end{mathpar}
    \caption{\protect\name Program and Store Typing Rules}
    \label{fig:programrules}
\end{figure}

Further, we must provide a typing rule that has the same form as the reduction relation (\ie a typing rule for $s;v;e$) to be able to get the type of a program during reduction.

\refrule{Code} checks that a sequence of instructions is well typed with an empty stack and the indexed types and constraints for the given local variables in the precondition.
Since local variables are values, we know that each one of them is equal to some constant, so \refrule{Code} is really just checking that the sequence of instructions has some postcondition reachable from the given local variables.
There is an optional return postcondition for \refrule{Code} because \refrule{Local} has as a premise a judgment of the exactly same form, except with a return postcondition.
\refrule{Program} uses \refrule{Code} without using the optional return postcondition, as well as \refrule{Store}, to ensure that a reducible \name program is well-typed.

\begin{figure}
    $$\boxed{S;C \vdash e^{*} : tfi}$$

    \begin{mathpar}
        \inferrule*[right=\defrule{Local}]{ %% local
            S;(ti^n;l_2;\phi_2) \vdash_i v_l^{*};e^{*} : ti^n;l_2;\phi_2
        } {
            S;C \vdash \<local> \{ i;v_l^{*} \} \; e^{*} \<end> : \epsilon;l_1;\phi_1 \rightarrow ti^n;l_1;\phi_1,\phi_2
        }

        \inferrule*[right=\defrule{Call-Cl}]{ %% call closure
            S \vdash cl : tfi
        } {
            S;C \vdash \<call> cl : tfi
        }

        \inferrule*[right=\defrule{Trap}]{ %% trap
        } {
            S;C \vdash \<trap> : tfi
        }

        \inferrule*[right=\defrule{Label}]{ %% label
            S;C\vdash e_0^{*} : ti_3^{*};l_3;\phi_3 \rightarrow ti_2^{*};l_2;\phi_2 \\
            S;C,\text{label } (ti_3^{*};l_3;\phi_3) \vdash e^{*} : \epsilon;l_1;\phi_1 \rightarrow ti_2^{*};l_2;\phi_2
        } {
            S;C \vdash \<label> \{ e_0^{*} \} \; e^{*} \<end> : \epsilon;l_1;\phi_1 \rightarrow ti_2^{*};l_2;\phi_2
        }
    \end{mathpar}
    \caption{\name Administrative Instruction Rules}
    \label{fig:adminrules}
\end{figure}

\autoref{fig:adminrules} extends the \name typing rules for instructions to include administrative instructions.
\refrule{Local} typechecks the local block using \refrule{Code} to ensure that the body is well typed with the indexed types and constraints for local variables provided by the local block as the precondition and any postcondition.
Since local blocks are inline expansions of function calls, we use the optional return postcondition functionality of \refrule{Code} to ensure that returning from inside the local block will be well-typed.

\refrule{Call-Cl} typechecks calling a closure by ensuring that the closure being called has the same type as the call instruction.
\refrule{Trap} is always well-typed under any precondition and postcondition.

\refrule{Label} typechecks the body of the label block with the precondition of the saved instructions pushed onto the label stack.
This was, if the label was generated by a loop, then the precondition of the saved values is the precondition of the loop, and we know the loop is well-typed.
Otherwise, the saved instructions will be an empty sequence and will be well typed from the precondition.

Additionally, we must keep track of the store context $S$ in all of the other typing rules.
We do this by ``silently'' adding $S$ to the typing judgment and rules in \autoref{sec:typesys}.

Given these additional typing judgments and rules, we can now show the metatheoretic properties mentioned above.

\section{Relationship Between \wasm and \name}
We want to show two properties about the relationship between \wasm and \name.
First, we want \name to be backwards compatible with \wasm.
It should be possible to convert well-typed \wasm programs into well-typed \name programs with no additional developer effort.
We demonstrate a simple yet naive way of embedding \wasm programs into \name in \autoref{subsec:embedding}.
Second, we want to show that well-typed \name programs can be turned into \wasm programs.
This is accomplished in \autoref{subsec:erasure} using an erasure function that turns \name programs and types into \wasm programs and types.

\subsection{Embedding \wasm in \name}
\label{subsec:embedding}
We present a way to embed \wasm programs in \name.
The embedding function takes a \wasm program and replaces all of the type annotations with indexed function types that have no constraints on the variables.
Intuitively, this is the only part of the surface syntax of \wasm that isn't in \name, so we must figure out a way to bring it over.
While this embedding requires no additional developer effort, it provides no information to the indexed type system beyond what can be inferred from the instructions in the program.
We conjecture that a well-typed \wasm program embedded in \name is also well-typed, but we have not proved it.

The embedding replaces all function types used within the \wasm syntax with \name indexed function types, and adds the function types for all of the functions in a table to the table's type declaration.
This occurs within blocks and indirect function calls, as shown in \autoref{def:embed-e}.
The indexed types simply have fresh index variables which are different in the precondition and postcondition, and the primitive types for the stack are known from the \wasm type $t_1^{*} \rightarrow t_2^{*}$
To know what the local variables are, we parameterize the embedding over the types of local variables ($t^{*}$).

We typeset \name instructions in a \tbsf{blue sans serif font} and \wasm instruction in a \trbf{bold red font} to set them apart.

\begin{definition}{$\embed[e]{e}^{t^{*}}=\mathbluesf{e}$}
    \label{def:embed-e}
    \begin{mathpar}
        %% SPACE HACKS
        \arraycolsep=2pt
        \begin{array}{rcl}
            embed_{e^{*}}({\begin{stackTL}
                \<wblock>
                {\begin{stackTL}
                    (t_1^{*}\rightarrow t_2^{*})\;
                    \\e^{*}
                \end{stackTL}}\\
            \<wend>)^{t^{*}}
            \end{stackTL}}
            &=& {\begin{stackTL}
                    \<block>
                    \\ \quad (\ti{t_1}{a_1}^{*};\ti{t}{a_3}^{*};(\circ,\ti{t_1}{a_1}^{*},\ti{t}{a_3}^{*})
                    \\ \quad\; \rightarrow \ti{t_2}{a_2}^{*};\ti{t}{a_4}^{*};(\circ,\ti{t_2}{a_2}^{*},\ti{t}{a_4}^{*}))
                    \\ \quad \embed[e^{*}]{e^{*}}^{t^{*}}
            \end{stackTL}} \\
            && \<nsend>\\

            embed_{e^{*}}({\begin{stackTL}
                \<wloop>
                {\begin{stackTL}
                    (t_1^{*}\rightarrow t_2^{*})\;
                    \\e^{*}
                \end{stackTL}}\\
            \<wend>)^{t^{*}}
            \end{stackTL}}
            &=& {\begin{stackTL}
                    \<loop>
                    \\ \quad (\ti{t_1}{a_1}^{*};\ti{t}{a_3}^{*};(\circ,\ti{t_1}{a_1}^{*},\ti{t}{a_3}^{*})
                    \\ \quad\; \rightarrow \ti{t_2}{a_2}^{*};\ti{t}{a_4}^{*};(\circ,\ti{t_2}{a_2}^{*},\ti{t}{a_4}^{*}))
                    \\ \quad \embed[e^{*}]{e^{*}}^{t^{*}}
            \end{stackTL}} \\
            && \<nsend>\\

            embed_{e^{*}}({\begin{stackTL}
                \<wif>
                {\begin{stackTL}
                    (t_1^{*}\rightarrow t_2^{*})\;
                    \\e^{*}
                \end{stackTL}}\\
            \<wend>)^{t^{*}}
            \end{stackTL}}
            &=& {\begin{stackTL}
                    \<if>
                    \\ \quad (\ti{t_1}{a_1}^{*};\ti{t}{a_3}^{*};(\circ,\ti{t_1}{a_1}^{*},\ti{t}{a_3}^{*})
                    \\ \quad\; \rightarrow \ti{t_2}{a_2}^{*};\ti{t}{a_4}^{*};(\circ,\ti{t_2}{a_2}^{*},\ti{t}{a_4}^{*}))
                    \\ \quad \embed[e]{e_1^{*}}^{t^{*}}\; \embed[e]{e_2^{*}}^{t^{*}}
                \end{stackTL}} \\
            && \<nsend>\\

            embed_{e^{*}}(
                {\begin{stackTL}
                    \<wcallindirect>
                    \\\quad (t_1^{*}\rightarrow t_2^{*}))^{t^{*}}
                \end{stackTL}}
            &=& {\begin{stackTL}
                \<callindirect>
                \\ \quad (\ti{t_1}{a_1}^{*};\ti{t}{a_3}^{*};(\circ,\ti{t_1}{a_1}^{*},\ti{t}{a_3}^{*})
                \\ \quad\; \rightarrow \ti{t_2}{a_2}^{*};\ti{t}{a_4}^{*};(\circ,\ti{t_2}{a_2}^{*},\ti{t}{a_4}^{*}))
            \end{stackTL}} \\

            \embed[e^{*}]{e}^{t^{*}} &=& e \text{, otherwise} \\
            \embed[e^{*}]{e^{*}}^{t^{*}} &=& (\embed[e^{*}]{e}^{t^{*}})^{*} \\
        \end{array}
    \end{mathpar}
\end{definition}

The embedding of functions, \autoref{def:embed-f}, both must construct an indexed function type for itself and embed its body.
Function bodies have their local variables defined by the function that they are enclosed in.
Thus, when the function body is embedded we pass the local types ($t_1^{*}\;t^{*}$) so the body knows how to constrain local variables.
The indexed function type that gets constructed has the precondition of the expected values on the stack turned into indexed types using fresh index variables and the types $t_1^{*}$ from the \wasm type.
The postcondition does the same for the stack with $t_2^{*}$.
We cannot embed imported functions because we have no way of accessing the types of the local variables of the function.

\begin{definition}{$\embed[f]{f}=\mathbluesf{f}$}
    \label{def:embed-f}
    \begin{mathpar}
        \begin{array}{rcl}
            embed_f(\<wfunc> {\begin{stackTL}
                (t_1^{*}\rightarrow t_2^{*})
                \\\<wlocal>\; t^{*}\; e^{*})
            \end{stackTL}}
            &=& \<func>\;
                {\begin{stackTL}
                    (\ti{t_1}{a_1}^{*};\epsilon;(\circ,\ti{t_1}{a_1}^{*})
                    \\ \rightarrow
                    \begin{stackTL}
                        \ti{t_2}{a_2}^{*};\ti{t_1}{a_3}^{*}\;\ti{t}{a_4}^{*};
                        \\ \quad (\circ,\ti{t_2}{a_2}^{*},\ti{t_1}{a_3}^{*}\ti{t}{a_4}^{*}))
                    \end{stackTL}
                    \\ t^{*}\; \embed[e^{*}]{e^{*}}^{(t_1^{*}\;t^{*})}
                \end{stackTL}} \\
            && \<nsend>\\
        \end{array}
    \end{mathpar}
\end{definition}

Tables in \name must also provide the indexed function types of all the functions they contain.
We do this by parameterizing the embedding of the table $tab$ with all of the declared functions $f^{*}$.
Then, we retrieve the indexed function type $tfi$ of the function pointed to by the function index $i$ in $f^{*}$ for every function index $i$ in the table.
We cannot embed imported tables because we have no way of accessing the types of the functions included in the table.

\begin{definition}{$\embed[tab]{tab}^{f^{*}}=\mathbluesf{tab}$}
    \label{def:embed-t}
    \begin{mathpar}
        \begin{array}{rcl}
            embed_{tab}(\<wtab> n\; i^{n})
            &=& \<tab> n\; tfi^{n} \\
            && \text{where } \forall i. f^{*}(i) = \<func> tfi\; \<local>\; t^{*}\; e^{*} \\
        \end{array}
    \end{mathpar}
\end{definition}

Finally, since tables and functions live in modules, which are the pinnacle syntactic object of the \wasm surface syntax hierarchy, we must embed modules.
Embedding a module $module$ means embedding all of the functions $f^{*}$ in the module, and embedding the table $tab$ parameterized with all of the function definitions $f^{*}$.

\begin{definition}{$\embed[module]{module}^{C}=\mathbluesf{module}$}
    \label{def:embed-t}
    \begin{mathpar}
        \begin{array}{rcl}
            embed_{module}(\<module> f^{*}\; glob^{*}\; tab^{?}\; mem^{?})
            &=& \<module>
            \begin{stackTL}
                \embed[f]{f^{*}}
                \\ glob^{*}
                \\ \embed[tab]{tab^{?}}^{f^{*}}
                \\ mem^{?}
            \end{stackTL} \\
        \end{array}
    \end{mathpar}
\end{definition}

These are not the only differences in the surface syntax between \wasm and \name: we also introduced four new instructions (the \prechk-tagged instructions).
The definition of embedding we have introduced has been entirely syntactic, but that will not work for replacing non-\prechk-tagged instructions with \prechk-tagged versions during embedding since we must be able to ensure that stronger guarantees are met.
Instead, one could, for example, check at every $\<div>$, $\<callindirect>$, $\<load>$, and $\<store>$ whether the \prechk-tagged version of the instruction is well-typed, and only if it is well-typed replace the instruction with the \prechk-tagged version.
\subsection{Erasing \name to \wasm}
\label{subsec:erasure}
We provide an erasure function for \name that transforms \name programs into \wasm programs by discarding the extra information from the \name type system and replacing \prechk-tagged instructions with their non-tagged counterparts.
Erasure is useful in the type safety proof because it lets us reuse much of the proof of progress from \wasm (see \autoref{subsec:progress}).
Therefore, we define erasure not just for the surface syntax, like we did for embedding, but also for typing constructs (such as the module type context), administrative instructions, and runtime data structures (such as the store).
We show that erasing a well-typed \name program produces a well-typed \wasm program.

As with the presentation of the embedding, we typeset \name instructions in a \tbsf{blue sans serif font} and \wasm instruction in a \trbf{bold red font}.

Erasing an indexed type function keeps only the primitive \wasm types ($t_1^{*}$ and $t_2^{*}$) from the indexed types representing the stack ($\ti{t_1}{a_1}^{*}$ and $\ti{t_2}{a_2}^{*}$), and discards everything else.

\begin{definition}{$\erase{tfi} = \mathredbold{tf}$}

    $\erase[tfi]{\ti{t_1}{a_1}^{*};l_1;\phi_1 \rightarrow \ti{t_2}{a_2}^{*};l_2;\phi_2} = t_1^{*} \rightarrow t_2^{*}$
\end{definition}

Erasing instructions involves erasing the indexed function types for every instruction that includes it as part of their syntax (blocks and indirect function calls).
We must also remove the \prechk tag from \prechk-tagged instructions to turn them into instructions that exist in \wasm.

\begin{definition}{$\erase[e]{e} = \mathredbold{e}$}
    \begin{mathpar}
        \begin{array}{rcl}
            \erase[e]{\<block>\; tfi\; e^{*} \<end>} &=& \<wblock> \erase[tfi]{tfi}\; \erase[e]{e^{*}} \<wend> \\

            \erase[e]{\<loop>\; tfi\; e^{*} \<end>} &=& \<wloop> \erase[tfi]{tfi}\; \erase[e]{e^{*}} \<wend> \\

            \erase[e]{\<if>\; tfi\; e_1^{*}\; e_2^{*} \<end>} &=& {\begin{stackTL}\<wif> {\begin{stackTL}\erase[tfi]{tfi} \\ \erase[e]{e_1^{*}} \\ \erase[e]{e_2^{*}} \end{stackTL}} \\ \<wend> \end{stackTL}} \\

            \erase[e]{\<callindirect> tfi} &=& \<wcallindirect> \erase[tfi]{tfi} \\

            \erase[e]{t.\<divpc>} &=& t.\<wdiv> \\

            \erase[e]{t.\<callindirectpc>} &=& t.\<wcallindirect> \\

            \erase[e]{t.\<storepc> tp^{?}\; align\; o} &=& t.\<wstore> tp^{?}\; align\; o \\

            \erase[e]{t.\<loadpc> (tp\_sx)^{?}\; align\; o} &=& t.\<wload> (tp\_sx)^{?}\; align\; o \\

            \erase[e]{e} &=& e \text{, otherwise} \\
        \end{array}
    \end{mathpar}
\end{definition}

To erase a module type context, we must erase all of the function types $tfi^{*}$, the table type $(n,tfi_2^{*})$ if one is present, and the postconditions in the label stack $(\ti{t_1}{a_1}^{*};l_1;\phi_1)^{*}$ and the return stack $(\ti{t_2}{a_2}^{*};l_2;\phi_2)^{?}$.
We erasing postconditions the same way we erase the postconditions of indexed function types.
Erasing a table type means discarding the type information about the functions in the table.

\begin{definition}{$\erase[c]{C} = \mathredbold{C}$}
    \begin{mathpar}
        \begin{array}{rcl}
            {\begin{stackTL} erase_C(\{
                {\begin{stackTL}
                    \text{func } tfi^{*}, \text{ global } tg^{*},
                    \\ \text{table } (n,tfi_2^{*})^{?},
                    \\ \text{memory } m^{?}, \text{ local } t^{*},
                    \\ \text{label } (\ti{t_1}{a_1}^{*};l_1;\phi_1)^{*},
                    \\ \text{return } (\ti{t_2}{a_2}^{*};l_2;\phi_2)^{?}\})
                \end{stackTL}}
            \end{stackTL}}
            &=&
            \{{\begin{stackTL}
                \text{func } \erase[tfi]{tfi^{*}},
                \\ \text{global } tg^{*}, \text{ table } n^{?},
                \\ \text{memory }\; m^{?}, \text{local } t^{*},
                \\ \text{label } (t_1^{*})^{*}, \text{ return } (t_2^{*})^{?}\}
            \end{stackTL}}
        \end{array}
    \end{mathpar}
\end{definition}

We show that erasing a \name instruction sequence $e^{*}$, that is well-typed under a module type context $C$, produces a \wasm instruction sequence $e'^{*}=\erase[e]{e^{*}}$ that is well-typed under the erased module type context $C'=\erase[C]{C}$.

\todo{erased-well-typed-typesys}
\todo{Do this one for the typing rules explained in \autoref{sec:wasmtyping}}

To erase a function definition $f$, we simply erase the type declaration $tfi$ and the body $e^{*}$.
We can also erase an imported function by erasing the declared type $tfi$.

\begin{definition}{$\erase[f]{f} = \mathredbold{f}$}
    \begin{mathpar}
        \begin{array}{rcl}
            \erase[f]{\<func> tfi\;\<local>\; t^{*}\; e^{*})}
            &=&
            \<wfunc> \erase[tfi]{tfi}\; \<wlocal>\; t^{*}\; \erase[e]{e^{*}} \\

            \erase[f]{\<func> tfi\; im}
            &=&
            \<wfunc> \erase[tfi]{tfi}\; im \\
        \end{array}
    \end{mathpar}
\end{definition}

We show that erasing a \name function $f$, that is well-typed under a module type context $C$, produces a \wasm function $\erase[f]{f}$ that is well-typed under the erased module type context $\erase[C]{C}$.
This is useful not just for erasing the surface syntax, but also because functions are a part of closures which are used at run-time (as part of module instances and tables).

\todo{erased-well-typed-func lemma}

Erasing a module instance erases all of the functions $f$ in the closures (which we have expanded inline to $\{\text{inst } i, \text{ func } f\}$) within the module instance.

\begin{definition}{$\erase[inst]{inst} = \mathredbold{inst}$}
    \begin{mathpar}
        \begin{array}{rcl}
            {\begin{stackTL} erase_inst(\{
                {\begin{stackTL}
                    \text{func } \{\text{inst } i, \text{ func } f\}^{*},
                    \\ \text{global } v^{*}, \text{ table } i^{?},
                    \\ \text{memory } j^{?}\})
                \end{stackTL}}
            \end{stackTL}}
            &=&
            \{{\begin{stackTL}
                \text{func } \{\text{inst } i, \text{ func } \erase{f}\}^{*},
                \\ \text{global } v^{*}, \text{ table } i^{?},
                \\ \text{memory } j^{?}\}
            \end{stackTL}}
        \end{array}
    \end{mathpar}
\end{definition}

We erase store contexts by erasing all of the module type instances $C^{*}$ and table types $(n,tfi^{*})^{*}$ within.

\begin{definition}{$\erase[S]{S} = \mathredbold{S}$}
    \begin{mathpar}
        \begin{array}{rcl}
            erase_S(\{ {\begin{stackTL}
                    \text{inst } C^{*},
                    \\ \text{tab } (n,tfi^{*})^{*}, \text{mem } m^{*}\}
                \end{stackTL}}
            &=& \{ {\begin{stackTL}
                \text{inst } \erase[c]{C}^{*},
                \\ \text{tab } n^{*}, \text{mem } m^{*}\} \end{stackTL}}
        \end{array}
    \end{mathpar}
\end{definition}

We now prove that if a \name module instance $inst$ has type $C$ under the store context $S$, then the erased \wasm instance $\erase{inst}$ will have the erased type $\erase[c]{C}$ under the erased store context $\erase[S]{S}$.
This will be useful for proving that a well-typed \name store $s$ erases to a well-typed \wasm store $\erase{s}{s}$ since stores contain many instances.

\begin{lemma}{(Erased-Well-Typed-Context)}
    If $S \vdash inst : C$, then $\erase{S} \vdash \erase{inst} : \erase{C}$
\end{lemma}
\begin{proof}
    We want to prove that erasing index information a \name runtime module instance $inst$ will result in a well-typed \wasm runtime module instance $\mathredbold{inst}$.
    To do this, we rely on the above lemmas to safely erase index information from function declarations and table declarations (globals and memory have the same type information in both \name and \wasm).

    \todo{Finish using erased-well-typed-func}
\end{proof}

\todo{erased-well-typed-store}

Before we can proof that erasing a well-typed \name program in reduction form $s;v^{*};e^{*}$ produces a well-typed \wasm program (so we can use erasure in progress), we must prove this property for the \name administrative instruction typing judgement $S;C\vdash e^{*}:tfi$.
Then, we can prove this of \refrule{Admin-Code} and finally of \refrule{Admin-Program}, which gives us the property we want.

\begin{lemma}{(Erased-Well-Typed-Admin)}

    If $S;C \vdash e^{*} : ti_1^{*};l_1;\phi_1 \rightarrow ti_2^{*};l_2;\phi_2$,
    \\ then $\erase{S};\erase{C} \vdash \erase{e^{*}} : \erase{\epsilon;l_1;\phi_1 \rightarrow \ti{t}{a}^{*};l_2;\phi_2}$
\end{lemma}
\begin{proof}
    \todo{Enumerate local, label, and call cl}

    We proceed by induction over typing rules. Most proof cases are omitted as they are simple, but we provide a few to give an idea of what the proofs look like.

    \begin{itemize}
        \item $S;C \vdash t.binop : \ti{t}{a_1}\;\ti{t}{a_2};l_1;\phi_1 \rightarrow \ti{t}{a_3};l_1;\phi_1,\ti{t}{a_3},(= a_3\; (binop\;a_1\;a_2))$

        $\erase{S};\erase{C} \vdash \erase{t.binop} : \erase{\ti{t}{a_1}\;\ti{t}{a_2};l_1;\phi_1 \rightarrow \ti{t}{a_3};l_1;\phi_1,\ti{t}{a_3},(= a_3\; (binop\;a_1\;a_2))}$ = $\erase{S};\erase{C} \vdash t.binop : t\;t \rightarrow t$, which holds under \wasm's type system.
        \item $S;C \vdash \<unreachable> : ti_1^{*};l_1;\phi_1 \rightarrow ti_2^{*};l_2;\phi_2$

        $\erase{S};\erase{C} \vdash \erase{\<unreachable>} : \erase{ti_1^{*};l_1;\phi_1 \rightarrow ti_2^{*};l_2;\phi_2}$ = $\erase{S};\erase{C} \vdash \<wunreachable> : t_1^{*} \rightarrow t_2$, which holds under \wasm's type system.

        \item $S;C \vdash \<drop> : \epsilon;l_1;\phi_1 \rightarrow \epsilon;l_1;\phi_1$

        $\erase{S};\erase{C} \vdash \erase{\<nop>} : \erase{\epsilon;l_1;\phi_1 \rightarrow \epsilon;l_1;\phi_1}$ = $\erase{S};\erase{C} \vdash \<wnop> : \epsilon \rightarrow \epsilon$, which holds under \wasm's type system.
    \end{itemize}
\end{proof}

\todo{erased well typed admin code}

\todo{erased well typed admin program}

\section{Type Safety}
\label{sec:typesafety}
\emph{Type safety} is the property that a well-typed program either reduces to another well-typed program, is an irreducible expression (in the case of \name, a sequence of values), or throws an error (trap, in the case of \name).
Thus, type safety assures us that the behavior of a well-typed program is always well-defined.
The type safety of \wasm guarantees a number of important properties, including memory safety.
Proving the type safety of \name gives us a high degree of assurance that it has the same level of safety as \wasm.

\subsection{Subject Reduction}
\label{subsec:subject-reduction}
\emph{Subject reduction}, also sometimes referred to as ``type preservation'', ensures that if a program has a specific type, then the program will have the same type after a reduction step.
Before we present the subject reduction proof, we first introduce a number of useful lemmas.

\reflemma{Inversion} tells us what typing rules can apply to a given \name instruction sequence, and therefore lets us reason about what the type of that sequence looks like.
For example, if we have a typing derivation, $D$ for $S;C \vdash t.\<const> c : ti_1^{*};l_1;\phi_1 \rightarrow ti_2^{*};l_2;\phi_2$, then we know that $D$ must have at its base \refrule{Const}, because that is the only way we have of typing constant instructions.
$D$ can also include any number of applications of \refrule{Subtyping} and \refrule{Stack-Poly}, because they can be applied to any well-typed sequence of instructions.
Thus, we do not know the exact types, since the typing rules are non-deterministic, but we can reason about the general shape given the base type on top of which \refrule{Subtyping} and \refrule{Stack-Poly} get applied.
Additionally, \refrule{Composition} can be used with the empty sequence and any well-typed single instruction.
However, the addition of \refrule{Composition} with the empty sequence is trivial because the postcondition of an empty instruction sequence must be immediately reachable from the precondition, and therefore the stack and local index store must be the same in both (and the postcondition index type context being reachable from the precondition index type context).

Most cases of \reflemma{Inversion} are omitted.
The complete definition can be found in the appendix (\autoref{sec:subreduxproof}).

\begin{lemma}{\deflemma{Inversion}}

    \begin{itemize}
        %% const
        \item If $S;C \vdash t.\<const> c : ti_1^{*};l_1;\phi_1 \rightarrow ti_2^{*};l_2;\phi_2$,
        then $ti_2^{*} = ti_1^{*}\;\ti{t}{a}$, $l_1 = l_2$,
        and $\phi_1,\ti{t}{a},(= a \; \ti{t}{c}) \implies \phi_2$.

        %% binop
        \item If $S;C \vdash t.binop : ti_1^{*};l_1;\phi_1 \rightarrow ti_2^{*};l_2;\phi_2$,
        then $ti_1^{*} = ti^{*} \; \ti{t}{a_1} \; \ti{t}{a_2}$, $ti_2^{*} = ti^{*} \; \ti{t}{a_3}$, $l_1 = l_2$,
        and $\phi_1,\ti{t}{a_3},(= a_3\;(binop\;a_1\;a_2)) \implies \phi_2$.

        %% block
        \item If $S;C \vdash \<block>\; {\begin{stackTL}(ti_3^{*};l_3;\phi_3 \rightarrow ti_4^m;l_4;\phi_4)\\ e^{*} \<end> : ti_1^{*};l_1;\phi_1 \rightarrow ti_2^{*};l_2;\phi_2\end{stackTL}}$
        \\ then $ti_1^{*} = ti_0^{*}\; ti_3^{*}$, $ti_2^{*}=ti_0^{*}\; ti_4^m$, $l_1=l_3$, $l_2=l_4$, $\phi_1 \implies \phi_3$, $\phi_4 \implies \phi_2$, and $S;C,\text{label}(ti_4^m;l_4;\phi_4) \vdash e^{*} : ti_3^{*};l_3;\phi_3 \rightarrow ti_4^m;l_4;\phi_4$.

        %% br
        \item If $S;C \vdash \<br> i : ti_1^{*};l_1;\phi_1 \rightarrow ti_2^{*};l_2;\phi_2$,
        then $ti_1^{*} = ti_3^{*}\;ti^{*}$, $C_\text{label}(i) = ti^{*};l_1;\phi_3$,
        and $\phi_1 \implies \phi_3$.

        %% call_indirect
        \item If $S;C \vdash \<callindirect> ti_3^{*};l_3;\phi_3 \rightarrow ti_4^{*};l_4;\phi_4 : ti_1^{*};l_1;\phi_1 \rightarrow ti_2^{*};l_2;\phi_2$,
        then $ti_1^{*} = ti_0^{*} \; ti_3^{*}$, $ti_2^{*} = ti_0^{*} \; ti_4^{*}$, $l_2=l_1$, $\phi_1 \implies \phi_3$, and $\phi_3,\phi_4 \implies \phi_2$.

        %% composition
        \item If $S;C \vdash e_1^{*} \; e_2 : ti_1^{*};l_1;\phi_1 \rightarrow ti_3^{*};l_3;\phi_3$,
        then $S;C \vdash e_1^{*} : ti_1^{*};l_1;\phi_1 \rightarrow ti_2^{*};l_2;\phi_2$,
        and $S;C \vdash e_2 : ti_2^{*};l_2;\phi_2 \rightarrow ti_3^{*};l_3;\phi_3$.

        \thought{This relies on a bit of a non-obvious idea that even if the precondition is changed through subtyping, the subtyping can be deferred until a later composition.}

        \thought{This also relies on the idea that even if subtyping appears in the derivation tree, it can equivalently be applied on the premises of the composition.}
    \end{itemize}
\end{lemma}
\begin{proof}
    Proof omitted, but follows from induction over typing derivations.
\end{proof}

The next lemma, \reflemma{Lift-Consts}, shows that if a sequence of constants, $v^n$, has a certain postcondition within a nested context, $L^j$, then it has the same postcondition outside of that context with the precondition of the context.
We use this rule for branching and returning when we have some values $v^n$ inside a reduction context $L^j$.

The intuition for the proof is that the nature of nested contexts are such that all of the instructions preceding $v^n$ are values and therefore only add fresh index variables which are constrained to be equal to constants.
Thus, we can pull $v^n$ outside of the nested context and know that we can still get to the postcondition because we can add back in, using implication, all of the fresh index variables that we would have added from the values preceding.

\begin{lemma}{\deflemma{Lift-Consts}}

    If $S;C \vdash v^n : \epsilon;l_3;\phi_3 \rightarrow ti^n;l_3;\phi_4$ is a subderivation of $S;C \vdash  L^j [v^n] : s_1;l_1;\phi_1 \rightarrow s_2;l_2;\phi_2$,
    \\then $S;C \vdash v^n : \epsilon;l_1;\phi_1 \rightarrow ti^n;l_3;\phi_4$ after reduction
\end{lemma}
\begin{proof}
    By induction on $j$.
    \begin{itemize}
        \item Base case: $j=0$

            We want to show that $S;C \vdash v^n : \epsilon;l_1;\phi_1 \rightarrow ti^n;l_3;\phi_4$ after reduction.

            We have $S;C \vdash v_0^{*} \; v^n \; e^{*} \<end> : s_1;l_1;\phi_1 \rightarrow s_2;l_2;\phi_2$ for some $v_0^{*}$ and $e^{*}$ by expanding $L^0$.

            Then, $S;C \vdash (t.\<const> c)^{*} : \epsilon;l_1;\phi_0 \rightarrow \ti{t}{a}^{*};l_1;\phi_0,\ti{t}{a}^{*},(\<eq> a \; \ti{t}{c})$ where $v_0^{*}=(t.\<const> c)^{*}$ and $\phi_1 \implies \phi_0$ by \reflemma{Inversion} on \refrule{Const}.

            Further, $S;C \vdash v^n : \epsilon;l_3;\phi_0,\ti{t}{a}^{*},(\<eq> a \; \ti{t}{c}) \rightarrow \ti{t}{a}^{*}\;ti^n;l_3;\phi_4$, by \reflemma{Inversion} on \refrule{Const}.

            We now have all the information we need to show what we want to show.

            We know $\phi_0,\ti{t}{a}^{*},(\<eq> a \; \ti{t}{c}) \implies \phi_3$.

            Recall that $S;C \vdash v^n : \epsilon;l_3;\phi_0,\ti{t}{a}^{*},(\<eq> a \; \ti{t}{c}) \rightarrow \ti{t}{a}^{*}\;ti^n;l_3;\phi_4$, then $$S;C \vdash v^n : \ti{t}{a}^{*};l_3;\phi_0,\ti{t}{a}^{*},(\<eq> a \; \ti{t}{c}) \rightarrow \ti{t}{a}^{*},\ti{t}{a}^{*}\;ti^n;l_3;\phi_4$$ by \refrule{Subtyping}.

            If $v_0^{*}$ are not executed (\ie they are not part of the reduced expression), then $a^{*}$ are fresh, so $\phi_0 \implies \phi_0,\ti{t}{a}^{*},(\<eq> a \; \ti{t}{c})$, and therefore $S;C \vdash v^n : \epsilon;l_1;\phi_0 \rightarrow ti^n;l_1;\phi_4$ by \refrule{Subtyping} and since $l_1=l_3$.

            Then, $S;C \vdash v^n : \epsilon;l_1;\phi_1 \rightarrow ti^n;l_1;\phi_4$ by $subtyping$.

        \item Induction case: $j=k+1$

            We want to show that $S;C \vdash v^n : \epsilon;l_1;\phi_1 \rightarrow ti^n;l_3;\phi_4$ after reduction.

            We have $S;C \vdash \<label>_n \{ e_0^{*} \} \; v_0^{*} \; L^k[v^n] \; e_1^{*} \<end> : s_1;l_1;\phi_1 \rightarrow s_2;l_2;\phi_2$ for some $v_0^{*}$, $e_0^{*}$, and $e_1^{*}$ by expanding $L^j$.

            Then, $S;C \vdash (t.\<const> c)^{*} : \epsilon;l_1;\phi_0 \rightarrow \ti{t}{a}^{*};l_1;\phi_0,\ti{t}{a}^{*},(\<eq> a \; \ti{t}{c})$ where $v_0^{*}=(t.\<const> c)^{*}$ and $\phi_1 \implies \phi_0$ by \reflemma{Inversion} on \refrule{Const}.

            Further, $S;C \vdash L^k[v^n] : \ti{t}{a}^{*};l_1;\phi_0,\ti{t}{a}^{*},(\<eq> a \; \ti{t}{c}) \rightarrow s_5;l_5;\phi_5$ for some $s_5;l_5;\phi_5$ by \reflemma{Inversion} on \refrule{Label}.

            Now we can prove want we wanted to show.

            We know $S;C \vdash v^n : \epsilon;l_1;\phi_0,\ti{t}{a}^{*},(\<eq> a \; \ti{t}{c}) \rightarrow ti^n;l_1;\phi_4$ by the inductive hypothesis.

            If $v_0^{*}$ are not executed (\ie after one reduction step), $a^{*}$ are fresh, so $\phi_0 \implies \ti{t}{a}^{*},(\<eq> a \; \ti{t}{c})$, and therefore $S;C \vdash v^n : \epsilon;l_1;\phi_0 \rightarrow \ti{t}{a}^{*}\;ti^n;l_3;\phi_5$ by \refrule{Subtyping} and since $l_1=l_3$.

            Then, $S;C \vdash v^n : \epsilon;l_1;\phi_1 \rightarrow \ti{t}{a}^{*}\;ti^n;l_3;\phi_3$ by \refrule{Subtyping}.

    \end{itemize}
\end{proof}


In many reduction cases, there are values on the stack that get consumed by reducing an instruction.
This creates a bit of a problem because those values represent intermediate state, and as such will introduce new index variables to the index type context in their postcondition.
After reduction, the intermediate state is no longer present, so we lose those index variables from the postconditions.

For example, $(t.\<const> c)\; \<drop>$ could be typed as $\epsilon;l;\phi \rightarrow \epsilon;l;\phi,\ti{t}{a},(= a\; \ti{t}{c})$ where $a$ represent the value on the stack $t.\<const> c$.
This would reduce to $\epsilon$, and then we lose the information about $a$ in the postcondition index type system.
However, this can be solved using implication, as we know $a$ is fresh from the const rule, and therefore we allow saying $\phi \implies \phi,\ti{t}{a},(= a\; \ti{t}{c})$ after reduction.
This pattern will appear in any case of the proof that consumes values.

\begin{theorem}{Subject Reduction}
  If $\vdash_i s;v^{*};e^{*} : ti^{*};l;\phi$ and $s;v^{*};e^{*} \hookrightarrow_i s';v'^{*};e'^{*}$ then $\vdash_i s';v'^{*};e'^{*} : ti^{*};l;\phi$.
\end{theorem}
\begin{proof}
By case analysis on the reduction rules.

\begin{itemize}
    %% Binop -> const
    \item $C\vdash (t.\<const> c_1)\; (t.\<const> c_2)\; t.binop : ti_1^{*};l_1;\phi_1 \rightarrow ti_2^{*};l_2;\phi_2$
    \\ $\land$ $(t.\<const> c_1)\; (t.\<const> c_2)\; t.binop \hookrightarrow t.\<const> c$ where $c=binop(c_1,c_2)$

        By $inversion$ on $const$ and $binop$, we know that $ti_2^{*} = ti_1^{*} \ti{t}{a_3}$, $l_2=l_1$, and that
        \begin{align*}
            \phi_1&,
            \begin{stackTL}
                \ti{t}{a_1}, (= a_1\; \ti{t}{c_1}), \\
                \ti{t}{a_2}, (= a_2\; \ti{t}{c_2}), \\
                \ti{t}{a_3}, (= a_3\; (binop\; a_1\; a_2))
            \end{stackTL} \\
            &\implies \phi_2
        \end{align*}

        By $const$, $C \vdash t.\<const> c :
            \begin{stackTL}
                \epsilon;l_1;\phi_1 \\
                \rightarrow \ti{t}{a_3};l_1;g_1;\phi_1,\ti{t}{a_3},(\<eq> a_3\;\ti{t}{c})
            \end{stackTL}$.

        Because $c=binop_t(c_1,c_2)$, then by $\implies$,
        \begin{align*}
            \phi_1,\ti{t}{a},(= a\; \ti{t}{c}) &\implies \phi_1,
            \begin{stackTL}
                \ti{t}{a_1}, (= a_1\; \ti{t}{c_1}), \\
                \ti{t}{a_2}, (= a_2\; \ti{t}{c_2}), \\
                \ti{t}{a_3}, (= a_3\; (binop\; a_1 a_2))
            \end{stackTL}
        \end{align*}

        Therefore, $C \vdash (t.\<const> c) : ti_1^{*};l_1;\phi_1 \rightarrow ti_1^{*}\; \ti{t}{a_3};l_1;\phi_2$, by $stack-poly$ and $sub-typing$

    %% Binop -> trap
    \item  $C\vdash (t.\<const> c_1)\; (t.\<const> c_2)\; t.binop : ti_1^{*};l_1;\phi_1 \rightarrow ti_2^{*};l_2;\phi_2$
    \\ $\land$ $(t.\<const> c_1)\; (t.\<const> c_2)\; t.binop \hookrightarrow \<trap>$

        Trivially, $C\vdash \<trap> : ti_1^{*};l_1;\phi_1 \rightarrow ti_2^{*};l_2;\phi_2$ by $trap$.

    %% Relop
    \item $C\vdash (t.\<const> c_1)\; (t.\<const> c_2)\; t.relop : ti_1^{*};l_1;\phi_1 \rightarrow ti_2^{*};l_2;\phi_2$
    \\ $\land$ $(t.\<const> c_1)\; (t.\<const> c_2)\; t.relop \hookrightarrow t.\<const> c$ where $c=relop(c_1,c_2)$

        Similar to $binop$.

    %% Testop
    \item $C\vdash (t.\<const> c)\; t.testop : ti_1^{*};l_1;\phi_1 \rightarrow ti_2^{*};l_2;\phi_2$
    \\ $\land$ $(t.\<const> c)\; t.testop \hookrightarrow \<ithreetwo>.\<const> c_2$ where $c_2=testop(c)$

        \todo{This is wonky}

        By $inversion$ on $const$ and $testop$, we know that $ti_2^{*}=ti_1^{*}\; \ti{t}{a_2}$, $l_2=l_1$, and that
        \begin{align*}
            \phi_1&,
            \begin{stackTL}
                \ti{t}{a_1}, (= a_1\;\ti{t}{c}), \\
                \ti{\<ithreetwo>}{a_2}, (= a_2\;(testop\;a_1))
            \end{stackTL} \\
            &\implies \phi_2
        \end{align*}

        By $const$, $C \vdash t.\<const> c :
            \begin{stackTL}
                \epsilon;l_1;g_1;\phi_1 \\
                \rightarrow \ti{\<ithreetwo>}{a_2};l_1;\phi_1,\ti{\<ithreetwo>}{a_2},(= a_2\;\ti{t}{c_2})
            \end{stackTL}$.

        Because $c_2=testop_t(c)$, then by $\implies$,
        \begin{align*}
            \phi_1,\ti{t}{a},(= a\;\ti{t}{c_2}) &\implies \phi_1,
            \begin{stackTL}
                \ti{t}{a_1}, (= a_1\;\ti{t}{c}), \\
                \ti{\<ithreetwo>}{a_2}, (= a_2\;(testop\;a_1))
            \end{stackTL}
        \end{align*}

    %% Unreachable
    \item $C\vdash \<unreachable> : ti_1^{*};l_1;\phi_1 \rightarrow ti_2^{*};l_2;\phi_2$
    \\ $\land$ $\<unreachable> \hookrightarrow \<trap>$

        Trivially, $C\vdash \<trap> : ti_1^{*};l_1;\phi_1 \rightarrow ti_2^{*};l_2;\phi_2$ by $trap$.

    %% Nop
    \item $C\vdash \<nop> : ti_1^{*};l_1;\phi_1 \rightarrow ti_2^{*};l_2;\phi_2$
    \\ $\land$ $\<nop> \hookrightarrow \epsilon$

        By $inversion$ on $nop$, we know that $ti_2^{*} = ti_1^{*}$, $l_2 = l_1$, and $\phi_1 \implies \phi_0$ and $\phi_0 \implies \phi_2$ for some $\phi_0$.

        $C\vdash \epsilon : \epsilon;l;g;\phi_0 \rightarrow \epsilon;l;g;\phi_0$ by $empty$.

        Then, $C \vdash \epsilon ti_1^{*};l;g;\phi_1 \rightarrow ti_1^{*};l;g;\phi_2$ by $stack-poly$ and $sub-typing$.

    %% Drop
    \item $C\vdash (t.\<const> c)\; \<drop> : ti_1^{*};l_1;\phi_1 \rightarrow ti_2^{*};l_2;\phi_2$
    \\ $\land$ $(t.\<const> c)\; \<drop> \hookrightarrow \epsilon$

        By $inversion$ on $compostion$, $const$, and $drop$, we know that $ti_2^{*} = ti_1^{*}$, $l_2 = l_1$, and $\phi_1 \implies \phi_0$ and $\phi_0 \implies \phi_2$ for some $\phi_0$.

        By $empty$, $C\vdash \epsilon : \epsilon;l_1;\phi_0 \rightarrow \epsilon;l_1;\phi_0$.

        Then, $C\vdash \epsilon : ti_1^{*};l_1;\phi_1 \rightarrow ti_1^{*};l_1;\phi_2$ by $stack-poly$ and $sub-typing$.

    %% Select
    \item Case: $C\; {\begin{stackTL}
        \vdash (t.\<const> c_1)\;(t.\<const> c_2)\;(\<ithreetwo>.\<const> 0)\;\<select>
        \\ : ti_1^{*};l_1;\phi_1 \rightarrow ti_2^{*};l_2;\phi_2
    \end{stackTL}}$
    \\ $\land$ $(t.\<const> c_1)\;(t.\<const> c_2)\;(\<ithreetwo>.\<const> 0)\;\<select> \hookrightarrow (t.\<const> c_2)$

        By $const$ and $select$, we know that $ti_2^{*} = ti_1^{*}\;\ti{a_3}$, $l_2 = l_1$, and
        $
        {\begin{stackTL}
            \phi_1, {\begin{stackTL}
                \ti{t}{a_1}, (= a_1\;\ti{t}{c_1}), \\
                \ti{t}{a_2}, (= a_2\;\ti{t}{c_2}), \\
                \ti{\<ithreetwo>}{a}, (= a\;\ti{\<ithreetwo>}{0}), \\
                \ti{t}{a_3},(if\; (= a\; \ti{\<ithreetwo>}{0})\; (= a_3\; a_2)\; (= a_3\; a_1))
            \end{stackTL}} \\
            \implies \phi_2
        \end{stackTL}}
        $

        By $const$, \\
        $ C \vdash (t.\<const> c_2) :
            {\begin{stackTL}
                \epsilon;l_1;\phi_1 \\
                \rightarrow \ti{t}{a_3};l_1;\phi_1,\ti{t}{a_3},(= a_3\; \ti{t}{c_2}) \\
            \end{stackTL}} $

        $C \vdash (t.\<const> c_2) : ti_1^{*};l_1;\phi_1 \rightarrow ti_1^{*}\;\ti{t}{a_3};l_1;\phi_1,\ti{t}{a_3},(= a_3 \; \ti{t}{c_2})$ by $stack-poly$.

        By $\implies$, we have \\
        $\phi_1,\ti{t}{a_3},(\<eq> a_3\; \ti{t}{c_2}) \implies \phi_1, {\begin{stackTL}
            \ti{t}{a_1}, (\<eq> a_1\; \ti{t}{c_1}), \\
            \ti{t}{a_2}, (\<eq> a_2\; \ti{t}{c_2}), \\
            \ti{\<ithreetwo>}{a}, (= a\;\ti{\<ithreetwo>}{0}), \\
            \ti{t}{a_3},(if\; (= a\; \ti{\<ithreetwo>}{0})\; (= a_3\; a_2)\; (= a_3\; a_1))
        \end{stackTL}} \\ $

        Therefore,
        $ C \vdash (t.\<const> c_2) :
        ti_1^{*};l_1;\phi_1
            \rightarrow ti_2^{*}\;\ti{t}{a_3};l_1;\phi_2$ by $sub-typing$

    %% Block
    \item Case: $C \vdash v^n \; \<block> tfi \; e^{*} \<end> : ti_1^{*};l_1;\phi_1 \rightarrow ti_2^{*};l_2;\phi_2$
    \\ $\land$ $v^n \; \<block> tfi \; e^{*} \<end> \hookrightarrow \<label>_m \{ \epsilon \} \; v^n \; e^{*} \<end>$

        Let $ti_3^n;l_3;\phi_3 \rightarrow ti_4^m;l_4;\phi_4=tfi$, $(t.\<const> c)^n=v^n$.

        $C \vdash \<block> tfi \; e^{*} \<end> : ti_1^{*}\; \ti{t}{a}^n;l_1;\phi_1,\ti{t}{a}^n,(\<eq> a \; \ti{t}{c})^n \rightarrow ti_2^{*};l_2;\phi_2$ by $inversion$ on $composition$ and $const$.

        Therefore, by $inversion$ on $block$, $l_1=l_3$ and $l_2=l_4$. We will use $l_1,l_2$ in place of $l_3,l_4$, respectively, for the remainder of the proof case.

        Further, $\ti{t}{a}^n=ti_3^n$, $ti_2^{*}=ti_1^{*}\; ti_4^m$, $\phi_1,\ti{t}{a}^n,(\<eq> a \; \ti{t}{c})^n \implies \phi_3$, and $\phi_4 \implies \phi_2$ by $inversion$ on $block$.

        $C,\text{label}(t_4^{m};l_2;\phi_4) \vdash (t.\<const> c)^n : \epsilon;l_1;\phi_1 \rightarrow \\ \ti{t}{a}^n;l_1;\phi_1,\ti{t}{a}^n,(\<eq> a \; \ti{t}{c})^n$ by $const$.

        $C,\text{label}(t_4^{m};l_2;g_2;\phi_4) \vdash (t.\<const> c)^n : \epsilon;l_1;\phi_1 \rightarrow \\ \ti{t}{a}^n;l_1;\phi_3$ by $sub-typing$.

        $C,\text{label}(t_4^{m};l_2;\phi_4) \vdash e^{*} : \ti{t}{a}^n;l_1;\phi_3 \rightarrow ti_4^m;l_2;\phi_4$ because it is a sub-derivation of $block$ which we have already assumed to hold.

        Then $C,\text{label}(t_4^{m};l_2;\phi_4) \vdash (t.\<const> c)^n\; e^{*} : \epsilon;l_1;\phi_1 \rightarrow \\ ti_4^m;l_2;\phi_4$ by $composition$.

        By $empty$ and $stack-poly$, $C \vdash \epsilon : ti_2^m;l_2;\phi_4 \rightarrow ti_2^m;l_2;\phi_4$.

        Therefore, $C \vdash \<label>_m \{ \epsilon \} \; v^n \; e^{*} \<end> : \epsilon;l_1;\phi_1 \rightarrow ti_2^m;l_2;\phi_4$ by $label$.

        $C \vdash \<label>_m \{ \epsilon \} \; v^n \; e^{*} \<end> : ti_1^{*};l_1;\phi_1 \rightarrow ti_1^{*}\; ti_4^m;l_2;\phi_2$ by $stack-poly$ and $sub-typing$.

    \item Case: $C \vdash v^n \; \<loop> tfi \; e^{*} \<end> : ti_1^{*};l_1;\phi_1 \rightarrow ti_2^{*};l_2;\phi_2$
    \\ $\land$ $v^n \; \<loop> tfi \; e^{*} \<end> \hookrightarrow \<label>_n \{ \<loop> tfi \; e^{*} \<end> \} \; v^n \; e^{*} \<end>$

        Let $ti_3^n;l_3;\phi_3 \rightarrow ti_4^m;l_4;\phi_4=tfi$, $(t.\<const> c)^n=v^n$.

        $C \vdash \<loop> tfi \; e^{*} \<end> : ti_1^{*}\; \ti{t}{a}^n;l_1;\phi_1,\ti{t}{a}^n,(\<eq> a \; \ti{t}{c})^n \rightarrow ti_2^{*};l_2;g_2;\phi_2$ by $inversion$ on $composition$ and $const$.

        Therefore, by $inversion$ on $loop$, $l_1=l_3$ and $l_2=l_4$. We will use $l_1,l_2$ in place of $l_3,l_4$, respectively, for the remainder of the proof case.

        Further, $\ti{t}{a}^n=ti_3^n$, $ti_2^{*}=ti_1^{*}\; ti_4^m$, $\phi_1,\ti{t}{a}^n,(\<eq> a \; \ti{t}{c})^n \implies \phi_3$, and $\phi_4 \implies \phi_2$ by $inversion$ on $loop$.

        $C,\text{label}(t_3^{n};l_1;\phi_3) \vdash (t.\<const> c)^n : \epsilon;l_1;\phi_1 \rightarrow \\ \ti{t}{a}^n;l_1;\phi_1,\ti{t}{a}^n,(\<eq> a \; \ti{t}{c})^n$ by $const$.

        $C,\text{label}(t_3^{n};l_1;\phi_3) \vdash (t.\<const> c)^n : \epsilon;l_1;\phi_1 \rightarrow \\ \ti{t}{a}^n;l_1;\phi_3$ by $sub-typing$.

        $C,\text{label}(t_3^{n};l_1;\phi_3) \vdash e^{*} : ti_1^n;l_1;\phi_3 \rightarrow ti_2^m;l_1;\phi_4$ because it is a sub-derivation of $loop$ which we have already assumed to hold.

        Then $C,\text{label}(t_3^{n};l_1;\phi_3) \vdash (t.\<const> c)^n\; e^{*} : \epsilon;l_1;\phi_1 \rightarrow \\ ti_4^m;l_2;\phi_4$ by $composition$.

        $C \vdash \<loop> tfi \; e^{*} \<end> : \ti{t}{a}^n;l_1;\phi_1,\ti{t}{a}^n,(\<eq> a \; \ti{t}{c})^n \rightarrow ti_4^{m};l_2;g_2;\phi_4$ by $loop$.

        Therefore, $C \vdash \<label>_m \{ \<loop> tfi \; e^{*} \<end> \} \; v^n \; e^{*} \<end> : \epsilon;l_1;\phi_1 \rightarrow ti_4^m;l_2;\phi_4$ by $label$.

        $C \vdash \<label>_m \{ \epsilon \} \; v^n \; e^{*} \<end> : ti_1^{*};l_1;\phi_1 \rightarrow ti_2^{*};l_2;\phi_2$ by $stack-poly$ and $sub-typing$.

    \item Case: $C \vdash (\<ithreetwo>.\<const> 0) \; \<if> tfi \; e_1^{*} \<else> e_2^{*} \<end> : ti_1^{*};l_1;\phi_1 \rightarrow ti_2^{*};l_2;\phi_2$
    \\ $\land$ $(\<ithreetwo>.\<const> 0) \; \<if> tfi \; e_1^{*} \<else> e_2^{*} \<end> \hookrightarrow \<block> tfi \; e_2^{*} \<end>$

        Let $tfi = ti_3^n \; \ti{<ithreetwo>}{a};l_3;\phi_3 \rightarrow ti_4^m;l_4;\phi_4$, \\ $tfi_1 = ti_3^n;l_3;\phi_3,\neg(= a\; \ti{\<ithreetwo>}{0}) \rightarrow ti_4^m;l_4;\phi_4$, \\
        and $tfi_2 = ti_3^n;l_3;\phi_3,(= a\; \ti{\<ithreetwo>}{0}) \rightarrow ti_4^m;l_4;\phi_4$.

        By $inversion$ on $composition$, $const$, and $if$, $ti_1^{*}=ti_0^{*}\; ti_3^{n}$ and $ti_2^{*}=ti_0^{*} \; ti_4^{m}$ for some $ti_0^{*}$, $l_1=l_3$, $l_2=l_4$, $\phi_1,\ti{\<ithreetwo>}{a},(\<eq> a\; 0) \implies \phi_3$, and $\phi_4 \implies \phi_2$.

        $C,\text{label}(ti_4^m;l_4;\phi_4) \vdash e_2^{*} : tfi_2$ because it is a sub-derivation of $if$ which we have assumed to hold.

        Then, $C \vdash \<block> tfi_2 \; e_2^{*} \<end>$ by $block$.

        Since $a$ is fresh after reduction, $\phi_1 \implies \phi_1,\ti{t}{a},(\<eqz> a)$ by $\implies$.

        Therefore, $C \vdash \<block> tfi_2\; e_2^{*} \<end> : \\ ti_0^{*}\; ti_3^n;l_1;\phi_1,\ti{t}{a},(\<eqz> a) \rightarrow s\; ti_0^{*}\;ti_4^m;l_2;\phi_2$ by $extension$ and $sub-typing$.

    \item Case: $C \vdash (\<ithreetwo>.\<const> k+1) \; \<if> tfi \; e_1^{*} \<else> e_2^{*} \<end> : ti_1^{*};l_1;\phi_1 \rightarrow ti_2^{*};l_2;\phi_2$
    \\ $\land$ $(\<ithreetwo>.\<const> k+1) \; \<if> tfi \; e_1^{*} \<else> e_2^{*} \<end> \hookrightarrow \<block> tfi \; e_1^{*} \<end>$

        Similar to above.

    \item Case: $C \vdash \<label>_n \{ e^{*} \} \; v^n \<end> : ti_1^{*};l_1;\phi_1 \rightarrow ti_2^{*};l_2;\phi_2$
    \\ $\land$ $\<label>_n \{ e^{*} \} \; v^n \<end> \hookrightarrow v^n$

        $C \vdash \<label>_n \{ e^{*} \} \; v^n \<end> : \epsilon;l_1;\phi_1 \rightarrow ti_4^{n};l_2;\phi_2$ by $inversion$ on $label$.

        By $inversion$, we know $ti_2^{*}=ti_1^{*}\;ti_4^{n}$.

        $C \vdash v^n : \epsilon;l_1;\phi_1 \rightarrow ti_4^{n};l_2;\phi_2$ because it is a premise of $label$ which we have assumed to hold.

        Therefore, $C \vdash v^n : ti_1^{*};l_1;\phi_1 \rightarrow ti_1^{*}\;ti_4^{n};l_1;\phi_2$ by $stack-poly$.

    \item Case: $C \vdash \<label>_n \{ e^{*} \} \; \<trap> \<end> : ti_1^{*};l_1;\phi_1 \rightarrow ti_2^{*};l_2;\phi_2$
    \\ $\land$ $\<label>_n \{ e^{*} \} \; \<trap> \<end> \hookrightarrow \<trap>$

        Trivially, $C\vdash \<trap> : ti_1^{*};l_1;\phi_1 \rightarrow ti_2^{*};l_2;\phi_2$ by $trap$.

    \item Case: $C \vdash \<label>_n \{ e^{*} \} \; L^j [v^n \; (\<br> j)] \<end> : ti_1^{*};l_1;\phi_1 \rightarrow ti_2^{*};l_2;\phi_2$
    \\ $\land$ $\<label>_n \{ e^{*} \} \; L^j [v^n \; (\<br> j)] \hookrightarrow v^n \; e^{*}$

        By $inversion$, $ti_2^{*}=ti_1^{*}\;ti_4^{*}$.

        Let $(t.\<const> c)^n = v^n$.

        $C,\text{label}(ti_1^n;l_3;\phi_5)^j \vdash v^n\; (\<br> j) : \epsilon;l_3;\phi_3 \rightarrow ti_\emptyset^{*};l_\emptyset;\phi_\emptyset$ for some $l_3$ and $\phi_3$, where $\phi_5=\phi_3,\ti{t}{a}^n,(= a\; \ti{t}{c})^n$, by $inversion$ on $label$ and $br$.

        $C,\text{label}(ti_1^n;l_3;\phi_5)^j \vdash (\<br> j) : ti_1^n;l_3;\phi_5 \rightarrow ti_\emptyset^{*};l_\emptyset;\phi_\emptyset$, by $inversion$ on $composition$ and $const$.

        Then, $C,\text{label}(ti_1^n;l_3;\phi_5)^j \vdash v^n : \epsilon;l_3;\phi_3 \rightarrow ti_1^n;l_3;\phi_5$ since it is a premise of $composition$ which we have assumed to hold.

        $C \vdash e^{*} : ti_1^n;l_3;\phi_5 \rightarrow ti_2^{*};l_2;\phi_4$ since it is a premise of $label$ which we have assumed to hold, and $\phi_4 \implies \phi_2$ by $inversion$ on $label$.

        Then, $C \vdash v^n \; e^{*} : \epsilon;l_1;\phi_1 \rightarrow ti_2^{*};l_2;\phi_4$ by $nested-type-preserved$ and $composition$.

        Finally, $C \vdash v^n \; e^{*} : ti_1^{*};l_1;\phi_1 \rightarrow ti_1^{*}\;ti_4^{*};l_2;\phi_2$ by $stack-poly$ and $sub-typing$.

    \item Case: $C \vdash (\<ithreetwo>.\<const> 0)\;(\<brif> j) : ti_1^{*};l_1;\phi_1 \rightarrow ti_2^{*};l_2;\phi_2$
    \\ $\land$ $(\<ithreetwo>.\<const> 0)\;(\<brif> j) \hookrightarrow \epsilon$

        $ti_1^{*}=ti_2^{*}$, $l_1=l_2$, and $\phi_1,\ti{\<ithreetwo>}{a},(\<eq> a\; \ti{\<ithreetwo>}{0}),(\<eqz> a) \implies \phi_2$ by $inversion$ on $composition$, $const$, and $br \_ if$.

        $C \vdash \epsilon : \epsilon;l_1;\phi_1 \rightarrow \epsilon;l_1;\phi_1$ by $empty$.

        $C \vdash \epsilon : ti_1^{*};l_1;\phi_1 \rightarrow ti_1^{*};l_1;\phi_1$ by $stack-poly$.

        $\phi_1 \implies \phi_1,\ti{\<ithreetwo>}{a},(\<eq> a\; \ti{\<ithreetwo>}{0}),(\<eqz> a)$ because $a$ is fresh after reduction, and therefore $\phi_1 \implies \phi_2$.

        Then, $C \vdash \epsilon : ti_1^{*};l_1;\phi_1 \rightarrow ti_1^{*};l_1;\phi_2$ by $sub-typing$.

    \item Case: $C \vdash (\<ithreetwo>.\<const> k+1)\;(\<brif> j) : ti_1^{*};l_1;\phi_1 \rightarrow ti_2^{*};l_2;\phi_2$
    \\ $\land$ $(\<ithreetwo>.\<const> k+1)\;(\<brif> j) \hookrightarrow \<br> j$

        $C_label(j)=ti_1^{*};l_1;\phi_1,\ti{t}{a},\neg(\<eqz> a)$ because it is a side condition of $br\_if$ which we have assumed to hold.

        $C \vdash \<br> j : ti_1^{*};l_1;\phi_1,\ti{t}{a},\neg(\<eqz> a) \rightarrow ti_2^{*};l_2;\phi_2$ by $br$.

        Because $a$ is fresh after reduction, $\phi_1 \implies \phi_1,\ti{\<ithreetwo>}{a},\neg(\<eqz> a)$.

        Therefore, $C \vdash \<br> j : ti_1^{*};l_1;\phi_1 \rightarrow ti_2^{*};l_2;\phi_2$ by $sub-typing$.

    \item Case: $C \vdash (\<ithreetwo>.\<const> k)\;(\<brtable> j_1^k\; j\; j_2^{*}) : ti_1^{*};l_1;\phi_1 \rightarrow ti_2^{*};l_2;\phi_2$
    \\ $\land$ $(\<ithreetwo>.\<const> k)\;(\<brtable> j_1^k\; j\; j_2^{*}) \hookrightarrow \<br> j$

        By $inversion$, we know that $C_\text{label}(j) = ti^{*};l_1;\phi_3$, $ti_1^{*} = ti_0^{*} \; ti^{*}$ for some $ti_0^{*}$, and $phi_1 \implies \phi_3$.

        $C \vdash \<br> j : ti_1^{*};l_1;\phi_3 \rightarrow ti_2^{*};l_2;\phi_2$ by $br$.

        $C \vdash \<br> j : ti_1^{*};l_1;\phi_1 \rightarrow ti_2^{*};l_2;\phi_2$ by $sub-typing$.

    \item Case: $C \vdash (\<ithreetwo>.\<const> k+n)\;(\<brtable> j_1^k\; j) : ti_1^{*};l_1;\phi_1 \rightarrow ti_2^{*};l_2;\phi_2$
    \\ $\land$ $(\<ithreetwo>.\<const> k+n)\;(\<brtable> j_1^k\; j) \hookrightarrow \<br> j$

        By $inversion$, we know that $C_\text{label}(j) = ti^{*};l_1;\phi_3$, $ti_1^{*} = ti_0^{*} \; ti^{*}$ for some $ti_0^{*}$, and $phi_1 \implies \phi_3$.

        $C \vdash \<br> j : ti_1^{*};l_1;\phi_3 \rightarrow ti_2^{*};l_2;\phi_2$ by $br$.

        $C \vdash \<br> j : ti_1^{*};l_1;\phi_1 \rightarrow ti_2^{*};l_2;\phi_2$ by $sub-typing$.

    \item Case: $S;C \vdash \<call> j : ti_1^{*};l_1;\phi_1 \rightarrow ti_2^{*};l_2;\phi_2$
    \\ $\land$ $s;\<call> j \hookrightarrow_i \<call> s_\text{func}(i,j)$

        By $inversion$, we know that $l_2 = l_1$, $ti_1^{*} = ti^{*} \; ti_3^{*}$, $ti_1^{*} = ti^{*} \; ti_4^{*}$, $\phi_1 \implies \phi_3$, and $\phi_3,\phi_4 \implies \phi_2$, where $ti_3^{*};l_3;\phi_3 \rightarrow ti_4^{*};l_4;\phi_4 = C_\text{func}(j)$.

        We know $S \vdash s_\text{inst}(i) : C$ since it is a premise of $\vdash s : S$ which we have assumed to hold.

        Then we know $S \vdash s_\text{func}(i,j) : ti_3^{*};l_3;\phi_3 \rightarrow ti_4^{*};l_4;\phi_4$ because it is a premise of $S \vdash s_\text{inst}(i) : C$.

        Therefore, $S;C\vdash \<call> s_\text{func}(i,j) : ti_3^{*};l_3;\phi_3 \rightarrow ti_4^{*};l_4;\phi_4$ by $call-cl$.

        $S;C\vdash \<call> s_\text{func}(i,j) : ti_1^{*};l_1;\phi_1 \rightarrow ti_2^{*};l_2;\phi_2$ by $stack-poly$ and $sub-typing$.

    \item Case: $S;C \vdash_i (\<ithreetwo>.\<const> j)\; \<callindirect> tfi : ti_1^{*};l_1;\phi_1 \rightarrow ti_2^{*};l_2;\phi_2$
    \\ $\land$ $s;(\<ithreetwo>.\<const> j)\; \<callindirect> tfi \hookrightarrow_i \<call> s_\text{tab}(i,j)$ where $s_\text{tab}(i,j)_\text{code}=(\<func> tfi_0\; \<local>\; t^{*}\; e^{*})$ and $tfi_0 <: tfi$

        Let $ti_3^{*};l_3;\phi_3 \rightarrow ti_4^{*};l_4;\phi_4 = tfi$

        By $inversion$ on $composition$, $const$, and $call-indirect$, we know that $ti_1^{*}=ti_0^{*}\; ti_3^{*}$ and $ti_2^{*}=ti_0^{*}\; ti_4^{*}$ for some $ti_0^{*}$, $l_1=l_3$, $l_2=l_4$, $\phi_1 \implies \phi_3$, and $\phi_4 \implies \phi_2$.

        $S \vdash s_\text{tab}(i,j) : tfi_0$ since it is a premise of $\vdash s : S$ which we have assumed to hold.

        Then, $S;C \vdash_i \<call> s_\text{tab}(i,j) : tfi_0$ by $call-cl$.

        $S;C \vdash_i \<call> s_\text{tab}(i,j) : tfi$ by $sub-typing$.

        Therefore, $S;C \vdash_i \<call> s_\text{tab}(i,j) : ti_0^{*}\;ti_1^{*};l_1;\phi_1 \rightarrow ti_0^{*}\;ti_1^{*};l_2;\phi_2$ by $stack-poly$.

    \item Case: $S;C \vdash_i (\<ithreetwo>.\<const> j)\; \<callindirect> tfi : ti_1^{*};l_1;\phi_1 \rightarrow ti_2^{*};l_2;\phi_2$
    \\ $\land$ $s;(\<ithreetwo>.\<const> j)\; \<callindirect> tfi \hookrightarrow_i \<trap>$.

        Trivially, $S;C \vdash_i \<trap> : ti_1^{*};l_1;\phi_1 \rightarrow ti_2^{*};l_2;\phi_2$ by $trap$.

    \item Case: $S;C \vdash_i v^n\; \<call> cl : ti_1^{*};l_1;\phi_1 \rightarrow ti_2^{*};l_2;\phi_2$
    \\ $\land$ $s;v^n\; \<call> cl \hookrightarrow_i \<local>_m\{j;v^n \; (t.\<const> 0)^k\} \; \<block> tfi_1\; e^{*} \<end> \<end>$
    \\ where $cl_\text{code} = \<func> tfi_2\; \<local>\; t^k \; e^{*}$ and $cl_\text{inst} = j$

        \todo{This needs an overhaul}

        Let $tfi_0 = ti_1^{*};l_1;\phi_1 \rightarrow ti_2^{*};l_2;\phi_2$, $tfi_1 = \epsilon;l_3;\phi_3 \rightarrow ti_4^{m};l_4;\phi_4$, and $tfi_2 = ti_3^{n};\ti{t_2}{a_2}^{n}\; \ti{t}{a}^k;\phi_3,\ti{t}{a}^k,(= a \;\ti{t}{0})^k \rightarrow ti_4^{m};l_4;\phi_4$ by $inversion$.

        $S;C \vdash (t_2 \<const> c)^n : ti_1^{*};l_1;\phi_1 \rightarrow ti_1^{*}\;ti_5^n;l_1;\phi_1,\ti{t_2}{a_2},(= a_2 \; \ti{t_2}{c})$, where $v^n=(t_2 \<const> c)^n$,
        and $S;C\vdash \<call> cl : ti_1^{*}\;ti_5^n;l_1;\phi_1,\ti{t_2}{a_2}^{n},(= a_2 \; \ti{t_2}{c})^{n} \rightarrow ti^{*}\;ti_2^m;l_2;\phi_2$ because they are premises of $composition$ which we have assumed to hold.

        By inversion, $l_2=l_1$, $ti_2^{*}=ti_1^{*}\;ti_4^m$, $\phi_1,\ti{t_2}{a_2},(\<eq> a_2 \; \ti{t_2}{c}) \implies \phi_3$, $\phi_4 \implies \phi_2$, and $S\vdash cl : tfi_1$.

        Therefore, $C \vdash \<func> tfi_1\; \<local>\; t^k \; e^{*} : tfi_1$ because it is a premise of $S \vdash cl : tfi_1$.

        $S;C,\text{local } t_2^n\; t^k,\text{label }(ti_4^{m};l_4;g_4;\phi_4),\text{return }(ti_4^{m};l_4;g_4;\phi_4) \vdash e^{*}: tfi_2$ because it is a premise of the above derivation.

        $S;C,\text{local } t_2^n\; t^k,\text{return }(ti_4^{m};l_4;g_4;\phi_4) \vdash \<block> tfi_2\; e^{*} \<end> : tfi_2$ by $block$.

        There are now two cases, based on whether or not the called closure was in the current module being called:

        \begin{itemize}
            \item Case: $i=j$
                By $inversion$, $g_1=g_3$ and $g_2=g_4$, so we will use $g_1$ instead of $g_3$ and $g_2$ instead of $g_4$.

                $C \vdash_j v^n \; (t \<const> 0)^k : \epsilon;l_3;g_1;\phi_1 \rightarrow \ti{t_2}{a_2}^n\;\ti{t}{a}^k ;l_3;g_1;\phi_1,\ti{t_2}{a_2},(\<eq> a_2 \; \ti{t_2}{c})^n,\ti{t}{a},(\<eq> a \; \ti{t}{0})^k$ by $const$.

                $\phi_1,\ti{t_2}{a_2},(\<eq> a_2 \; \ti{t_2}{c})^n \implies \phi_3$, and therefore $\phi_1,\ti{t_2}{a_2},(\<eq> a_2 \; \ti{t_2}{c})^n,\ti{t}{a},(\<eq> a \; \ti{t}{0})^k \implies \phi_3,\ti{t}{a}^k,(\<eq> a \;\ti{t}{0})^k$.

                $S;C,\text{local } t_2^n\; t^k,\text{return }(ti_4^{m};l_4;g_2;\phi_4) \vdash \<block> tfi_2\; e^{*} \<end> :  ti_3^{n};\ti{t_2}{a_2}^{n}\; \ti{t}{a}^k;g_1;\phi_1,\ti{t_2}{a_2},(\<eq> a_2 \; \ti{t_2}{c})^n,\ti{t}{a},(\<eq> a \; \ti{t}{0})^k \rightarrow ti_4^{m};l_4;g_2;\phi_4$ by $sub-typing$.

                $S;(ti_4^{m};l_4;g_2;\phi_4) \vdash_j v^n \; (t \<const> 0)^k;\<block> tfi_2\; e^{*} \<end> : \epsilon;l_3;g_1;\phi_1 \rightarrow ti_4^{m};l_4;g_2;\phi_4$ by $with-return$.

                $S;C \vdash_i \<local>_m\{j;v^n \; (t.\<const> 0)^k\} \; \<block> tfi_2\; e^{*} \<end> \<end> : \epsilon;l_1;g_1;\phi_1 \rightarrow \epsilon\;ti_4^m;l_1;g_2;\phi_4$ by $local-same-inst$.

                $S;C \vdash_i \<local>_m\{j;v^n \; (t.\<const> 0)^k\} \; \<block> tfi_2\; e^{*} \<end> \<end> : tfi_0$ by $stack-poly$ and $sub-typing$.

            \item Case: $i \neq j$
                By $inversion$, $g_2=\ti{t_g}{a_3}^{*}$ where $C_\text{global}=(mut?\; t_g)^{*}$ and $a_3^{*}$ are fresh.

                $C \vdash_j v^n \; (t \<const> 0)^k : \epsilon;l_3;g_3;\phi_1 \rightarrow \ti{t_2}{a_2}^n\;\ti{t}{a}^k ;l_3;g_3;\phi_1,\ti{t_2}{a_2},(\<eq> a_2 \; \ti{t_2}{c})^n,\ti{t}{a},(\<eq> a \; \ti{t}{0})^k$ by $const$

                $\phi_1,\ti{t_2}{a_2},(\<eq> a_2 \; \ti{t_2}{c})^n \implies \phi_3$, and therefore $\phi_1,\ti{t_2}{a_2},(\<eq> a_2 \; \ti{t_2}{c})^n,\ti{t}{a},(\<eq> a \; \ti{t}{0})^k \implies \phi_3,\ti{t}{a}^k,(\<eq> a \;\ti{t}{0})^k$.

                $S;C,\text{local } t_2^n\; t^k,\text{return }(ti_4^{m};l_4;g_4;\phi_4) \vdash \<block> tfi_2\; e^{*} \<end> :  ti_3^{n};\ti{t_2}{a_2}^{n}\; \ti{t}{a}^k;g_3;\phi_1,\ti{t_2}{a_2},(\<eq> a_2 \; \ti{t_2}{c})^n,\ti{t}{a},(\<eq> a \; \ti{t}{0})^k \rightarrow ti_4^{m};l_4;g_4;\phi_4$ by $sub-typing$.

                $S;(ti_4^{m};l_4;g_4;\phi_4) \vdash_j v^n \; (t \<const> 0)^k;\<block> tfi_2\; e^{*} \<end> : \epsilon;l_3;g_3;\phi_1 \rightarrow ti_4^{m};l_4;g_4;\phi_4$ by $with-return$.

                $S;C \vdash_i \<local>_m\{j;v^n \; (t.\<const> 0)^k\} \; \<block> tfi_2\; e^{*} \<end> \<end> : \epsilon;l_1;g_1;\phi_1 \rightarrow \epsilon\;ti_4^m;l_1;\ti{t_g}{a_3}^{*};\phi_4$ by $local-diff-inst$.

                $S;C \vdash_i \<local>_m\{j;v^n \; (t.\<const> 0)^k\} \; \<block> tfi_2\; e^{*} \<end> \<end> : tfi_0$ by $stack-poly$ and $sub-typing$.
        \end{itemize}

    \item Case: $S;C \vdash \<local>_n \{ i;v_l^{*} \} \; v^n \<end> : ti_1^{*};l_1;\phi_1 \rightarrow ti_2^{*};l_2;\phi_2$
    \\ $\land$ $\<local>_n \{ i;v^{*}_l \} \; v^n \<end> \hookrightarrow_j \; v^n$

        By \reflemma{Inversion} on \refrule{Local}, $ti_2^{*} = ti_2^{*} \; ti^n$, $l_1 = l_2$,
        $S;(ti^n;l_3;\phi_3) \vdash_i v_l^{*};v^n : ti^n;l_3;\phi_3$,
        and $\phi_1,\phi_3 \implies \phi_2$.

        $(\vdash v_l : ti_l;\phi_l)^{*}$ and $S;C_l \vdash v^n : \epsilon:ti_l^{*};\phi_l^{*} \rightarrow ti^n;l_3;\phi_3$
        because they are premises of \refrule{Code} which we have assumed to hold.

        $\phi_l^{*} = \circ,\ti{t}{a}^{*},(= a\;\ti{t}{c})^{*}$ because it is a premise of \refrule{Admin-Const} which we have assumed to hold.

        By \reflemma{Inversion} on \refrule{Const}, $\phi_l^{*},\phi_v^n \implies \phi_3$.

        Since $a^{*}$ are fresh, $\phi_v^n \implies \phi_3$.

        \thought{Is this enough justification?}

        $S;C \vdash v^n : \epsilon;l_1;\phi_1 \rightarrow ti^n;l_2;\phi_1,\phi_v^n$ by \refrule{Const}.

        $S;C \vdash v^n : \epsilon;l_1;\phi_1 \rightarrow ti^n;l_2;\phi_1,\phi_3$ by \refrule{Implies}.

        $S;C \vdash v^n : \epsilon;l_1;\phi_1 \rightarrow ti^n;l_2;\phi_2$ by \refrule{Implies}.

        Therefore $S;C \vdash v^n : ti_1^{*};l_1;\phi_1 \rightarrow ti_2^{*};l_2;\phi_2$ by \refrule{Stack-Poly}.

    \item Case: $S;C \vdash \<local>_n \{ i;v_l^{*} \} \; \<trap> \<end> : ti_1^{*};l_1;\phi_1 \rightarrow ti_2^{*};l_2;\phi_2$
    \\ $\land$ $\<local>_n \{ i;v_l^{*} \} \; \<trap> \<end> \hookrightarrow \; \<trap>$

        Trivially, $S;C \vdash \<trap> : ti_1^{*};l_1;\phi_1 \rightarrow ti_2^{*};l_2;\phi_2$ by \refrule{Trap}.

    \item Case: $S;C \vdash \<local>_n \{ i;v_l^{*} \} \; L^k[v^n \; \<return>] \<end> : ti_1^{*};l_1;\phi_1 \rightarrow ti_2^{*};l_2;\phi_2$
    \\ $\land$ $\<local>_n \{ i;v_l^{*} \} \; L^k[v^n \; \<return>] \<end> \hookrightarrow_j \; v^n$

        By \reflemma{Inversion} on \refrule{Local}, $ti_2^{*} = ti_1^{*} \; ti^n$, $l_1 = l_2$,
        $S;(ti^n;l_3;\phi_3) \vdash_i v_l^{*};L^k[v^n \; \<return>] : ti^n;l_3;\phi_3$,
        and $\phi_1,\phi_3 \implies \phi_2$.

        $(\vdash v_l : ti_l;\phi_l)^{*}$ and $S;C_l \vdash L^k[v^n \; \<return>] : \epsilon;ti_l^{*};\phi_l^{*} \rightarrow ti^n;l_3;\phi_3$,
        where $C_l = S_\text{inst}(i),\text{local} \; t^{*}, \text{return} \; (ti^n;l_3;\phi_3)$,
        because they are premises of \refrule{Code} that we have assumed to hold.

        $ti_l^{*} = \ti{t_l}{a_l}^{*}$ because it is a premise of \refrule{Admin-Const} which we have assumed to hold.

        By \reflemma{Inversion} on \refrule{Composition},
        $C_l \vdash v^n : ti_4^{*};l_4;\phi_4 \rightarrow ti_5^{*};l_5;\phi_5$,
        and $C_l \vdash \<return> : ti_5^{*};l_5;\phi_5 \rightarrow ti_6^{*};l_6;\phi_6$.

        By \reflemma{Inversion} on \refrule{Return},
        $ti_5^{*} = ti_7^{*}\;ti^n$, $l_5 = l_3$, and $\phi_5 \implies \phi_3$.

        By \reflemma{Inversion} on \refrule{Const},
        $ti_5^{*} = ti_4^{*}\;ti^n$, $l_4 = l_5$,
        and $\phi_4,\phi_v^n \implies \phi_5$.

        $C_l \vdash v^n : \epsilon;l_3;\phi_4 \rightarrow ti^n;l_3;\phi_4,\phi_v^n$ by \refrule{Const}.

        $C_l \vdash v^n : \epsilon;ti_l^{*};\phi_l^{*} \rightarrow ti^n;l_3;\phi_4,\phi_v^n$ by \reflemma{Lift-Consts}.

        By \reflemma{Inversion} on \refrule{Const}, $\phi_l^{*} \implies \phi_4$.

        \thought{This is kinda non-obvious}

        Since $a_l^{*}$ are fresh, $\circ \implies \phi_4$.

        $C_l \vdash v^n : \epsilon;l_1;\phi_1 \rightarrow ti^n;l_2;\phi_1,\phi_v^n$ by \refrule{Const}.

        $C_l \vdash v^n : \epsilon;l_1;\phi_1 \rightarrow ti^n;l_2;\phi_1,\phi_3$ by \refrule{Implies}.

        \thought{This is using like 5 different implication arrows.}

        $C_l \vdash v^n : \epsilon;l_1;\phi_1 \rightarrow ti^n;l_2;\phi_2$ by \refrule{Implies}.

        Therefore, $C_l \vdash v^n : ti_1^{*};l_1;\phi_l^{*} \rightarrow ti_2^{*};l_2;\phi_2$ by \refrule{Stack-Poly}.

        \thought{There has to be a better way to do this proof...}

    \item Case: $S;\epsilon \vdash_i v_1^j\;v\;v_2^k;\<getlocal> j : ti^{*};l;\phi$
    \\ $\land$ $v_1^j\;v\;v_2^k;\<getlocal> j \hookrightarrow v$

        $\vdash v : \ti{t}{a};\phi_v$ and $S;C \vdash \<getlocal> j : \epsilon;l_1;\phi_1^j,\phi_v,\phi_2^k \rightarrow ti^{*};l;\phi$,
        because they are premises of \refrule{Code} that we have assumed to hold.

        $t.\<const> c = v$, and $\phi_v = \circ,\ti{t}{a},(= a\;\ti{t}{c})$, because it is a premise of \refrule{Admin-Const} that we have assumed to hold.

        By \reflemma{Inversion} on \refrule{Get-Local},
        $ti^{*} = \ti{t}{a_2}$, $l_1 = l$, and $\phi_1^j,\phi_v,\phi_2^k,\ti{t}{a_2},(= a_2\;a) \implies \phi$.

        $C \vdash v : \epsilon;l;\phi_1^j,\phi_v,\phi_2^k \rightarrow \ti{t}{a_2};l;\phi_1^j,\phi_v,\phi_2^k,\ti{t}{a_2},(= a_2\;\ti{t}{c})$ by \refrule{Const}.

        $\phi_v,\ti{t}{a_2},(= a_2\;\ti{t}{c}) \iff \phi_v,\ti{t}{a_2},(= a_2\;a)$ trivially.

        $C \vdash v : \epsilon;l;\phi_1^j,\phi_v,\phi_2^k \rightarrow \ti{t}{a_2};l;\phi$ by \refrule{Implies}.

        Therefore, $S;\epsilon \vdash_i v_1^j\;v\;v_2^k;v : ti^{*};l;\phi$ by \refrule{Code}.

    \item Case: $S;\epsilon \vdash_i v_1^j \; v \; v_2^k; v' \; (\<setlocal> j) : ti^{*};l;\phi$
    \\ $\land$ $v_1^j \; v \; v_2^k;v' \; \<setlocal> j \hookrightarrow v_1^j \; v' \; v_2^k;\epsilon$

        $\vdash v : \ti{t}{a};\phi_v$,
        $S;C \vdash v' \; (\<setlocal> j) : \epsilon:l_1;\phi_1^j,\phi_v,\phi_2^k \rightarrow ti^{*};l;\phi$,
        $l_1(j) = \ti{t}{a}$, and $C_\text{local}(j) = t$ because they are premises of \refrule{Code} that we have assumed to hold.

        $t.\<const> c = v$ and $\phi_v = \circ,\ti{t}{a},(= a\;\ti{t}{c})$ because they are premises of \refrule{Admin-Const} that we have assumed to hold.

        By \reflemma{Inversion} on \refrule{Composition},
        $C \vdash v' : \epsilon;l_1;\phi_1^j,\phi_v,\phi_2^k \rightarrow ti_3^{*};l_3;\phi_3$,
        $C \vdash \<setlocal> j : ti_3^{*};l_3;\phi_3 \rightarrow ti^{*};l;\phi$.

        Recalling that $t = C_\text{local}(j)$;
        by \reflemma{Inversion} on \refrule{Set-Local},
        $ti_3^{*} = ti^{*} \; \ti{t}{a'}$,
        $l = l_3[j := \ti{t}{a'}]$,
        and $\phi_3 \implies \phi$.

        By \reflemma{Inversion} on \refrule{Const},
        $t.\<const> c' = v'$, $ti^{*} = \epsilon$, $l_1 = l_3$, and\\
        $\phi_1^j,\phi_v,\phi_2^k,\ti{t}{a'},(= a'\;\ti{t}{c'}) \implies \phi_3$.

        $C \vdash \epsilon : \epsilon;l;\phi \rightarrow \epsilon;l;\phi$ by \refrule{Empty}.

        $C \vdash \epsilon : \epsilon;l;\phi_1^j,\phi_v,\phi_2^k,\ti{t}{a'},(= a'\;\ti{t}{c'}) \rightarrow \epsilon;l;\phi$ by \refrule{Implies}.

        % Since $a$ is fresh in the typing derivation, $C \vdash \epsilon : \epsilon;l;\phi_1^j,\phi_2^k,\ti{t}{a'},(= a'\;\ti{t}{c'}) \rightarrow \epsilon;l;\phi$.

        Since $a$ is fresh, $\phi_1^j,\phi_2^k,\ti{t}{a'},(= a'\;\ti{t}{c'}) \implies \phi_1^j,\phi_v,\phi_2^k,\ti{t}{a'},(= a'\;\ti{t}{c'})$.

        $C \vdash \epsilon : \epsilon;l;\phi_1^j,\phi_2^k,\ti{t}{a'},(= a'\;\ti{t}{c'}) \rightarrow \epsilon;l;\phi$ by \refrule{Implies}.

        $\vdash v' : \ti{t}{a'};\circ,\ti{t}{a'},(= a'\;\ti{t}{c'})$ by \refrule{Admin-Const}.

        Therefore, $S;\epsilon \vdash_i v_1^j\;v'\;v_2^k;\epsilon : ti^n;l;\phi$ by \refrule{Code}.

        \thought{I set out to write a very verbose proof case, but I didn't expect it to be this verbose.}

    \item Case: $C \vdash v \; (\<teelocal> j) : ti_1^{*};l_1;\phi_1 \rightarrow ti_2^{*};l_2;\phi_2$
    \\ $\land$ $v \; (\<teelocal> j) \hookrightarrow v\;v\;(\<setlocal> j)$

        By \reflemma{Inversion} on \refrule{Composition},
        $C \vdash v : ti_1^{*};l_1;\phi_1 \rightarrow ti_3^{*};l_3;\phi_3$,
        and $C \vdash \<teelocal> j : ti_3^{*};l_3;\phi_3 \rightarrow ti_2^{*};l_2;\phi_2$.

        By \reflemma{Inversion} on \refrule{Tee-Local},
        $ti_3^{*} = ti^{*} \; \ti{t}{a}$, $ti_2^{*} = ti^{*} \; \ti{t}{a_2}$, $l_2 = l_3[j := \ti{t}{a}]$,
        and $\phi_3,\ti{t}{a_2},(= a_2\;a) \implies \phi_2$.

        By \reflemma{Inversion} on \refrule{Const},
        $t.\<const> c = v$, $ti_1^{*} = ti^{*}$, $l_3 = l_1$,
        and $\phi_1,\ti{t}{a},(= a\;\ti{t}{c}) \implies \phi_3$.

        $C \vdash v\;v : \epsilon;l_1;\phi_1 \rightarrow \ti{t}{a_2}\;\ti{t}{a};l_1;\phi_1,\ti{t}{a_2},(= a_2\;\ti{t}{c}),\ti{t}{a},(= a\;\ti{t}{c})$ by \refrule{Const}.

        $\ti{t}{a_2},(= a_2\;\ti{t}{c}),\ti{t}{a},(= a\;\ti{t}{c}) \iff \ti{t}{a_2},(= a_2\;a),\ti{t}{a},(= a\;\ti{t}{c})$ trivially.

        $C \vdash v\;v : \epsilon;l_1;\phi_1 \rightarrow \ti{t}{a_2}\;\ti{t}{a};l_1;\phi_2$ by \refrule{Implies}.

        $C \vdash \<setlocal> j : \ti{t}{a};l_1;\phi_2 \rightarrow \epsilon;l_1[j := \ti{t}{a}];\phi_2$ by \refrule{Set-Local}.

        Therefore, $C \vdash v\;v\;(\<setlocal> j) : ti_1^{*};l_1;\phi_1 \rightarrow ti_2^{*};l_2;\phi_2$ by \refrule{Composition} and \refrule{Stack-Poly}.

    \item Case: $C \vdash \<getglobal> j : ti_1^{*};l_1;\phi_1 \rightarrow ti_2^{*};l_2;\phi_2$
    \\ $\land$ $s;\<getglobal> j \hookrightarrow_i s_\text{glob}(i,j)$

        $\vdash s : S$ and $S;\epsilon \vdash_i v^{*};\<getglobal> j : ti^{*};l;\phi$ because they are premises of \refrule{Program} that we have assumed to hold.

        $S;C \vdash \<getglobal> j : \epsilon;l_1;\phi_v^{*} \rightarrow ti^{*};l;\phi$ because it is a premise of \refrule{Code} that we have assumed to hold.

        By \reflemma{Inversion} on \refrule{Get-Global}, $ti^{*} = \ti{t}{a}$, $l = l_1$, $C_\text{global}(j) = \text{mut}^{?} t$,
        and $\phi_v^{*},\ti{t}{a} \implies \phi$.

        $S \vdash s_\text{inst}(i) : C$ because it is a premise of \refrule{Store} that we have assumed to hold.

        Recalling that $C_\text{global}(j) = \text{mut}^{?} t$;
        $\vdash s_\text{glob}(i,j) : \ti{t}{a_1};\phi_1$ because it is a premise of \refrule{Instance} that we have assumed to hold.

        $S;C \vdash t.\<const> c : \epsilon;l;\phi_v^{*} \rightarrow \ti{t}{a};l;\phi_v^{*},\ti{t}{a},(= a \; \ti{t}{c})$ by \refrule{Const},
        where $t.\<const> c = s_\text{glob}(i,j)$.

        $\ti{t}{a},(= a\;\ti{t}{c}) \implies \ti{t}{a}$ trivially.

        $S;C \vdash s_\text{glob}(i,j) : \epsilon;l;\phi_v^{*} \rightarrow \ti{t}{a};l;\phi$ by \refrule{Implies}.

        $S;\epsilon \vdash_i v^{*};s_\text{glob}(i,j) : \ti{t}{a};l;\phi$ by \refrule{Code}, having assumed that the other premises hold.

        \todo{Phrase this better.}

        Therefore, $\vdash_i s;v^{*};s_\text{glob}(i,j) : \ti{t}{a};l;\phi$ by \refrule{Program}.

    \item Case: $\vdash_i s;v_l^{*};v \; (\<setglobal> j) : ti^{*};l;\phi$
    \\ $\land$ $s;v \; (\<setglobal> j) \hookrightarrow_i s';\epsilon$, where $s' = s$ with $\text{glob}(i,j) = v$

        $\vdash s : S$ and $S;\epsilon \vdash_i v_l^{*};v \; (\<setglobal> j) : ti^{*};l;\phi$ because they are premises of \refrule{Program} that we have assumed to hold.

        $S;C \vdash v \; (\<setglobal> j) : \epsilon;l_1;\phi_1 \rightarrow ti^{*};l;\phi$ because it is a premise of \refrule{Code} that we have assumed to hold.

        By \reflemma{Inversion} on \refrule{Composition}, \refrule{Set-Global}, and \refrule{Const},
        $t.\<const> c = v$, $ti^{*} = \epsilon$, $l_1 = l$, $C_\text{global}(j) = \text{mut}\;t$,
        and $\phi_1,\ti{t}{a},(= a\;\ti{t}{c}) \implies \phi$.

        % By \reflemma{Inversion} on \refrule{Composition},
        % $C \vdash v : \epsilon;l_1;\phi_1 \rightarrow ti_2^{*};l_2;\phi_2$,
        % and $C \vdash \<setglobal> j : ti_2^{*};l_2;\phi_2 \rightarrow ti^{*};l;\phi$.

        % By \reflemma{Inversion} on \refrule{Set-Global},
        % $ti_2^{*} = ti^{*} \; \ti{t}{a}$, $l_2 = l$, $C_\text{global}(j) = \text{mut}\;t$,
        % and $\phi_2, \implies \phi$.

        % By \reflemma{Inversion} on \refrule{Const},
        % $t.\<const> c = v$, $ti_2^{*} = \ti{t}{a}$, $l_1 = l_2$,
        % and $\phi_1,\ti{t}{a},(= a\;\ti{t}{c}) \implies \phi_2$.

        $S;C \vdash \epsilon : \epsilon;l;\phi \rightarrow \epsilon;l;\phi$ by \refrule{Empty}.

        Since $a$ is fresh, $\phi_1 \implies \phi_1,\ti{t}{a},(= a\;\ti{t}{c})$.

        $S;C \vdash \epsilon : \epsilon;l;\phi_1 \rightarrow \epsilon;l;\phi$ by \refrule{Implies}.

        $S;\epsilon \vdash_i v_l^{*};\epsilon : ti^{*};l;\phi$ by \refrule{Code}.

        $S \vdash s_\text{inst}(i) : C$ because it is a premise of \refrule{Store} that we have assumed to hold.

        $\vdash s_\text{glob}(i,j) : \ti{t}{a_g};\phi_g$ because it is a premise of \refrule{Instance} that we have assumed to hold.

        $s' : S$ by \refrule{Instance} and \refrule{Store}.

        \thought{This might be skipping too much.}

        Therefore, $\vdash s';\epsilon : ti^{*};l;\phi$ by \refrule{Program}.

    \item Case: $\vdash_i s;v^{*};(\<ithreetwo>.\<const> k)\;(t.\<load> align\;o) : ti^{*};l;\phi$
    \\ $\land$ $s;(\<ithreetwo>.\<const> k)\;(t.\<load> align\;o) \hookrightarrow_i s;t.\<const> \text{const}_t(b^{*})$, where $s_\text{mem}(i,k+o,|t|) = b^{*}$

        $S;\epsilon \vdash_i v^{*};(\<ithreetwo>.\<const> k)\;(t.\<load> align\;o) : ti^{*};l;\phi$ and $\vdash s : S$ because they are premises of \refrule{Program} which we have assumed to hold.

        $S;C \vdash (\<ithreetwo>.\<const> k)\;(t.\<load> align\;o) : \epsilon;l_1;\phi_1 \rightarrow ti^{*};l;\phi$ because it is a premise of \refrule{Code} which we have assumed to hold.

        By \reflemma{Inversion} on \refrule{Composition}, \refrule{Const}, \refrule{Mem-Load},
        $ti^{*} = \ti{t}{a}$, $l_1 = l$, and $\phi_1,\ti{t}{a} \implies \phi$.

        $C \vdash t.\<const> \text{const}_t(b^{*}) : \epsilon;l;\phi_1 \rightarrow \ti{t}{a};l;\phi_1,\ti{t}{a},(= a\;\ti{t}{c})$ by \refrule{Const}.

        $C \vdash t.\<const> \text{const}_t(b^{*}) : \epsilon;l;\phi_1 \rightarrow \ti{t}{a};l;\phi$ by \refrule{Implies}.

        $S;\epsilon \vdash_i v^{*};t.\<const> \text{const}_t(b^{*}) : ti^{*};l;\phi$ by \refrule{Code}.

        Therefore, $s;t.\<const> \text{const}_t(b^{*}) : ti^{*};l;\phi$ by \refrule{Program}.

    \item Case: $\vdash_i s;v^{*};(\<ithreetwo>.\<const> k)\;(t.\<load> tp_sx\;align\;o) : ti^{*};l;\phi$
    \\ $\land$ $s;(\<ithreetwo>.\<const> k)\;(t.\<load> tp_sx\;align\;o) \hookrightarrow_i s;t.\<const> \text{const}_t^{sx}(b^{*})$, where $s_\text{mem}(i,k+o,|tp|) = b^{*}$

        Similar to \refrule{Mem-Load} non-packed.

    \item Case: $C \vdash (\<ithreetwo>.\<const> k)\;(t.\<load> tp\_sx^{?}\;align\;o) : ti_1^{*};l_1;\phi_1 \rightarrow ti_2^{*};l_2;\phi_2$
    \\ $\land$ $s;(\<ithreetwo>.\<const> k)\;(t.\<load> tp\_sx^{?}\;align\;o) \hookrightarrow_i \<trap>$

        Trivially, $C \vdash \<trap> : ti_1^{*};l_1;\phi_1 \rightarrow ti_2^{*};l_2;\phi_2$ by \refrule{Trap}.

    \item Case: $\vdash_i s;v^{*};(\<ithreetwo>.\<const> k)\;(t.\<const> c)\;(t.\<store> align\;o) : ti^{*};l;\phi$
    \\ $\land$ $s;(\<ithreetwo>.\<const> k)\;(t.\<const> c)\;(t.\<store> align\;o) \hookrightarrow_i s';\epsilon$, where $s' = s \text{ with } \text{mem}(i,k+o,|t|) = \text{bits}_t^{|t|}(c)$

        $S;\epsilon \vdash_i v^{*};(\<ithreetwo>.\<const> k)\;(t.\<const> c)\;(t.\<store> align\;o) : ti^{*};l;\phi$ and $\vdash s : S$ because they are premises of \refrule{Program} which we have assumed to hold.

        $S;C \vdash (\<ithreetwo>.\<const> k)\;(t.\<const> c)\;(t.\<store> align\;o) : \epsilon;l_1;\phi_1 \rightarrow ti^{*};l;\phi$ because it is a premise of \refrule{Code} which we have assumed to hold.

        By \reflemma{Inversion} on \refrule{Composition}, \refrule{Const}, and \refrule{Mem-Store},
        $ti^{*} = \epsilon$, $l_1 = l$, and $\phi_1,\ti{\<ithreetwo>}{a_1},(= a_1\;\ti{\<ithreetwo>}{k}),\ti{t}{a_2},(= a_2\;\ti{t}{c}) \implies \phi$.

        Since $a_1$ and $a_2$ are fresh, $\phi_1 \implies \phi$.

        $C \vdash \epsilon : \epsilon;l;\phi_1 \rightarrow \epsilon;l;\phi_1$ by \refrule{Empty}.

        $C \vdash \epsilon : \epsilon;l;\phi_1 \rightarrow \epsilon;l;\phi$ by \refrule{Implies}.

        $s' : S$ trivially.

        \thought{Maybe justify more? Intuitively the contents of the mem function doesn't affect the store typing.}

        $S;\epsilon \vdash_i v^{*};\epsilon : ti^{*};l;\phi$ by \refrule{Code}.

        Therefore, $\vdash_i s';\epsilon : ti^{*};l;\phi$ by \refrule{Program}.

    \item Case: $C \vdash (\<ithreetwo>.\<const> k)\;(t.\<const> c)\;(t.\<store> tp\;align\;o) : ti_1^{*};l_1;\phi_1 \rightarrow ti_2^{*};l_2;\phi_2$
    $\land$ $\vdash s : S$
    \\ $\land$ $s;(\<ithreetwo>.\<const> k)\;(t.\<const> c)\;(t.\<store> tp\;align\;o) \hookrightarrow_i s';\epsilon$, where $s' = s \text{ with } \text{mem}(i,k+o,|tp|)=\text{bits}_t^{|tp|}(c)$

        Similar to \refrule{Mem-Store} non-packed.

    \item Case: $C \vdash (\<ithreetwo>.\<const> k)\;(t.\<const> c)\;(t.\<store> tp^{?}\;align\;o) : ti_1^{*};l_1;\phi_1 \rightarrow ti_2^{*};l_2;\phi_2$
    \\ $\land$ $s;(\<ithreetwo>.\<const> k)\;(t.\<const> c)\;(t.\<store> tp^{?}\;align\;o) \hookrightarrow_i \<trap>$

        Trivially, $C \vdash \<trap> : ti_1^{*};l_1;\phi_1 \rightarrow ti_2^{*};l_2;\phi_2$ by \refrule{Trap}.

    \item Case: $\vdash_i s;v^{*};\<currentmemory> : ti^{*};l;\phi$
    \\ $\land$ $s;\<currentmemory> \hookrightarrow_i \<ithreetwo>.\<const> |s_\text{mem}(i,*)|/64\text{Ki}$

        $S;\epsilon \vdash_i v^{*};\<currentmemory> : ti^{*};l;\phi$, and $\vdash s : S$ because they are premises of \refrule{Program} which we have assumed to hold.

        $S;C \vdash \<currentmemory> : \epsilon;l_1;\phi_1 \rightarrow ti^{*};l;\phi$ because it is a premise of \refrule{Code} which we have assumed to hold.

        By \reflemma{Inversion} on \refrule{Current-Memory}, $ti^{*} = \ti{\<ithreetwo>}{a}$, $l_1 = l$, and $\phi_1,\ti{\<ithreetwo>}{a} \implies \phi$.

        Let $c = |s_\text{mem}(i,*)|/64\text{Ki}$.

        $S;C \vdash \<ithreetwo>.\<const> c : \epsilon;l;\phi_1 \rightarrow \ti{\<ithreetwo>}{a};l;\phi_1,\ti{\<ithreetwo>}{a},(= a\;\ti{\<ithreetwo>}{c})$ by \refrule{Const}.

        $S;C \vdash \<ithreetwo>.\<const> c : \epsilon;l;\phi_1 \rightarrow \ti{\<ithreetwo>}{a};l;\phi$ by \refrule{Implies}.

        $S;\epsilon \vdash_i v^{*};\<ithreetwo>.\<const> c : ti^{*};l;\phi$ by \refrule{Code}.

        Therefore, $\vdash_i s;\<ithreetwo>.\<const> |s_\text{mem}(i,*)|/64\text{Ki} : ti^{*};l;\phi$ by \refrule{Program}.

    \item Case: $\vdash_i s;v^{*};(\<ithreetwo>.\<const> k) \; \<growmemory> : ti^{*};l;\phi$
    \\ $\land$ $s;(\<ithreetwo>.\<const> k) \; \<growmemory> \hookrightarrow_i s';\<ithreetwo>.\<const> |s_\text{mem}(i,*)|/64\text{Ki}$, where $s' = s \text{ with } \text{mem}(i,*) = s_\text{mem}(i,*)(0)^{k \cdot 64\text{Ki}}$

        $S;\epsilon \vdash_i v^{*};(\<ithreetwo>.\<const> k) \; \<growmemory> : ti^{*};l;\phi$, and $\vdash s : S$ because they are premises of \refrule{Program} which we have assumed to hold.

        $S;C \vdash (\<ithreetwo>.\<const> k) \; \<growmemory> : \epsilon;l_1;\phi_1 \rightarrow ti^{*};l;\phi$ because it is a premise of \refrule{Code} which we have assumed to hold.

        By \reflemma{Inversion} on \refrule{Composition}, \refrule{Const}, and \refrule{Grow-Memory},
        $ti^{*} = \ti{\<ithreetwo>}{a_1}$, $l_1 = l$, and $\phi_1,\ti{\<ithreetwo>}{a_2},(= a_2\;\ti{\<ithreetwo>}{k}),\ti{\<ithreetwo>}{a_1} \implies \phi$.

        Since $a_2$ is fresh, $\phi_1,\ti{\<ithreetwo>}{a_1} \implies \phi$.

        $S_\text{mem}(i) \leq |s_\text{mem}(i,*)|$ because it is a premise of \refrule{Store} on $\vdash s : S$ which we have assumed to hold.

        $s' : S$ by \refrule{Store}, using $S_\text{mem}(i) \leq |s_\text{mem}(i,*)(0)^{k \cdot 64\text{Ki}}|$.

        \thought{Maybe explicitly lay out the $\leq$ relationship we're using?}

        Let $c = \<ithreetwo>.\<const> |s_\text{mem}(i,*)|/64\text{Ki}$.

        $C \vdash \<ithreetwo>.\<const> c : \epsilon;l;\phi_1 \rightarrow \ti{\<ithreetwo>}{a_1};l;\phi_1,\ti{\<ithreetwo>}{a_1},(= a_1\;\ti{\<ithreetwo>}{c})$ by \refrule{Const}.

        $C \vdash \<ithreetwo>.\<const> c : \epsilon;l;\phi_1 \rightarrow \ti{\<ithreetwo>}{a_1};l;\phi$ by \refrule{Implies}.

        $S;\epsilon \vdash_i \<ithreetwo>.\<const> c : ti^{*};l;\phi$ by \refrule{Code}.

        Therefore, $\vdash_i s';\<ithreetwo>.\<const> |s_\text{mem}(i,*)|/64\text{Ki} : ti^{*};l;\phi$ by \refrule{Program}.

    \item Case: $\vdash_i s;v^{*};(\<ithreetwo>.\<const> k) \; \<growmemory> : ti^{*};l;\phi$
    \\ $\land$ $s;(\<ithreetwo>.\<const> k) \; \<growmemory> \hookrightarrow_i \<ithreetwo>.\<const> (-1)$

        $S;\epsilon \vdash_i v^{*};(\<ithreetwo>.\<const> k) \; \<growmemory> : ti^{*};l;\phi$, and $\vdash s : S$ because they are premises of \refrule{Program} which we have assumed to hold.

        $S;C \vdash (\<ithreetwo>.\<const> k) \; \<growmemory> : \epsilon;l_1;\phi_1 \rightarrow ti^{*};l;\phi$ because it is a premise of \refrule{Code} which we have assumed to hold.

        By \reflemma{Inversion} on \refrule{Composition}, \refrule{Const}, \refrule{Grow-Memory},
        $ti^{*} = \ti{\<ithreetwo>}{a_1}$, $l_1 = l$, and $\phi_1,\ti{\<ithreetwo>}{a_2},(= a_2\;\ti{\<ithreetwo>}{k}),\ti{\<ithreetwo>}{a_1} \implies \phi$.

        Since $a_2$ is fresh, $\phi_1,\ti{\<ithreetwo>}{a_1} \implies \phi$.

        $C \vdash \<ithreetwo>.\<const> (-1) : \epsilon;l;\phi_1 \rightarrow \ti{\<ithreetwo>}{a_1};l;\phi_1,\ti{\<ithreetwo>}{a_1},(= a_1\;\ti{\<ithreetwo>}{(-1)})$ by \refrule{Const}.

        $C \vdash \<ithreetwo>.\<const> (-1) : \epsilon;l;\phi_1 \rightarrow \ti{\<ithreetwo>}{a_1};l;\phi$ by \refrule{Implies}.

        $S;\epsilon \vdash_i \<ithreetwo>.\<const> (-1) : ti^{*};l;\phi$ by \refrule{Code}.

        Therefore, $\vdash_i s;\<ithreetwo>.\<const> (-1) : ti^{*};l;\phi$ by \refrule{Program}.

\end{itemize}
\end{proof}

\subsection{Progress}
\label{subsec:progress}
\emph{Progress} ensures that if a program is well typed then it either: entirely consists of values, traps, or is reducible (\ie there exists another program that it reduces to).
Proving progress for \name is the key metatheoretic property that ensures that our claim that \name is as safe as \wasm is valid.
This is because it connects the static guarantees of the type system to the dynamic assumptions of \prechk-tagged instructions.
By proving that well-typed \prechk-tagged instructions will always be reducible, we prove that the static guarantees are sufficient to ensure that they will not trap and therefore the dynamic checks are unnecessary.

Since most \name instructions have the same semantics as in \wasm, and every \name type includes all the information of a \wasm type, we can reuse the \wasm proof for those instructions by using the erasure function from Section \ref{subsec:erasure}.
The intuition for this is that the \name indexed type system provides strictly more information than the \wasm type system.
However, for \name instructions that do not have the same semantics as in \wasm, specifically \prechk-tagged instructions, we still must prove those cases.

\begin{theorem}{Progress}
    If $\vdash_i s;v^{*};e^{*} : ti^{*};l;\phi$ then either $e^{*} = v'^{*}$, $e^{*}= \<trap>$, or $s;v^{*};e^{*} \hookrightarrow_i s';v'^{*};e'^{*}$.
\end{theorem}
\begin{proof}
    We proceed by induction on $\vdash_i s;v^{*};e^{*} : ti^{*};l;\phi$.

    Because $\vdash_i s;v^{*};e^{*} : ti^{*};l;\phi$, we know that $\vdash s : S$ for some $S$, and that $S; \epsilon \vdash_i v^{*};e^{*}:ti^{*};l;\phi$ because they are premises of \refrule{Program} which we have assumed to hold.

    Then we know that $(\vdash v: \ti{t_v}{a_v};\phi_v)^{*}$ and $S;S_\text{inst}(i),\text{local } t_v^{*} \vdash e^{*} : \epsilon;\ti{t_v}{a_v}^{*};\phi_v^{*} \rightarrow ti^{*};l;\phi$ because they are premises of \refrule{Code} which we have assumed to hold.

    \begin{itemize}
        \item Case: $\vdash_i s;v^{*};(t.\<const> c_1)\;(t.\<const> c_2)\;t.\<divpc>$

        We must show that $(t.\<const> c_1)\;(t.\<const> c_2)\;t.\<divpc> \hookrightarrow e'^{*}$ for some $e'^{*}$.

        We have $S;() \vdash_i v^{*};(t.\<const> c_1)\;(t.\<const> c_2)\;t.\<divpc> : ti^{*};l;\phi$ for some $ti^{*}$, $l$, and $\phi$ because it is a premise of \refrule{Program} which we have assumed to hold.

        Then, $(\vdash v : \ti{t_v}{a_v};\phi_v)^{*}$ for some $\ti{t_v}{a_v}^{*}$ and $\phi_v^{*}$, since it is a premise of \refrule{Code} which we have assumed to hold.

        It is important to note that $\phi_v^{*}$ cannot contain a contradiction because it contains a single equality constraint per fresh index variable (see \refrule{Admin-Const}).

        Further,
        $$S;S_\text{inst}(i),\text{local } t_v^{*}\;
        {\begin{stackTL}
            \vdash (t.\<const> c_1)\;(t.\<const> c_2)\;t.\<divpc>
            \\: \epsilon;\ti{t_v}{a_v}^{*});\phi_v^{*} \rightarrow ti^{*};l;\phi
        \end{stackTL}}$$
        because it too is a premise of \refrule{Code}.

        Then,
        $$S_\text{inst}(i)
        {\begin{stackTL}
            \vdash (t.\<const> c_1)\;(t.\<const> c_2)
            \\ : \epsilon;\ti{t_v}{a_v}^{*});\phi_v^{*}
            \\ \;\; \rightarrow \ti{t}{a_1}\;\ti{t}{a_2};\ti{t_v}{a_v}^{*});\phi_v^{*},
            {\begin{stackTL}
                \ti{t}{a_1},(= a_1\; \ti{t}{c_1}),
                \\ \ti{t}{a_2},(= a_2\; \ti{t}{c_2})
            \end{stackTL}}
        \end{stackTL}}$$
        where $\phi_v^{*},\ti{t}{a_1},(= a_1\; \ti{t}{c_1}),\ti{t}{a_2},(= a_2\; \ti{t}{c_2}) \implies \neg(= a_2\; \ti{t}{0})$ by \reflemma{Inversion} on \refrule{Composition} and \refrule{Div-Prechk}.

        Therefore, it must be the case that $c_2\neq 0$, and therefore there must exist some $c_3$ such that $c_3=div(c_1,c_2)$ since $div(c_1,c_2)$ is well-defined when $c_2$ is non-zero.
        Then, $s;(t.\<const> c_1)\;(t.\<const> c_2)\;t.\<divpc> \hookrightarrow_i (t.\<const> c_3)$.

        \item Case: $\vdash_i s;v^{*};(\<ithreetwo>.\<const> k)\;(t.\<loadpc> (tp\_sx)\; align\;o)$

        We must show that $s;(\<ithreetwo>.\<const> k)\;(t.\<loadpc> (tp\_sx)\; align\;o) \hookrightarrow e'^{*}$ for some $e'^{*}$.

        We have $S;\epsilon \vdash_i v^{*};(\<ithreetwo>.\<const> k)\;(t.\<loadpc> (tp\_sx)\; align\;o) : ti^{*};l;\phi$ for some $ti^{*}$, $l$, and $\phi$ because it is a premise of \refrule{Program} which we have assumed to hold.

        We also have that $\vdash s : S$, and therefore $(n \leq |b^{*}|)^{*}$ where $S_\text{tab}=n^{*}$ and $s_\text{mem}=(b^{*})^{*}$.

        Then, $(\vdash v : \ti{t_v}{a_v};\phi_v)^{*}$ for some $\ti{t_v}{a_v}^{*}$ and $\phi_v^{*}$, since it is a premise of \refrule{Code} which we have assumed to hold.

        It is important to note that $\phi_v^{*}$ cannot contain a contradiction because it contains a single equality constraint per fresh index variable (see \refrule{Admin-Const}).

        Further, we have that
        $$S;S_\text{inst}(i),\text{local } t_v^{*}\;
        {\begin{stackTL}
            \vdash (\<ithreetwo>.\<const> k)\;(t.\<loadpc> (tp\_sx)\; align\;o)
            \\ : \epsilon;\ti{t_v}{a_v}^{*};\phi_v^{*} \rightarrow ti^{*};l;\phi
        \end{stackTL}}$$
        because it too is a premise of \refrule{Code}.

        Then,
        $$S_\text{inst}(i) \vdash (\<ithreetwo>.\<const> k) :
        {\begin{stackTL}
            \epsilon;\ti{t_v}{a_v}^{*};\phi_v^{*}
            \\ \rightarrow \ti{\<ithreetwo>}{a};\ti{t_v}{a_v}^{*};\phi_v^{*}, \ti{\<ithreetwo>}{a},(= a\; \ti{\<ithreetwo>}{k})
        \end{stackTL}}$$
        where
        $$\phi_v^{*},\ti{\<ithreetwo>}{a},(= a\; \ti{\<ithreetwo>}{k}) \implies
        {\begin{stackTL}
            (\<ge> (\<add> a\; \ti{\<ithreetwo>}{o}) \ti{\<ithreetwo>}{0}),
            \\ (\<le>
            {\begin{stackTL}
                (\<add> a\; (\<add> \ti{\<ithreetwo>}{o+width}))
                \\ \ti{\<ithreetwo>}{n_2*64 \text{Ki}})
            \end{stackTL}}
        \end{stackTL}}$$ and $n_2*64 \text{Ki} = S_\text{mem}(i,j)$
        by \reflemma{Inversion} on \refrule{Composition} and \refrule{Store-Prechk}.

        Because we have
        $$\phi_v^{*},\ti{\<ithreetwo>}{a},(= a\; \ti{\<ithreetwo>}{k}) \implies
        {\begin{stackTL}
            (\<ge> (\<add> a\; \ti{\<ithreetwo>}{o}) \ti{\<ithreetwo>}{0}),
            \\ (\<le>
            {\begin{stackTL}
                (\<add> a\; (\<add> \ti{\<ithreetwo>}{o+width}))
                \\ \ti{\<ithreetwo>}{n_2*64 \text{Ki}})
            \end{stackTL}}
        \end{stackTL}}$$, then we must have $k + o \geq 0$ and $k+o+|tp| \leq n_2*64 \text{Ki}$.

        Recall $\vdash s : S$.
        Then, since $n_2*64 \text{Ki} = S_\text{mem}(i,j)$, we have $s_\text{mem}(i,j)=b_2^{*}$ where $n_2*64 \text{Ki} \leq |b_2^{*}|$.

        Therefore, it must be the case that $k+o \geq 0$ and $k+o+|tp|<|b_2^{*}|$, and therefore $s_\text{mem}(i,k+o,|tp|)=b_3^{*}$ for some $b_3^{*}$ that is a subsequence of $b_2^{*}$.
        Then, $s;(\<ithreetwo>.\<const> k)\;(t.\<loadpc> (tp\_sx)\; align\;o) \hookrightarrow_i t.\<const> \text{const}_t^{sx}(b_3^{*})$.

        \item Case: $\vdash_i s;v^{*};(\<ithreetwo>.\<const> k)\;t.\<loadpc> align\;o$

        Same as above, except with $|t|$ replacing $|tp|$ and $\text{const}_t(b_3^{*})$ instead of $\text{const}_t^{sx}(b_3^{*})$.

        \item Case: $\vdash_i s;v^{*};(\<ithreetwo>.\<const> k)\;(t.\<const> c)\;(t.\<storepc> tp\; align\; o)$

        We must show that $s;(\<ithreetwo>.\<const> k)\;(t.\<storepc> tp\; align\;o) \hookrightarrow e'^{*}$ for some $e'^{*}$.

        We have $$S;() \vdash_i v^{*};(\<ithreetwo>.\<const> k)\;(t.\<storepc> (tp\_sx)\; align\;o) : ti^{*};l;\phi$$ for some $ti^{*}$, $l$, and $\phi$ because it is a premise of \refrule{Program} which we have assumed to hold.

        We also have that $\vdash s : S$, and therefore $(n \leq |b^{*}|)^{*}$ where $S_\text{tab}=n^{*}$ and $s_\text{mem}=(b^{*})^{*}$.

        Then, $(\vdash v : \ti{t_v}{a_v};\phi_v)^{*}$ for some $\ti{t_v}{a_v}^{*}$ and $\phi_v^{*}$, since it is a premise of \refrule{Code} which we have assumed to hold.

        It is important to note that $\phi_v^{*}$ cannot contain a contradiction because it contains a single equality constraint per fresh index variable (see \refrule{Admin-Const}).

        Further, we have that
        $$S;S_\text{inst}(i),\text{local } t_v^{*}\;
        {\begin{stackTL}
            \vdash (\<ithreetwo>.\<const> k)\;(t.\<storepc> tp\; align\;o)
            \\ : \epsilon;\ti{t_v}{a_v}^{*};\phi_v^{*} \rightarrow ti^{*};l;\phi
        \end{stackTL}}$$
        because it too is a premise of \refrule{Code}.

        Then,
        $$S_\text{inst}(i)
        {\begin{stackTL}
            \;\vdash (\<ithreetwo>.\<const> k)\;(t.\<const> c) :
            \\ {\begin{stackTL}
                \epsilon ; \ti{t_v}{a_v}^{*};\phi_v^{*}
                \\ \rightarrow \ti{\<ithreetwo>}{a}\;\ti{t}{a_2};\ti{t_v}{a_v}^{*};\phi_v^{*},
                {\begin{stackTL}
                    \ti{\<ithreetwo>}{a},(= a\; \ti{\<ithreetwo>}{k}),
                    \\ \ti{t}{a_2},(= a_2\;\ti{t}{c})
                \end{stackTL}}
        \end{stackTL}}
        \end{stackTL}}$$
        where
        $$\phi_v^{*},{\begin{stackTL}
            \ti{\<ithreetwo>}{a},(= a\; \ti{\<ithreetwo>}{k}),
            \\ \ti{t}{a_2},(= a_2\;\ti{t}{c})
        \end{stackTL}} \implies
        {\begin{stackTL}
            (\<ge> (\<add> a\; \ti{\<ithreetwo>}{o}) \ti{\<ithreetwo>}{0}),
            \\ (\<le>
            {\begin{stackTL}
                (\<add> a\; (\<add> \ti{\<ithreetwo>}{o+width}))
                \\ \ti{\<ithreetwo>}{n_2*64 \text{Ki}})
            \end{stackTL}}
        \end{stackTL}}$$ and $n_2*64 \text{Ki} = S_\text{mem}(i,j)$
        by \reflemma{Inversion} on \refrule{Composition} and \refrule{Load-Prechk}.

        Because we have
        $${\begin{stackTL}
            \ti{\<ithreetwo>}{a},(= a\; \ti{\<ithreetwo>}{k}),
            \\ \ti{t}{a_2},(= a_2\;\ti{t}{c})
        \end{stackTL}} \implies
        {\begin{stackTL}
            (\<ge> (\<add> a\; \ti{\<ithreetwo>}{o}) \ti{\<ithreetwo>}{0}),
            \\ (\<le>
            {\begin{stackTL}
                (\<add> a\; (\<add> \ti{\<ithreetwo>}{o+width}))
                \\ \ti{\<ithreetwo>}{n_2*64 \text{Ki}})
            \end{stackTL}}
        \end{stackTL}}$$, then we must have $k + o \geq 0$ and $k+o+|tp| \leq n_2*64 \text{Ki}$.

        Recall $\vdash s : S$.
        Then, since $n_2*64 \text{Ki} = S_\text{mem}(i,j)$, we have $s_\text{mem}(i,j)=b_2^{*}$ where $n_2*64 \text{Ki} \leq |b_2^{*}|$.

        It must be the case that $k+o \geq 0$ and $k+o+|tp|<|b_2^{*}|$, and therefore $s_\text{mem}(i,k+0,|tp|)=b_3^{*}$ for some $b_3^{*}$ that is a subsequence of $b_2^{*}$
        Then, we can construct $s'= s$ with $s'_\text{mem}(i,k+o,|tp|)=bits_t^{|tp|}(c)$ because $|bits_t^{|tp|}(c)|=|b_3^{*}|$.
        Then, $$s;(\<ithreetwo>.\<const> k)\;(\<ithreetwo>.\<const> c)\;(t.\<storepc> tp\; align\;o) \hookrightarrow_i s';\epsilon$$

        \item Case: $\vdash_i s;v^{*};(\<ithreetwo>.\<const> c)\;(t.\<storepc> align\;o)$

        Same as above, except with $|t|$ replacing $|tp|$.

        \item Case: $\vdash_i (\<ithreetwo>.\<const> c)\;\<callindirect> ti_1^{*};l_1;\phi_1 \rightarrow ti_2^{*};l_2;\phi_2$

        We must show that $(\<ithreetwo>.\<const> c)\;\<callindirect> ti_1^{*};l_1;\phi_1 \rightarrow ti_2^{*};l_2;\phi_2 \hookrightarrow e'^{*}$ for some $e'^{*}$.

        We have $S;() \vdash_i v^{*};(\<ithreetwo>.\<const> c)\;\<callindirect> ti_1^{*};l_1;\phi_1 \rightarrow ti_2^{*};l_2;\phi_2 : ti^{*};l;\phi$ for some $ti^{*}$, $l$, and $\phi$ because it is a premise of \refrule{Program} which we have assumed to hold.

        We also have that $\vdash s : S$, and therefore $S_\text{tab}(i)=(n,tfi^{n})$ and $(S \vdash cl : tfi)^{*}$ where $s_\text{tab}(i)=cl^{*}$ and $n\leq |cl^{*}|$.

        Then, $(\vdash v : \ti{t_v}{a_v};\phi_v)^{*}$ for some $\ti{t_v}{a_v}^{*}$ and $\phi_v^{*}$, since it is a premise of \refrule{Code} which we have assumed to hold.

        It is important to note that $\phi_v^{*}$ cannot contain a contradiction because it contains a single equality constraint per fresh index variable (see \refrule{Admin-Const}).

        Then,
        $$S_\text{inst}(i)
        {\begin{stackTL}
            \;\vdash (\<ithreetwo>.\<const> c) :
            \\ {\begin{stackTL}
                \epsilon ; \ti{t_v}{a_v}^{*};\phi_v^{*}
                \\ \rightarrow \ti{\<ithreetwo>}{a};\ti{t_v}{a_v}^{*};\phi_v^{*},\ti{\<ithreetwo>}{a},(= a\; \ti{\<ithreetwo>}{c})
        \end{stackTL}}
        \end{stackTL}}$$
        where $\phi_v^{*},\ti{\<ithreetwo>}{a},(= a\; \ti{\<ithreetwo>}{c}) \implies (\<gt>\; n\; a) \land (\<le> \ti{\<ithreetwo>}{0}\; a) $
        by \reflemma{Inversion} on \refrule{Composition} and \refrule{Call-Indirect-Prechk}.

        We have
        $$\forall i.\; (\phi \implies \neg (= \ti{\<ithreetwo>}{i}\; a)) \lor\; tfis(i) <: ti_1^{*};l_1;\phi_1 \rightarrow ti_2^{*};l_2;\phi_2$$
        where $tfis=tfi^{n}$, because it is a premise of \refrule{Call-Indirect-Prechk} which we have assumed to hold by \reflemma{Inversion}.
        Since $\ti{\<ithreetwo>}{a};\ti{t_v}{a_v}^{*};\phi_v^{*},\ti{\<ithreetwo>}{a},(= a\; \ti{\<ithreetwo>}{c}) \implies (= \ti{\<ithreetwo>}{c}\; a)$, then it has to be the case that $tfis(c) <: ti_1^{*};l_1;\phi_1 \rightarrow ti_2^{*};l_2;\phi_2$.

        Let, $\{\text{inst } j, \text{ func } f\} = s_\text{tab}(i,c)$.
        Recall from before that $(S\vdash cl: tfi)^{*}$.
        Then, $S \vdash \{\text{inst } j, \text{ func } f\} : tfi_2$ for some $tfi_2$.

        $S_\text{inst}(j) \vdash f : tfi_2$, as it is a premise of $S \vdash \{\text{inst } j, \text{ func } f\} : tfi_2$.

        Then, we know that $f= \<func> tfi_2\; \<local> \dots$ because it is a premise of $S_\text{inst}(j) \vdash f : tfi_2$, and we know that $tfi_2<:ti_1^{*};l_1;\phi_1 \rightarrow ti_2^{*};l_2;\phi_2$.

        Thus, ${\begin{stackTL}s;(\<ithreetwo>.\<const> c)\;\<callindirect> ti_1^{*};l_1;\phi_1 \rightarrow ti_2^{*};l_2;\phi_2
            \\ \hookrightarrow \<call> \{\text{inst } j, \text{ func } f\}\end{stackTL}}$.

    \item Otherwise, we reuse the \wasm proof, which we can do thanks to \autoref{thm:programerasure}.
    \end{itemize}
\end{proof}


\chapter{Discussion}
\label{chp:discussion}

\section{Implementation Details}
In practice, we reason about implication using the Z3 theorem prover \todo{citation needed}.
To test whether the satisfiability of one index type context $\phi_1$ implies that some other index type context $\phi_2$ is satisfiable, we first generate Z3 constraints based on the propositions in both contexts.
Then, we assert that the constraints generated for the first context must hold.
Finally, we ask Z3 to find an assignment to the variables declared in the type index contexts where the constraints from the second context do not hold (a counterexample).
If a counterexample cannot be found then the implication must hold, otherwise it does not hold.

\paragraph{Impact of using Z3}
Our choice of using Z3 has impacted \name in several ways.
\begin{itemize}
    \item The biggest impact is that we currently do not supporting floating point values and certain unary and binary operators because Z3 is unable to reason about them.
    \item Because of how we test implication, we require that the index type contexts use the same names to refer to the same index variables.
    \item The requirement of adding explicit type annotations for index variables and constants that appear in type indices comes from needing to know what width the variable will be when we convert it to a Z3 bitvector.
\end{itemize}

\chapter{Conclusion}
\label{chp:conclusion}
We have introduced \name, a low-level language that uses an expressive type system to potentially improve performance via the elimination of unnecessary run-time checks.
\name is based on \wasm, a real world language commonly used in performance-critical and untrusted contexts, where both safety and performance are critical.
To ensure the safety of \name, we have proven the type safety of the \name language as well as showing a sound type erasure to \wasm, demonstrating that \name is at least as safe as \wasm.
We have shown that an indexed type system can be used in a low-level language to reduce the number of dynamic checks required, without sacrificing safety and security guarantees or increasing the programmer's proof burden.
We built a reference interpreter for \name to demonstrate the practicality of implementing a type checker for \name.
This demonstrates the usefulness of using expressive type systems as a practical tool to improve performance and ensure safety for low-level languages in real use cases.


\bibliographystyle{plainnat}
\bibliography{citations}

\end{document}

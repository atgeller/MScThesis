\documentclass[draft,msc]{ubcthesis}
\usepackage[usenames]{color}
\usepackage{savesym}
\usepackage{amsmath}
\usepackage{amsthm}
\savesymbol{program}
\savesymbol{@program}
\usepackage{semantic}   % Tools for typesetting PL semantics
\usepackage{braket}     % Easy angle-bracket notation
\restoresymbol{}{program}
\restoresymbol{}{@program}
\usepackage{syntax}
\usepackage{mathpartir}
\usepackage{xspace}
\usepackage{dblfloatfix}
\usepackage{multicol}
\usepackage{subcaption}
\usepackage{tikz}
\usepackage{stmaryrd}

% Always Use these
\usepackage{hyperref}
\usepackage{microtype}
\usepackage[utf8]{inputenc}
\usepackage[T1]{fontenc}

\newcommand{\shrug}[1][]{%
\begin{tikzpicture}[baseline,x=0.8\ht\strutbox,y=0.8\ht\strutbox,line width=0.125ex,#1]
\def\arm{(-2.5,0.95) to (-2,0.95) (-1.9,1) to (-1.5,0) (-1.35,0) to (-0.8,0)};
\draw \arm;
\draw[xscale=-1] \arm;
\def\headpart{(0.6,0) arc[start angle=-40, end angle=40,x radius=0.6,y radius=0.8]};
\draw \headpart;
\draw[xscale=-1] \headpart;
\def\eye{(-0.075,0.15) .. controls (0.02,0) .. (0.075,-0.15)};
\draw[shift={(-0.3,0.8)}] \eye;
\draw[shift={(0,0.85)}] \eye;
% draw mouth
\draw (-0.1,0.2) to [out=15,in=-100] (0.4,0.95); 
\end{tikzpicture}}
\newcommand{\thought}[1]{\textcolor{red}{\textit{(Thought: #1 \shrug)}}}
\newcommand{\todo}[1]{\textcolor{red}{\textit{(TODO: #1)}}}
\newcommand{\feedback}[2]{#2}

\newcommand{\prechk}[0]{$prechk$\xspace}
\newcommand{\name}[0]{Wasm-prechk\xspace}
\newcommand{\wasm}[0]{Wasm\xspace}
\newcommand{\dtal}[0]{DTAL\xspace}
\newcommand{\ie}[0]{\emph{i.e.,}\xspace}
\newcommand{\eg}[0]{\emph{e.g.,}\xspace}

\newcommand{\erase}[1]{\llbracket #1 \rrbracket}

\newtheorem{definition}{Definition}[section]
\newtheorem{theorem}{Theorem}[section]
\newtheorem{lemma}[section]{Lemma}
%\Crefname{lemma}{Lemma}{Lemmas}

% Stack formatting
\newenvironment{stackTL}{
    \setlength{\arraycolsep}{0pt}
    \begin{array}[t]{l}\ignorespacesafterend
} {
    \end{array}\ignorespacesafterend
}

% Some custom notations for object language stuff
% Typeset object language notation in blue with sans serif font
\newcommand{\tbbf}[1]{\textbf{\color{blue}#1}}
\reservestyle{\keywords}{\tbbf}
\keywords{ithreetwo[i32], isixfour[i64], binop[binop], testop[testop], relop[relop], const[const\;], eq[eq\;], neq[neq\;], eqz[eqz\;], unreachable[unreachable], nop[nop\;], drop[drop], select[select], block[block\;], end[\;end], loop[loop\;], if[if\;], else[\;else\;], br[br\;], brif[br\rule{1ex}{.4pt}if\;], brtable[br\rule{1ex}{.4pt}table\;], return[return], call[call\;], callindirect[call\rule{1ex}{.4pt}indirect\;], getlocal[get\rule{1ex}{.4pt}local\;], setlocal[set\rule{1ex}{.4pt}local\;], teelocal[tee\rule{1ex}{.4pt}local\;], getglobal[get\rule{1ex}{.4pt}global\;], setglobal[set\rule{1ex}{.4pt}global\;], trap[trap], func[func\;], local[local], global[global\;], table[table\;], memory[memory\;], label[label], div[div], load[load\;], store[store\;], divpc[div\textsubscript{prechk}], callindirectpc[call\rule{1ex}{.4pt}indirect\textsubscript{prechk}\;], loadpc[load\textsubscript{prechk}\;], storepc[store\textsubscript{prechk}\;]}

\mathlig{;}{\tbbf{;\;}}
\mathlig{:}{:}

\hyphenation{Web-Assembly}

\newcommand{\typerule}[2]{C \vdash #1:#2}
%% stack ; locals ; globals ; index context
\newcommand{\ti}[2]{(#1\;#2)}
\newcommand{\type}[4]{#1;#2;#3;#4}
\newcommand{\insttype}[2]{#1 \rightarrow #2}

\includeonly{
    Preamble/acknowledgements,
    Preamble/dedication,
    Preamble/abstract,
    Chapters/introduction,
    Chapters/related-work,
    Chapters/wasm-prechk,
    Chapters/type-safety,
    Chapters/discussion,
    Chapters/conclusion,
}

%%%%%%%%%%%%%%%%%%%%%%%%%%%%%%%%%%%%%%%%%%%%%%%%%%%%%%%%%%%%%%%

\author{Adam T. Geller}
\title{An Indexed Type System for Faster and Safer WebAssembly}
%% \subtitle{With a Subtitle}

\institution{The University Of British Columbia}
\faculty{The Faculty of Graduate Studies}
\institutionaddress{Vancouver}
\program{Computer Science}
\department{Computer Science}

%%\previousdegree{AAS-DTA (With High Distinction), Bellevue College, 2016}
%%\previousdegree{B.Sc., The University of Washington, 2018}

\copyrightyear{2020}
\submitdate{\monthname\ \number\year} % The "\ " is required after
                                      % \monthname to prevent the
                                      % command from eating the space.
%%%%%%%%%%%%%%%%%%%%%%%%%%%%%%%%%%%%%%%%%%%%%%%%%%%%%%%%%%%%%%%

\begin{document}
\frontmatter

\maketitle

\begin{abstract}
    Downloading and executing untrusted code is inherently unsafe, but also something that happens often on the internet.
    Therefore, untrusted code often requires run-time checks to ensure safety during execution.
    These checks compromise performance and may be unnecessary.
    We present the \name language, an assembly language based on WebAssembly that is intended to ensure safety while justifying the static elimination of run-time checks.
\end{abstract}

\maketitle

\chapter{Acknowledgements}
\todo{This is pretty out of date, also quite long}
This section proceeds through my life in reverse chronological order.

\section*{University of British Columbia}
First and foremost I would like to thank my wonderful advisors, who made this whole process much easier and significantly contributed to my edification.
\begin{itemize}
    \item William Bowman - Despite being a new professor, William has been a wonderful advisor and provided a large amount of guidance when I was struggling.
    He has been a fantastic resource for all of the work here.
    \item Ivan Beschastnikh - Ivan helped me navigate through the system when I was a wide-eyed first-year and learn from various mistakes I made along the way.
    He was a great source of wisdom and gave me the freedom to pursue the topics I found most interesting.
\end{itemize}

I would also like to thank various professors for providing guidance and advice, some of whom are listed below.
\begin{itemize}
    \item Ron Garcia - Ron is definitely one of the nicest people I have met and has forgotten way more about PL than I have ever learned.
    \item Margo Seltzer - Margo is a kickass professor who is insanely smart and cares a lot about students.
\end{itemize}

Finally, all of my fellow graduate students and lab mates who have become my friends and helped me in various ways.
\begin{itemize}
    \item Puneet Mehrotra, Nico Ritschel, Chris Chen, Felipe Banados Schwerter, Paulette Koronkevich, Clement Fung, Vaastav Anand, Joey Eremondi, Anny Gakhokidze, Jonathan Chan, and countless others.
\end{itemize}

\section*{University of Washington}
During my undergrad, I spent a little over a year working on the Cassius project.
This was my first research experience and I think I learned just as much from that experience as the rest of my undergrad combined.
\begin{itemize}
    \item Pavel Panchekha - Pavel was very patient and provided a lot of help getting me off the ground and learning about what doing research is like.
    \item Michael D. Ernst - As well as advising me throughout my undergrad research, Mike guided me through the process of applying to grad school.
    \item Zach Tatlock - Zach is super nice and always made sure I wasn't lost.
    \item Shoaib Kamil - For all his help in research and helping me find my way at my first conference alone.
    \item James Wilcox - For introducing me to PL and helping me find and get started on the Cassius project.
    \item All the other profs and graduate students of UWPLSE, who were extremely friendly and welcoming.
\end{itemize}

\section*{Various Others}
Many different people have helped me on my path at various times. Below I list a few notable ones.
\begin{itemize}
    \item B.J. Unti - For first giving me the idea to go to grad school.
    \item Xander Veerhoff - For first teaching me how to program, which was not easy.
\end{itemize}

\section*{My Parents}
Last but certainly not least, I'd like to thank my awesome parents for everything they have and continue to do for me.
I'm not sure one could ask for better parents than I have been given (even if I did not ask for them).
They have both always been there for me, especially my mom who spent a huge amount of effort and time taking care of every issue that came up in my early years, as well as homeschooling me.
My Dad has always been my role model and is the smartest person I know.
He tried to introduce me to set theory when I was around 4 and calculus when I was about 13 (the first one stuck pretty well, the second one not so much).


\chapter{Dedications}
This is dedicated to Tabby, Koko, and Socrates.
These have been my best (feline) friends who have always helped me maintain my mental health.


\tableofcontents                %% Mandatory
\listoftables                   %% Mandatory if thesis has tables
\listoffigures                  %% Mandatory if thesis has figures

\mainmatter

\chapter{Introduction}
\label{chp:intro}

\section{Unsafe Code}
Browsers and Internet-of-Things (IoT) require running untrusted code, that may have been downloaded from anywhere.
It is crucial to ensure the safety of the code being executed in these contexts.
\todo{Find some examples of javascript exploits}
Typically, \emph{sandboxing} and/or \emph{dynamic safety checks} are used to ensure the safety of untrusted code.

Sandboxing involves placing untrusted code into a secure environment to contain the damage caused by unsafe behavior ~\cite{sandboxes}.
For example, Mozilla's Firefox places untrusted code in separate processes so that unsafe code cannot access the address space of other websites or the broswer ~\cite{foxbox}.
However, sandboxing requires additional run-time resources, as processes require overhead in most OSes.

Dynamic safety checks are run-time checks that catch any attempted unsafe operations.
For example, WebAssembly (\wasm) is a relatively new low-level language designed to be both safe and fast to use in place of JavaScript for performance-critical applications in browsers.
While \wasm is type safe and its semantics enforce the separation of control flow and data, it still relies on dynamic checks to ensure certain type and memory safety properties at run-time.
These dynamic checks potentially slow down programs by introducing unnecessary instructions to perform the checks.
We chose \wasm because it is used in browsers and IoT devices, so both performance and safety are critical concerns.

We have designed an extension to \wasm, called \name, that adds new instructions which do not require dynamic safety checks.
However, under the existing \wasm model the new \name instructions have potentially unsafe semantics, as they require stronger static guarantees than \wasm can provide to ensure safety.
These instructions are guaranteed to be faster than their \wasm counterparts because they do not require the addition of instructions by the compiler/interpreter to perform checks.
To provide these additional static guarantees, we equip \name with a more advanced type system.

\section{Type Systems}
Types systems are useful for reasoning about programs.
They can be used to reason about the correctness of programs, usually in the form of safety guarantees.
For example, type safety is the property that a well-typed program will never become \emph{stuck}, that is, it will always be able to reduce the current expression or the current expression is a well-formed irreducible value.
The safety guarantees of type systems provide a degree of trust in programs, as a well-typed program implicitly contains a checkable proof that it will only exhibit limited behavior.

More expressive type systems that can encode richer invariants, enabling ruling out more bad behaviors with static checks alone.
Generally, such type systems are attached to high-level languages, where explicit abstractions make it easy to reason about programs.
Conversely, using expressive type systems in low-level languages often requires reasoning about program state and unstructured control flow (\ie $goto$), which introduces more complexity into the type system.
However, prior work has attached expressive type systems, that permit complex correctness guarantees, to simple low-level languages.

\todo{I like the next two paragraphs (commented out), but I think they should be moved to the conclusion/discussion, as they don't really fit here anymore.}
%%Using such an expressive type system for a low-level language, we can alleviate overhead otherwise required to ensure safety of untrusted low-level programs.
%%Typically, executing code in an untrusted context requires dynamic safety checks which introduce potentially unnecessary instructions, slowing down execution.
%%However, with the safety guarantees provided by the type system, we can determine when these checks are unnecessary and remove them.
%%This would allow low-level programs to be downloaded, checked, and executed safely and efficiently.

%%Type systems can be used to alleviate the need for safety checks, reducing overhead, by providing safety guarantees about programs before they are run.
%%The idea of using type systems to ensure the safety of low-level code is not a new one.
%%Several projects have attached expressive type systems to low-level languages to attach proofs of correctness to low-level programs.
%%However, the focus in these cases is on correctness, not on performance.

We have built such a type system, based on prior work, for \name.
The \name type system tracks the values of some computations in the types, and constraints between those values can be statically checked.
It can provide the static guarantees necessary for the new \name instructions.
Therefore, with the new type system and new instructions, \name has equivalent safety guarantees to \wasm, with potentially improved performance thanks to fewer instructions.

\section{Thesis Statement}
\begin{adjustwidth}{1cm}{1cm}
    Using a more expressive type system, we can ensure safety \emph{and} improve the performance of low-level code in untrusted environments.
\end{adjustwidth}

\section{Contributions}
We want to use types to improve performance while ensuring safety in real-world low-level programs.
Towards that goal, we introduce \name, an extension of the WebAssembly (\wasm) language.
\name introduces new versions of \wasm instructions which are faster than their \wasm counterparts, but also require stronger type-level safety guarantees (Section ~\ref{sec:newinstructions}).
To facilitate type-checking these new instructions, \name uses an indexed type system which is able to encode linear constraints on program variables and therefore ensure complex safety properties (Section ~\ref{sec:typesys}).
We ensure that \name is as safe as \wasm by providing a type safety proof of the \name indexed type system (Section ~\ref{chp:typesafety}).
Together, these additions mean that \name is as safe as \wasm while potentially improving performance.

\section{Related Work}
\label{sec:relwork}
\todo{Exact positioning TBA (probably use to drive discussion of type systems)}
%% TIL
Using type information to improve compiler optimizations is not a new idea.
In 1996, a paper by Tarditi et al. used strongly typed intermediate languages (TIL) to improve optimizations of SML code ~\cite{TIL}.
Compiling SML involves many translations to intermediate languages, and by preserving type information through those translations and in the intermediate language Tarditi et al. were able to safely perform additional compiler operations.
Using TIL in the compilation of programs led to significantly faster programs.
TIL focuses on compiler optimizations and eventually translates into untyped languages and finally runnable assembly, so the guarantees of the type system are lost along the way.

%% PCC
\todo{Blah}
Proof carrying code (PCC) was an idea introduced in 1997 by Necula et al. ~\cite{PCC}.
While typed assembly languages carry implicit proofs in their types, PCC attached explicit proofs that low-level code satisfies some safety properties.
The proof can then be quickly checked to ensure the safety of the code before it is run.
This provides a way to verify that untrusted code will not violate correctness or security invariants of a program.
The author provides a detailed example of invariants for extensions to TIL to ensure type safety of compiled code.
The example uses the Edinburgh Logical Framework (LF) to encode the proof.
A type safety proof of a LF program is a proof of correctness.
However, PCC puts a burden on developers to formally specify safety and correctness properties, and encoded proofs may be quite large requiring extra time to transmit.

%% FToTAL
Morrisett et al. showed how types could provably be preserved during five different compilation passes to get from System F all the way down to a typed assembly language (TAL) ~\cite{FtoTAL}.
The purpose of TAL was much more focused on safety than on optimizations.
Although Morrisett et al. argued that the type-preserving compilation passes would permit similar optimizations to TIL, they didn't include further optimizations based on TAL.
However, Morrisett et al. did argue that the guarantees of TAL were sufficient to allow untrusted code to be safely executed.

%% DTAL
Xi and Harper created a much more expressive type system for an assembly language which had the potential to allow more compiler optimizations ~\cite{DTAL}.
Their language, a dependently typed assembly language (\dtal), used a limited dependent type system, which enabled safely removing some run-time checks, including array bounds checks.
The goal of \dtal, similar to TAL, was to support type-preserving compilation from a high-level language for both optimizations and safety.
\dtal intended to support type-preserving compilation from Dependent ML as well as SML.

%% LTAL
\todo{LTAL}

%% Wasm
After almost two decades of JavaScript being the dominant language in browsers, it was decided that an alternative was necessary for performance-critical code.
The alternative that was jointly created by the major browser developers was WebAssembly (\wasm) ~\cite{WASM}.
\wasm is a stack-based assembly language with structured control flow.
It is designed to be safe as well as performant, with a small binary footprint.
The \wasm type system is simple, only encoding primitive types, but strong enough to ensure type safety.
Memory safety in \wasm is enforced using run-time checks.
\wasm is supported by most major browsers, and is increasingly used in IoT devices due to its portability and safety.
\chapter{Related Work}
\label{chp:relwork}

%% TIL
Using type information to improve compiler optimizations is not a new idea.
In 1996, a paper by Tarditi et al. used strongly typed intermediate languages (TIL) to improve optimizations of SML code ~\cite{TIL}.
Compiling SML involves many translations to intermediate languages, and by preserving type information through those translations and in the intermediate language Tarditi et al. were able to safely perform additional compiler operations.
Using TIL in the compilation of programs led to significantly faster programs.
TIL focuses on compiler optimizations and eventually translates into untyped languages and finally runnable assembly, so the guarantees of the type system are lost along the way.

%% FToTAL
\todo{Is TAL actually the first typed assembly language? From what I remember of 539B this is true, but I may be forgetting something}
The idea of a typed assembly language was first introduced in 1999 by Morrisett et al. with the strongly typed assembly language (TAL) ~\cite{FToTAL}.
Morrisett et al. showed how types could provably be preserved during five different compilation passes to get from System F to TAL.
The purpose of TAL was much more focused on safety than on optimizations.
Although Morrisett et al. argued that the type-preserving compilation passes would permit similar optimizations to TIL, they didn't include further optimizations based on TAL.
However, Morrisett et al. did argue that the guarantees of TAL were sufficient to allow untrusted code to be safely executed.

%% LTAL
\todo{LTAL}

%% PC
\todo{Blah}
Proof carrying code (PCC) was an idea introduced in 1997 by Necula et al. ~\cite{PCC}.
While typed assembly languages carry implicit proofs in their types, PCC attached explicit proofs that low-level code satisfies some safety properties.
The proof can then be quickly checked to ensure the safety of the code before it is run.
This provides a way to verify that untrusted code will not violate correctness or security invariants of a program.
The author provides a detailed example of invariants for extensions to TIL to ensure type safety of compiled code.
The example uses the Edinburgh Logical Framework (LF) to encode the proof.
A type safety proof of a LF program is a proof of correctness.
However, PCC puts a burden on developers to formally specify safety and correctness properties.

%% DTAL
Xi and Harper created a much more expressive type system for an assembly language which had the potential to allow more compiler optimizations ~\cite{dtal}.
Their language, a dependently typed assembly language (\dtal), used a limited dependent type system to safely remove some run-time checks, including array bounds checks.
The goal of \dtal, similar to TAL, was to support type-preserving compilation from a high-level language for both optimizations and safety.
\dtal intended to support type-preserving compilation from Dependent ML as well as SML.
However, these goals were never realized in a practical system.

%% Wasm
After almost two decades of JavaScript being the dominant language in browsers, it was decided that an alternative was necessary for performance-critical code.
The alternative that was jointly created by the major browser developers was WebAssembly (\wasm) ~\cite{Wasm}.
\wasm is a stack-based assembly language with structured control flow.
It is designed to be safe as well as performant.
The \wasm type system is simple, only encoding primitive types, but strong enough to ensure type safety.
Memory safety in \wasm is enforced using run-time checks.
\wasm is supported by most major browsers, and is increasingly used in IoT devices due to its portability and safety.

%%\begin{mathpar}
%%a ::= variable-not-otherwise-mentioned
%%(x y ::= a (t c) (binop x y))
%%(P ::= (testop x) (relop x y) (not P) (and P P) (or P P))
%%;(γ ::= t (a : γ P)) TODO: I don't think we really need these? Syntactic sugar
%%(φ ::= empty (φ (t a)) (φ P))

%%(ti ::= (t a))
%%  (mut? ::= boolean)
%%  (tgi ::= (mut? ti))
%%  ;; Index-type pre/post-condition: types on stack, locals, globals, and constraint context
%%  (locals ::= (ti ...))
%%  (globals ::= (ti ...))
%%  (ticond ::= ((ti ...) locals globals φ))
%%  (tfi ::= (ticond -> ticond))
%%\end{mathpar}

\chapter{\name}
\label{chp:prechk}
The goal of \name is to be used to eliminate unnecessary dynamic checks.
To accomplish this, it must (1) have instructions that do not require dynamic checks and (2) statically prove that the assumptions of those instructions are met.
In \name, we extend \wasm with new instructions that explicitly do not require dynamic checks, and design an indexed type system to reason about the safety of removing checks.

\section{\name Syntax}
The syntax of \name is highly similar to that of \wasm except for two additions.
First, \name introduces four additional instructions, which are referred to as ``\prechk-tagged'' instructions.
Second, \name changes the representation of types within \wasm instructions and functions.

Recall from \autoref{sec:wasmsemantics} that there are four \wasm instructions that require run-time checks: integer division, indirect function calls, and memory loads and stores.
``\prechk-tagged'' instructions refer to a set of four \name instructions, listed in \autoref{fig:newinstructions}, that are counterparts to these four \wasm instructions.
Intuitively, we are simply adding a tag to the instruction to show that it doesn't require run-time checks.
Formally, however, these are different instructions with different semantics and different typing rules, as explained below.

\begin{figure}[t]
    \begin{align*}
        &t.\<divpc> \mid
        t.\<callindirectpc> \mid
        \\
        &t.\<loadpc> (tp\_sx)^{?}\; a\;o \mid
        t.\<storepc> tp^{?\;} a\;o
    \end{align*}
    \caption{The four \prechk-tagged instructions}
    \label{fig:newinstructions}
\end{figure}

\subsection{The \name Index Language}
\label{subsec:indexlang}
\name uses an indexed type system.
An indexed type language uses an index language in the type system to encode information within types.
We use the index language to encode linear constraints on program variables within types.
\autoref{fig:itsyntax} shows the syntax for the index type language.
Remember, syntax written in a \tbsf{blue sans serif font} denotes a \wasm keyword.
Below is a quick overview of each of the terms.

\begin{figure}[t]
    \begin{math}
        \begin{array}{rcl}
            t &:: & \<ithreetwo> \mid \<isixfour> \\
            a &::= & Index\; Variable \\
            x\;y &::=& a \mid \ti{t}{c} \mid (\<binop>\;x\;y) \mid (\<testop>\;x) \mid (\<relop>\;x\;y) \\
            P &::=& (=\; x \; y) \mid (\text{if}\; P\; P\; P) \mid \neg P \mid P \land P \mid P \lor P \\
            \phi &::=& \circ \mid \phi, \ti{t}{a} \mid \phi, P \\
        \end{array}
    \end{math}
    \caption{Syntax of the \name index type language}
    \label{fig:itsyntax}
\end{figure}

\begin{itemize}
    \item $t$ represents a primitive \wasm type.
    We do not reason about floating points, so it is either a 32-bit integer ($i32$) or a 64-bit integer ($i64$).
    \item $a$ is a type index variable, which is used to track constraints on program variables.
    \item $x$ and $y$ are type indices, they can be an index type variable, a constant with an explicit type, or a \wasm operation on a type index.
    \item $P$ is a proposition about type indices which can encode equality constraints on type indices, or combine propositions using common first-order logic operators.
    \item $\phi$ is the type index context which stores index type variable declarations and propositions.
    Essentially, it contains all of the knowledge we have about all of the index variables.
\end{itemize}

\begin{figure}[t]
    \begin{math}
        \begin{array}{rcl}
            ti &::=& \ti{t}{a} \\
            l &::=& ti^{*} \\
            tfi &::=& ti^{*};l;\phi \rightarrow ti^{*};l;\phi \\
            C &::=& \{
                {\begin{stackTL}
                    \text{func } \; tfi^{*}, \text{ global } \; (\text{mut}^{?} \; t)^{*}, \text{ table} \; n^{?}, \text{ memory }  m^{?}, \\
                    \text{ local } \; t^{*}, \text{ label}(ti^{*};l;\phi)^{*}, \text{ return}\;(ti^{*};l;\phi)^{? }\}
                \end{stackTL}}
        \end{array}
    \end{math}
    \caption{\name indexed function types}
    \label{fig:tfisyntax}
\end{figure}

Indexed types, $ti$, are used to associate index variables $a$ with values in the program.
\autoref{fig:tfisyntax} shows the form of an indexed type, which includes both the type $t$ and an index variable $a$.
In \name, we represent the shape of the stack as a sequence of indexed types $ti^{*}$.

The index local store associates index variables with local variables.
It has an identical form to the stack: a sequence of indexed types to associate index variables with local variables.
We use the shorthand $l$ to refer to the index local store since we rarely reason about it but rather thread it through typing rules.
The index type context $\phi$ is the mechanism that is used to reason about the possible values of computations.
It stores constraints on and between program variables tracked by indexed types representing the stack and index local store.

\name uses indexed function types $tfi$, which, similar to \wasm's function types, are just a precondition and postcondition.
However, indexed function types include much more information in their precondition and postcondition!
They represent the stack using a sequence of indexed types and track local variables using the index local store, and include $\phi$ which contains constraints about those values.
We will see how this information is used in \autoref{subsec:checkelim}.

We retain $C$ to refer to the module type context in \name, although the representation of module types is slightly different.
\wasm function types are replaced with \name indexed function types.
Further, the postconditions in the label stack and return stack are replaced with \name indexed postconditions including indexed types, the local index store, and the index type context.

We can now introduce the \name typing judgement for instructions.
It is similar to the \wasm typing judgment, but uses indexed function types which include much more information by tracking constraints about program values.

\begin{mathpar}
    \boxed{C \vdash e^{*} : tfi}
\end{mathpar}

Recall that certain \wasm instructions (such as $\<block>$ and $\<callindirect>$) include \wasm function types to declare the expected types of their bodies.
In \name, we replace those function types with indexed function types.

\section{\name Dynamic Semantics}
\name uses the same reduction relation with the same structure as \wasm (explained in detail in \autoref{sec:wasmsemantics}).
All the reduction rules for all of the \name instructions are the same as they are for \wasm, except that indexed function types are used instead of \wasm function types.
We also have four new instructions, for which we introduce new reduction rules.

\label{sec:newinstructions}
\begin{figure}[t]
    \begin{mathpar}
        \boxed{s;v^{*};e^{*} \rightarrow s';v'^{*};e'^{*}}
    \end{mathpar}

    \begin{math}
        \arraycolsep=1.4pt
        \begin{array}{rcl}
            (t.\<const> c_1)\; (t.\<const> c_2)\; t.binop &\hookrightarrow& t.\<const> c \\
            \text{if } c=binop(c_1,c_2) && \\ %% binop

            (t.\<const> c_1)\; (t.\<const> c_2)\; t.binop &\hookrightarrow& \<trap> \\ %% binop to trap
            otherwise && \\

            s;\<callindirect> j &\hookrightarrow_i& s_\text{tab}(i,j) \\
            \text{if } s_\text{tab}(i,j)_\text{code}=(\<func> tf\; \<local>\; t^{*}\; e^{*}) && \\

            s;\<callindirect> j &\hookrightarrow_i& \<trap> \\
            \text{otherwise} && \\

            s;(\<ithreetwo>.\<const> k) &&\\
            (t.\<load> tp\_sx\; align\; o) &\hookrightarrow_i& s;(t.\<const> const_t(b^{*})) \\
            \text{if } s_\text{mem}(i,k+o,|t|)=b^{*} && \\

            &&\\

            s;(\<ithreetwo>.\<const> k) &&\\
            (t.\<load> tp\_sx\; align\; o) &\hookrightarrow_i& \<trap> \\
            \text{otherwise} && \\

            &&\\

            s;(\<ithreetwo>.\<const> k)\; (t.\<const> c) && \\
            (t.\<store> tp\_sx\; align\; o) &\hookrightarrow_i& s';\epsilon \\
            \text{if } s'=s &\text{ with }& \text{mem}(i,k+o,|t|)=bits_t(c) \\

            &&\\

            s;(\<ithreetwo>.\<const> k)\; (t.\<const> c) && \\
            (t.\<store> tp\_sx\; align\; o) &\hookrightarrow_i& \<trap> \\
            \text{otherwise} && \\
        \end{array}
    \end{math}
    \caption{\wasm instructions that have preconditions for reduction}
    \label{fig:checked}
\end{figure}

The formal reason certain \wasm instructions require run-time checks is because they have preconditions as part of their semantics, and if the preconditions are not met then those instructions trap to avoid undefined behavior (we've reproduced the reduction rules for those instructions in \autoref{fig:checked}).
The \wasm type system is not expressive enough to ensure these preconditions statically, so they instead must be checked at run-time.
Thus, ``\prechk-tagged'' instructions can assume that the preconditions on their behavior hold because it is enforced by the \name type system.
This can be seen in the reduction rules for the ``\prechk-tagged'' instructions in Figure \autoref{fig:prechkredux}, where they do not have rules to trap when their preconditions do not hold.

\begin{figure}[t]
    \begin{mathpar}
        \boxed{s;v^{*};e^{*} \rightarrow s';v'^{*};e'^{*}}
    \end{mathpar}

    \begin{math}
        \arraycolsep=1.4pt
        \begin{array}{rcl}
            (t.\<const> c_1)\;(t.\<const> c_2) && \\
            t.\<divpc> & \hookrightarrow & c \\
            && \text{where } c_2 \neq 0 \land c=c_1/c_2 \\
            s;(t.\<const> j) && \\
            t.\<callindirectpc> & \hookrightarrow_i & \<call> s_{tab}(i,j) \\
            &&\text{where } s_{tab}(i,j) = \\
            && \<func> tfi\; \<local>\; t^{*}\;e^{*} \\
            s;(\<ithreetwo>.\<const> k) && \\
            (t.\<loadpc> (tp\_sx)^{?}\; a\;o) & \hookrightarrow_i & t.\<const> const_t(b^{*}) \\
            && \text{where } s_{mem}(i,k+o,|t|)=b^{*} \\
            s;(\<ithreetwo>.\<const> k)\;(t.\<const> c) && \\
            (t.\<storepc> tp^{?}\; a\;o) & \hookrightarrow_i & s';\epsilon \\
            && \text{where } s'=s \\
            && \text{with mem}(i,k+o,|t|)=bits_t^{|t|}(c) \\
        \end{array}
    \end{math}
    \caption{Behavior of new \prechk-tagged instructions}
    \label{fig:prechkredux}
\end{figure}

All other \name instructions have equivalent semantics to their \wasm versions as presented in \autoref{sec:wasmsemantics}.

\section{The \name Indexed Type System}
\label{sec:typesys}
The \name type system is designed to provide sufficient information to safely eliminate dynamic checks (\ie to ensure that the required preconditions are met to \prechk-tag an instruction).
As explained in ~\ref{subsec:indexlang}, the \name type system can encode linear constraints on program variables in the preconditions and postconditions of instructions.
We will now show how these constraints are added and used.

Recall the form of the \name typing judgement for instructions.
\begin{mathpar}
    \boxed{C \vdash e^{*} : ti_1^{*};l_1;\phi_1 \rightarrow ti_2^{*};l_2;\phi_2}
\end{mathpar}

Under $C$, the module type context, $e^{*}$ has the precondition $ti_1^{*};l_1;\phi_1$ and postcondition $ti_2^{*};l_2;\phi_2$.
We use the abbreviation $tfi ::= ti_1^{*};l_1;\phi_1 \rightarrow ti_2^{*};l_2;\phi_2$ as shorthand for the precondition and postcondition of an instruction.

As in \wasm, \name generally has two kinds of typing rules.
Most rules are for inferring or checking the types of instructions (in which case $e^{*}$ will be a single instruction).
There are also a few rules to compose together instruction sequences.
We present the typing rules mixed with discussion of those rules.
The typing judgment in its entirety is reproduced in the appendix.

Here are some of the simpler rules.
These are rules don't use or modify index type information.
\refrule{Unreachable} accepts any precondition and guarantees any postcondition since it just causes a trap.
In \refrule{Nop}, no changes are made from the precondition to the post condition because the instruction does nothing.
\refrule{Drop} consumes the top value from the stack (without caring about its type) and does not change the local index store or index type context.
\begin{mathpar}
    \inferrule*[right=\defrule{Unreachable}]{ }{ %% unreachable
        C \vdash \<unreachable> : tfi
    }

    \inferrule*[right=\defrule{Nop}]{ }{ %% nop
        C \vdash \<nop> : \epsilon;l;\phi \rightarrow \epsilon;l;\phi
    }

    \inferrule*[right=\defrule{Drop}]{ }{ %% drop
        C \vdash \<drop> : \ti{t}{a};l;\phi \rightarrow \epsilon;l;\phi
    }
\end{mathpar}

The constant instruction is a simple example of how indexed types work.
It adds a new indexed type onto the stack to track the new program variable $\ti{t}{a}$, declares the new indexed type in the index type context, and constrains that indexed type to be equal to the constant $(= a\; \ti{t}{c})$.
We require $a$ to be fresh, that is, we require that $a$ is a new index variable present no where else in any of the types for the program.
This is a common pattern to see in rules which introduce new index variables.
The local index store is unchanged between the precondition and postcondition.
\begin{mathpar}
    \inferrule*[right=\defrule{Const}]{ a \text{ is fresh} }{ %% const
        C \vdash t.\<const> c : \epsilon;l;\phi \rightarrow \ti{t}{a};l;\phi,\ti{t}{a},(= a \ti{t}{c})
    }
\end{mathpar}

There are several different kinds of operations, but they all work similarly.
The binary operator instruction adds constraints between new and old program values, since the result of the instruction is a new program values, while the consumed values may already be constrained.
A binary operation consumes two values from the stack, whiuch have associated indexed types $\ti{t}{a_1}$ and $\ti{t}{a_2}$, and produces a value which is associated with the fresh indexed type $\ti{t}{a_3}$.
The index type declaration $\ti{t}{a_3}$ is added to the index type context $\phi$ and $a_3$ is constrained to be equal to the binary operator applied to the index variables that correspond to the input $(= a_3\;(\|binop\|\;a_1\;a_2)$.
As a side note, we use $\|binop\|$ to indicate that we are using the $binop$ (or relop or testop) from the index language.
Binary operators do not affect or use local variables, they simply propagate, so the local index store. $l$, is the same in the precondition and postcondition.
\begin{mathpar}
    \inferrule*[right=\defrule{Binop}]{a_3 \text{ is fresh}}{ %% binop
        C \vdash t.binop : \ti{t}{a_1}\;\ti{t}{a_2};l;\phi \rightarrow \ti{t}{a_3};l;\phi,\ti{t}{a_3},(= a_3\;(\|binop\|\;a_1\;a_2))
    }

    \inferrule*[right=\defrule{Testop}]{a_2 \text{ is fresh}}{ %% testop
        C \vdash t.testop : \ti{t}{a_1}\;l;\phi \rightarrow \ti{\<ithreetwo>}{a_2};l;\phi,\ti{t}{a_2},(= a_2\;(\|testop\|\;a_1))
    }

    \inferrule*[right=\defrule{Relop}]{a_3 \text{ is fresh}}{ %% relop
        C \vdash t.relop : \ti{t}{a_1}\;\ti{t}{a_2};l;\phi \rightarrow \ti{t}{a_3};l;\phi,\ti{t}{a_3},(= a_3\;(\|relop\|\;a_1\;a_2))
    }
\end{mathpar}

\refrule{Select} constrains indexed types in a rather complex way.
Select consumes three values from the stack, it returns the second value if the third value is zero, and otherwise returns the first value (similar to C's ternary operator).
For this rule we use the type-level ``if'' to allow the constraint on the result to depend on the third value consumed: $(\text{if}\; (= a\; \ti{\<ithreetwo>}{0})\; (= a_3\;a_2)\; (= a_3\;a_1))$.
\begin{mathpar}
    \inferrule*[right=\defrule{Select}]{a_3 \text{ is fresh}}{ %% select
        C \vdash \<select> : {\begin{stackTL}
            \ti{t}{a_1}\;\ti{t}{a_2}\;\ti{i32}{a};l;\phi
            \\ \rightarrow \ti{t}{a_3};l;\phi,\ti{t}{a_3},
                (if\; (= a\; \ti{\<ithreetwo>}{0})\; (= a_3\;a_2)\; (= a_3\;a_1))
        \end{stackTL}}
    }
\end{mathpar}

The rules for the three different kinds of blocks (\<block>, \<loop>, and \<if>) are similar to \wasm.
They simply ensure that the interior instruction sequence has the expected type under the context with the expected postcondition (or precondition in the case of loop) appended to the local stack.
If blocks are able to make extra assumptions about the consumed value in the subsequences (that it is non-zero in the first sequence and zero in the second), because those constraints must be true for that sequence to be executed.
While \<if> and \<block> append their postcondition to the label stack for type checking branching instructions within the block, \<loop> appends its precondition because branching to a loop means running the loop again.

\begin{mathpar}
    \inferrule*[right=\defrule{Block}]{ %% block
        C_2,\text{label } (ti_2^{*};l_2;\phi_2) \vdash e^{*} : ti_1^{*};l_1;\phi_1 \rightarrow ti_2^{*};l_2;\phi_2 \\
    }
    {
        C \vdash \<block>\; (ti_1^{*};l_1;\phi_1 \rightarrow ti_2^{*};l_2;\phi_2)\; e^{*} \<end> : ti_1^{*};l_1;\phi_1 \rightarrow ti_2^{*};l_2;\phi_2
    }

    \inferrule*[right=\defrule{Loop}]{ %% loop
        C_2,\text{label } (ti_1^{*};l_1;\phi_1)^{*} \vdash e^{*} : ti_1^{*};l_1;\phi_1 \rightarrow ti_2^{*};l_2;\phi_2 \\
    }
    {
        C \vdash \<loop>\; (ti_1^{*};l_1;\phi_1 \rightarrow ti_2^{*};l_2;\phi_2)\; e^{*} \<end> : ti_1^{*};l_1;\phi_1 \rightarrow ti_2^{*};l_2;\phi_2
    }

    \inferrule*[right=\defrule{If}]{ %% if
        C_2,\text{label } (ti_2^{*};l_2;\phi_2) \vdash e_1^{*} : ti_1^{*};l_1;\phi_1, \neg(= a\; \ti{\<ithreetwo>}{0}) \rightarrow ti_2^{*};l_2;\phi_2 \\
        C_2,\text{label } (ti_2^{*};l_2;\phi_2) \vdash e_2^{*} : ti_1^{*};l_1;\phi_1, (= a\; \ti{\<ithreetwo>}{0})) \rightarrow ti_2^{*};l_2;\phi_2 \\
    }
    {
        C \vdash \<if>\; (ti_1^{*};l_1;\phi_1 \rightarrow ti_2^{*};l_2;\phi_2)\; e_1^{*} \<else> e_2^{*} \<end> : ti_1^{*};l_1;\phi_1 \rightarrow ti_2^{*};l_2;\phi_2
    }
\end{mathpar}

One thing to note is that all three of these rules include their expected preconditions and postconditions as part of their syntax.
We consider the index variables in these indexed function types to be unification variables rather then literals, allowing them to match any literal as long as the types unify.
Intuitively, this is very similar to alpha equivalence, where the precondition matches any preceding postcondition with the same structure as long as the variable can be renamed to match.
The postcondition appended to the label stack also has unification variables instead of the supplied literals.

The rules for branching instructions and return are similar to \wasm.
However, $\<brif>$ adds the assumption that the consumed value is zero to its postcondition.
This assumption can be safely added because the value must be zero for execution to continue without a branch occurring.
If the consumed value is constrained to be non-zero in the indexed type system, then this will cause a contradiction in the constraints of the index type context $\phi$.
However, that is fine since this means that no instructions following the $\<brif>$ will be executed.
Also remember the above note that the postconditions on the label stack contain unification variables, not literals.

Recall from \autoref{sec:wasmsemantics} that $\<brtable>$ branches to one of many different labels.
Thus, we must ensure that every possible branching postcondition it might branch to is implied by the precondition.

\begin{mathpar}
    \inferrule*[right=\defrule{Br}]{ %% br
        C_{\text{label}}(i) = ti^{*};l_1;\phi_1
    }
    {
        C \vdash \<br> i : ti_1^{*}\;ti^{*};l_1;\phi_1 \rightarrow ti_2^{*};l_2;\phi_2
    }

    \inferrule*[right=\defrule{Return}]{ %% return
        C_{\text{return}} = ti^{*};l_1;\phi_1
    }
    {
        C \vdash \<return> : ti_1^{*}\;ti^{*};l_1;\phi_1 \rightarrow ti_2^{*};l_2;\phi_2
    }

    \inferrule*[right=\defrule{Br-If}]{ %% br_if
        C_{\text{label}}(i) = ti^{*};l_1;\phi_1,\neg(= a\; \ti{\<ithreetwo>}{0})
    }
    {
        C \vdash \<brif> i : ti^{*}\;\index{i32}{a};l_1;\phi_1 \rightarrow ti^{*};l_1;\phi_1,(= a\; \ti{\<ithreetwo>}{0})
    }

    \inferrule*[right=\defrule{Br-Table}]{ %% br_table
        (C_{\text{label}}(i) = ti^{*};l_1;\phi_i)^{+} \and
        (\phi_1 \implies \phi_i)^n
    }
    {
        C \vdash \<brtable> i^{+} : ti_1^{*}\;ti^{*}\;\index{i32}{a};l_1;\phi_1 \rightarrow ti_2^{*};l_2;\phi_2
    }
\end{mathpar}

Recall that functions are declared within the module with a specific type $tfi$, that is a precondition and postcondition.
These declared indexed function types are placed inside the module type context $C$.
Direct function calls $\<call> i$ have the same type as the declared indexed function type of the function they are calling with two differences.
First, the local index store is unchanged, since the called function will have been turned into a closure that operates on separate local variables in a local block.
Second, the index type context in the postcondition is extended with the declarations and constraints from the precondition.
The precondition and postcondition of a function can only contain constraints about the arguments supplied to that function, so simply copying the postcondition of the function would result in the loss of information about all other index variables.

Indirect function calls $\<callindirect> ti_1^{*};l_1;\phi_1 \rightarrow ti_2^{*};l_2;\phi_2$ include the expected indexed function type $ti_1^{*};l_1;\phi_1 \rightarrow ti_2^{*};l_2;\phi_2$ provided as part of their syntax (the same note about index variables being unification variables from above holds).
Remember that indirect function calls perform a run time type check against the closure that they end up calling, so we assume statically that the check will proceed because if it does not the program will trap and not be able to do any harm.
The same two differences described above between the expected indexed function type $tfi$ and the type of the $\<callindirect> tfi$ instruction also hold.

\begin{mathpar}
    \inferrule*[right=\defrule{Call}]{ %% call
        C_\text{func}(i) = ti_1^{*};l_1;\phi_1 \rightarrow ti_2^{*};l_2;\phi_2
    }
    {
        C \vdash \<call> i : ti_1^{*};l;\phi_1 \rightarrow ti_2^{*};l;\phi_1,\phi_2
    } \and
    \inferrule*[right=\defrule{Call-Indirect}]{ %% call_indirect
        C_\text{table}(i) = (j, tfi_2^{*}) \and
    }
    {
        C \vdash \<callindirect> ti_1^{*};l_1;\phi_1 \rightarrow ti_2^{*};l_2;\phi_2 : ti_1^{*}\;\ti{i32}{a};l;\phi_1 \rightarrow ti_2^{*};l;\phi_1,\phi_2
    } \and
\end{mathpar}

The only instructions that actually mutate the local index store are those that operate on local variables.
$\<getlocal>$ produces a fresh indexed type $\ti{t}{a_2}$ that is constrained to be equal to the index variable associated with the local being retrieved.
$\<setlocal>$ works in the reverse direction, replacing the index variable associated with the local being set with a fresh indexed type constrained to be equal to the value consumed.
Finally, $\<teelocal>$ is effectively a $\<setlocal>$ that consumes and immediately regurgitates the indexed type back onto the stack, like the Unix tool ``tee''.
\begin{mathpar}
    \inferrule*[right=\defrule{Get-Local}]{ %% get_local
        C_{\text{local}}(i) = t \and
        l(i) = \ti{t}{a} \and
    }
    {
        C \vdash \<getlocal> i : \epsilon;l;\phi \rightarrow \ti{t}{a};l;\phi
    }

    \inferrule*[right=\defrule{Set-Local}]{ %% set_local
        C_{\text{local}}(i) = t \and
        l_2 = l_1[i := \ti{t}{a}] \and
    }
    {
        C \vdash \<setlocal> i : \ti{t}{a};l_1;\phi \rightarrow \epsilon;l_2;\phi
    }

    \inferrule*[right=\defrule{Tee-Local}]{ %% tee_local
        C_{\text{local}}(i) = t \and
        l_2 = l_1[i := \ti{t}{a_2}] \and
    }
    {
        C \vdash \<teelocal> i : \ti{t}{a};l_1;\phi \rightarrow \ti{t}{a};l_2;\phi
    }
\end{mathpar}

Instructions for getting and setting globals produce and consume unconstrained values respectively.
Global variables are difficult to reason about in the type system since they are different between modules.
At compile-time, before linking, a module has no information about globals from another module which would be necessary for reasoning about the types of functions imported from the other module.
Therefore, we do not track index variables for globals.
We do still ensure that the value is of the correct type and in the case of setting a global variable that the global variable is mutable (has the ``mut'' flag in its type).
\begin{mathpar}
    \inferrule*[right=\defrule{Get-Global}]{ %% get_global
        C_\text{global}(i) = \text{mut}^{?}\; t \and
        a \text{ is fresh}
    }
    {
        C \vdash \<getglobal> i : \epsilon;l;\phi \rightarrow \ti{t}{a};l;\phi,\ti{t}{a}
    }

    \inferrule*[right=\defrule{Set-Global}]{ %% set_global
        C_\text{global}(i) = \text{mut } t
    }
    {
        C \vdash \<setglobal> i : \ti{t}{a};l;\phi \rightarrow \epsilon;l;\phi
    }
\end{mathpar}

The typing rules for memory instructions are very similar to \wasm as we do not reason about the contents of memory or how its size can change throughout a program.
As in \wasm, there are many small details related to how exactly values are loaded and store that are not particularly important to the understanding of the type system, but they are explained with the reduction rules for these values in \autoref{sec:wasmsemantics}.
One thing that does not appear in the \wasm reduction rules but mysteriously appears in the typing rules without much explanation is $align$.
According to the \wasm paper, $align$ is an ``alignment exponent''.
It is checked against the size of the type of the value being stored/loaded $|t|$ (or optionally $|tp|$, which should be less than $|t|$) in the premise $2^{align} \leq (|tp| <)^{?} |t|$.

\begin{mathpar}
    \inferrule*[right=\defrule{Mem-Load}]{ %% memory load
        C_\text{memory} = n \and
        2^{align} \leq (|tp| <)^{?} |t| \and
        a_2 \text{ is fresh}
    }
    {
        C \vdash t.\<load> (tp\_sx)^{?}\; align\; o : \ti{\<ithreetwo>}{a_1};l;\phi \rightarrow \ti{t}{a_2};l;\phi,\ti{t}{a_2}
    }

    \inferrule*[right=\defrule{Mem-Store}]{ %% memory store
        C_\text{memory} = n \and
        2^{align} \leq (|tp| <)^{?} |t|
    }
    {
        C \vdash t.\<store> tp^{?}\; align\; o : \ti{\<ithreetwo>}{a_1}\;\ti{t}{a_2};l;\phi \rightarrow \epsilon;l;\phi
    }

    \inferrule*[right=\defrule{Current-Memory}]{ %% current mem
        C_\text{memory} = n \and
        a \text{ is fresh}
    }
    {
        C \vdash \<currentmemory> : \epsilon;l;\phi \rightarrow \ti{\<ithreetwo>}{a};l;\phi
    }

    \inferrule*[right=\defrule{Grow-Memory}]{ %% grow mem
        C_\text{memory} = n \and
        a_2 \text{ is fresh}
    }
    {
        C \vdash \<growmemory> : \ti{\<ithreetwo>}{a_1};l;\phi \rightarrow \ti{\<ithreetwo>}{a_2};l;\phi
    }
\end{mathpar}

The last rules are the ones that can be used to compose sequences of instructions.
The first rule is for the empty instruction sequence $\epsilon$, which, similar to in \wasm, simply has the same precondition and postcondition $\epsilon;l;\phi$.
Second, we have \refrule{Stack-Poly} to add stack polymorphism (see \autoref{subsec:stackpoly}).
Third, there is a rule to compose a sequence of instructions $e_1^{*}$ with another instruction $e_2$.
\begin{mathpar}
    \inferrule*[right=\defrule{Empty}]{ %% empty
    }
    {
        C \vdash \epsilon : \epsilon;l;\phi \rightarrow \epsilon;l;\phi
    }

    \inferrule*[right=\defrule{Stack-Poly}]{ %% extra vars
        C \vdash e^{*} : ti_1^{*};l_1;\phi_1 \rightarrow ti_2^{*};l_2;\phi_2
    }
    {
        C \vdash e^{*} : ti^{*}\;ti_1^{*};l_1;\phi_1 \rightarrow ti^{*}\;ti_2^{*};l_2;\phi_2
    }

    \inferrule*[right=\defrule{Composition}]{ %% combine
        C \vdash e_1^{*} : ti_1^{*};l_1;\phi_1 \rightarrow ti_2^{*};l_2;\phi_2 \\
        C \vdash e_2 : ti_2^{*};l_2;\phi_2 \rightarrow ti_3^{*};l_3;\phi_3
    }
    {
        C \vdash e_1^{*}\;e_2 : ti_1^{*};l_1;\phi_1 \rightarrow ti_3^{*};l_3;\phi_3
    }
\end{mathpar}

\subsection{Subtyping, Implication, and Constraint Satisfaction}
\label{subsec:subtyping}
One issue with adding the index type context $\phi$ to preconditions and postconditions is that the postcondition of one instruction and the precondition of the next instruction might not match up exactly.
For example, one instruction may ensure a value is greater than ten, but the next just wants the value to be greater than zero.
Intuitively, if a value, ``x'', is greater than ten it must also be greater than zero, and we want the \name type system to be able to figure this out as well.
However, computers as of yet are unable to use intuition, so we must instead formalize this.

Our formalization of this problem is to allow \emph{strengthening} preconditions and \emph{weakening} postconditions.
Strengthening and weakening is based on implication ($\implies$).
We say that $\phi_1 \implies \phi_2$ when the following holds: if $\phi_1$ is satisfied, then $\phi_2$ must also be satisfied.
If $\phi_1 \implies \phi_2$, then we consider $\phi_1$ to be stronger than $\phi_2$, and $\phi_2$ to be weaker than $\phi_1$.
This solves the aforementioned problem because we can weaken ``x is greater than 10'' to ``x is greater than 0'' (or equivalently strengthen ``x is greater than 0'' to ``x is greater than 10'').

To fit strengthening and weakening into the type system, we define a subtyping judgment based on implication.
The \refrule{Implies} says that if an indexed function type $tfi_1$ has a stronger precondition and weaker postcondition than some other indexed function type $tfi_2$, and is otherwise equivalent, then $tfi_1$ is a subtype of $tfi_2$.

\begin{mathpar}
    \inferrule*[right=\defrule{Implies}]{
        \phi_0 \implies \phi_1 \and
        \phi_2 \implies \phi_3
    }{
        ti_1^{*};l_1;\phi_0 \rightarrow ti_2^{*};l_2;\phi_3 <: ti_1^{*};l_1;\phi_1 \rightarrow ti_2^{*};l_2;\phi_2
    }
\end{mathpar}

We then use this in the \name type system by adding a typing rule that allows the indexed function type for a list of instructions to be replaced by a subtype of that indexed function type.

\[
    \inferrule*[right=\defrule{SubTyping}]{
        ti_1^{*};l_1;\phi_0 \rightarrow ti_2^{*};l_2;\phi_3 <: ti_1^{*};l_1;\phi_1 \rightarrow ti_2^{*};l_2;\phi_2 \\
        C \vdash e^{*} \rightarrow ti_1^{*};l_1;\phi_1 \rightarrow ti_2^{*};l_2;\phi_2
    }{
        C \vdash e^{*} \rightarrow ti_1^{*};l_1;\phi_0 \rightarrow ti_2^{*};l_2;\phi_3
    }
\]

\subsection{Using Types for Check Elimination}
\label{subsec:checkelim}
In Section \ref{sec:newinstructions} we explained why \prechk-tagged instructions do not need dynamic checks because of their static guarantees of the \name type system.
Now, we will see how the \name type system provides those guarantees by looking at the typing rules for each of the \prechk-tagged instructions.

Integer division simply requires that the second argument is non-zero.
The premise $\phi \implies \neg(=\ a_2\ 0)$ requires that the index constraints satisfy the proposition $a_2 \neq 0$ for the pre-checked instruction to be safe.
Therefore, since a divide-by-zero is provably absent, it is safe to use the \prechk-tagged division instruction.
As an aside, recall that $a_3 \not\in \phi$ ensures that $a_3$ is fresh.
\begin{mathpar}
    \inferrule*[right=\defrule{Div-Prechk}]{
        \phi \implies \neg(=\ a_2\ 0) \and
        a_3 \not\in \phi
    }{
        C \vdash t.\<divpc> : \ti{t}{a_1}\;\ti{t}{a_2};l;\phi \rightarrow \ti{t}{a_3};l;\phi,\ti{t}{a_3},(= a_3\;(div\;a_1\;a_2))
    }
\end{mathpar}

Tagging memory loads and stores with \prechk requires ensuring that the memory index is valid.
Since \wasm and \name use linear memory which is a contiguous block of memory, we simply have to ensure that the index is within those bounds.
The initial memory size is the number of $64$ Ki pages ($65,536$ bytes), so we check that the constraints in the index type context ensure that the memory index plus the static offset is between $0$ and $65,536-width$.
We use $width$ as a shorthand to denote the number of bytes that is being stored/loaded, it is equal to $|t|/8$ if $tp^{?}=\epsilon$, and otherwise equal to $|tp|/8$.

Unfortunately, while the size of memory may be grown during program execution, we are currently unable to reason about changing memory size.
Therefore, we just use the initial memory size.
\begin{mathpar}
    \inferrule*[right=\defrule{Load-Prechk}]{ %% memory load
        C_\text{memory} = n \and
        2^{align} \leq (|tp| <)^{?} |t| \and
        a_3 \not\in \phi \\
        \phi \implies
        {\begin{stackTL}
            (\<ge> (\<add> a_1\; \ti{\<ithreetwo>}{o}) \ti{\<ithreetwo>}{0}), 
            \\ (\<le> (\<add> a_1\; (\<add> \ti{\<ithreetwo>}{o+width}))\; \ti{\<ithreetwo>}{n*64 \text{Ki}})
        \end{stackTL}}
    }
    {
        C \vdash t.\<loadpc> (tp\_sx)^{?}\; align\; o : \ti{\<ithreetwo>}{a_1};l;\phi \rightarrow \ti{t}{a_2};l;\phi,\ti{t}{a_2}
    }

    \inferrule*[right=\defrule{Store-Prechk}]{ %% memory store
        C_\text{memory} = n \and
        2^{align} \leq (|tp| <)^{?} |t| \\
        \phi \implies
        {\begin{stackTL}
            (\<ge> (\<add> a_1\; \ti{\<ithreetwo>}{o}) \ti{\<ithreetwo>}{0}), 
            \\ (\<le> (\<add> a_1\; (\<add> \ti{\<ithreetwo>}{o+width}) \ti{\<ithreetwo>}{0}) \ti{\<ithreetwo>}{n*64\text{Ki}})
        \end{stackTL}}
    }
    {
        C \vdash t.\<storepc> tp^{?}\; align\; o : \ti{\<ithreetwo>}{a_1}\;\ti{t}{a_2};l;\phi \rightarrow \epsilon;l;\phi
    }
\end{mathpar}

Indirect function calls in \wasm require a dynamic check to ensure that the index into the table points to a function of a suitable type (recall the explanation of tables and \<callindirect> from \autoref{sec:wasmsemantics}).
Proving the safety of an indirect function call involves showing that every possible function that could be called will not cause a run-time type error.
We ensure this by requiring that the type of every function at every possible index value has a subtype of the expected type: $\forall 0 \leq i < j. (\phi \implies \neg (= \ti{\<ithreetwo>}{i}\; a)) \lor tfis(i) <: tfi$ where $tfis=(tfi_2 ...)$.
Further, we must show that the provided table index is within the table boundaries: $\phi \implies 0 \leq c < j$.
\begin{mathpar}
    \inferrule*[right=\defrule{Call-Indirect-Prechk}]{ %% call_indirect
        C_{table}(i) = (j, (tfi_2 ...)) \and
        \phi \implies 0 \leq c < j \\
        tfi = ti_1^{*};l_1;\phi_1 \rightarrow ti_2^{*};l_2;\phi_2 \\
        \forall 0 \leq i < j.\;
        {\begin{stackTL}
            (\phi \implies \neg (= \ti{\<ithreetwo>}{i}\; a)) 
            \\ \lor\; tfis(i) <: tfi$ where $tfis=(tfi_2 ...)
        \end{stackTL}}
    }
    {
        C \vdash \<callindirectpc> tfi : ti_1^{*}\;\ti{i32}{a};l;\phi_1 \rightarrow ti_2^{*};l;\phi_1,\phi_2
    } \and
\end{mathpar}

\subsection{Module Types}
The complete module typing rules are in Figure \ref{fig:modulerules} (not that $im$ is an import and $ex$ is an export).
Functions $f$, typecheck their body $e^{*}$ under the module type context $C$ with the expected postcondition $ti_2^{*};l_2;\phi_2$ in the label stack and return position, and with the local index store $\ti{t_1}{a_1}^{*}\;\ti{t}{a_2}^{*}$ constructed from the function's arguments $\ti{t_1}{a_1}^{*}$ and declared locals $\ti{t}{a_2}^{*}$.
Global variables $glob$ must ensure that their initialization instructions $e^{*}$ produce a value of the proper type $t$.
Exported global variables cannot be mutable, if there are any exports defined, the global cannot have the mutable tag $mut$: $ex^{*}=\epsilon \lor tg=t$.
Tables $tab$ ensure that the indices $i^{n}$ refer to well-typed functions and there are exactly as many indices as the expected size $n$.
Memory $mem$ simply has its declared initial size $n$ from which it can only grow bigger.
All imported functions, globals, tables, and memories are expected to have their declared type.
They are type checked during linking.

Type checking a module involves typechecking every component of the module.
Functions, $f$, are typechecked under the module type context, $C$, containing the entirety of the module.
This means that functions can refer to themselves, other functions, all globals, the table, and memory.
This may seem to be a circular definition, but the type of the module is declared statically (as the combined declared types of all the module components), so it is just checking against the expected module index type context.
Globals, $glob$, are typechecked under the module index context containing only the global variable declarations preceding the current declaration.

\begin{figure}[h]
    \begin{mathpar}
        \inferrule*[]{ %% local function
            tfi = \ti{t_1}{a_1}^{*};\epsilon;\phi_1 \rightarrow ti_2^{*};l_2;\phi_2 \\
            C_2 = C,\text{local } t_1^{*}\;t^{*},\text{label } (ti_2^{*};l_2;\phi_2),\text{return } (ti_2^{*},l_2,\phi_2) \\
            C_2 \vdash e^{*} : \epsilon ;\ti{t_1}{a_1}^{*}\;\ti{t}{a_2}^{*};\phi_1 \rightarrow ti_2^{*};l_2;\phi_2
        } {
            C \vdash ex^{*}\; \<func> tfi\; \<local> t^{*}\; e^{*} : ex^{*}\; tfi
        } \and

        \inferrule*[]{ %% imported function
        } {
            C \vdash ex^{*}\; \<func> tfi\; im : ex^{*}\; tfi
        } \\

        \inferrule*[]{ %% local global
            tg = mut^{?}\;t \and
            ex^{*} = \epsilon \lor tg = t \and
            C \vdash e^{*} : \epsilon; \epsilon; \phi_1 \rightarrow \ti{t}{a};\epsilon; \phi_2
        } {
            C \vdash ex^{*}\; \<global> tg\; e^{*} : ex^{*}\; tg
        } \and

        \inferrule*[]{ %% imported global
            tg = t \and
        } {
            C \vdash ex^{*}\; \<global> tg\; im : ex^{*}\; tg
        } \and

        \inferrule*[]{ %% local table
            (C_{\text{func}}(i) = tfi)^n
        } {
            C \vdash ex^{*}\; \<table> n\; i^n : ex^{*}\; (n,tfi^n)
        } \and

        \inferrule*[]{ %% imported table
        } {
            C \vdash ex^{*}\; \<table> (n,tfi^n)\; im : ex^{*}\; (n,tfi^n)
        } \and

        \inferrule*[]{ %% local memory
        } {
            C \vdash ex^{*}\; \<memory> n : ex^{*}\; n
        }

        \inferrule*[]{ %% imported memory
        } {
            C \vdash ex^{*}\; \<memory> n\; im : ex^{*}\; n
        }

        \inferrule*[]{ %% module
            (C\vdash f : ex_f^{*}\; tfi)^{*} \and
            (C_i \vdash glob_i : ex_g^{*}\; tg_i)^{*} \\
            (C \vdash tab : ex_t^{*}\; (n,tfi^n))^{?} \and
            (C \vdash mem : ex_m^{*}\; n)^{?} \\
            (C_i=\{\text{global } tg^{i-1}\})_i^{*} \and
            ex_f^{*\;*}\; ex_g^{*\;*}\; ex_t^{*\;?}\; ex_m^{*\;?} \text{ distinct} \\
            C = \{\text{func } tfi^{*}, \text{global } tg^{*}, \text{table } (n,tfi^n)^{?}, \text{memory } n^{?}\}  
        } {
            \vdash \<module> f^{*}\; glob^{*}\; tab^{?}\; mem^{?}
        }
    \end{mathpar}
    \caption{Indexed Module Typing Rules}
    \label{fig:modulerules}
\end{figure}
\chapter{Type Safety}
\label{chp:typesafety}

\todo{The leamms seem to be messing with the section counters somehow...}

\section{Subject Reduction}
\section{Subject Reduction}

\begin{theorem}{Subject Reduction}
  Bla bla bla
\end{theorem}
\begin{proof}
By case analysis on the reduction rules.
\todo{I want to omit locals and globals, but I also think it's good to keep them for completeness.}
Note that the locals and globals are omitted where they are unnecessary and trivially unchanged.

\thought{I'm like 90\% sure that a good chunk of these proofs are only understandable by me, to the extent that I understand what I'm doing.}

\thought{I keep appealing to basically an inversion lemma that is never stated but seems obvious. It would be a good idea to formalize this eventually (shouldn't be too difficult). In the meanwhile assume I mean that when I say "by x, we know y" where y is not a premise of the typing rule x or where I say "y is of the form z".}

\todo{Type rule names based on code from William's dissertation to match those used here}

\begin{itemize}
    %% Binop -> const
    \item $C\vdash (t.\<const> c_1)\; (t.\<const> c_2)\; t.\<binop> : s_1;l_1;g_1;\phi_1 \rightarrow s_2;l_2;g_2;\phi_2$ 
    \\ $\land$ $(t.\<const> c_1)\; (t.\<const> c_2)\; t.\<binop> \hookrightarrow t.\<const> c$ if $c=\<binop>(c_1,c_2)$

        By $const$ and $binop$, we know that $s_2$ has the form $s_1 \ti{t}{a_3}$ and that
        \begin{align*}
            \phi_1&, 
            \begin{stackTL}
                \ti{t}{a_1}, (\<eq> a_1\;\ti{t}{c_1}), \\
                \ti{t}{a_2}, (\<eq> a_2 (t c_2)), \\
                \ti{t}{a_3}, (\<eq> a_3\;(\<binop>\; a_1\; a_2))
            \end{stackTL} \\    
            &\implies \phi_2
        \end{align*}

        By $const$, $C \vdash t.\<const> c :
            \begin{stackTL}
                \epsilon;l_1;g_1;\phi_1 \\ 
                \rightarrow \ti{t}{a_3};l_1;g_1;\phi_1,\ti{t}{a_3},(\<eq> a_3\;\ti{t}{c})
            \end{stackTL}$.

        Because $c=binop_t(c_1,c_2)$, then by $\implies$,
        \begin{align*}
            \phi_1,\ti{t}{a},(\<eq> a\;\ti{t}{c}) &\implies \phi_1,
            \begin{stackTL}
                \ti{t}{a_1}, (\<eq> a_1\; \ti{t}{c_1}), \\
                \ti{t}{a_2}, (\<eq> a_2\; \ti{t}{c_2}), \\
                \ti{t}{a_3}, (\<eq> a_3\;(\<binop>\;a_1 a_2))
            \end{stackTL}
        \end{align*}

        Therefore, $C \vdash (t.\<const> c) : s_1;l_1;g_1;\phi_1 \rightarrow s_1\; \ti{t}{a_3};l_1;g_1;\phi_2$, by $stack-poly$ and $weakening$

    %% Binop -> trap
    \item  $C\vdash (t.\<const> c_1)\; (t.\<const> c_2)\; t.\<binop> : s_1;l_1;g_1;\phi_1 \rightarrow s_2;l_2;g_2;\phi_2$
    \\ $\land$ $(t.\<const> c_1)\; (t.\<const> c_2)\; t.\<binop> \hookrightarrow \<trap>$

        Trivially, $C\vdash \<trap> : s_1;l_1;g_1;\phi_1 \rightarrow s_2;l_2;g_2;\phi_2$ by $trap$.

    %% Relop
    \item $C\vdash (t.\<const> c_1)\; (t.\<const> c_2)\; t.\<relop> : s_1;l_1;g_1;\phi_1 \rightarrow s_2;l_2;g_2;\phi_2$
    \\$\land$ $(t.\<const> c_1)\; (t.\<const> c_2)\; t.\<relop> \hookrightarrow t.\<const> c$ if $c=\<relop>(c_1,c_2)$

        \todo{I think right now it gets translated a little differently because the redex-model typing rule expands the index constraint a bit more to resemble what gets passed to Z3. I think this is better to fix in the redex-model so it resembles WASM execution more closely than Z3 constraints.}
        Similar to $\<binop>$.

    %% Testop
    \item $C\vdash (t.\<const> c)\; t.\<testop> : s_1;l_1;g_1;\phi_1 \rightarrow s_2;l_2;g_2;\phi_2$ 
    \\ $\land$ $(t.\<const> c)\; (t.\<const> c)\; t.\<testop> \hookrightarrow t.\<const> c_2$ where $c_2=\<testop>(c)$

        By $const$ and $testop$, we know that $s_2$ has the form $s_1 \ti{t}{a_2}$, $l_2=l_1$, $g_2=g_1$, and that
        \begin{align*}
            \phi_1&, 
            \begin{stackTL}
                \ti{t}{a_1}, (\<eq> a_1\;\ti{t}{c}), \\
                \ti{t}{a_2}, (\<eq> a_2\;(\<testop>\;a_1))
            \end{stackTL} \\
            &\implies \phi_2
        \end{align*}

        By $const$, $C \vdash t.\<const> c :
            \begin{stackTL}
                \epsilon;l_1;g_1;\phi_1 \\ 
                \rightarrow \ti{t}{a_2};l_1;g_1;\phi_1,\ti{t}{a_2},(\<eq> a_2\;\ti{t}{c_2})
            \end{stackTL}$.

        Because $c_2=testop_t(c)$, then by $\implies$,
        \begin{align*}
            \phi_1,\ti{t}{a},(\<eq> a\;\ti{t}{c_2}) &\implies \phi_1,
            \begin{stackTL}
                \ti{t}{a_1}, (\<eq> a_1\;\ti{t}{c}), \\
                \ti{t}{a_2}, (\<eq> a_2\;(\<testop>\;a_1))
            \end{stackTL}
        \end{align*}

    %% Unreachable
    \item $C\vdash \<unreachable> : s_1;l_1;g_1;\phi_1 \rightarrow s_2;l_2;g_2;\phi_2$
    \\ $\land$ $\<unreachable> \hookrightarrow \<trap>$
    
        Trivially, $C\vdash \<trap> : s_1;l_1;g_1;\phi_1 \rightarrow s_2;l_2;g_2;\phi_2$ by $trap$.

    %% Nop
    \item $C\vdash \<nop> : \epsilon;l;g;\phi \rightarrow \epsilon;l;g;\phi$ 
    \\ $\land$ $\<nop> \hookrightarrow \epsilon$
    
        Trivially, $C\vdash \epsilon : \epsilon;l;g;\phi \rightarrow \epsilon;l;g;\phi$ by $empty$. \todo{Not quite this trivial: what about extension and weakening?}

    %% Drop
    \item $C\vdash (t.\<const> c)\; \<drop> : s_1;l_1;g_1;\phi_1 \rightarrow s_2;l_2;g_2;\phi_2$
    \\ $\land$ $(t.\<const> c)\; \<drop> \hookrightarrow \epsilon$

        By $const$ and $drop$, we know that $s_2 = s_1$, $l_2 = l_1$, $g_2 = g_1$, and $\phi_1 \implies \phi_2$.

        By $empty$, $C\vdash \epsilon : \epsilon;l_1;g_1;\phi_1 \rightarrow \epsilon;l_1;g_1;\phi_1$

        By $stack-poly$, $C\vdash \epsilon : s_1;l_1;g_1;\phi_1 \rightarrow s_1;l_1;g_1;\phi_1$

        Since $\phi_1 \implies \phi_2$, then $C\vdash \epsilon : s_1;l_1;g_1;\phi_1 \rightarrow s_1;l_1;g_1;\phi_2$ by $weakening$.

    %% Select
    \item Case: $C\; {\begin{stackTL}
        \vdash (t.\<const> c_1)\;(t.\<const> c_2)\;(\<ithreetwo>.\<const> 0)\;\<select> 
        \\ : s_1;l_1;g_1;\phi_1 \rightarrow s_2;l_2;g_2;\phi_2
    \end{stackTL}}$
    \\ $\land$ $(t.\<const> c_1)\;(t.\<const> c_2)\;(\<ithreetwo>.\<const> 0)\;\<select> \hookrightarrow (t.\<const> c_2)$

        By $const$ and $select$, we know that $s_2$ has the form $s_1\;\ti{a_3}$, $l_2 = l_1$, $g_2 = g_1$, and
        $
        {\begin{stackTL}
            \phi_1, {\begin{stackTL}
                \ti{t}{a_1}, (\<eq> a_1\;\ti{t}{c_1}), \\
                \ti{t}{a_2}, (\<eq> a_2\;\ti{t}{c_2}), \\
                \ti{\<ithreetwo>}{a}, (\<eq> a\;\ti{\<ithreetwo>}{0}), \\
                \ti{t}{a_3}, (
                {\begin{stackTL}
                    (\<eqz> a) \land (\<eq> a_3\;a_2)) \\
                    \lor (\neg(\<eqz> a) \land (\<eq> a_3\;a_1)))
                \end{stackTL}} \\
            \end{stackTL}} \\
            \implies \phi_2
        \end{stackTL}}
        $
        
        By $const$, \\
        $ C \vdash (t.\<const> c_2) :
            {\begin{stackTL}
                \epsilon;l_1;g_1;\phi_1 \\
                \rightarrow \ti{t}{a_3};l_1;g_1;\phi_1,\ti{t}{a_3},(\<eq> a_3\; \ti{t}{c_2}) \\
            \end{stackTL}} $

        Then, by $stack-poly$,\\
        $ C \vdash (t.\<const> c_2) : 
            {\begin{stackTL}
                s_1;l_1;g_1;\phi_1 \\
                \rightarrow s_1\;\ti{t}{a_3};l_1;g_1;\phi_1,\ti{t}{a_3},(\<eq> a_3 \; \ti{t}{c_2}) \\
            \end{stackTL}} $

        By $implies$, we have \\
        $\phi_1,\ti{t}{a_3},(\<eq> a_3\; \ti{t}{c_2}) \implies \phi_1, {\begin{stackTL}
            \ti{t}{a_1}, (\<eq> a_1\; \ti{t}{c_1}), \\
            \ti{t}{a_2}, (\<eq> a_2\; \ti{t}{c_2}), \\
            \ti{\<ithreetwo>}{a}, (\<eq> a\;\ti{\<ithreetwo>}{0}), \\
            \ti{t}{a_3}, (
            {\begin{stackTL}
                (\<eqz> a) \land (\<eq> a_3\;a_2)) \\
                \lor (\neg(\<eqz> a) \land (\<eq> a_3\;a_1)))
            \end{stackTL}} \\
        \end{stackTL}} \\$

        Therefore,
        $ C \vdash (t.\<const> c_2) :
            s_1;l_1;g_1;\phi_1 
            \rightarrow s_1\;\ti{t}{a_3};l_1;g_1;\phi_2$, by $weakening$

    %% Block
    \item Case: $C \vdash v^n \; \<block> tfi \; e^{*} \<end> : s_1;l_1;g_1;\phi_1 \rightarrow s_2;l_2;g_2;\phi_2$
    \\ $\land$ $v^n \; \<block> tfi \; e^{*} \<end> \hookrightarrow \<label>_m \{ \epsilon \} \; v^n \; e^{*} \<end>$

        \thought{Technically we can't yet have $\ti{t}{a}^n=ti_1^{n}$ as we will derive that by $composition$ on $const$ and $block$. However, I'm lazy and will address this later because it is a minor detail and I want to focus on major details right now.}

        Let $ti_1^n;l_3;g_3;\phi_3 \rightarrow ti_2^m;l_4;g_4;\phi_4=tfi$, $(t.\<const> c)^n=v^n$, and $\ti{t}{a}^n=ti_1^{n}$.

        $C \vdash (t.\<const> c)^n : \epsilon;l_1;g_1;\phi_1 \rightarrow \\ \; \ti{t}{a}^n;l_1;g_1;\phi_1,\ti{t}{a}^n,(\<eq> a \; \ti{t}{c})^n$ by $const$.

        $C \vdash (t.\<const> c)^n : s_1;l_1;g_1;\phi_1 \rightarrow \\ \; s_1\; \ti{t}{a}^n;l_1;g_1;\phi_1,\ti{t}{a}^n,(\<eq> a \; \ti{t}{c})^n$ by $stack-poly$.
    
        $C \vdash \<block> tfi \; e^{*} \<end> : s_1 \; \ti{t}{a}^n;l_1;g_1;\phi_1,\ti{t}{a}^n,(\<eq> a \; \ti{t}{c})^n \rightarrow s_2;l_2;g_2;\phi_2$ by $composition$.

        \todo{1: we should derive $l_1=l_3$,$g_1=g_3$, etc... and 2: add a parenthetical about usage}.

        $C \vdash \<block> tfi \; e^{*} \<end> : s_1 \; \ti{t}{a}^n;l_1;g_1;\phi_1,\ti{t}{a}^n,(\<eq> a \; \ti{t}{c})^n \rightarrow s_1\; ti_2^{m};l_2;g_2;\phi_2$ by inversion, and therefore $s_2=s_1\; ti_2^{m}$.

        Further, $\phi_1,\ti{t}{a}^n,(\<eq> a \; \ti{t}{c})^n \implies \phi_3$ by $weakening$, and $\phi_4 \implies \phi_2$ by $block$ and $weakening$.

        $C,\text{label}(t_2^{m};l_4;g_4;\phi_4) \vdash (t.\<const> c)^n : \epsilon;l_1;g_1;\phi_1 \rightarrow \\ \ti{t}{a}^n;l_1;g_1;\phi_1,\ti{t}{a}^n,(\<eq> a \; \ti{t}{c})^n$ by $const$.

        $C,\text{label}(t_2^{m};l_4;g_4;\phi_4) \vdash e^{*} : ti_1^n;l_1;g_1;\phi_3 \rightarrow ti_2^m;l_4;g_4;\phi_4$ because it is a sub-derivation of $block$ which we have already assumed to hold.

        Then $C,\text{label}(t_2^{m};l_4;g_4;\phi_4) \vdash (t.\<const> c)^n\; e^{*} : \epsilon;l_1;g_1;\phi_1 \rightarrow \\ ti_2^m;l_4;g_4;\phi_4$ by $composition$.

        By $empty$ and $stack-poly$, $C \vdash \epsilon : ti_2^m;l_4;g_4;\phi_4 \rightarrow ti_2^m;l_4;g_4;\phi_4$.

        Therefore, $C \vdash \<label>_m \{ \epsilon \} \; v^n \; e^{*} \<end> : \epsilon;l_1;g_1;\phi_1 \rightarrow ti_2^m;l_4;g_4;\phi_4$ by $label$.

        Since $s_2 = s_1\; ti_2^m$, $l_2 = l_4$, $g_2 = g_4$, and $\phi_4 \implies \phi_2$, then by $stack-poly$ and $weakening$ we have: $C \vdash \<label>_m \{ \epsilon \} \; v^n \; e^{*} \<end> : s_1;l_1;g_1;\phi_1 \rightarrow ti_2^m;l_2;g_2;\phi_2$

    \item Case: $C \vdash v^n \; \<loop> tfi \; e^{*} \<end> : s_1;l_1;g_1;\phi_1 \rightarrow s_2;l_2;g_2;\phi_2$
    \\ $\land$ $v^n \; \<loop> tfi \; e^{*} \<end> \hookrightarrow \<label>_n \{ \<loop> tfi \; e^{*} \<end> \} \; v^n \; e^{*} \<end>$

        \thought{Same as above about $\ti{t}{a}^n=ti_1^{n}$.}

        Let $ti_1^n;l_3;g_3;\phi_3 \rightarrow ti_2^m;l_3;g_3;\phi_4=tfi$, $(t.\<const> c)^n=v^n$, and $\ti{t}{a}^n=ti_1^{n}$.

        $C \vdash (t.\<const> c)^n : \epsilon;l_1;g_1;\phi_1 \rightarrow \\ \; \ti{t}{a}^n;l_1;g_1;\phi_1,\ti{t}{a}^n,(\<eq> a \; \ti{t}{c})^n$ by $const$.

        $C \vdash (t.\<const> c)^n : s_1;l_1;g_1;\phi_1 \rightarrow \\ \; s_1\; \ti{t}{a}^n;l_1;g_1;\phi_1,\ti{t}{a}^n,(\<eq> a \; \ti{t}{c})^n$ by $stack-poly$.
    
        $C \vdash \<loop> tfi \; e^{*} \<end> : s_1 \; \ti{t}{a}^n;l_1;g_1;\phi_1,\ti{t}{a}^n,(\<eq> a \; \ti{t}{c})^n \rightarrow s_2;l_2;g_2;\phi_2$ by $composition$, and therefore $l_1=l_3$ and $g_1=g_3$ (Note: since $l_1=l_3$ and $g_1=g_3$, I will use $l_1$ and $g_1$ in place of $l_3$ and $g_3$, respectively, for the rest of the proof).

        \thought{Do we need this next type rule statement? We can just get rid of it and have ``By inversion ...''}

        $C \vdash \<loop> tfi \; e^{*} \<end> : s_1 \; \ti{t}{a}^n;l_1;g_1;\phi_1,\ti{t}{a}^n,(\<eq> a \; \ti{t}{c})^n \rightarrow s_1\; ti_2^{m};l_2;g_2;\phi_2$ by inversion, and therefore $s_2=s_1\; ti_2^{m}$, $l_4=l_2$, and $g_4=g_2$ (Note: since $l_2=l_4$ and $g_2=g_4$, I will use $l_4$ and $g_4$ in place of $l_2$ and $g_2$, respectively, for the rest of the proof).

        Further, $\phi_1,\ti{t}{a}^n,(\<eq> a \; \ti{t}{c})^n \implies \phi_3$ by $weakening$, and $\phi_4 \implies \phi_2$ by $loop$ and $weakening$.

        $C,\text{label}(t_1^{n};l_1;g_1;\phi_1) \vdash (t.\<const> c)^n : \epsilon;l_1;g_1;\phi_1 \rightarrow \\ \ti{t}{a}^n;l_1;g_1;\phi_1,\ti{t}{a}^n,(\<eq> a \; \ti{t}{c})^n$ by $const$.

        $C,\text{label}(t_1^{n};l_1;g_1;\phi_1) \vdash e^{*} : ti_1^n;l_1;g_1;\phi_3 \rightarrow ti_2^m;l_1;g_1;\phi_4$ because it is a sub-derivation of $loop$ which we have already assumed to hold.

        Then $C,\text{label}(t_1^{n};l_1;g_1;\phi_1) \vdash (t.\<const> c)^n\; e^{*} : \epsilon;l_1;g_1;\phi_1 \rightarrow \\ ti_2^m;l_4;g_4;\phi_4$ by $composition$.

        $C \vdash \<loop> tfi \; e^{*} \<end> : \ti{t}{a}^n;l_1;g_1;\phi_1,\ti{t}{a}^n,(\<eq> a \; \ti{t}{c})^n \rightarrow ti_2^{m};l_4;g_4;\phi_4$ by $loop$.

        Therefore, $C \vdash \<label>_m \{ \<loop> tfi \; e^{*} \<end> \} \; v^n \; e^{*} \<end> : \epsilon;l_1;g_1;\phi_1 \rightarrow ti_2^m;l_4;g_4;\phi_4$ by $label$.

        Since $s_2 = s_1\; ti_2^m$, $l_2 = l_4$, $g_2 = g_4$, and $\phi_4 \implies \phi_2$, then by $stack-poly$ and $weakening$ we have: $C \vdash \<label>_m \{ \epsilon \} \; v^n \; e^{*} \<end> : s_1;l_1;g_1;\phi_1 \rightarrow ti_2^m;l_2;g_2;\phi_2$

    \item Case: $C \vdash (\<ithreetwo>.\<const> 0) \; \<if> tfi \; e_1^{*} \<else> e_2^{*} \<end> : s_1;l_1;g_1;\phi_1 \rightarrow s_2;l_2;g_2;\phi_2$
    \\ $\land$ $(\<ithreetwo>.\<const> 0) \; \<if> tfi \; e_1^{*} \<else> e_2^{*} \<end> \hookrightarrow \<block> tfi \; e_2^{*} \<end>$

        Let $tfi = ti_1^n \; \ti{<ithreetwo>}{a};l_3;g_3;\phi_3 \rightarrow ti_2^m;l_4;g_4;\phi_4$, \\ $tfi_1 = ti_1^n;l_3;g_3;\phi_3,\neg(\<eqz> a) \rightarrow ti_2^m;l_4;g_4;\phi_4$, \\
        and $tfi_2 = ti_1^n;l_3;g_3;\phi_3,(\<eqz> a) \rightarrow ti_2^m;l_4;g_4;\phi_4$.

        $C \vdash \<if> tfi \; e_1^{*} \<else> e_2^{*} \<end> : tfi_1$ by $if$.

        By $inversion$, $s_1=s \; ti_1^{n}$ and $s_2=s \; ti_2^{m}$ for some $s$, $l_1=l_3$, $g_1=g_3$, $l_2=l_4$, $g_2=g_4$, $\phi_1,\ti{\<ithreetwo>}{a},(\<eq> a\; 0) \implies \phi_3$, and $\phi_4 \implies \phi_2$.

        \todo{Might want to spell this out a bit more, there's like three different levels of $inversion$ used to get here, which is valid but might be overly obfuscated}

        $C,\text{label}(ti_2^m;l_4;g_4;\phi_4) \vdash e_2^{*} : tfi_2$ because it is a sub-derivation of $if$ which we have assumed to hold by $inversion$.

        Then, $C \vdash \<block> tfi_2 \; e_2^{*} \<end>$ by $block$.

        \thought{We can appeal to implication because $\ti{\<ithreetwo>}{a}$ is freshly introduced by $const$, this is an important point that should be mentioned somewhere, also, since $a$ is fresh, we know this won't cause a contradiction in $\phi$}.

        \thought{That being said, maybe we want use $\equiv$ instead of $\implies$ to handle this just as a presentation detail.}

        $\phi_1 \implies \phi_1,\ti{t}{a},(\<eqz> a)$ by $\implies$.

        Therefore, $C \vdash \<block> tfi_2\; e_2^{*} \<end> : \\ s \; ti_1^n;l_1;g_1;\phi_1,\ti{t}{a},(\<eqz> a) \rightarrow s\; ti_2^m;l_2;g_2;\phi_2$ by $extension$ and $weakening$.

    \item Case: $C \vdash (\<ithreetwo>.\<const> k+1) \; \<if> tfi \; e_1^{*} \<else> e_2^{*} \<end> : s_1;l_1;g_1;\phi_1 \rightarrow s_2;l_2;g_2;\phi_2$
    \\ $\land$ $(\<ithreetwo>.\<const> k+1) \; \<if> tfi \; e_1^{*} \<else> e_2^{*} \<end> \hookrightarrow \<block> tfi \; e_1^{*} \<end>$

        Similar to above.

    \item Case: $C \vdash \<label>_n \{ e^{*} \} \; v^n \<end> : s_1;l_1;g_1;\phi_1 \rightarrow s_2;l_2;g_2;\phi_2$
    \\ $\land$ $\<label>_n \{ e^{*} \} \; v^n \<end> \hookrightarrow v^n$

        $C \vdash \<label>_n \{ e^{*} \} \; v^n \<end> : \epsilon;l_1;g_1;\phi_1 \rightarrow ti_2^{*};l_2;g_2;\phi_2$ by $label$ and $inversion$.

        By $inversion$, we know $s_2=s_1\;ti_2^{*}$.

        $C \vdash v^n : \epsilon;l_1;g_1;\phi_1 \rightarrow ti_2^{*};l_2;g_2;\phi_2$ because it is a premise of $label$ which we have assumed to hold.

        Therefore, $C \vdash v^n : s_1;l_1;g_1;\phi_1 \rightarrow s_1\; ti_2^{*};l_1;g_1;\phi_2$ by $stack-poly$.

    \item Case: $C \vdash \<label>_n \{ e^{*} \} \; \<trap> \<end> : s_1;l_1;g_1;\phi_1 \rightarrow s_2;l_2;g_2;\phi_2$
    \\ $\land$ $\<label>_n \{ e^{*} \} \; \<trap> \<end> \hookrightarrow \<trap>$

        Trivially, $C\vdash \<trap> : s_1;l_1;g_1;\phi_1 \rightarrow s_2;l_2;g_2;\phi_2$ by $trap$.

    \item Case: $C \vdash \<label>_n \{ e^{*} \} \; L^k [v^n \; (\<br> j)] \<end> : s_1;l_1;g_1;\phi_1 \rightarrow s_2;l_2;g_2;\phi_2$
    \\ $\land$ $\<label>_n \{ e^{*} \} \; L^k [v^n \; (\<br> j)] \hookrightarrow v^n \; e^{*}$

        \thought{Intuitively, this follows because \<br> is typed based on the label in $C$ from the $label$ rule, and everything follows from that. I'm not sure exactly how hand-wavy I can get away with being describing that though.}

        \todo{The original \wasm proof addresses this using a lemma about labels which will be easy to recreate. In the meantime I will refer to that (which is the precise description of what I just said).}

        By $inversion$, $s_2=s_1\;ti_2^{*}$.

        $C,\text{label}(ti_1^n;l_1;g_1;\phi_3)^j \vdash v^n\; (\<br> j) : \epsilon;l_1;g_1;\phi_1 \rightarrow s_2;l_2;g_2;\phi_2$ since it is a premise of $label$ which we have assumed to hold.

        $C,\text{label}(ti_1^n;l_1;g_1;\phi_3)^j \vdash (\<br> j) : ti_1^n;l_1;g_1;\phi_3 \rightarrow s_2;l_2;g_2;\phi_2$, by $label-filled$.

        Then, $C,\text{label}(ti_1^n;l_1;g_1;\phi_3)^j \vdash v^n : \epsilon;l_1;g_1;\phi_1 \rightarrow ti_1^n;l_1;g_1;\phi_3$ since it is a premise of $composition$ which we have assumed to hold.

        $C \vdash e^{*} : ti_1^n;l_1;g_1;\phi_3 \rightarrow ti_2^{*};l_2;g_2;\phi_4$ since it is a premise of $label$ which we have assumed to hold, and $\phi_4 \implies \phi_2$ by $inversion$.

        Then, $C \vdash v^n \; e^{*} : \epsilon;l_1;g_1;\phi_1 \rightarrow ti_2^{*};l_2;g_2;\phi_4$ by $composition$.

        Finally, $C \vdash v^n \; e^{*} : s_1;l_1;g_1;\phi_1 \rightarrow s_1\;ti_2^{*};l_2;g_2;\phi_2$ by $stack-poly$ and $weakening$.

    \item Case: $C \vdash (\<ithreetwo>.\<const> 0)\;(\<brif> j) : s_1;l_1;g_1;\phi_1 \rightarrow s_2;l_2;g_2;\phi_2$
    \\ $\land$ $(\<ithreetwo>.\<const> 0)\;(\<brif> j) \hookrightarrow \epsilon$

        By $br \_ if$, $C \vdash (\<brif> j) : s_1\; \ti{\<ithreetwo>}{a};l_1;g_1;\phi_1,\ti{t}{a},(\<eq> a\; \ti{\<ithreetwo>}{0}) \rightarrow s_1;l_1;g_1;\phi_1,\ti{t}{a},(\<eq> a\; \ti{\<ithreetwo>}{0}),(\<eqz> a)$.

        $C \vdash (\<ithreetwo>.\<const> 0) : s_1;l_1;g_1;\phi_1 \rightarrow s\;\ti{\<ithreetwo>}{a};l;g;\phi,\ti{\<ithreetwo>}{a},(\<eq> a\; \ti{\<ithreetwo>}{0})$ by $const$.

        By $composition$, $C \vdash (\<ithreetwo>.\<const> 0)\;(\<brif> j) : s_1;l_1;g_1;\phi_1 \rightarrow s_1;l_1;g_1;\phi_1,\ti{\<ithreetwo>}{a},(\<eq> a\; \ti{\<ithreetwo>}{0}),(\<eqz> a)$.

        Then, by $inversion$, $s_1=s_2$, $l_1=l_2$, $g_1=g_2$, and $\phi_1,\ti{\<ithreetwo>}{a},(\<eq> a\; \ti{\<ithreetwo>}{0}),(\<eqz> a) \implies \phi_2$.

        $C \vdash \epsilon : \epsilon;l_1;g_1;\phi_1 \rightarrow \epsilon;l_1;g_1;\phi_1$ by $empty$.

        $C \vdash \epsilon : s_1;l_1;g_1;\phi_1 \rightarrow s_1;l_1;g_1;\phi_1$ by $stack-poly$.

        \thought{I'm not super keen on this way of presenting because the two completely different universes used before and after reduction may be easily muddled into one by a reader. I think the key is to just be really explicit that these are two different type universes and different things fly in different places.}

        $\phi_1 \implies \phi_1,\ti{\<ithreetwo>}{a},(\<eq> a\; \ti{\<ithreetwo>}{0}),(\<eqz> a)$ because $a$ is fresh, and therefore $\phi_1 \implies \phi_2$.

        Then, $C \vdash \epsilon : s_1;l_1;g_1;\phi_1 \rightarrow s_1;l_1;g_1;\phi_2$ by $weakening$.

    \item Case: $C \vdash (\<ithreetwo>.\<const> k+1)\;(\<brif> j) : s_1;l_1;g_1;\phi_1 \rightarrow s_2;l_2;g_2;\phi_2$
    \\ $\land$ $(\<ithreetwo>.\<const> k+1)\;(\<brif> j) \hookrightarrow \<br> j$

        $C \vdash (\<ithreetwo>.\<const> k+1) : s_1;l_1;g_1;\phi_1 \rightarrow s\;\ti{\<ithreetwo>}{a};l;g;\phi,\ti{\<ithreetwo>}{a},(\<eq> a\; \ti{\<ithreetwo>}{k+1})$ by $const$.

        By $br \_ if$, $C \vdash (\<brif> j) : s_1\; \ti{\<ithreetwo>}{a};l_1;g_1;\phi_1,\ti{t}{a},(\<eq> a\; \ti{\<ithreetwo>}{k+1}) \rightarrow s_1;l_1;g_1;\phi_1,\ti{t}{a},(\<eq> a\; \ti{\<ithreetwo>}{k+1}),(\<eqz> a)$.

        \thought{We're not rederiving the type of $br\_if$ above, we already have it as part of the derivation.}

        By $composition$, $C \vdash (\<ithreetwo>.\<const> k+1)\;(\<brif> j) : s_1;l_1;g_1;\phi_1 \rightarrow s_1;l_1;g_1;\phi_1,\ti{\<ithreetwo>}{a},(\<eq> a\; \ti{\<ithreetwo>}{k+1}),(\<eqz> a)$.

        $C_label(j)=s_1;l_1;g_1;\phi_1,\ti{t}{a},\neq(\<eqz> a)$ because it is a side condition of $br\_if$ which we have assumed to hold by $inversion$.

        $C \vdash \<br> j : s_1;l_1;g_1;\phi_1,\ti{t}{a},\neq(\<eqz> a) \rightarrow s_2;l_2;g_2;\phi_2$ by $br$.

        Because $a$ is fresh, $\phi_1 \implies \phi_1,\ti{\<ithreetwo>}{a},\neg(\<eqz> a)$.

        Therefore, $C \vdash \<br> j : s_1;l_1;g_1;\phi_1 \rightarrow s_2;l_2;g_2;\phi_2$ by $weakening$.

    \item Case: $\<local>_n \{ i;v^{*}_l \} \; v^n \<end> \hookrightarrow_j \; v^n$

        \begin{itemize}
            \item Case: $i = j$
            \\ $\land$ $S;C \vdash_j \<local>_n \{ j;v_l^{*} \} \; v^n \<end> : s_1;l_1;g_1;\phi_1 \rightarrow s_2;l_1;g_2;\phi_2$

                By $inversion$, $s_2=s_1\;ti^n$.

                $S;(ti^n;l_2;g_2;\phi_2) \vdash_j v_l^{*};v^n : \epsilon;\epsilon;g_1;\phi_1 \rightarrow ti^n;l_2;g_2;\phi_2$ because it is a premise of $local-same-inst$ which we have assumed to hold.

                $(\vdash v_l : ti_l;\epsilon;\epsilon;\phi_v)^{*}$\\ $\land$
                $S;C_l \vdash_j v^n : \epsilon;ti_l^{*};g_1;\phi_1,\phi_v^{*} \rightarrow ti^n;l_2;g_2;\phi_2$ because they are premises of $admin-code$ which we have assumed to hold.

                $\phi_v^{*} = (\ti{t}{a},(\<eq> a \; \ti{t}{c}))^{*}$, where $\ti{t}{a}^{*} = ti_l^{*}$, $(t.\<const> c)^{*} = v_l^{*}$ by $admin-const$.

                Because $a^{*}$ are fresh, $\phi_1 \implies \phi_1,(\ti{t}{a},(\<eq> a \; \ti{t}{c}))^{*}$.

                $S;C_l \vdash_j v^n : \epsilon;ti_l^{*};g_1;\phi_1 \rightarrow ti^n;l_2;g_2;\phi_2$ by $weakening$.

                By $const$, $g_1 = g_2$.

                $S;C \vdash_j v^n : \epsilon;l_1;g_1;\phi_1 \rightarrow ti^n;l_1;g_2;\phi_2$ by $const$.

                Therefore, $S;C \vdash_j v^n : s_1;l_1;g_1;\phi_1 \rightarrow s_2;l_1;g_2;\phi_2$ by $stack-poly$.

            \item Case: $i \neq j$
            \\ $\land$ $S;C \vdash_j \<local>_n \{ i;v_l^{*} \} \; v^n \<end> : s_1;l_1;g_1;\phi_1 \rightarrow s_2;l_1;g_2;\phi_2$

                By $inversion$, $s_2=s_1\;ti^n$.

                $S;(ti^n;l_2;g_4;\phi_2) \vdash_i v_l^{*};v^n : \epsilon;\epsilon;g_3;\phi_1 \rightarrow ti^n;l_2;g_4;\phi_2$ because it is a premise of $local-diff-inst$ which we have assumed to hold.

                $(\vdash v_l : ti_l;\epsilon;\epsilon;\phi_v)^{*}$\\ $\land$
                $S;C_l \vdash_i v^n : \epsilon;ti_l^{*};g_1;\phi_1,\phi_v^{*} \rightarrow ti_n;l_2;g_4;\phi_2$ because they are premises of $admin-code$ which we have assumed to hold.

                $\phi_v^{*} = (\ti{t}{a},(\<eq> a \; \ti{t}{c}))^{*}$, where $\ti{t}{a}^{*} = ti_l^{*}$, $(t.\<const> c)^{*} = v_l^{*}$ by $admin-const$.

                Because $a^{*}$ are fresh, $\phi_1 \implies \phi_1,(\ti{t}{a},(\<eq> a \; \ti{t}{c}))^{*}$.

                $S;C_l \vdash_i v^n : \epsilon;ti_l^{*};g_3;\phi_1 \rightarrow ti^n;l_2;g_4;\phi_2$ by $weakening$.

                $S;C \vdash_j v^n : \epsilon;l_1;g_1;\phi_1 \rightarrow ti^n;l_1;g_1;\phi_2$ by $const$.

                \thought{$g_1$ and $g_2$ are not provably the same, so the above substitution might not be valid.}

                \todo{That substitution is not valid, need to appeal to new $fresh-globals$ rule.}

                Therefore, $S;C \vdash_j v^n : s_1;l_1;g_1;\phi_1 \rightarrow s_2;l_1;g_2;\phi_2$ by $stack-poly$.

        \end{itemize}

    \item Case: $S;C \vdash \<local>_n \{ i;v_l^{*} \} \; \<trap> \<end> : s_1;l_1;g_1;\phi_1 \rightarrow s_2;l_2;g_2;\phi_2$
    \\ $\land$ $\<local>_n \{ i;v_l^{*} \} \; \<trap> \<end> \hookrightarrow \; \<trap>$

        Trivially, $S;C \vdash \<trap> : s_1;l_1;g_1;\phi_1 \rightarrow s_2;l_2;g_2;\phi_2$ by $trap$.

    \item Case: $\<local>_n \{ i;v^{*}_l \} \; L^k[v^n \; \<return>] \<end> \hookrightarrow_j \; v^n$

        \begin{itemize}
            \item Case: $i = j$
            \\ $\land$ $S;C \vdash_j \<local>_n \{ j;v_l^{*} \} \; L^k[v^n \; \<return>] \<end> : s_1;l_1;g_1;\phi_1 \rightarrow s_2;l_1;g_2;\phi_2$

            By $inversion$, $s_2=s_1\;ti^n$.

            $S;(ti^n;l_2;g_2;\phi_2) \vdash_j v_l^{*};L^k[v^n \; \<return>] : \epsilon;\epsilon;g_1;\phi_1 \rightarrow ti^n;l_2;g_2;\phi_2$ because it is a premise of $local-same-inst$ which we have assumed to hold.

            $(\vdash v_l : ti_l;\epsilon;\epsilon;\phi_v)^{*}$\\ $\land$
            $S;C_l \vdash_j L^k[v^n \; \<return>] : \epsilon;ti_l^{*};g_1;\phi_1,\phi_v^{*} \rightarrow ti_n;l_2;g_2;\phi_2$,
            where $C_l = S_{\text{inst}}(j),\text{local} \; t_v^{*}, \text{return} \; ti^n;l_2;g_2;\phi_2$
            because they are premises of $admin-code$ which we have assumed to hold.

            $\phi_v^{*} = \circ,(\ti{t}{a},(\<eq> a \; \ti{t}{c}))^{*}$, where $\ti{t}{a}^{*} = ti_l^{*}$, $(t.\<const> c)^{*} = v_l^{*}$ by $admin-const$.

            Because $a^{*}$ are fresh, $\phi_1 \implies \phi_1,(\ti{t}{a},(\<eq> a \; \ti{t}{c}))^{*}$.

            $S;C_l \vdash_j L^k[v^n \; \<return>] : \epsilon;ti_l^{*};g_1;\phi_1 \rightarrow ti^n;l_2;g_2;\phi_2$ by $weakening$.

            $S;C_l \vdash_j \<return> : s_3 \; ti^n;l_2;g_2;\phi_3 \rightarrow s_4;l_4;g_4;\phi_4$, and $\phi_3 \implies \phi_2$ by $return$.

            \thought{We know such a $\phi_3$ exists because the expression is well-typed? Do we need to appeal to $inversion$ since we're gaining information?}

            $S;C_l \vdash_j v^n : \epsilon;l_2;g_2;\phi_5 \rightarrow ti^n;l_2;g_2;\phi_3$ by $const$ and $composition$.

            $S;C_l \vdash_j v^n : \epsilon;ti_l^{*};g_1;\phi_1 \rightarrow ti^n;ti_l^{*};g_1;\phi_3$, and $g_1 = g_2$ by (Nested-Type-Preserved).

            $S;C_l \vdash_j v^n : \epsilon;ti_l^{*};g_1;\phi_1 \rightarrow ti^n;ti_l^{*};g_1;\phi_2$ by $subtyping$.

            $S;C \vdash_j v^n : \epsilon;l_1;g_1;\phi_1 \rightarrow ti^n;l_1;g_2;\phi_2$ by $const$.

            Therefore, $S;C \vdash_j v^n : s_1;l_1;g_1;\phi_1 \rightarrow s_2;l_1;g_2;\phi_2$ by $stack-poly$.

            \item Case: $i \neq j$
            \\ $\land$ $S;C \vdash_j \<local>_n \{ i;v_l^{*} \} \; L^k[v^n \; \<return>] \<end> : s_1;l_1;g_1;\phi_1 \rightarrow s_2;l_1;g_2;\phi_2$

            \todo{...}

            Therefore, $S;C \vdash_j v^n : s_1;l_1;g_1;\phi_1 \rightarrow s_2;l_1;g_2;\phi_2$ by \todo{}.

        \end{itemize}

\end{itemize}
\end{proof}


\section{Progress}
\section{Progress}
We proceed by case analysis on the typing rules.

\begin{itemize}
    \item $\inferrule[]{ }{ %% const
    \typerule{t.\<const>c} {
        \insttype{\type{\epsilon}{l}{g}{\phi}}
            {\type{\ti{t}{a}}{l}{g}{\phi,\ti{t}{a},(\<eq>a\;(t\;c))}}}
    }$

    \proof Trivial, since $t.\<const> c$ is a value.

    \item $\inferrule[]{ }{ %% binop
        C \vdash t.\<binop> :
        {\begin{stackTL}
            \ti{t}{a_1}\ti{t}{a_2};l;g;\phi \\
            \rightarrow \ti{t}{a_3};l;g;\phi,\ti{t}{a_3},(\<eq>a_3\;(\<binop>\;a_1\;a_2))
        \end{stackTL}}
    }$
    
    \proof 

\end{itemize}

\chapter{Discussion}
\label{chp:discussion}
 
By creating \name, we have taken the first step towards creating a practical system in which an expressive type system is used with a low-level language for safety and performance.
This is a first step in the sense that it provides the scaffolding to build such a system: unlike prior work, \name provides \emph{concrete} ways to use type information for compiler optimization at the assembly language level.
However, there are still a number of unanswered questions.
We have a number of future ideas for this work some based on what we think is necessary to realize our eventual goal of making \name practical in the real-world, and others based on problems identified during the course of the project so far.

\paragraph{Empirical Evaluation}
The first step would be to implement a compiler for name so we are able to perform experiments and measure the real performance benefit provided by \name.
This would allow us to empirically test whether \name actually improves performance.
Our plan is to implement \name in Rust building on the CraneLift compilier.
This will also require constructing an algorithm to build typing derivations as well as checking them.

\paragraph{More Optimizations}
There is also the potential to find other optimizations we can perform with the additional type information.
For example, remember from \autoref{sec:typesys} that an $\<if>$ may have a contradiction in the index type context one one of its branches.
In this case, that branch will never be executed, and therefore the other branch must always be taken, so we can safely replace the $\<if>$ instruction with the other branch.
We can do similar optimizations with $\<brif>$ and $\<select>$.

\paragraph{Types Annotations}
Recall that the embedding of \wasm into \name from Section \ref{subsec:embedding} does not take advantage of the possibility of using type annotations on functions and blocks to check stronger guarantees about programs.
Type annotations can be added by the developer, who will then get stronger guarantees of correctness along with the potential for more optimizations.
However, we would prefer for the developer not to have to hand-annotate compiled \wasm.
Instead, we could use static analysis to try to find the weakest preconditions that guarantee the safety of \prechk-tagging instructions.
We could also attempt to have have a compiler from a higher-level language to \wasm add annotations as a form of type preserving compilation similar to Sytem F to Typed Assembly Language \cite{FtoTAL}.

\paragraph{Reasoning About Global Variables}
Reasoning about global variables is made difficult because static typechecking is restricted to within the module we are checking.
Thus, it is difficult to reason about global variables imported from another module.
Concretely, imagine, in the $j$th module calling a function $f$ that was imported from the $i$th module.
The call instruction will be reduced to $\<call> \{\text{inst } i, \text{ func } f\}$ where $i$ is the module index for the module instance where $f$ is defined.
Theoretically, $f$ should not change the global variables in the $j$th module.
However, it may call a function in the $j$th module which could change the globals in the $j$th module, and since we do not know what the behavior of $f$ is statically within $j$, we have to assume the worst and can make no assumptions about the global variables after $f$ returns.

\paragraph{Handlind the Dynamic Resizing of Memory}
While linear memory chunks are initialized with a static size, \wasm supports dynamically growing memory using the $\<growmemory>$ instruction.
Currently, \name only supports erasing memory bounds checks based on the static size.
However, it should be possible to reason about the size of memory being increased by inserting a dependency on the result of the $\<growmemory>$ instruction.
If the result is -1, we know that the memory will not have grown and remains the same size.
Otherwise, the result will be equal to the new memory size.
This would require more dependency in the type system then we currently have with indexed types, since static type values would depend on dynamic control flow.

\paragraph{Support Streaming Compilation}
The format of \wasm code allows compilation and execution to begin with only part of the program downloaded.
Similar streaming compilation is theoretically possible with \name, but there are unanswered questions about how to work in typechecking in such a compilation pipeline.
Here are two examples of problems that we expect to face implementing such a system.
First of all, we must make sure such a system is safe, which is complicated by the fact that we may begin executing code before we have finished type checking.
This should not be too much of an issue as long as we ensure code is type checked before we can execute it, so we only execute well-typed code and if we come across code that is not well typed we halt execution and throw a type error.
Second of all, this will require highly efficient type checking, preferably performed in parallel to type check many functions at one.
We could also try to be clever and prioritize type checking on functions that we expect to be executed sooner.

\chapter{Conclusion}
\label{chp:conclusion}

We have introduced \name, a low-level language that uses an expressive type system to potentially improve performance via the elimination of unnecessary run-time checks.
To ensure the safety of \name, we have proven the type safety of the \name language as well as showing a sound type erasure to \wasm, demonstrating that \name is at least as safe as \wasm.
Further, \name is based on \wasm, a real-world language commonly used in performance-critical and untrusted contexts, where both safety and performance are critical.
This demonstrates the usefulness of using expressive type systems as a practical tool to improve performance and ensure safety for low-level languages in real use cases.


\end{document}

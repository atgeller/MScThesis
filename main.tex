\documentclass[msc]{ubcthesis}
\PassOptionsToPackage{pdftex,pdfpagelabels}{hyperref}
\usepackage{nameref}
\usepackage{hyperref}
\usepackage{letltxmacro}
\LetLtxMacro{\rulelabel}{\label}
\LetLtxMacro{\lemmalabel}{\label}

\usepackage[usenames]{color}
\usepackage{savesym}
\usepackage{amsmath}
\usepackage{amsthm}
\usepackage{amssymb}
\savesymbol{program}
\savesymbol{@program}
\usepackage{semantic}   % Tools for typesetting PL semantics
\usepackage{braket}     % Easy angle-bracket notation
\restoresymbol{}{program}
\restoresymbol{}{@program}
\usepackage{syntax}
\usepackage{mathpartir}
\usepackage{xspace}
\usepackage{dblfloatfix}
\usepackage{multicol}
\usepackage{subcaption}
\usepackage{tikz}
\usepackage{stmaryrd}
\usepackage{changepage}
\usepackage{multirow}
\usepackage{mathtools}
\usepackage{calc}
\usepackage{natbib}

% Always Use these
\usepackage{microtype}
\usepackage[utf8]{inputenc}
\usepackage[T1]{fontenc}

\newcommand{\shrug}[1][]{%
\begin{tikzpicture}[baseline,x=0.8\ht\strutbox,y=0.8\ht\strutbox,line width=0.125ex,#1]
\def\arm{(-2.5,0.95) to (-2,0.95) (-1.9,1) to (-1.5,0) (-1.35,0) to (-0.8,0)};
\draw \arm;
\draw[xscale=-1] \arm;
\def\headpart{(0.6,0) arc[start angle=-40, end angle=40,x radius=0.6,y radius=0.8]};
\draw \headpart;
\draw[xscale=-1] \headpart;
\def\eye{(-0.075,0.15) .. controls (0.02,0) .. (0.075,-0.15)};
\draw[shift={(-0.3,0.8)}] \eye;
\draw[shift={(0,0.85)}] \eye;
% draw mouth
\draw (-0.1,0.2) to [out=15,in=-100] (0.4,0.95); 
\end{tikzpicture}}
\newcommand{\thought}[1]{\textcolor{red}{\textit{(Thought: #1 \shrug)}}}
\newcommand{\todo}[1]{\textcolor{red}{\textit{(TODO: #1)}}}
\newcommand{\feedback}[2]{#2}

\newcommand{\prechk}[0]{$prechk$\xspace}
\newcommand{\name}[0]{Wasm-prechk\xspace}
\newcommand{\wasm}[0]{Wasm\xspace}
\newcommand{\dtal}[0]{DTAL\xspace}
\newcommand{\ie}[0]{\emph{i.e.,}\xspace}
\newcommand{\eg}[0]{\emph{e.g.,}\xspace}

\newcommand{\erase}[1]{erase(#1)}
\newcommand{\embed}[1]{embed(#1)}

\theoremstyle{definition}
\newtheorem{definition}{\theoremstyle{definition}Definition}
\newtheorem{theorem}{Theorem}
\theoremstyle{plain}
\newtheorem{lemma}{Lemma}
%\Crefname{lemma}{Lemma}{Lemmas}

% Stack formatting
\newenvironment{stackTL}{
    \setlength{\arraycolsep}{0pt}
    \begin{array}[t]{l}\ignorespacesafterend
} {
    \end{array}\ignorespacesafterend
}

% Some custom notations for object language stuff
% Typeset object language notation in blue with sans serif font
\newcommand{\tbsf}[1]{\textsf{\color{blue}#1}}
\reservestyle{\keywords}{\tbsf}
\keywords{ithreetwo[i32], isixfour[i64], binop[binop], testop[testop], relop[relop], const[const\;], eq[eq\;], ne[ne\;], eqz[eqz\;], le[le\;], add[add\;], ge[ge\;], unreachable[unreachable], nop[nop], drop[drop], select[select], block[block], end[\;end], loop[loop], if[if], else[\;else\;], br[br\;], brif[br\rule{1ex}{.4pt}if\;], brtable[br\rule{1ex}{.4pt}table\;], return[return], call[call\;], callindirect[call\rule{1ex}{.4pt}indirect\;], getlocal[get\rule{1ex}{.4pt}local\;], setlocal[set\rule{1ex}{.4pt}local\;], teelocal[tee\rule{1ex}{.4pt}local\;], getglobal[get\rule{1ex}{.4pt}global\;], setglobal[set\rule{1ex}{.4pt}global\;], trap[trap], func[func\;], local[local], global[global\;], table[table\;], memory[memory\;], label[label], div[div], load[load\;], store[store\;], currentmemory[current\rule{1ex}{.4pt}memory\;], growmemory[grow\rule{1ex}{.4pt}memory\;], divpc[div\textsubscript{prechk}], callindirectpc[call\rule{1ex}{.4pt}indirect\textsubscript{prechk}\;], loadpc[load\textsubscript{prechk}\;], storepc[store\textsubscript{prechk}\;], module[module\;], fthreetwo[f32], fsixfour[f64], ieight[i8], isixteen[i16], clz[clz], ctz[ctz], popcnt[popcnt],
sub[sub], shl[shl], or[or], gt[gt], import[import\;], export[export\;], convert[convert], reinterpret[reinterpret], nsend[end], tab[tab\;]}

\mathlig{;}{\tbsf{;\;}}
\mathlig{:}{:}

\newcommand{\mathredbold}[1]{\textcolor{red}{\mathbf{#1}}}
\newcommand{\trbf}[1]{\textcolor{red}{\textbf{#1}}}
\reservestyle{\wkeywords}{\trbf}
\wkeywords{withreetwo[i32], wisixfour[i64], wbinop[binop], wtestop[testop], wrelop[relop], wconst[const\;], weq[eq\;], wneq[neq\;], weqz[eqz\;], wle[le\;], wadd[add\;], wge[ge\;], wunreachable[unreachable], wnop[nop], wdrop[drop], wselect[select], wblock[block\;], wend[\;end], wloop[loop\;], wif[if\;], welse[\;else\;], wbr[br\;], wbrif[br\rule{1ex}{.4pt}if\;], wbrtable[br\rule{1ex}{.4pt}table\;], wreturn[return], wcall[call\;], wcallindirect[call\rule{1ex}{.4pt}indirect\;], wgetlocal[get\rule{1ex}{.4pt}local\;], wsetlocal[set\rule{1ex}{.4pt}local\;], wteelocal[tee\rule{1ex}{.4pt}local\;], wgetglobal[get\rule{1ex}{.4pt}global\;], wsetglobal[set\rule{1ex}{.4pt}global\;], wtrap[trap], wfunc[func\;], wlocal[local], wglobal[global\;], wtable[table\;], wmemory[memory\;], wlabel[label], wdiv[div], wload[load\;], wstore[store\;], wcurrentmemory[current\rule{1ex}{.4pt}memory\;], wgrowmemory[grow\rule{1ex}{.4pt}memory\;], wdivpc[div\textsubscript{prechk}], wcallindirectpc[call\rule{1ex}{.4pt}indirect\textsubscript{prechk}\;], wloadpc[load\textsubscript{prechk}\;], wstorepc[store\textsubscript{prechk}\;], wmodule[module\;], wtab[tab\;]}

\hyphenation{Web-Assembly}

\newcommand{\typerule}[2]{C \vdash #1:#2}
%% stack ; locals ; globals ; index context
\newcommand{\ti}[2]{(#1\;#2)}
\newcommand{\type}[4]{#1;#2;#3;#4}
\newcommand{\insttype}[2]{#1 \rightarrow #2}

\makeatletter

\newcommand{\techprefix}{}
\newcommand{\rulename}[1]{{\scshape #1}}

% For rules in particular
\newcommand{\@defruleStar}[3][\techprefix]{\phantomsection{\rulename{#3}}\expandafter\rulelabel{rule:#1:#2}}
\newcommand{\@defruleNoStar}[2][\techprefix]{\@defruleStar[#1]{#2}{#2}}
\newcommand{\defrule}{\@ifstar\@defruleStar\@defruleNoStar}

\newcommand{\@refruleStar}[3][\techprefix]{\hyperref[rule:#1:#2]{Rule \rulename{#3}}}
\newcommand{\@refruleNoStar}[2][\techprefix]{\@refruleStar[#1]{#2}{#2}}
\newcommand{\refrule}{\@ifstar\@refruleStar\@refruleNoStar}

\newcommand{\lemmaname}[1]{{\scshape #1}}

% For lemmas in particular
\newcommand{\@deflemmaStar}[3][\techprefix]{\phantomsection{\lemmaname{#3}}\expandafter\lemmalabel{lemma:#1:#2}}
\newcommand{\@deflemmaNoStar}[2][\techprefix]{\@deflemmaStar[#1]{#2}{#2}}
\newcommand{\deflemma}{\@ifstar\@deflemmaStar\@deflemmaNoStar}

\newcommand{\@reflemmaStar}[3][\techprefix]{\hyperref[lemma:#1:#2]{Lemma \rulename{#3}}}
\newcommand{\@reflemmaNoStar}[2][\techprefix]{\@reflemmaStar[#1]{#2}{#2}}
\newcommand{\reflemma}{\@ifstar\@reflemmaStar\@reflemmaNoStar}

\makeatother

\newcommand{\satisfies}[3]{#1 \xRightarrow{#2} #3}

\includeonly{
    Preamble/acknowledgements,
    Preamble/dedication,
    Preamble/abstract,
    Chapters/introduction,
    Chapters/background,
    Chapters/wasm-prechk,
    Chapters/metatheory,
    Chapters/discussion,
    Chapters/conclusion,
}

%%%%%%%%%%%%%%%%%%%%%%%%%%%%%%%%%%%%%%%%%%%%%%%%%%%%%%%%%%%%%%%

\author{Adam T. Geller}
\title{An Indexed Type System for Faster and Safer WebAssembly}
%% \subtitle{With a Subtitle}

\institution{The University Of British Columbia}
\faculty{The Faculty of Graduate Studies}
\institutionaddress{Vancouver}
\program{Computer Science}
\department{Computer Science}

%%\previousdegree{AAS-DTA (With High Distinction), Bellevue College, 2016}
%%\previousdegree{B.Sc., The University of Washington, 2018}

\copyrightyear{2020}
\submitdate{\monthname\ \number\year} % The "\ " is required after
                                      % \monthname to prevent the
                                      % command from eating the space.
%%%%%%%%%%%%%%%%%%%%%%%%%%%%%%%%%%%%%%%%%%%%%%%%%%%%%%%%%%%%%%%

\begin{document}
\frontmatter

\maketitle

\begin{abstract}
    Downloading and executing untrusted code is inherently unsafe, but also something that happens often on the internet.
    Therefore, untrusted code often requires run-time checks to ensure safety during execution.
    These checks compromise performance and may be unnecessary.
    We present the \name language, an assembly language based on WebAssembly that is intended to ensure safety while justifying the static elimination of run-time checks.
\end{abstract}

\maketitle

\chapter{Acknowledgements}
\todo{This is pretty out of date, also quite long}
This section proceeds through my life in reverse chronological order.

\section*{University of British Columbia}
First and foremost I would like to thank my wonderful advisors, who made this whole process much easier and significantly contributed to my edification.
\begin{itemize}
    \item William Bowman - Despite being a new professor, William has been a wonderful advisor and provided a large amount of guidance when I was struggling.
    He has been a fantastic resource for all of the work here.
    \item Ivan Beschastnikh - Ivan helped me navigate through the system when I was a wide-eyed first-year and learn from various mistakes I made along the way.
    He was a great source of wisdom and gave me the freedom to pursue the topics I found most interesting.
\end{itemize}

I would also like to thank various professors for providing guidance and advice, some of whom are listed below.
\begin{itemize}
    \item Ron Garcia - Ron is definitely one of the nicest people I have met and has forgotten way more about PL than I have ever learned.
    \item Margo Seltzer - Margo is a kickass professor who is insanely smart and cares a lot about students.
\end{itemize}

Finally, all of my fellow graduate students and lab mates who have become my friends and helped me in various ways.
\begin{itemize}
    \item Puneet Mehrotra, Nico Ritschel, Chris Chen, Felipe Banados Schwerter, Paulette Koronkevich, Clement Fung, Vaastav Anand, Joey Eremondi, Anny Gakhokidze, Jonathan Chan, and countless others.
\end{itemize}

\section*{University of Washington}
During my undergrad, I spent a little over a year working on the Cassius project.
This was my first research experience and I think I learned just as much from that experience as the rest of my undergrad combined.
\begin{itemize}
    \item Pavel Panchekha - Pavel was very patient and provided a lot of help getting me off the ground and learning about what doing research is like.
    \item Michael D. Ernst - As well as advising me throughout my undergrad research, Mike guided me through the process of applying to grad school.
    \item Zach Tatlock - Zach is super nice and always made sure I wasn't lost.
    \item Shoaib Kamil - For all his help in research and helping me find my way at my first conference alone.
    \item James Wilcox - For introducing me to PL and helping me find and get started on the Cassius project.
    \item All the other profs and graduate students of UWPLSE, who were extremely friendly and welcoming.
\end{itemize}

\section*{Various Others}
Many different people have helped me on my path at various times. Below I list a few notable ones.
\begin{itemize}
    \item B.J. Unti - For first giving me the idea to go to grad school.
    \item Xander Veerhoff - For first teaching me how to program, which was not easy.
\end{itemize}

\section*{My Parents}
Last but certainly not least, I'd like to thank my awesome parents for everything they have and continue to do for me.
I'm not sure one could ask for better parents than I have been given (even if I did not ask for them).
They have both always been there for me, especially my mom who spent a huge amount of effort and time taking care of every issue that came up in my early years, as well as homeschooling me.
My Dad has always been my role model and is the smartest person I know.
He tried to introduce me to set theory when I was around 4 and calculus when I was about 13 (the first one stuck pretty well, the second one not so much).


\chapter{Dedications}
This is dedicated to Tabby, Koko, and Socrates.
These have been my best (feline) friends who have always helped me maintain my mental health.


\tableofcontents                %% Mandatory
\listoffigures                  %% Mandatory if thesis has figures

\mainmatter

\chapter{Introduction}
\label{chp:intro}

\section{Unsafe Code}
Browsers and Internet-of-Things (IoT) require running untrusted code, that may have been downloaded from anywhere.
It is crucial to ensure the safety of the code being executed in these contexts.
\todo{Find some examples of javascript exploits}
Typically, \emph{sandboxing} and/or \emph{dynamic safety checks} are used to ensure the safety of untrusted code.

Sandboxing involves placing untrusted code into a secure environment to contain the damage caused by unsafe behavior ~\cite{sandboxes}.
For example, Mozilla's Firefox places untrusted code in separate processes so that unsafe code cannot access the address space of other websites or the broswer ~\cite{foxbox}.
However, sandboxing requires additional run-time resources, as processes require overhead in most OSes.

Dynamic safety checks are run-time checks that catch any attempted unsafe operations.
For example, WebAssembly (\wasm) is a relatively new low-level language designed to be both safe and fast to use in place of JavaScript for performance-critical applications in browsers.
While \wasm is type safe and its semantics enforce the separation of control flow and data, it still relies on dynamic checks to ensure certain type and memory safety properties at run-time.
These dynamic checks potentially slow down programs by introducing unnecessary instructions to perform the checks.
We chose \wasm because it is used in browsers and IoT devices, so both performance and safety are critical concerns.

We have designed an extension to \wasm, called \name, that adds new instructions which do not require dynamic safety checks.
However, under the existing \wasm model the new \name instructions have potentially unsafe semantics, as they require stronger static guarantees than \wasm can provide to ensure safety.
These instructions are guaranteed to be faster than their \wasm counterparts because they do not require the addition of instructions by the compiler/interpreter to perform checks.
To provide these additional static guarantees, we equip \name with a more advanced type system.

\section{Type Systems}
Types systems are useful for reasoning about programs.
They can be used to reason about the correctness of programs, usually in the form of safety guarantees.
For example, type safety is the property that a well-typed program will never become \emph{stuck}, that is, it will always be able to reduce the current expression or the current expression is a well-formed irreducible value.
The safety guarantees of type systems provide a degree of trust in programs, as a well-typed program implicitly contains a checkable proof that it will only exhibit limited behavior.

More expressive type systems that can encode richer invariants, enabling ruling out more bad behaviors with static checks alone.
Generally, such type systems are attached to high-level languages, where explicit abstractions make it easy to reason about programs.
Conversely, using expressive type systems in low-level languages often requires reasoning about program state and unstructured control flow (\ie $goto$), which introduces more complexity into the type system.
However, prior work has attached expressive type systems, that permit complex correctness guarantees, to simple low-level languages.

\todo{I like the next two paragraphs (commented out), but I think they should be moved to the conclusion/discussion, as they don't really fit here anymore.}
%%Using such an expressive type system for a low-level language, we can alleviate overhead otherwise required to ensure safety of untrusted low-level programs.
%%Typically, executing code in an untrusted context requires dynamic safety checks which introduce potentially unnecessary instructions, slowing down execution.
%%However, with the safety guarantees provided by the type system, we can determine when these checks are unnecessary and remove them.
%%This would allow low-level programs to be downloaded, checked, and executed safely and efficiently.

%%Type systems can be used to alleviate the need for safety checks, reducing overhead, by providing safety guarantees about programs before they are run.
%%The idea of using type systems to ensure the safety of low-level code is not a new one.
%%Several projects have attached expressive type systems to low-level languages to attach proofs of correctness to low-level programs.
%%However, the focus in these cases is on correctness, not on performance.

We have built such a type system, based on prior work, for \name.
The \name type system tracks the values of some computations in the types, and constraints between those values can be statically checked.
It can provide the static guarantees necessary for the new \name instructions.
Therefore, with the new type system and new instructions, \name has equivalent safety guarantees to \wasm, with potentially improved performance thanks to fewer instructions.

\section{Thesis Statement}
\begin{adjustwidth}{1cm}{1cm}
    Using a more expressive type system, we can ensure safety \emph{and} improve the performance of low-level code in untrusted environments.
\end{adjustwidth}

\section{Contributions}
We want to use types to improve performance while ensuring safety in real-world low-level programs.
Towards that goal, we introduce \name, an extension of the WebAssembly (\wasm) language.
\name introduces new versions of \wasm instructions which are faster than their \wasm counterparts, but also require stronger type-level safety guarantees (Section ~\ref{sec:newinstructions}).
To facilitate type-checking these new instructions, \name uses an indexed type system which is able to encode linear constraints on program variables and therefore ensure complex safety properties (Section ~\ref{sec:typesys}).
We ensure that \name is as safe as \wasm by providing a type safety proof of the \name indexed type system (Section ~\ref{chp:typesafety}).
Together, these additions mean that \name is as safe as \wasm while potentially improving performance.

\section{Related Work}
\label{sec:relwork}
\todo{Exact positioning TBA (probably use to drive discussion of type systems)}
%% TIL
Using type information to improve compiler optimizations is not a new idea.
In 1996, a paper by Tarditi et al. used strongly typed intermediate languages (TIL) to improve optimizations of SML code ~\cite{TIL}.
Compiling SML involves many translations to intermediate languages, and by preserving type information through those translations and in the intermediate language Tarditi et al. were able to safely perform additional compiler operations.
Using TIL in the compilation of programs led to significantly faster programs.
TIL focuses on compiler optimizations and eventually translates into untyped languages and finally runnable assembly, so the guarantees of the type system are lost along the way.

%% PCC
\todo{Blah}
Proof carrying code (PCC) was an idea introduced in 1997 by Necula et al. ~\cite{PCC}.
While typed assembly languages carry implicit proofs in their types, PCC attached explicit proofs that low-level code satisfies some safety properties.
The proof can then be quickly checked to ensure the safety of the code before it is run.
This provides a way to verify that untrusted code will not violate correctness or security invariants of a program.
The author provides a detailed example of invariants for extensions to TIL to ensure type safety of compiled code.
The example uses the Edinburgh Logical Framework (LF) to encode the proof.
A type safety proof of a LF program is a proof of correctness.
However, PCC puts a burden on developers to formally specify safety and correctness properties, and encoded proofs may be quite large requiring extra time to transmit.

%% FToTAL
Morrisett et al. showed how types could provably be preserved during five different compilation passes to get from System F all the way down to a typed assembly language (TAL) ~\cite{FtoTAL}.
The purpose of TAL was much more focused on safety than on optimizations.
Although Morrisett et al. argued that the type-preserving compilation passes would permit similar optimizations to TIL, they didn't include further optimizations based on TAL.
However, Morrisett et al. did argue that the guarantees of TAL were sufficient to allow untrusted code to be safely executed.

%% DTAL
Xi and Harper created a much more expressive type system for an assembly language which had the potential to allow more compiler optimizations ~\cite{DTAL}.
Their language, a dependently typed assembly language (\dtal), used a limited dependent type system, which enabled safely removing some run-time checks, including array bounds checks.
The goal of \dtal, similar to TAL, was to support type-preserving compilation from a high-level language for both optimizations and safety.
\dtal intended to support type-preserving compilation from Dependent ML as well as SML.

%% LTAL
\todo{LTAL}

%% Wasm
After almost two decades of JavaScript being the dominant language in browsers, it was decided that an alternative was necessary for performance-critical code.
The alternative that was jointly created by the major browser developers was WebAssembly (\wasm) ~\cite{WASM}.
\wasm is a stack-based assembly language with structured control flow.
It is designed to be safe as well as performant, with a small binary footprint.
The \wasm type system is simple, only encoding primitive types, but strong enough to ensure type safety.
Memory safety in \wasm is enforced using run-time checks.
\wasm is supported by most major browsers, and is increasingly used in IoT devices due to its portability and safety.
\chapter{Background: \wasm}
\label{sec:wasm}
Here we present an overview of \wasm so readers have some familiarity with it for when we present \name.
We do not cover the entirety of the \wasm language as presented in the 2017 paper \cite{WASM}, but rather present selected important facets of the syntax, semantics, and type system.
It is recommended that the reader first skim this chapter to understand the basics and then refer back while reading \autoref{chp:prechk} and \autoref{chp:metatheory}.

\section{\wasm Syntax}
\autoref{fig:wasmsyntaxtypes} shows the types of \wasm.
Primitive \wasm types, represented as $t$, include 32- and 64-bit floats and integers.
Packed types, $tp$, include 8-, 16-, and 32- bit integers, are used in memory operations to load/store a smaller payload (\eg $\<ieight>$ loads/stores just one byte).
\wasm is a stack-based language, so the type of an instruction in \wasm consists of a precondition and postcondition on the shape of the stack, which is what a \wasm function type $tf$ is encoding.
This can be viewed as though instructions \emph{consume} certain values from the stack and then \emph{produce} values to be pushed on the stack.
Thus, function types, $tf$, are just syntax used in certain instructions, function declarations, and the \wasm typing judgment, not function types in the traditional sense.
Lastly, global types consist of a primitive type $t$ and an optional mutable flag (the $\empty^{?}$ form is explained more below).

\begin{figure}
\begin{math}
\begin{array}{rcl}
    t & ::= & \<ithreetwo> \mid \<isixfour> \mid \<fthreetwo> \mid \<fsixfour> \\
    tp & ::= & \<ieight> \mid \<isixteen> \mid \<ithreetwo> \\
    tf & ::= & t^{*} \rightarrow t^{*} \\
    tg & ::= & \text{mut}^{?}\; t
\end{array}
\end{math}
\caption{\wasm Types}
\label{fig:wasmsyntaxtypes}
\end{figure}

The \wasm syntax uses the Kleene star within its BNF (\eg $t^{*}$) to denote possibly empty sequences.
For example, $t^{*}$ matches $\epsilon$ (the empty sequence, which is an empty sequence of anything), $\<ithreetwo>\; \<ithreetwo>$, and $\<ithreetwo>\; \<isixfour>\; \<ithreetwo>$.
Instructions, represented by the metavariable $e$, are usually grouped into sequences $e^{*}$ which are possibly empty $\epsilon$.
As a further point on metavariables, $e_1$ and $e_2$, both instruction metavariables, may happen to be the same instruction, or not, we do not know.
Similarly, $e_1^{*}$ and $e_2^{*}$ refer to different sequence metavariables that may or may not be the same; we can make no assumptions about them.

We can use different annotations in place of the Kleene star to add additional information.
The Kleene star may be replaced with an exact value $n$ when we know that the sequence has length $n$ (\eg a sequence of 3 types be phrased as $t^{3}$).
We can also use a question mark to represent either an empty sequence ($\epsilon$), or a sequence with exactly one item (\eg $v^{?}=v' \lor v^{?}=\epsilon$).

There is no requirement that a sequence of non-terminals, $e_1^{*}$, be made up of entirely the same pattern, unless it is explicitly written out as in $(t.\<const> c)^{*}$.
For example, $e_1^{*}$ matches $(t.\<const> c_1)\; (t.\<const> c_2)\; (t.binop)$.
Further, we may separate out subsequences: from $(t.\<const> c)^{*}$ we may separate out $t^{*}$ and $c^{*}$ to refer to the sequences of types and constant values respectively.

With this notation in mind, we can now look over the \wasm instructions in \autoref{fig:wasminstructions} (we will discuss the instructions in \autoref{sec:wasmsemantics}).
Syntax written in a \tbsf{blue sans serif font} denotes a keyword, while text written in $italics$ represents a metavariable.
Throughout the \wasm syntax there are many metavariables used to represent natural numbers: $n$ and $m$ are usually used for the table and memory sizes, $i$ and $j$ are often used as indexes (\eg to reference a local variable), $o$ and $align$ are used within memory operations (we replace $a$ with $align$ for clarity and since we use $a$ elsewhere), and lastly $c$ is used as a constant metavariable (which could also be a float).
$iN$ is used to annotate operations that support integers, and $fN$ is used to annotate operations that support floats.

Some instructions, such as $\<loop>\; \mathit{tf}\; e^{*} \<end>$ include a sequence of instructions $e^{*}$.
We refer to such instructions as block instructions, since they define control flow blocks for the instructions inside (not to be confused with the $\<block>$ instruction, which is a block instruction).
In a block instruction, you will see one or more instruction sequences $e^{*}$ as part of the syntax before $\<nsend>$, we refer to this as the body.
Further, many block instructions also include an explicit type annotation $\mathit{tf}$ declaring their precondition and postcondition.

\begin{figure}
    \begin{math}
    \begin{array}{rcl}
        unop_{iN} & ::= & \<clz> \mid \<ctz> \mid \<popcnt> \\
        testop_{iN} & ::= & \<eqz> \\
        binop_{iN} & ::= & \<add> \mid \<sub> \mid \<shl> \mid \<or> \mid ... \\
        relop_{iN} & ::= & \<eq> \mid \<ne> \mid \<gt> \mid \<ge> \mid ... \\
        cvtop & ::= & \<convert> \mid \<reinterpret> \\
    \end{array}
    \end{math}

    \begin{math}
    \begin{array}{rcl}
        e & ::= & \<unreachable> \mid \<nop> \mid \<drop> \mid \<select> \mid \\
        && \<block>\; \mathit{tf}\; e^{*} \<end> \mid \<loop>\; \mathit{tf}\; e^{*} \<end> \mid \<if>\; \mathit{tf}\; e^{*} \<else> e^{*} \<end> \mid \\
        && \<br> i \mid \<brif> i \mid \<brtable> i^{+} \mid \<return> \mid \<call> i \mid \<callindirect> tf \mid \\
        && \<getlocal> i \mid \<setlocal> i \mid \<teelocal> i \mid \<getglobal> i \mid \\
        && \<setglobal> i \mid t.\<load> (tp\_sx)^{?}\; align\; o \mid t.\<store> tp^{?}\; align\; o \mid \\
        && \<currentmemory> \mid \<growmemory> \mid t.\<const> c \mid \\
        && t.unop_t \mid t.binop_t \mid t.testop_t \mid t.relop_t \mid t.cvtop\; t\_sx^{?} \mid ... \\
    \end{array}
    \end{math}
    \caption{\wasm Instructions}
    \label{fig:wasminstructions}
\end{figure}

\wasm has modules that include functions ($f$), global variables ($glob$), an optional function table ($tab$), and an optional linear memory chunk ($mem$), as seen in \autoref{fig:wasmmodules}.
Functions, globals, the table, and memory can be imported, using $\<import> "name_1"\; "name_2"$, which imports $name_2$ from the file $name_1$.
Similarly, they can also be exported under any number of names using $\<export> "name"$.

Functions include a list of local variable declarations to use within the body (a sequence of instructions).
Additionally, function arguments are accessible as local variables within the body of functions.
Global variables may be mutable (although, exported global variables, which are accessible in other modules, cannot be mutable, as we will see later), and are initialized via a sequence of instructions.
Function tables store references to functions that can be called using indirect function calls; they are used to more safely represent function pointers.
Indirect function calls take an index and use it to lookup a function in the function table and call it.
They must supply a function type annotation, $\mathit{tf}$ that gets checked against the function that ends up being called at run-time.
Linear memory, $mem$, is a continuous chunk of memory.
Memory load and store operations operate within the linear memory chunk.

\begin{figure}
    \begin{math}
    \begin{array}{lrcl}
        \text{(imports)} & \mathit{im} &::=& \<import> "name"\; "name" \\
        \text{(exports)} & \mathit{ex} &::=& \<export> "name" \\
        \text{(functions)} &f &::=& \mathit{ex}^{*}\; \<func> \mathit{tf}\; \<local>\; t^{*}\; e^{*} \mid \mathit{ex}^{*}\; \<func> \mathit{tf}\; \mathit{im} \\
        \text{(globals)} &\mathit{glob} &::=& ex^{*}\; \<global> \mathit{tg}\; e^{*} \mid ex^{*}\; \<global> \mathit{tg}\; \mathit{im} \\
        \text{(table)} &\mathit{tab} &::=& ex^{*}\; \<table> n\; i^{*} \mid \mathit{ex}^{*}\; \<table> n\; im \\
        \text{(memory)} &\mathit{mem} &::=& ex^{*}\; \<memory> n\; \mid \mathit{ex}^{*}\; \<memory> n\; im \\
        \text{(modules)} &\mathit{module} &::=& \<module> f^{*}\; \mathit{glob}^{*}\; \mathit{tab}^{?}\; \mathit{mem}^{?}
    \end{array}
    \end{math}
    \caption{\wasm Module Definitions}
    \label{fig:wasmmodules}
\end{figure}

\section{\wasm Dynamic Semantics}
\label{sec:wasmsemantics}
\wasm is a stack-based assembly language specified using reduction semantics \footnote{For those unfamiliar with reduction semantics, I highly recommended these notes by Ron Garcia: \hyperlink{https://www.cs.ubc.ca/~rxg/cpsc509/05-reduction.pdf}{https://www.cs.ubc.ca/\textasciitilde rxg/cpsc509/05-reduction.pdf}}.
Before we introduce the \wasm semantics, we first must introduce some administrative structures and instructions that are used in the reduction relation to keep track of information.
Administrative instructions are not part of the surface syntax of a language (\eg you cannot put a local block in a \name program), and can only appear as an intermediate term during reduction.
\autoref{fig:wasmadmin} shows the new administrative instructions and run-time structures.

\begin{figure}
    \begin{math}
    \begin{array}{lrcl}
        \text{(closures)} & \mathit{cl} &::=& \{\text{inst } i, \text{func } f\} \\
        \text{(bytes)} & b &::=& 0x00, 0x01, ..., 0x\texttt{ff} \\
        \text{(table instances)} & \mathit{tabinst} &::=& cl^{*} \\
        \text{(memory instances)} & \mathit{meminst} &::=& b^{*} \\
        \text{(modules instances)} & \mathit{inst} &::=& \{\text{func } cl^{*}, \text{glob } v^{*}, \text{tab } i^{?},\text{mem } i^{?}\} \\
        \text{(stores)} & s &::=& \{{\begin{stackTL}\text{inst } \mathit{inst}^{*}, \text{tab } \mathit{tabinst}^{*},\\\text{mem } \mathit{meminst}^{*}\}\end{stackTL}} \\
        \text{(values)} & v &::=& t.\<const> c \\
        \text{(admin. instrs.)}&e &::=& ... \mid \<trap> \mid \<call> cl \mid \<label>_n\{ e^{*}\}\; e^{*}\<end> \mid \\
        &&& \<local>_n\{ i;v^{*}\}\; e^{*}\<end>\\
        \text{(reduction contexts)} & L^0 &::=& v^{*}\; \square \; e^{*} \\
        &L^{k+1} &::=& v^{*}\; \<label>_n\{ e^{*}\}\; L^{k}\<end> \; e^{*} \\
    \end{array}
    \end{math}
    \caption{\wasm Administrative Instructions and Run-Time Structures}
    \label{fig:wasmadmin}
\end{figure}

The runtime store, $s$, includes runtime instances for every module ($inst^{*}$), as well as all of the tables ($tabinst^{*}$), and memory chunks ($meminst^{*}$).
In other words, $s$ includes an instantiation of every module.
Module instances, $inst$, represent \wasm modules after linking.
They refer to their table and memory (if they have either), by indexing into the list of runtime instances of tables and memory chunks in the store $s$.
A table instance $tabinst$ contains a list of closures that can be called.
$b$ represents a byte.
A memory instance $meminst$ is a sequence of bytes representing a contiguous memory chunk.
\wasm closures, $cl$, intuitively represents a function closed under linking.
Closures include the module instance that the function is defined in, as well as the function definition (which cannot be an import) with any exports erased.

There are a few final notational digressions we must make before describing the reduction relation.
Firstly, objects such as $s=\{\text{inst } inst^{*}, \dots \}$ can be dereferenced using their keywords (\eg ``inst'').
For example, $s_\text{inst}=inst^{*}$ given the above definition of $s$.
Secondly, we can index into a sequence to get a specific element (\eg $inst^{*}(i)$ returns the $i$th $inst$ in $inst^{*}$).
Lastly, \wasm uses several shorthands to get information out of module instances in $s$: $s_\text{func}(i,j)=s_\text{inst}(i)_\text{func}(j)$.
Essentially, this allows us to implicitly dereference the $i$th module instance to get the $j$th function inside of the instance.
This shorthand is used similarly for glob, tab, and mem.

Constant instructions $t.\<const> c$ represent values, and are denoted by the metavariable $v$, when they should be interpreted that way.
They produce a constant value (known statically).
This leads to a particular representation of the stack, as discussed in \autoref{subsec:wasmredux}.
A \emph{trap} ($\<trap>$) is the \wasm term for a run-time error.
$\<call> cl$ is a function call on a closure.
As we will see, it is an intermediate step for performing both direct and indirect function calls.

Two types of block instructions are introduced.
The first, the label block, is used in handling control flow.
Specifically, they are used to handle branching.
All block instructions (\<block>, \<loop>, and \<if>) reduce to label blocks.
Label blocks can store instructions ($e^{*}$ inside the curly braces), and the annotation $n$ is equal to the expected number of inputs to those instructions.
This is explained more when we describe how branching works.

The second block instruction is the local block.
A $\<local>$ is the result of reducing a closure call; it is used to reduce a function body within the closed environment of the closure.
They introduce an environment consisting of the module instance and local variables inside which their body is reduced.

Finally, we introduce reduction contexts, $L^{k}$, where $k$ is the nesting depth.
Reduction contexts are defined using label blocks, so $L^{k}$ contains $k$ nested label blocks.
As well as nested label blocks, reduction contexts contain preceding values $v^{*}$ (\ie a stack), and proceeding instructions $e^{*}$ that are next to be executed after the nested label block finishes reducing.

\subsection{The \wasm Reduction Relation}
\label{subsec:wasmredux}
The \wasm Reduction Relation works on \emph{configurations} that include the store $s$, local variables (represented as a sequence of values $v^{*}$), and the instruction sequence $e^{*}$.
Reduction is relative the the current module index $i$, which is used to know which module instance in the store to look at when dereferencing the store.
The store, local variables, and module index are omitted when not used.
We present all the reduction rules below.

\begin{mathpar}
    \boxed{s;v^{*};e^{*} \hookrightarrow_i s';v'^{*};e'^{*}}
\end{mathpar}

Instructions are reduced in place by decomposing the program using reduction contexts.
Intuitively, we pull out the next instruction to execute, reduce it, and push the result on top of the stack.
The ``stack'' is just the sequence of values (\ie constant instructions) preceding the first reducible instruction.
When an instruction reduces to a value, that value becomes the new top of the stack and the next instruction is reduced.
\emph{This method of decomposing ensures that all of the instructions preceding the instruction currently being reduced have already been reduced to values.}

Binary and relation operations consume two values from the stack and either push back onto the stack the specified operation applied to those values, or trap if the operation on the values is not defined (in the case of dividing by zero).
Test operators only consume one value, and do not trap, but are otherwise similar.
The reduction rules for these operators use metafunctions (\eg $testop_t(c)$) to compute the result of applying the operator for the produced value.

The instruction \<unreachable> causes a trap (it is similar to \eg \texttt{assert false}), \<nop> reduces to the empty sequence, and \<drop> consumes one value and reduces to the empty sequence (\ie it discards the value on top of the stack).
\<select> is a ternary operator (like $?:$ in C) that consumes three values and produces either the first or the second depending on the third value.
The true/non-zero case of select returns the first value consumed ($k+1$ is a common shorthand for a non-zero value), and the false/zero case returns the second value consumed.

\begin{mathpar}
    \begin{array}{rcl}
        (t.\<const> c_1)\; (t.\<const> c_2)\; t.binop &\hookrightarrow& t.\<const> c \\
        && \text{if } c=binop(c_1,c_2) \\ %% binop

        (t.\<const> c_1)\; (t.\<const> c_2)\; t.binop &\hookrightarrow& \<trap> \\ %% binop to trap
        && otherwise \\

        (t.\<const> c_1)\; (t.\<const> c_2)\; t.relop &\hookrightarrow& t.\<const> relop(c_1,c_2) \\ %% relop

        (t.\<const> c)\; t.testop &\hookrightarrow& \<ithreetwo>.testop_t(c) \\

        \<unreachable> &\hookrightarrow& \<trap> \\

        \<nop> &\hookrightarrow& \epsilon \\

        v\;\<drop> &\hookrightarrow& \epsilon \\

        v_1\;v_2\;(\<ithreetwo>.\<const> 0)\;\<select> &\hookrightarrow& v_2 \\

        v_1\;v_2\;(\<ithreetwo>.\<const> k+1)\;\<select> &\hookrightarrow& v_1 \\
    \end{array}
\end{mathpar}

Block instructions define a control flow environment used by branching instructions inside which their bodies are reduced.
The true case of an \<if> block reduces to the first body inside of a block; the false case does the same but with the second body.
Both \<block> and \<loop> reduces to label blocks.
Stored instructions are only added when reducing a \<loop>, in which case it stores the loop code so it can run the loop again.
If the body of a label block is a \<trap> or a sequence of values then the \<trap>/values replace the block.
Since decomposition happens on label blocks, we have included the inductive reduction rule, which intuitively pulls instructions out of the context, reduces them outside the context, and then plugs them back in.

\begin{mathpar}
    \inferrule[]{
        s;v^{*};e^{*} \hookrightarrow s';v'^{*};e'^{*}
    } {
        s;v^{*};L^k[e^{*}] \hookrightarrow s';v'^{*};L^k[e'^{*}]
    } \\

    \begin{array}{rcl}
        L^{0}[\<trap>] &\hookrightarrow& \<trap> \\

        v^n\;\<block>\; (t_1^{n}\rightarrow t_2^{m})\; e^{*} \<end> &\hookrightarrow& \<label>_m \{\} v^n\;e^{*} \<end> \\

        v^n\;\<loop>\; (t_1^{n}\rightarrow t_2^{m})\; e^{*} \<end> &
        \hookrightarrow&
        {\begin{stackTL}
            \<label>_n
            {\begin{stackTL}
                \{ \<loop>\; (t_1^{n}\rightarrow t_2^{m})\; e^{*}
                \\ \<end> \}
                \\ v^n\;e^{*}
            \end{stackTL}} \\
            \<end>
        \end{stackTL}} \\

        (\<ithreetwo>.\<const> 0)\; \<if>\; tf\; e_1^{*} \<else> e_2^{*} \<end> &\hookrightarrow& \<block>\; tf\; e_2^{*} \<end> \\

        (\<ithreetwo>.\<const> k+1)\; \<if>\; tf\; e_1^{*} \<else> e_2^{*} \<end> &\hookrightarrow& \<block>\; tf\; e_1^{*} \<end> \\

        \<label>_n\; \{ e_0^{*} \}\; v^{*} \<end> &\hookrightarrow& v^{*} \\

        \<label>_n\; \{ e_0^{*} \}\; \<trap> \<end> &\hookrightarrow& \<trap> \\
    \end{array}
\end{mathpar}

Branching ($\<br> j$) intuitively jumps to the $j+1$th outer control flow block (\ie a label block).
More concretely, a $\<br> j$ inside a label block (which, you may recall, are used as control flow blocks) jumps to the surrounding label block with nesting depth $j+1$ (essentially peeling back $j$ layers).
After branching, execution continues with the values $v^n$ consumed by the \<br> and the stored instructions $e_0^{*}$ of the $j+1$th outer label block (this is in place to support loops, as jumping to the label block introduced by \<loop> is what causes the next iteration to be performed).
Extra stack values beyond those consumed are discarded.
\autoref{fig:branching} has several examples of branching in action.

\wasm also has a conditional branch instruction.
This instruction, $\<brif> j$, consumes a value and reduces to $\<br> j$ if the value is non-zero, otherwise it reduces to the empty sequence.
Table branches, $\<brtable>$, has a list of one or more numbers, $i^{+}$ that may be used for a branch.
It consumes a $\<ithreetwo>\; k$ and reduces to $\<br>$ with the $k$th number, or the last number if there is no $k$th number.

\begin{mathpar}
    \begin{array}{rcl}
        \<label>_n\; \{ e_0^{*} \}\; L^j[v^{n}\; \<br> j] \<end> &\hookrightarrow& v^n\; e_0^{*} \\

        (\<ithreetwo>.\<const> 0)\; \<brif> j &\hookrightarrow& \epsilon \\

        (\<ithreetwo>.\<const> k+1)\; \<brif> j &\hookrightarrow& \<br> j \\

        (\<ithreetwo>.\<const> k)\; \<brtable> j_1^{k}\;j\;j_2^{*} &\hookrightarrow& \<br> j \\

        (\<ithreetwo>.\<const> k+n)\; \<brtable> j_1^{k} j &\hookrightarrow& \<br> j \\
    \end{array}
\end{mathpar}

\begin{figure}
\begin{math}
\begin{array}{l}
    \<label>_0 \{ \<loop> \dots \<end> \}\; \<br> 0 \<end> \\
    \hookrightarrow \<loop> \dots \<end> \\\\
    \<label>_0
    \begin{stackTL}
        \{ \}\; \\
        \<label>_0 \{ \<loop> \dots \<end> \}\; \<br> 1 \<end>
    \end{stackTL} \\
    \<nsend> \\
    \hookrightarrow \epsilon\\\\
    {\begin{stackTL}
        \<label>_0
        {\begin{stackTL}
            \{\} \\
            \<label>_0 {\begin{stackTL}
                \{\} \\
                \<label>_0 \{\}\; \<br> 1 \<end>
            \end{stackTL}} \\
            \<nsend>
        \end{stackTL}} \\
        \<nsend>
    \end{stackTL}} \\
    \hookrightarrow \<label>_0 \{\}\; \<end>
\end{array}
\end{math}
\caption{Branching Examples}
\label{fig:branching}
\end{figure}

Direct and indirect function calls are expanded in two steps.
First, the associated closure is fetched either from the current module instance (for direct calls) or from the table (for indirect calls, which traps if the type of the fetched closure doesn't match the expected type).
This step reduces a direct or indirect call to a $\<call> cl$.
Then, the closure body is placed into a local block with the arguments from the stack and locals declared by the function ($t^k$), which are zero-initialized, being used as the local variables.

\begin{mathpar}
    \begin{array}{rcl}
        s;\<call> j &\hookrightarrow_i& \<call> s_\text{func}(i,j) \\

        s;\<callindirect> j &\hookrightarrow_i& s_\text{tab}(i,j) \\
        && \text{if } s_\text{tab}(i,j)_\text{code}=(\<func> tf\; \<local>\; t^{*}\; e^{*}) \\

        s;\<callindirect> j &\hookrightarrow_i& \<trap> \\
        && \text{otherwise} \\

        v^{n}\;(\<call> cl) &\hookrightarrow_i&
        {\begin{stackTL}
            \<local>_m
            {\begin{stackTL}
                \{
                    {\begin{stackTL}
                        cl_\text{inst}; \\
                        v^{n}\;(t.\<const> 0)^{k}\}
                    \end{stackTL}} \\
                \<block>\;(\epsilon \rightarrow t_2^{m})\; e^{*} \\
                \<nsend>
            \end{stackTL}} \\
            \<nsend> \\
        \end{stackTL}} \\
        && \text{where } cl_\text{func} = (\<func> tf\; \<local>\; t^{*}\; e^{*})\\
    \end{array}
\end{mathpar}

The local block has the same module index, $i$, as the closure, so the body of the local block is reduced within the module that the closure is defined in and thus uses the global variables, table, and memory of that module instance.
This is handled by the inductive reduction rule (which has much more of a structural operational small-step semantics flavor).
In general, $\<return>$ can be thought of as $\<br> k$, where $k$ is the context depth.
A label block is added inside of the local block when expanded a function call, so at the top level of a function $\<br> 0$ is essentially equivalent to $\<return>$, except with an additional reduction step.
\todo{Was that note confusing? Was that sentence confusing?}
Returning, somewhat similarly to branching, replaces the local block with the arguments to the return instruction, except that it skips over any label blocks.
If the body of a local block is a \<trap> or sequence of values equal to the number annotation on the local block, then that is what the local block reduces to, similar to branching (also similar to branching, any extra values on the stack are discarded).

\begin{mathpar}
    \inferrule[]{
            s;v^{*};e^{*} \hookrightarrow_i s';v'^{*};e'^{*}
        } {
            s;v_0^{*};\<local>_n \{ i;v^{*} \} e^{*} \<end> \hookrightarrow_j s';v_0^{*};\<local>_n \{i;v'^{*}\} e'^{*} \<end>
        } \\

    \begin{array}{rcl}
        \<local>_n \{ i;v_l^{*} \} v^{n} &\hookrightarrow& v^{n} \\
        \<local>_n \{ i;v_l^{*} \} \<trap> &\hookrightarrow& \<trap> \\
        \<local>_n \{ i;v_l^{*} \} L^{k}[v^n \<return>] &\hookrightarrow& v^{n} \\
    \end{array}
\end{mathpar}

Local variables are represented as a list of values at run time.
They are get/set by indexing into them, like everything else in \wasm.
The same is true of global variables, except there is an extra step since they are stored in the current module instance inside the store $s$.

\begin{mathpar}
    \begin{array}{rcl}
        v_1^{j}\;v\;v_2^{};\<getlocal> j &\hookrightarrow& v \\

        v_1^{j}\;v\;v_2^{};v'\; (\<setlocal> j) &\hookrightarrow& v_1^{j}\;v'\;v_2^{};\epsilon \\

        v_1^{j}\;v\;v_2^{};v'\; (\<teelocal> j) &\hookrightarrow& v_1^{j}\;v'\;v_2^{};v' \\

        s;\<getglobal> j &\hookrightarrow_i& s_\text{glob}(i,j) \\

        s; v;\;(\<setglobal> j) &\hookrightarrow_i& s';\epsilon \\

        && \text{where } s' = s \text{ with } glob(i,j)=v' \\
    \end{array}
\end{mathpar}

Finally, there are the memory instructions.
One can load or store a value from or to memory, get the current memory size, or try to grow the memory.
$|t|$ is used to represent the size of the type (\eg $|\<isixfour>| = 8$ bytes).
We omit two rules, one each for store and load, that include the ability to use packed types to load/store smaller values and to load signed/unsigned.
There is a lot of minutiae detail, but none of it is particularly important.
For example, $tp$ is an optional packed type which allows storing values smaller than the normal size of the type of the value (\eg storing eight bits $\<ieight>$ of a thirtytwo bit integer $\<ithreetwo>$).
Loading from memory can optionally be signed or unsigned using $sx$, which represented signed or unsigned.
The ``alignment exponent'' $align$ is a mysterious variable that is not used during reduction, and is only used during typechecking without any explanation.
Two metafunctions, $const_t$ and $bits_t$, are used to convert bits to values and vice versa.
The key high level takeaway is that load and store will trap if the supplied index $k$ plus the static offset $o$ is out of bounds.

\begin{mathpar}
    \begin{array}{rcl}
        s;(\<ithreetwo>.\<const> k) &&\\
        (t.\<load> tp\_sx\; align\; o) &\hookrightarrow_i& s;(t.\<const> const_t(b^{*})) \\
        && \text{if } s_\text{mem}(i,k+o,|t|)=b^{*} \\


        s;(\<ithreetwo>.\<const> k) &&\\
        (t.\<load> tp\_sx\; align\; o) &\hookrightarrow_i& \<trap> \\
        && \text{otherwise} \\


        s;(\<ithreetwo>.\<const> k)\; (t.\<const> c) && \\
        (t.\<store> tp\_sx\; align\; o) &\hookrightarrow_i& s';\epsilon \\
        && \text{if } s'=s \text{ with } \text{mem}(i,k+o,|t|)=bits_t(c) \\


        s;(\<ithreetwo>.\<const> k)\; (t.\<const> c) && \\
        (t.\<store> tp\_sx\; align\; o) &\hookrightarrow_i& \<trap> \\
        && \text{otherwise} \\

    \end{array}
\end{mathpar}

\begin{mathpar}
    \begin{array}{rcl}
        s;\<currentmemory> &\hookrightarrow_i& \<ithreetwo>.\<const> |s_\text{mem}(i,*) | / 64\text{Ki} \\

        s;(\<ithreetwo>.\<const> k)&&\\
        \<growmemory> &\hookrightarrow_i& s';\<ithreetwo>.\<const> | s'_\text{mem}(i,*) | / 64\text{Ki} \\
        \text{if } s'=s &\text{ with }& \text{mem}(i,*)=s_\text{mem}(i,*)(0)^{k*64\text{Ki}} \\

        s;(\<ithreetwo>.\<const> k)&&\\
        \<growmemory> &\hookrightarrow_i& \<ithreetwo>.\<const> (-1) \\
        \text{otherwise} && \\
    \end{array}
\end{mathpar}

\section{The \wasm Type System}
\label{sec:wasmtyping}
Instructions in \wasm are typed under a module type context $C$.
$C$ keeps track of various module-level types: functions, globals, the table, memory, locals, the label stack (\ie the expected types for branching instructions), and the return stack (\ie the expected type of the return instruction).

$$ C::= \{ {\begin{stackTL}
    \text{func } tf^{*}, \text{ global } tg^{*}, \text{ table } n^{?}, \text{ memory } m^{?},
    \\ \text{local } t^{*}, \text{ label } (t^{*})^{*}, \text{ return } (t^{*})^{?} \}
\end{stackTL}} $$

Here is an example of a \wasm typing rule, a binary operation of some type $t$ consumes two values of the given type $t$ on the stack and produces a value of type $t$:

\[
    \inferrule{ }{C \vdash t.binop : t\; t \rightarrow t}
\]

The above example shows what a typical \wasm typing rule looks like.
The type associated with the instruction $t.binop$ is a \wasm function type, which is just the precondition ($t\;t$ on the left of the $\rightarrow$) and postcondition ($t$ on the right of the $\rightarrow$) on the stack.
In the precondition, the top of the stack is the rightmost type (for example, in $t_1\;t_2\;t_3$, $t_3$ is the top of the stack), since that represents the value closest to the instruction getting reduced.
The precondition and postcondition represent the shape of the stack before and after executing a sequence of instructions.
Intuitively, they represent the ``state of the world'' before and after the instruction sequence is executed: they require the world to be in a certain state, and then transform it into some other state.
Thus, the static \wasm typing judgement is as follows:

$$\boxed{C \vdash e^{*} : tf}$$

In addition to this typing judgment, \wasm also includes typing judgments for administrative instructions (which require additional type information about runtime structures, so the judgment has a different form) and a typing judgment in the form of the reduction relation for the \wasm type safety proof.
\wasm also has typing judgments for modules and module-level declarations.

We reproduce and explain a few selected typing rules from \wasm using the static typing judgement.
Most typing rules are for a single instruction and there are a few rules which can combine rules.
The rule for typing a block, \refrule{Wasm-Block} typechecks the body $e^{*}$ under the module type context with the postcondition $t_2^{m}$ appended to the label stack.
This is yet another common notational shorthand where $x,y$ means $x$ extended with $y$.
The branch rule, \refrule{Wasm-Br}, accepts any precondition, extended with the $i$th postcondition on the label stack (counting backwards), and returns to any postcondition.
A branch will return the $n$ values before it, so it is ok if there are more values on the stack, as they will be discarded.
Execution does not proceed after branching, so the postcondition can be anything.
For function calls we lookup the type of the function in the context (\refrule{Wasm-Call}).
Recall that local variables are represented by a list of values at runtime.
Thus, the typing rule for \<setlocal> checks that the value consumed by \<setlocal>, which will replace the $i$th local in the list, has the correct type that is given by looking up the type of the $i$th local in the context (\refrule{Wasm-Set-Local}).

\begin{mathpar}
    \inferrule*[right=\defrule{Wasm-Binop}]{ }{C \vdash t.binop : t\; t \rightarrow t}

    \inferrule*[right=\defrule{Wasm-Block}]{
        tf = t_1^{n} \rightarrow t_2^{m} \and
        C,\text{label}(t_2^{m})\vdash e^{*} : tf
    }{
        C \vdash \<block>\; tf\; e^{*} \<end> : tf
    }

    \inferrule*[right=\defrule{Wasm-Br}]{
        C_\text{label}(i) = t^{n}
    }{
        C \vdash \<br> i : t_1^{*}\;t^{n} \rightarrow t_2^{*}
    }

    \inferrule*[right=\defrule{Wasm-Call}]{
        C_\text{func}(i) = tf
    }{
        C \vdash \<call> i : tf
    }

    \inferrule*[right=\defrule{Wasm-Set-Local}]{
        C_\text{local}(i) = t
    }{
        C \vdash \<setlocal> i : t \rightarrow \epsilon
    }
\end{mathpar}

The empty instruction sequence has an empty precondition and postcondition (\refrule{Wasm-Empty}).
An instruction $e_2$ can be appended to a sequence of instructions $e_1^{*}$ if the precondition of $e_2$ is the same as the postcondition of $e_1^{*}$ (\refrule{Wasm-Composition}).
Then, the precondition of the full sequence $e_1^{*}\;e_2$ is the precondition of $e_1^{*}$ and the postcondition of $e_1^{*}\;e_2$ is the postcondition of $e_2$.

\begin{mathpar}
    \inferrule*[right=\defrule{Wasm-Empty}]{ }{C \vdash \epsilon : \epsilon \rightarrow \epsilon}

    \inferrule*[right=\defrule{Wasm-Composition}]{
        C \vdash e_1^{*} : t_1^{*} \rightarrow t_2^{*} \and
        C \vdash e_2 : t_2^{*} \rightarrow t_3^{*}
    }{
        C \vdash e_1^{*}\;e_2 : t_1^{*} \rightarrow t_3^{*}
    }
\end{mathpar}

\subsection{Stack Polymorphism}
\label{subsec:stackpoly}
To compose together the types of many instructions, it is necessary to carry around extra type information about the rest of the stack while type-checking instructions.
\emph{Stack polymorphism} allows extending the precondition and postcondition with the same data to thread unmodified parts of the stack through a list of instructions.
Intuitively, this allows you to ``forget'' the rest of the stack and focus only on the part being manipulated by the instruction being checked, after which point the ``forgotten'' part can be re-added.

For example, if the stack has the shape $\<isixfour>\; \<ithreetwo>\; \<ithreetwo>$, then stack polymorphism allows us to ignore $\<isixfour>$ and typecheck $\<ithreetwo>.binop$ with $\<ithreetwo>\;\<ithreetwo>$.
Then the stack would look like $\<ithreetwo>$, at which point we add $\<isixfour>$ back to the postcondition to get $\<isixfour>\; \<ithreetwo>$ after executing $\<ithreetwo>.binop$.

\chapter{\name}
\label{chp:prechk}
The goal of \name is to be used to eliminate unnecessary dynamic checks.
To accomplish this, it must (1) have instructions that do not require dynamic checks and (2) statically prove that the assumptions of those instructions are met.
In \name, we extend \wasm with new instructions that explicitly do not require dynamic checks, and design an indexed type system to reason about the safety of removing checks.

\section{\name Syntax}
The syntax of \name is highly similar to that of \wasm except for two additions.
First, \name introduces four additional instructions, which are referred to as ``\prechk-tagged'' instructions.
Second, \name changes the representation of types within \wasm instructions and functions.

Recall from \autoref{sec:wasmsemantics} that there are four \wasm instructions that require run-time checks: integer division, indirect function calls, and memory loads and stores.
``\prechk-tagged'' instructions refer to a set of four \name instructions, listed in \autoref{fig:newinstructions}, that are counterparts to these four \wasm instructions.
Intuitively, we are simply adding a tag to the instruction to show that it doesn't require run-time checks.
Formally, however, these are different instructions with different semantics and different typing rules, as explained below.

\begin{figure}[t]
    \begin{align*}
        &t.\<divpc> \mid
        t.\<callindirectpc> \mid
        \\
        &t.\<loadpc> (tp\_sx)^{?}\; a\;o \mid
        t.\<storepc> tp^{?\;} a\;o
    \end{align*}
    \caption{The four \prechk-tagged instructions}
    \label{fig:newinstructions}
\end{figure}

\subsection{The \name Index Language}
\label{subsec:indexlang}
\name uses an indexed type system.
An indexed type language uses an index language in the type system to encode information within types.
We use the index language to encode linear constraints on program variables within types.
\autoref{fig:itsyntax} shows the syntax for the index type language.
Remember, syntax written in a \tbsf{blue sans serif font} denotes a \wasm keyword.
Below is a quick overview of each of the terms.

\begin{figure}[t]
    \begin{math}
        \begin{array}{rcl}
            t &:: & \<ithreetwo> \mid \<isixfour> \\
            a &::= & Index\; Variable \\
            x\;y &::=& a \mid \ti{t}{c} \mid (\<binop>\;x\;y) \mid (\<testop>\;x) \mid (\<relop>\;x\;y) \\
            P &::=& (=\; x \; y) \mid (\text{if}\; P\; P\; P) \mid \neg P \mid P \land P \mid P \lor P \\
            \phi &::=& \circ \mid \phi, \ti{t}{a} \mid \phi, P \\
        \end{array}
    \end{math}
    \caption{Syntax of the \name index type language}
    \label{fig:itsyntax}
\end{figure}

\begin{itemize}
    \item $t$ represents a primitive \wasm type.
    We do not reason about floating points, so it is either a 32-bit integer ($i32$) or a 64-bit integer ($i64$).
    \item $a$ is a type index variable, which is used to track constraints on program variables.
    \item $x$ and $y$ are type indices, they can be an index type variable, a constant with an explicit type, or a \wasm operation on a type index.
    \item $P$ is a proposition about type indices which can encode equality constraints on type indices, or combine propositions using common first-order logic operators.
    \item $\phi$ is the type index context which stores index type variable declarations and propositions.
    Essentially, it contains all of the knowledge we have about all of the index variables.
\end{itemize}

\begin{figure}[t]
    \begin{math}
        \begin{array}{rcl}
            ti &::=& \ti{t}{a} \\
            l &::=& ti^{*} \\
            tfi &::=& ti^{*};l;\phi \rightarrow ti^{*};l;\phi \\
            C &::=& \{
                {\begin{stackTL}
                    \text{func } \; tfi^{*}, \text{ global } \; (\text{mut}^{?} \; t)^{*}, \text{ table} \; n^{?}, \text{ memory }  m^{?}, \\
                    \text{ local } \; t^{*}, \text{ label}(ti^{*};l;\phi)^{*}, \text{ return}\;(ti^{*};l;\phi)^{? }\}
                \end{stackTL}}
        \end{array}
    \end{math}
    \caption{\name indexed function types}
    \label{fig:tfisyntax}
\end{figure}

Indexed types, $ti$, are used to associate index variables $a$ with values in the program.
\autoref{fig:tfisyntax} shows the form of an indexed type, which includes both the type $t$ and an index variable $a$.
In \name, we represent the shape of the stack as a sequence of indexed types $ti^{*}$.

The index local store associates index variables with local variables.
It has an identical form to the stack: a sequence of indexed types to associate index variables with local variables.
We use the shorthand $l$ to refer to the index local store since we rarely reason about it but rather thread it through typing rules.
The index type context $\phi$ is the mechanism that is used to reason about the possible values of computations.
It stores constraints on and between program variables tracked by indexed types representing the stack and index local store.

\name uses indexed function types $tfi$, which, similar to \wasm's function types, are just a precondition and postcondition.
However, indexed function types include much more information in their precondition and postcondition!
They represent the stack using a sequence of indexed types and track local variables using the index local store, and include $\phi$ which contains constraints about those values.
We will see how this information is used in \autoref{subsec:checkelim}.

We retain $C$ to refer to the module type context in \name, although the representation of module types is slightly different.
\wasm function types are replaced with \name indexed function types.
Further, the postconditions in the label stack and return stack are replaced with \name indexed postconditions including indexed types, the local index store, and the index type context.

We can now introduce the \name typing judgement for instructions.
It is similar to the \wasm typing judgment, but uses indexed function types which include much more information by tracking constraints about program values.

\begin{mathpar}
    \boxed{C \vdash e^{*} : tfi}
\end{mathpar}

Recall that certain \wasm instructions (such as $\<block>$ and $\<callindirect>$) include \wasm function types to declare the expected types of their bodies.
In \name, we replace those function types with indexed function types.

\section{\name Dynamic Semantics}
\name uses the same reduction relation with the same structure as \wasm (explained in detail in \autoref{sec:wasmsemantics}).
All the reduction rules for all of the \name instructions are the same as they are for \wasm, except that indexed function types are used instead of \wasm function types.
We also have four new instructions, for which we introduce new reduction rules.

\label{sec:newinstructions}
\begin{figure}[t]
    \begin{mathpar}
        \boxed{s;v^{*};e^{*} \rightarrow s';v'^{*};e'^{*}}
    \end{mathpar}

    \begin{math}
        \arraycolsep=1.4pt
        \begin{array}{rcl}
            (t.\<const> c_1)\; (t.\<const> c_2)\; t.binop &\hookrightarrow& t.\<const> c \\
            \text{if } c=binop(c_1,c_2) && \\ %% binop

            (t.\<const> c_1)\; (t.\<const> c_2)\; t.binop &\hookrightarrow& \<trap> \\ %% binop to trap
            otherwise && \\

            s;\<callindirect> j &\hookrightarrow_i& s_\text{tab}(i,j) \\
            \text{if } s_\text{tab}(i,j)_\text{code}=(\<func> tf\; \<local>\; t^{*}\; e^{*}) && \\

            s;\<callindirect> j &\hookrightarrow_i& \<trap> \\
            \text{otherwise} && \\

            s;(\<ithreetwo>.\<const> k) &&\\
            (t.\<load> tp\_sx\; align\; o) &\hookrightarrow_i& s;(t.\<const> const_t(b^{*})) \\
            \text{if } s_\text{mem}(i,k+o,|t|)=b^{*} && \\

            &&\\

            s;(\<ithreetwo>.\<const> k) &&\\
            (t.\<load> tp\_sx\; align\; o) &\hookrightarrow_i& \<trap> \\
            \text{otherwise} && \\

            &&\\

            s;(\<ithreetwo>.\<const> k)\; (t.\<const> c) && \\
            (t.\<store> tp\_sx\; align\; o) &\hookrightarrow_i& s';\epsilon \\
            \text{if } s'=s &\text{ with }& \text{mem}(i,k+o,|t|)=bits_t(c) \\

            &&\\

            s;(\<ithreetwo>.\<const> k)\; (t.\<const> c) && \\
            (t.\<store> tp\_sx\; align\; o) &\hookrightarrow_i& \<trap> \\
            \text{otherwise} && \\
        \end{array}
    \end{math}
    \caption{\wasm instructions that have preconditions for reduction}
    \label{fig:checked}
\end{figure}

The formal reason certain \wasm instructions require run-time checks is because they have preconditions as part of their semantics, and if the preconditions are not met then those instructions trap to avoid undefined behavior (we've reproduced the reduction rules for those instructions in \autoref{fig:checked}).
The \wasm type system is not expressive enough to ensure these preconditions statically, so they instead must be checked at run-time.
Thus, ``\prechk-tagged'' instructions can assume that the preconditions on their behavior hold because it is enforced by the \name type system.
This can be seen in the reduction rules for the ``\prechk-tagged'' instructions in Figure \autoref{fig:prechkredux}, where they do not have rules to trap when their preconditions do not hold.

\begin{figure}[t]
    \begin{mathpar}
        \boxed{s;v^{*};e^{*} \rightarrow s';v'^{*};e'^{*}}
    \end{mathpar}

    \begin{math}
        \arraycolsep=1.4pt
        \begin{array}{rcl}
            (t.\<const> c_1)\;(t.\<const> c_2) && \\
            t.\<divpc> & \hookrightarrow & c \\
            && \text{where } c_2 \neq 0 \land c=c_1/c_2 \\
            s;(t.\<const> j) && \\
            t.\<callindirectpc> & \hookrightarrow_i & \<call> s_{tab}(i,j) \\
            &&\text{where } s_{tab}(i,j) = \\
            && \<func> tfi\; \<local>\; t^{*}\;e^{*} \\
            s;(\<ithreetwo>.\<const> k) && \\
            (t.\<loadpc> (tp\_sx)^{?}\; a\;o) & \hookrightarrow_i & t.\<const> const_t(b^{*}) \\
            && \text{where } s_{mem}(i,k+o,|t|)=b^{*} \\
            s;(\<ithreetwo>.\<const> k)\;(t.\<const> c) && \\
            (t.\<storepc> tp^{?}\; a\;o) & \hookrightarrow_i & s';\epsilon \\
            && \text{where } s'=s \\
            && \text{with mem}(i,k+o,|t|)=bits_t^{|t|}(c) \\
        \end{array}
    \end{math}
    \caption{Behavior of new \prechk-tagged instructions}
    \label{fig:prechkredux}
\end{figure}

All other \name instructions have equivalent semantics to their \wasm versions as presented in \autoref{sec:wasmsemantics}.

\section{The \name Indexed Type System}
\label{sec:typesys}
The \name type system is designed to provide sufficient information to safely eliminate dynamic checks (\ie to ensure that the required preconditions are met to \prechk-tag an instruction).
As explained in ~\ref{subsec:indexlang}, the \name type system can encode linear constraints on program variables in the preconditions and postconditions of instructions.
We will now show how these constraints are added and used.

Recall the form of the \name typing judgement for instructions.
\begin{mathpar}
    \boxed{C \vdash e^{*} : ti_1^{*};l_1;\phi_1 \rightarrow ti_2^{*};l_2;\phi_2}
\end{mathpar}

Under $C$, the module type context, $e^{*}$ has the precondition $ti_1^{*};l_1;\phi_1$ and postcondition $ti_2^{*};l_2;\phi_2$.
We use the abbreviation $tfi ::= ti_1^{*};l_1;\phi_1 \rightarrow ti_2^{*};l_2;\phi_2$ as shorthand for the precondition and postcondition of an instruction.

As in \wasm, \name generally has two kinds of typing rules.
Most rules are for inferring or checking the types of instructions (in which case $e^{*}$ will be a single instruction).
There are also a few rules to compose together instruction sequences.
We present the typing rules mixed with discussion of those rules.
The typing judgment in its entirety is reproduced in the appendix.

Here are some of the simpler rules.
These are rules don't use or modify index type information.
\refrule{Unreachable} accepts any precondition and guarantees any postcondition since it just causes a trap.
In \refrule{Nop}, no changes are made from the precondition to the post condition because the instruction does nothing.
\refrule{Drop} consumes the top value from the stack (without caring about its type) and does not change the local index store or index type context.
\begin{mathpar}
    \inferrule*[right=\defrule{Unreachable}]{ }{ %% unreachable
        C \vdash \<unreachable> : tfi
    }

    \inferrule*[right=\defrule{Nop}]{ }{ %% nop
        C \vdash \<nop> : \epsilon;l;\phi \rightarrow \epsilon;l;\phi
    }

    \inferrule*[right=\defrule{Drop}]{ }{ %% drop
        C \vdash \<drop> : \ti{t}{a};l;\phi \rightarrow \epsilon;l;\phi
    }
\end{mathpar}

The constant instruction is a simple example of how indexed types work.
It adds a new indexed type onto the stack to track the new program variable $\ti{t}{a}$, declares the new indexed type in the index type context, and constrains that indexed type to be equal to the constant $(= a\; \ti{t}{c})$.
We require $a$ to be fresh, that is, we require that $a$ is a new index variable present no where else in any of the types for the program.
This is a common pattern to see in rules which introduce new index variables.
The local index store is unchanged between the precondition and postcondition.
\begin{mathpar}
    \inferrule*[right=\defrule{Const}]{ a \text{ is fresh} }{ %% const
        C \vdash t.\<const> c : \epsilon;l;\phi \rightarrow \ti{t}{a};l;\phi,\ti{t}{a},(= a \ti{t}{c})
    }
\end{mathpar}

There are several different kinds of operations, but they all work similarly.
The binary operator instruction adds constraints between new and old program values, since the result of the instruction is a new program values, while the consumed values may already be constrained.
A binary operation consumes two values from the stack, whiuch have associated indexed types $\ti{t}{a_1}$ and $\ti{t}{a_2}$, and produces a value which is associated with the fresh indexed type $\ti{t}{a_3}$.
The index type declaration $\ti{t}{a_3}$ is added to the index type context $\phi$ and $a_3$ is constrained to be equal to the binary operator applied to the index variables that correspond to the input $(= a_3\;(\|binop\|\;a_1\;a_2)$.
As a side note, we use $\|binop\|$ to indicate that we are using the $binop$ (or relop or testop) from the index language.
Binary operators do not affect or use local variables, they simply propagate, so the local index store. $l$, is the same in the precondition and postcondition.
\begin{mathpar}
    \inferrule*[right=\defrule{Binop}]{a_3 \text{ is fresh}}{ %% binop
        C \vdash t.binop : \ti{t}{a_1}\;\ti{t}{a_2};l;\phi \rightarrow \ti{t}{a_3};l;\phi,\ti{t}{a_3},(= a_3\;(\|binop\|\;a_1\;a_2))
    }

    \inferrule*[right=\defrule{Testop}]{a_2 \text{ is fresh}}{ %% testop
        C \vdash t.testop : \ti{t}{a_1}\;l;\phi \rightarrow \ti{\<ithreetwo>}{a_2};l;\phi,\ti{t}{a_2},(= a_2\;(\|testop\|\;a_1))
    }

    \inferrule*[right=\defrule{Relop}]{a_3 \text{ is fresh}}{ %% relop
        C \vdash t.relop : \ti{t}{a_1}\;\ti{t}{a_2};l;\phi \rightarrow \ti{t}{a_3};l;\phi,\ti{t}{a_3},(= a_3\;(\|relop\|\;a_1\;a_2))
    }
\end{mathpar}

\refrule{Select} constrains indexed types in a rather complex way.
Select consumes three values from the stack, it returns the second value if the third value is zero, and otherwise returns the first value (similar to C's ternary operator).
For this rule we use the type-level ``if'' to allow the constraint on the result to depend on the third value consumed: $(\text{if}\; (= a\; \ti{\<ithreetwo>}{0})\; (= a_3\;a_2)\; (= a_3\;a_1))$.
\begin{mathpar}
    \inferrule*[right=\defrule{Select}]{a_3 \text{ is fresh}}{ %% select
        C \vdash \<select> : {\begin{stackTL}
            \ti{t}{a_1}\;\ti{t}{a_2}\;\ti{i32}{a};l;\phi
            \\ \rightarrow \ti{t}{a_3};l;\phi,\ti{t}{a_3},
                (if\; (= a\; \ti{\<ithreetwo>}{0})\; (= a_3\;a_2)\; (= a_3\;a_1))
        \end{stackTL}}
    }
\end{mathpar}

The rules for the three different kinds of blocks (\<block>, \<loop>, and \<if>) are similar to \wasm.
They simply ensure that the interior instruction sequence has the expected type under the context with the expected postcondition (or precondition in the case of loop) appended to the local stack.
If blocks are able to make extra assumptions about the consumed value in the subsequences (that it is non-zero in the first sequence and zero in the second), because those constraints must be true for that sequence to be executed.
While \<if> and \<block> append their postcondition to the label stack for type checking branching instructions within the block, \<loop> appends its precondition because branching to a loop means running the loop again.

\begin{mathpar}
    \inferrule*[right=\defrule{Block}]{ %% block
        C_2,\text{label } (ti_2^{*};l_2;\phi_2) \vdash e^{*} : ti_1^{*};l_1;\phi_1 \rightarrow ti_2^{*};l_2;\phi_2 \\
    }
    {
        C \vdash \<block>\; (ti_1^{*};l_1;\phi_1 \rightarrow ti_2^{*};l_2;\phi_2)\; e^{*} \<end> : ti_1^{*};l_1;\phi_1 \rightarrow ti_2^{*};l_2;\phi_2
    }

    \inferrule*[right=\defrule{Loop}]{ %% loop
        C_2,\text{label } (ti_1^{*};l_1;\phi_1)^{*} \vdash e^{*} : ti_1^{*};l_1;\phi_1 \rightarrow ti_2^{*};l_2;\phi_2 \\
    }
    {
        C \vdash \<loop>\; (ti_1^{*};l_1;\phi_1 \rightarrow ti_2^{*};l_2;\phi_2)\; e^{*} \<end> : ti_1^{*};l_1;\phi_1 \rightarrow ti_2^{*};l_2;\phi_2
    }

    \inferrule*[right=\defrule{If}]{ %% if
        C_2,\text{label } (ti_2^{*};l_2;\phi_2) \vdash e_1^{*} : ti_1^{*};l_1;\phi_1, \neg(= a\; \ti{\<ithreetwo>}{0}) \rightarrow ti_2^{*};l_2;\phi_2 \\
        C_2,\text{label } (ti_2^{*};l_2;\phi_2) \vdash e_2^{*} : ti_1^{*};l_1;\phi_1, (= a\; \ti{\<ithreetwo>}{0})) \rightarrow ti_2^{*};l_2;\phi_2 \\
    }
    {
        C \vdash \<if>\; (ti_1^{*};l_1;\phi_1 \rightarrow ti_2^{*};l_2;\phi_2)\; e_1^{*} \<else> e_2^{*} \<end> : ti_1^{*};l_1;\phi_1 \rightarrow ti_2^{*};l_2;\phi_2
    }
\end{mathpar}

One thing to note is that all three of these rules include their expected preconditions and postconditions as part of their syntax.
We consider the index variables in these indexed function types to be unification variables rather then literals, allowing them to match any literal as long as the types unify.
Intuitively, this is very similar to alpha equivalence, where the precondition matches any preceding postcondition with the same structure as long as the variable can be renamed to match.
The postcondition appended to the label stack also has unification variables instead of the supplied literals.

The rules for branching instructions and return are similar to \wasm.
However, $\<brif>$ adds the assumption that the consumed value is zero to its postcondition.
This assumption can be safely added because the value must be zero for execution to continue without a branch occurring.
If the consumed value is constrained to be non-zero in the indexed type system, then this will cause a contradiction in the constraints of the index type context $\phi$.
However, that is fine since this means that no instructions following the $\<brif>$ will be executed.
Also remember the above note that the postconditions on the label stack contain unification variables, not literals.

Recall from \autoref{sec:wasmsemantics} that $\<brtable>$ branches to one of many different labels.
Thus, we must ensure that every possible branching postcondition it might branch to is implied by the precondition.

\begin{mathpar}
    \inferrule*[right=\defrule{Br}]{ %% br
        C_{\text{label}}(i) = ti^{*};l_1;\phi_1
    }
    {
        C \vdash \<br> i : ti_1^{*}\;ti^{*};l_1;\phi_1 \rightarrow ti_2^{*};l_2;\phi_2
    }

    \inferrule*[right=\defrule{Return}]{ %% return
        C_{\text{return}} = ti^{*};l_1;\phi_1
    }
    {
        C \vdash \<return> : ti_1^{*}\;ti^{*};l_1;\phi_1 \rightarrow ti_2^{*};l_2;\phi_2
    }

    \inferrule*[right=\defrule{Br-If}]{ %% br_if
        C_{\text{label}}(i) = ti^{*};l_1;\phi_1,\neg(= a\; \ti{\<ithreetwo>}{0})
    }
    {
        C \vdash \<brif> i : ti^{*}\;\index{i32}{a};l_1;\phi_1 \rightarrow ti^{*};l_1;\phi_1,(= a\; \ti{\<ithreetwo>}{0})
    }

    \inferrule*[right=\defrule{Br-Table}]{ %% br_table
        (C_{\text{label}}(i) = ti^{*};l_1;\phi_i)^{+} \and
        (\phi_1 \implies \phi_i)^n
    }
    {
        C \vdash \<brtable> i^{+} : ti_1^{*}\;ti^{*}\;\index{i32}{a};l_1;\phi_1 \rightarrow ti_2^{*};l_2;\phi_2
    }
\end{mathpar}

Recall that functions are declared within the module with a specific type $tfi$, that is a precondition and postcondition.
These declared indexed function types are placed inside the module type context $C$.
Direct function calls $\<call> i$ have the same type as the declared indexed function type of the function they are calling with two differences.
First, the local index store is unchanged, since the called function will have been turned into a closure that operates on separate local variables in a local block.
Second, the index type context in the postcondition is extended with the declarations and constraints from the precondition.
The precondition and postcondition of a function can only contain constraints about the arguments supplied to that function, so simply copying the postcondition of the function would result in the loss of information about all other index variables.

Indirect function calls $\<callindirect> ti_1^{*};l_1;\phi_1 \rightarrow ti_2^{*};l_2;\phi_2$ include the expected indexed function type $ti_1^{*};l_1;\phi_1 \rightarrow ti_2^{*};l_2;\phi_2$ provided as part of their syntax (the same note about index variables being unification variables from above holds).
Remember that indirect function calls perform a run time type check against the closure that they end up calling, so we assume statically that the check will proceed because if it does not the program will trap and not be able to do any harm.
The same two differences described above between the expected indexed function type $tfi$ and the type of the $\<callindirect> tfi$ instruction also hold.

\begin{mathpar}
    \inferrule*[right=\defrule{Call}]{ %% call
        C_\text{func}(i) = ti_1^{*};l_1;\phi_1 \rightarrow ti_2^{*};l_2;\phi_2
    }
    {
        C \vdash \<call> i : ti_1^{*};l;\phi_1 \rightarrow ti_2^{*};l;\phi_1,\phi_2
    } \and
    \inferrule*[right=\defrule{Call-Indirect}]{ %% call_indirect
        C_\text{table}(i) = (j, tfi_2^{*}) \and
    }
    {
        C \vdash \<callindirect> ti_1^{*};l_1;\phi_1 \rightarrow ti_2^{*};l_2;\phi_2 : ti_1^{*}\;\ti{i32}{a};l;\phi_1 \rightarrow ti_2^{*};l;\phi_1,\phi_2
    } \and
\end{mathpar}

The only instructions that actually mutate the local index store are those that operate on local variables.
$\<getlocal>$ produces a fresh indexed type $\ti{t}{a_2}$ that is constrained to be equal to the index variable associated with the local being retrieved.
$\<setlocal>$ works in the reverse direction, replacing the index variable associated with the local being set with a fresh indexed type constrained to be equal to the value consumed.
Finally, $\<teelocal>$ is effectively a $\<setlocal>$ that consumes and immediately regurgitates the indexed type back onto the stack, like the Unix tool ``tee''.
\begin{mathpar}
    \inferrule*[right=\defrule{Get-Local}]{ %% get_local
        C_{\text{local}}(i) = t \and
        l(i) = \ti{t}{a} \and
    }
    {
        C \vdash \<getlocal> i : \epsilon;l;\phi \rightarrow \ti{t}{a};l;\phi
    }

    \inferrule*[right=\defrule{Set-Local}]{ %% set_local
        C_{\text{local}}(i) = t \and
        l_2 = l_1[i := \ti{t}{a}] \and
    }
    {
        C \vdash \<setlocal> i : \ti{t}{a};l_1;\phi \rightarrow \epsilon;l_2;\phi
    }

    \inferrule*[right=\defrule{Tee-Local}]{ %% tee_local
        C_{\text{local}}(i) = t \and
        l_2 = l_1[i := \ti{t}{a_2}] \and
    }
    {
        C \vdash \<teelocal> i : \ti{t}{a};l_1;\phi \rightarrow \ti{t}{a};l_2;\phi
    }
\end{mathpar}

Instructions for getting and setting globals produce and consume unconstrained values respectively.
Global variables are difficult to reason about in the type system since they are different between modules.
At compile-time, before linking, a module has no information about globals from another module which would be necessary for reasoning about the types of functions imported from the other module.
Therefore, we do not track index variables for globals.
We do still ensure that the value is of the correct type and in the case of setting a global variable that the global variable is mutable (has the ``mut'' flag in its type).
\begin{mathpar}
    \inferrule*[right=\defrule{Get-Global}]{ %% get_global
        C_\text{global}(i) = \text{mut}^{?}\; t \and
        a \text{ is fresh}
    }
    {
        C \vdash \<getglobal> i : \epsilon;l;\phi \rightarrow \ti{t}{a};l;\phi,\ti{t}{a}
    }

    \inferrule*[right=\defrule{Set-Global}]{ %% set_global
        C_\text{global}(i) = \text{mut } t
    }
    {
        C \vdash \<setglobal> i : \ti{t}{a};l;\phi \rightarrow \epsilon;l;\phi
    }
\end{mathpar}

The typing rules for memory instructions are very similar to \wasm as we do not reason about the contents of memory or how its size can change throughout a program.
As in \wasm, there are many small details related to how exactly values are loaded and store that are not particularly important to the understanding of the type system, but they are explained with the reduction rules for these values in \autoref{sec:wasmsemantics}.
One thing that does not appear in the \wasm reduction rules but mysteriously appears in the typing rules without much explanation is $align$.
According to the \wasm paper, $align$ is an ``alignment exponent''.
It is checked against the size of the type of the value being stored/loaded $|t|$ (or optionally $|tp|$, which should be less than $|t|$) in the premise $2^{align} \leq (|tp| <)^{?} |t|$.

\begin{mathpar}
    \inferrule*[right=\defrule{Mem-Load}]{ %% memory load
        C_\text{memory} = n \and
        2^{align} \leq (|tp| <)^{?} |t| \and
        a_2 \text{ is fresh}
    }
    {
        C \vdash t.\<load> (tp\_sx)^{?}\; align\; o : \ti{\<ithreetwo>}{a_1};l;\phi \rightarrow \ti{t}{a_2};l;\phi,\ti{t}{a_2}
    }

    \inferrule*[right=\defrule{Mem-Store}]{ %% memory store
        C_\text{memory} = n \and
        2^{align} \leq (|tp| <)^{?} |t|
    }
    {
        C \vdash t.\<store> tp^{?}\; align\; o : \ti{\<ithreetwo>}{a_1}\;\ti{t}{a_2};l;\phi \rightarrow \epsilon;l;\phi
    }

    \inferrule*[right=\defrule{Current-Memory}]{ %% current mem
        C_\text{memory} = n \and
        a \text{ is fresh}
    }
    {
        C \vdash \<currentmemory> : \epsilon;l;\phi \rightarrow \ti{\<ithreetwo>}{a};l;\phi
    }

    \inferrule*[right=\defrule{Grow-Memory}]{ %% grow mem
        C_\text{memory} = n \and
        a_2 \text{ is fresh}
    }
    {
        C \vdash \<growmemory> : \ti{\<ithreetwo>}{a_1};l;\phi \rightarrow \ti{\<ithreetwo>}{a_2};l;\phi
    }
\end{mathpar}

The last rules are the ones that can be used to compose sequences of instructions.
The first rule is for the empty instruction sequence $\epsilon$, which, similar to in \wasm, simply has the same precondition and postcondition $\epsilon;l;\phi$.
Second, we have \refrule{Stack-Poly} to add stack polymorphism (see \autoref{subsec:stackpoly}).
Third, there is a rule to compose a sequence of instructions $e_1^{*}$ with another instruction $e_2$.
\begin{mathpar}
    \inferrule*[right=\defrule{Empty}]{ %% empty
    }
    {
        C \vdash \epsilon : \epsilon;l;\phi \rightarrow \epsilon;l;\phi
    }

    \inferrule*[right=\defrule{Stack-Poly}]{ %% extra vars
        C \vdash e^{*} : ti_1^{*};l_1;\phi_1 \rightarrow ti_2^{*};l_2;\phi_2
    }
    {
        C \vdash e^{*} : ti^{*}\;ti_1^{*};l_1;\phi_1 \rightarrow ti^{*}\;ti_2^{*};l_2;\phi_2
    }

    \inferrule*[right=\defrule{Composition}]{ %% combine
        C \vdash e_1^{*} : ti_1^{*};l_1;\phi_1 \rightarrow ti_2^{*};l_2;\phi_2 \\
        C \vdash e_2 : ti_2^{*};l_2;\phi_2 \rightarrow ti_3^{*};l_3;\phi_3
    }
    {
        C \vdash e_1^{*}\;e_2 : ti_1^{*};l_1;\phi_1 \rightarrow ti_3^{*};l_3;\phi_3
    }
\end{mathpar}

\subsection{Subtyping, Implication, and Constraint Satisfaction}
\label{subsec:subtyping}
One issue with adding the index type context $\phi$ to preconditions and postconditions is that the postcondition of one instruction and the precondition of the next instruction might not match up exactly.
For example, one instruction may ensure a value is greater than ten, but the next just wants the value to be greater than zero.
Intuitively, if a value, ``x'', is greater than ten it must also be greater than zero, and we want the \name type system to be able to figure this out as well.
However, computers as of yet are unable to use intuition, so we must instead formalize this.

Our formalization of this problem is to allow \emph{strengthening} preconditions and \emph{weakening} postconditions.
Strengthening and weakening is based on implication ($\implies$).
We say that $\phi_1 \implies \phi_2$ when the following holds: if $\phi_1$ is satisfied, then $\phi_2$ must also be satisfied.
If $\phi_1 \implies \phi_2$, then we consider $\phi_1$ to be stronger than $\phi_2$, and $\phi_2$ to be weaker than $\phi_1$.
This solves the aforementioned problem because we can weaken ``x is greater than 10'' to ``x is greater than 0'' (or equivalently strengthen ``x is greater than 0'' to ``x is greater than 10'').

To fit strengthening and weakening into the type system, we define a subtyping judgment based on implication.
The \refrule{Implies} says that if an indexed function type $tfi_1$ has a stronger precondition and weaker postcondition than some other indexed function type $tfi_2$, and is otherwise equivalent, then $tfi_1$ is a subtype of $tfi_2$.

\begin{mathpar}
    \inferrule*[right=\defrule{Implies}]{
        \phi_0 \implies \phi_1 \and
        \phi_2 \implies \phi_3
    }{
        ti_1^{*};l_1;\phi_0 \rightarrow ti_2^{*};l_2;\phi_3 <: ti_1^{*};l_1;\phi_1 \rightarrow ti_2^{*};l_2;\phi_2
    }
\end{mathpar}

We then use this in the \name type system by adding a typing rule that allows the indexed function type for a list of instructions to be replaced by a subtype of that indexed function type.

\[
    \inferrule*[right=\defrule{SubTyping}]{
        ti_1^{*};l_1;\phi_0 \rightarrow ti_2^{*};l_2;\phi_3 <: ti_1^{*};l_1;\phi_1 \rightarrow ti_2^{*};l_2;\phi_2 \\
        C \vdash e^{*} \rightarrow ti_1^{*};l_1;\phi_1 \rightarrow ti_2^{*};l_2;\phi_2
    }{
        C \vdash e^{*} \rightarrow ti_1^{*};l_1;\phi_0 \rightarrow ti_2^{*};l_2;\phi_3
    }
\]

\subsection{Using Types for Check Elimination}
\label{subsec:checkelim}
In Section \ref{sec:newinstructions} we explained why \prechk-tagged instructions do not need dynamic checks because of their static guarantees of the \name type system.
Now, we will see how the \name type system provides those guarantees by looking at the typing rules for each of the \prechk-tagged instructions.

Integer division simply requires that the second argument is non-zero.
The premise $\phi \implies \neg(=\ a_2\ 0)$ requires that the index constraints satisfy the proposition $a_2 \neq 0$ for the pre-checked instruction to be safe.
Therefore, since a divide-by-zero is provably absent, it is safe to use the \prechk-tagged division instruction.
As an aside, recall that $a_3 \not\in \phi$ ensures that $a_3$ is fresh.
\begin{mathpar}
    \inferrule*[right=\defrule{Div-Prechk}]{
        \phi \implies \neg(=\ a_2\ 0) \and
        a_3 \not\in \phi
    }{
        C \vdash t.\<divpc> : \ti{t}{a_1}\;\ti{t}{a_2};l;\phi \rightarrow \ti{t}{a_3};l;\phi,\ti{t}{a_3},(= a_3\;(div\;a_1\;a_2))
    }
\end{mathpar}

Tagging memory loads and stores with \prechk requires ensuring that the memory index is valid.
Since \wasm and \name use linear memory which is a contiguous block of memory, we simply have to ensure that the index is within those bounds.
The initial memory size is the number of $64$ Ki pages ($65,536$ bytes), so we check that the constraints in the index type context ensure that the memory index plus the static offset is between $0$ and $65,536-width$.
We use $width$ as a shorthand to denote the number of bytes that is being stored/loaded, it is equal to $|t|/8$ if $tp^{?}=\epsilon$, and otherwise equal to $|tp|/8$.

Unfortunately, while the size of memory may be grown during program execution, we are currently unable to reason about changing memory size.
Therefore, we just use the initial memory size.
\begin{mathpar}
    \inferrule*[right=\defrule{Load-Prechk}]{ %% memory load
        C_\text{memory} = n \and
        2^{align} \leq (|tp| <)^{?} |t| \and
        a_3 \not\in \phi \\
        \phi \implies
        {\begin{stackTL}
            (\<ge> (\<add> a_1\; \ti{\<ithreetwo>}{o}) \ti{\<ithreetwo>}{0}), 
            \\ (\<le> (\<add> a_1\; (\<add> \ti{\<ithreetwo>}{o+width}))\; \ti{\<ithreetwo>}{n*64 \text{Ki}})
        \end{stackTL}}
    }
    {
        C \vdash t.\<loadpc> (tp\_sx)^{?}\; align\; o : \ti{\<ithreetwo>}{a_1};l;\phi \rightarrow \ti{t}{a_2};l;\phi,\ti{t}{a_2}
    }

    \inferrule*[right=\defrule{Store-Prechk}]{ %% memory store
        C_\text{memory} = n \and
        2^{align} \leq (|tp| <)^{?} |t| \\
        \phi \implies
        {\begin{stackTL}
            (\<ge> (\<add> a_1\; \ti{\<ithreetwo>}{o}) \ti{\<ithreetwo>}{0}), 
            \\ (\<le> (\<add> a_1\; (\<add> \ti{\<ithreetwo>}{o+width}) \ti{\<ithreetwo>}{0}) \ti{\<ithreetwo>}{n*64\text{Ki}})
        \end{stackTL}}
    }
    {
        C \vdash t.\<storepc> tp^{?}\; align\; o : \ti{\<ithreetwo>}{a_1}\;\ti{t}{a_2};l;\phi \rightarrow \epsilon;l;\phi
    }
\end{mathpar}

Indirect function calls in \wasm require a dynamic check to ensure that the index into the table points to a function of a suitable type (recall the explanation of tables and \<callindirect> from \autoref{sec:wasmsemantics}).
Proving the safety of an indirect function call involves showing that every possible function that could be called will not cause a run-time type error.
We ensure this by requiring that the type of every function at every possible index value has a subtype of the expected type: $\forall 0 \leq i < j. (\phi \implies \neg (= \ti{\<ithreetwo>}{i}\; a)) \lor tfis(i) <: tfi$ where $tfis=(tfi_2 ...)$.
Further, we must show that the provided table index is within the table boundaries: $\phi \implies 0 \leq c < j$.
\begin{mathpar}
    \inferrule*[right=\defrule{Call-Indirect-Prechk}]{ %% call_indirect
        C_{table}(i) = (j, (tfi_2 ...)) \and
        \phi \implies 0 \leq c < j \\
        tfi = ti_1^{*};l_1;\phi_1 \rightarrow ti_2^{*};l_2;\phi_2 \\
        \forall 0 \leq i < j.\;
        {\begin{stackTL}
            (\phi \implies \neg (= \ti{\<ithreetwo>}{i}\; a)) 
            \\ \lor\; tfis(i) <: tfi$ where $tfis=(tfi_2 ...)
        \end{stackTL}}
    }
    {
        C \vdash \<callindirectpc> tfi : ti_1^{*}\;\ti{i32}{a};l;\phi_1 \rightarrow ti_2^{*};l;\phi_1,\phi_2
    } \and
\end{mathpar}

\subsection{Module Types}
The complete module typing rules are in Figure \ref{fig:modulerules} (not that $im$ is an import and $ex$ is an export).
Functions $f$, typecheck their body $e^{*}$ under the module type context $C$ with the expected postcondition $ti_2^{*};l_2;\phi_2$ in the label stack and return position, and with the local index store $\ti{t_1}{a_1}^{*}\;\ti{t}{a_2}^{*}$ constructed from the function's arguments $\ti{t_1}{a_1}^{*}$ and declared locals $\ti{t}{a_2}^{*}$.
Global variables $glob$ must ensure that their initialization instructions $e^{*}$ produce a value of the proper type $t$.
Exported global variables cannot be mutable, if there are any exports defined, the global cannot have the mutable tag $mut$: $ex^{*}=\epsilon \lor tg=t$.
Tables $tab$ ensure that the indices $i^{n}$ refer to well-typed functions and there are exactly as many indices as the expected size $n$.
Memory $mem$ simply has its declared initial size $n$ from which it can only grow bigger.
All imported functions, globals, tables, and memories are expected to have their declared type.
They are type checked during linking.

Type checking a module involves typechecking every component of the module.
Functions, $f$, are typechecked under the module type context, $C$, containing the entirety of the module.
This means that functions can refer to themselves, other functions, all globals, the table, and memory.
This may seem to be a circular definition, but the type of the module is declared statically (as the combined declared types of all the module components), so it is just checking against the expected module index type context.
Globals, $glob$, are typechecked under the module index context containing only the global variable declarations preceding the current declaration.

\begin{figure}[h]
    \begin{mathpar}
        \inferrule*[]{ %% local function
            tfi = \ti{t_1}{a_1}^{*};\epsilon;\phi_1 \rightarrow ti_2^{*};l_2;\phi_2 \\
            C_2 = C,\text{local } t_1^{*}\;t^{*},\text{label } (ti_2^{*};l_2;\phi_2),\text{return } (ti_2^{*},l_2,\phi_2) \\
            C_2 \vdash e^{*} : \epsilon ;\ti{t_1}{a_1}^{*}\;\ti{t}{a_2}^{*};\phi_1 \rightarrow ti_2^{*};l_2;\phi_2
        } {
            C \vdash ex^{*}\; \<func> tfi\; \<local> t^{*}\; e^{*} : ex^{*}\; tfi
        } \and

        \inferrule*[]{ %% imported function
        } {
            C \vdash ex^{*}\; \<func> tfi\; im : ex^{*}\; tfi
        } \\

        \inferrule*[]{ %% local global
            tg = mut^{?}\;t \and
            ex^{*} = \epsilon \lor tg = t \and
            C \vdash e^{*} : \epsilon; \epsilon; \phi_1 \rightarrow \ti{t}{a};\epsilon; \phi_2
        } {
            C \vdash ex^{*}\; \<global> tg\; e^{*} : ex^{*}\; tg
        } \and

        \inferrule*[]{ %% imported global
            tg = t \and
        } {
            C \vdash ex^{*}\; \<global> tg\; im : ex^{*}\; tg
        } \and

        \inferrule*[]{ %% local table
            (C_{\text{func}}(i) = tfi)^n
        } {
            C \vdash ex^{*}\; \<table> n\; i^n : ex^{*}\; (n,tfi^n)
        } \and

        \inferrule*[]{ %% imported table
        } {
            C \vdash ex^{*}\; \<table> (n,tfi^n)\; im : ex^{*}\; (n,tfi^n)
        } \and

        \inferrule*[]{ %% local memory
        } {
            C \vdash ex^{*}\; \<memory> n : ex^{*}\; n
        }

        \inferrule*[]{ %% imported memory
        } {
            C \vdash ex^{*}\; \<memory> n\; im : ex^{*}\; n
        }

        \inferrule*[]{ %% module
            (C\vdash f : ex_f^{*}\; tfi)^{*} \and
            (C_i \vdash glob_i : ex_g^{*}\; tg_i)^{*} \\
            (C \vdash tab : ex_t^{*}\; (n,tfi^n))^{?} \and
            (C \vdash mem : ex_m^{*}\; n)^{?} \\
            (C_i=\{\text{global } tg^{i-1}\})_i^{*} \and
            ex_f^{*\;*}\; ex_g^{*\;*}\; ex_t^{*\;?}\; ex_m^{*\;?} \text{ distinct} \\
            C = \{\text{func } tfi^{*}, \text{global } tg^{*}, \text{table } (n,tfi^n)^{?}, \text{memory } n^{?}\}  
        } {
            \vdash \<module> f^{*}\; glob^{*}\; tab^{?}\; mem^{?}
        }
    \end{mathpar}
    \caption{Indexed Module Typing Rules}
    \label{fig:modulerules}
\end{figure}
\chapter{Metatheory}
\label{chp:metatheory}

Now that we have introduced \name and shown how it can be used for reasoning, it is time to reason about \name itself.
First, we will take a look at the relationship between \wasm and \name, by showing methods to translate \wasm programs to \name programs and vice versa.
Then, we will prove the type safety of \name, to ensure that our claim that \name is as safe as \wasm is valid.
However, before we can do any of that, we must ``complete'' our reasoning ability by creating a way to connect the reduction relation form with the type system.

\section{Administrative Typing Rules}
While we have shown the \name typing rules for instructions within a static context, we still need typing rules for administrative instructions and the store used in reduction.
\emph{Administrative instructions} are introduced for reduction to keep track of information during reduction.
For example, $\<local>$ is the result of reducing a closure call; it is used to reduce a function body within the closed environment of the closure.
They are not part of the surface syntax of a language (\eg you cannot put a local block in a \name program), and can only appear as an intermediate term during reduction.
\autoref{fig:storerules} shows the \name typing rules for module instances $inst$, the run time store $s$, and various data structures contained within $s$.
There are many different judgments being introduced, so we explicitly state the form of the judgment before stating the rule for that judgment.

During reduction, we use \refrule{Program} (\autoref{fig:programrules}) to ensure that a \name program state (consisting of the store $s$, local variables $v^{*}$, and instruction sequence $e^{*}$) is well typed (notice that it has the same form as the reduction relation).
It uses \refrule{Code} and relies on the store being well-typed (\refrule{Store} in \autoref{fig:storerules}), to ensure that a reducible \name program is well typed.
\refrule{Code} checks that a sequence of instructions is well typed with an empty stack, the indexed types and constraints for the given local variables in the precondition, and an optional return postcondition (not used by \refrule{Program}).
Since local variables are values, we know that each one of them is equal to some constant, so \refrule{Code} is really just checking that the sequence of instructions has some postcondition reachable from the given local variables.
There is an optional return postcondition for \refrule{Code} because the typing rule for local blocks (as seen in \refrule{Local} in \autoref{fig:adminrules}) has as a premise a judgment of the exactly same form, except with a return postcondition.

\begin{figure}
    $$ S ::= \{ \text{inst } C^{*}, \text{ tab } n^{*}, \text{ mem } m^{*} \} $$

    $ \boxed{\vdash s;v^{*};e^{*}} $

    \begin{mathpar}
        \inferrule*[right=\defrule{Program}]{ %% admin program
            \vdash s : S \and
            S;\epsilon \vdash_i v^{*};e^{*} : ti^{*};l;\phi
        } {
            \vdash_i s;v^{*};e^{*} : ti^{*};l;\phi
        }
    \end{mathpar}

    $ \boxed{S;(ti^{*};l;\phi)^{?} \vdash_i v^{*};e^{*} : ti^{*};l;\phi} $

    \begin{mathpar}
    \inferrule*[right=\defrule{Code}]{ %% admin code
        (\vdash v : \ti{t}{a};\phi_v)^{*}\\
        C = S_{\text{inst}}(i),\text{local} \; t^{*}, \text{return} \; (ti^n;l;\phi)^{?}\\
        S;C \vdash e^{*} : \epsilon:\ti{t}{a}^{*};\phi_v^{*} \rightarrow ti^n;l;\phi
    } {
        S;(ti^n;l;\phi)^{?} \vdash_i v^{*};e^{*} : ti^n;l;\phi
    }
    \end{mathpar}
    \caption{\name Program Typing Rules}
    \label{fig:programrules}
\end{figure}

In addition to getting the type of the instructions being reduced, we also need to know the type of the store $s$ since it is part of the reduction relation.
\refrule{Store} checks that a run-time store, $s$ is well typed by the store context $S$.
The store context $S$ is to $s$ as $C$ is to $inst$.
That is, it contains the type information for everything in $s$.
\refrule{Store} ensures that every module instance $inst$ in $s$ has the type of the index module context $C$ in $S$ using \refrule{Instance}.
Further, \refrule{Store} ensures that all of the closures in all of the tables in $s$ are well typed, and the the sizes of all the tables and memory chunks in $S$ do not exceed the actual size of their implementations.

To get the type of the store, we in turn have to know the types of each of the various run-time data structures.
\refrule{Instance} checks that a module instance is well-typed by the index module context under the store context $S$.
It checks all of the closures $cl^{*}$ against their expected types $tfi^{*}$ in $C$, and similarly for all of the globals ($v^{*}$ and $(\text{mut}^{?}\; t)^{*}$).
The table and memory indices ($i$ and $j$, respectively) are used to look up the the relevant types ($(n,tfi^{*})$ and $m$, respectively) in the store context $S$.
Closures are typechecked by \refrule{Closure}, which falls back on the module typing rules from \autoref{fig:modulerules} to typecheck the function definition inside of the closure.
\refrule{Admin-Const} gets the postcondition indexed types and constraints on values; it is used to typecheck local and global variables.

\begin{figure}
    $ \boxed{\vdash s : S} $

    \begin{mathpar}
        \inferrule*[right=\defrule{Store}]{ %% store
            S = \{ \text{inst} \; C^{*}, \text{tab} \; n^{*}, \text{mem} \; m^{*} \}\\
            (S \vdash inst : C)^{*} \and
            ((S \vdash cl : tfi)^{*})^{*} \\
            (n \leq |cl^{*}|)^{*} \and
            (m \leq |b^{*}|)^{*}
        } {
            \vdash \{ \text{inst} \; inst^{*}, \text{tab} \; (cl^{*})^{*}, \text{mem} \; (b^{*})^{*} \} : S
        }
    \end{mathpar}

    $ \boxed{S \vdash inst : C} $

    \begin{mathpar}
        \inferrule*[right=\defrule{Instance}]{ %% instance
            (S \vdash cl : tfi)^{*} \and
            (\vdash v : \ti{t}{a},\phi_v)^{*} \\
            (S_\text{tab}(i) = n)^{?} \and
            (S_\text{mem}(j) = m)^{?}
        } {
            S \vdash
            {\begin{stackTL}
                \{ \text{func} \; cl^{*}, \text{glob} \; v^{*}, \text{tab} \; i^{?}, \text{mem} \; j^{?} \}
                \\ : \{ \text{func} \; tfi^{*}, \text{global} \; (\text{mut}^{?} \; t)^{*}, \text{table} \; n^{?}, \text{memory} \; m^{?} \}
            \end{stackTL}}
        }
    \end{mathpar}

    $ \boxed{\vdash v : ti;\phi} $

    \begin{mathpar}
        \inferrule*[right=\defrule{Admin-Const}]{ %% admin const
        } {
            \vdash t.\<const> c : \ti{t}{a};\circ,\ti{t}{a},(\<eq> a \; \ti{t}{c})
        }
    \end{mathpar}

    $ \boxed{S \vdash cl : tfi} $

    \begin{mathpar}
        \inferrule*[right=\defrule{Closure}]{ %% closure
            S_\text{inst}(i) \vdash f : tfi
        } {
            S \vdash \{ \text{inst} \; i, \text{code} \; f \} : tfi
        }
    \end{mathpar}
    \caption{\name Store Typing Rules}
    \label{fig:storerules}
\end{figure}

\begin{figure}
    $\boxed{S;C \vdash e^{*} : tfi}$

    \begin{mathpar}
        \inferrule*[right=\defrule{Local}]{ %% local
            S;(ti^n;l_2;\phi_2) \vdash_i v_l^{*};e^{*} : ti^n;l_2;\phi_2
        } {
            S;C \vdash \<local> \{ i;v_l^{*} \} \; e^{*} \<end> : \epsilon;l_1;\phi_1 \rightarrow ti^n;l_1;\phi_1,\phi_2
        }

        \inferrule*[right=\defrule{Call-Cl}]{ %% call closure
            S \vdash cl : tfi
        } {
            S;C \vdash \<call> cl : tfi
        }

        \inferrule*[right=\defrule{Trap}]{ %% trap
        } {
            S;C \vdash \<trap> : tfi
        }

        \inferrule*[right=\defrule{Label}]{ %% label
            S;C\vdash e_0^{*} : ti_3^{*};l_3;\phi_3 \rightarrow ti_2^{*};l_2;\phi_2 \\
            S;C,\text{label } (ti_3^{*};l_3;\phi_3) \vdash e^{*} : \epsilon;l_1;\phi_1 \rightarrow ti_2^{*};l_2;\phi_2
        } {
            S;C \vdash \<label> \{ e_0^{*} \} \; e^{*} \<end> : \epsilon;l_1;\phi_1 \rightarrow ti_2^{*};l_2;\phi_2
        }
    \end{mathpar}
    \caption{\name Administrative Instruction Rules}
    \label{fig:adminrules}
\end{figure}

Now we will introduce the typing rules for administrative instructions, and the administrative typing judgment in \autoref{fig:adminrules}.
The administrative typing judgment $S;C \vdash e^{*} : tfi$ extends the \name typing rules for instructions to include administrative instructions and the store context $S$.
Every rule of the judgment $C \vdash e^{*} : tfi$ (recall the rules enumerated in \autoref{sec:typesys}) is implicitly added to the administrative judgment by accepting any $S$.

Most of the rules for administrative instructions check against extra information provided by the administrative typing judgment.
\refrule{Local} typechecks a local block using \refrule{Code} to ensure that the body $e^{*}$ is well typed with the indexed types and constraints for local variables provided by the local block as the precondition and any postcondition.
Since local blocks are inline expansions of function calls, we use the optional return postcondition functionality of \refrule{Code} to ensure that returning from inside the local block will be well typed.
\refrule{Call-Cl} typechecks calling a closure by ensuring that the closure $cl$ being called has the same type as the call instruction $\<call> cl$ in $S$.
\refrule{Trap} is always well typed under any precondition and postcondition.
\refrule{Label} typechecks the body of the label block with the precondition of the saved instructions pushed onto the label stack.
If the label was generated by a loop, then the precondition of the saved values is the precondition of the loop, and we know the loop is well typed.
Otherwise, the saved instructions will be an empty sequence and will be well typed from the precondition.

Given these additional typing judgments and rules, we can now show the metatheoretic properties mentioned above.

\section{Relationship Between \wasm and \name}
We want to show two properties about the relationship between \wasm and \name.
First, we want \name to be backwards compatible with \wasm.
It should be possible to convert well-typed \wasm programs into well-typed \name programs with no additional developer effort.
We demonstrate a simple yet naive way of embedding \wasm programs into \name in \autoref{subsec:embedding}.
Second, we want to show that well-typed \name programs can be turned into \wasm programs.
This is accomplished in \autoref{subsec:erasure} using an erasure function that turns \name programs and types into \wasm programs and types.

\subsection{Embedding \wasm in \name}
\label{subsec:embedding}
We present a way to embed \wasm programs in \name.
The embedding function takes a \wasm program and replaces all of the type annotations with indexed function types that have no constraints on the variables.
Intuitively, this is the only part of the surface syntax of \wasm that isn't in \name, so we must figure out a way to bring it over.
While this embedding requires no additional developer effort, it provides no information to the indexed type system beyond what can be inferred from the instructions in the program.
We conjecture that a well typed \wasm program embedded in \name is also well typed, but we have not proved it.

We typeset \name instructions in a \tbsf{blue sans serif font} and \wasm instruction in a \trbf{bold red font} to set them apart.

\begin{conjecture}{Well Typed \wasm Programs Embedded in \name are Well Typed}

    If $\vdash \<wmodule> f^{*}\;glob^{*}\;tab^{?}\;mem^{?}$,
    \\then $\vdash \embed[module]{\<wmodule> f^{*}\;glob^{*}\;tab^{?}\;mem^{?}}$
\end{conjecture}

In order to support embedding of both local and imported functions we need a syntactic way of extracting the type annotation from the function.

\begin{definition}{\fbox{$annotation(f)=\tfi$}}
    \begin{mathpar}
        \begin{array}{rcl}
            annotation(ex^{*}\; \<func> \tfi\; \<local>\; t^{*}\; e^{*})
            &=& \tfi\\
            annotation(ex^{*}\; \<func> \tfi\; im)
            &=& \tfi\\
        \end{array}
    \end{mathpar}
\end{definition}

Embedding works purely over the surface syntax of the languages.
As such, we define embedding over modules: the pinnacle syntactic objects of both the \wasm and \name surface syntax hierarchies.
Embedding a module $module$ means embedding all of the functions $f^{*}$ in the module, and embedding the table $tab$ parameterized with all of the function definitions $f^{*}$.
We do not have to embed globals $glob^{*}$ or the memory $mem^{?}$ as they have the same syntax in both \wasm and \name.
We explain how to embed tables $tab$ in \autoref{def:embed-t}, and functions $f$ in \autoref{def:embed-f}.

\begin{definition}{\fbox{$\embed[module]{module}=\mathbluesf{module}$}}
    \label{def:embed-m}
    \begin{mathpar}
        \begin{array}{rcl}
            embed_{module}(\<wmodule> f^{*}\; glob^{*}\; tab^{?}\; mem^{?})
            &=& \<module>
            \begin{stackTL}
                \embed[f]{f}^{*}
                \\ glob^{*}
                \\ \embed[tab]{tab^{?}}^{annotation(f)^{*}}
                \\ mem^{?}
            \end{stackTL} \\
        \end{array}
    \end{mathpar}
\end{definition}

Tables in \name must also provide the indexed function types of all the functions they contain, so to embed them we must include those types.
We do this by parameterizing the embedding of the table $tab$ with all of the function types $\tfi^{*}$.
Then, we retrieve the indexed function type $ti_1;l_1;\phi_1 \rightarrow ti_2;l_2;\phi_2$ of the function pointed to by the function index $i$ in $\tfi^{*}$ for every function index $i$ in the table.
We cannot embed imported tables because we have no way of accessing the types of the functions included in the table.

\begin{definition}{\fbox{$\embed[tab]{tab}^{\tfi^{*}}=\mathbluesf{tab}$}}
    \label{def:embed-t}
    \begin{mathpar}
        \begin{array}{rcl}
            embed_{tab}(ex^{*}\; \<wtable> n\; i^{n})
            &=& ex^{*}\; \<table> n\; (ti_1;\epsilon;\phi_1 \rightarrow ti_2;\epsilon;\phi_2)^{n} \\
            && \text{where } \forall i. \tfi^{*}(i) = ti_1;\epsilon;\phi_1 \rightarrow ti_2;\epsilon;\phi_2
        \end{array}
    \end{mathpar}
\end{definition}

The embedding of functions, \autoref{def:embed-f}, both must construct an indexed function type for itself and embed its body.
Function bodies have their local variables defined by the function that they are enclosed in.
Thus, when the function body is embedded we pass the local types ($t_1^{*}\;t^{*}$) so the body knows how to constrain local variables.
We construct an indexed function type that has the precondition of the expected values on the stack turned into indexed types using fresh index variables and the types $t_1^{*}$ from the \wasm type, and do the same with the postcondition and $t_2^{*}$.

\begin{definition}{\fbox{$\embed[f]{f}=\mathbluesf{f}$}}
    \label{def:embed-f}
    \begin{mathpar}
        \begin{array}{rcl}
            embed_f(ex^{*}\; \<wfunc>
            \begin{stackTL}
                (t_1^{*} \rightarrow t_2^{*})
                \\ \<wlocal>\; t^{*}\; e^{*})
            \end{stackTL}
            &=& ex^{*}\; \<func>\;
                \begin{stackTL}
                    (\ti{t_1}{a_1}^{*};\epsilon;(\circ,\ti{t_1}{a_1}^{*})
                    \\ \rightarrow
                    \ti{t_2}{a_2}^{*};\epsilon;(\circ,\ti{t_2}{a_2}^{*}))
                    \\ \<local>\; t^{*}\; \embed[e]{e}^{(t_1^{*}\;t^{*})})^{*}
                \end{stackTL}\\
            embed_f(ex^{*}\; \<wfunc> (t_1^{*} \rightarrow t_2^{*})\; im)
            &=& ex^{*}\; \<func>\;
                \begin{stackTL}
                    (\ti{t_1}{a_1}^{*};\epsilon;(\circ,\ti{t_1}{a_1}^{*})
                    \\ \rightarrow
                    \ti{t_2}{a_2}^{*};\epsilon;(\circ,\ti{t_2}{a_2}^{*}))
                    \\ im
                \end{stackTL}\\
        \end{array}
    \end{mathpar}
\end{definition}

Embedding instructions replaces all function types used within the \wasm syntax with \name indexed function types, and adds the function types for all of the functions in a table to the table's type declaration.
This occurs within blocks and indirect function calls, as shown in \autoref{def:embed-e}.
The indexed types simply have fresh index variables that are different in the precondition and postcondition, and the primitive types for the stack are known from the \wasm type $t_1^{*} \rightarrow t_2^{*}$.
To know what the local variables are, we parameterize the embedding over the types of local variables ($t^{*}$).

\begin{definition}{\fbox{$\embed[e]{e}^{t^{*}}=\mathbluesf{e}$}}
    \label{def:embed-e}
    \begin{mathpar}
        %% SPACE HACKS
        \arraycolsep=2pt
        \begin{array}{rcl}
            embed_{e^{*}}({\begin{stackTL}
                \<wblock>
                {\begin{stackTL}
                    (t_1^{*} \rightarrow t_2^{*})\;
                    \\e^{*}
                \end{stackTL}}\\
            \<wend>)^{t^{*}}
            \end{stackTL}}
            &=& {\begin{stackTL}
                    \<block>
                    \\ \quad (\ti{t_1}{a_1}^{*};\ti{t}{a_3}^{*};(\circ,\ti{t_1}{a_1}^{*},\ti{t}{a_3}^{*})
                    \\ \quad\; \rightarrow \ti{t_2}{a_2}^{*};\ti{t}{a_4}^{*};(\circ,\ti{t_2}{a_2}^{*},\ti{t}{a_4}^{*}))
                    \\ \quad \embed[e^{*}]{e^{*}}^{t^{*}}
            \end{stackTL}} \\
            && \<nsend>\\

            embed_{e^{*}}({\begin{stackTL}
                \<wloop>
                {\begin{stackTL}
                    (t_1^{*}\rightarrow t_2^{*})\;
                    \\e^{*}
                \end{stackTL}}\\
            \<wend>)^{t^{*}}
            \end{stackTL}}
            &=& {\begin{stackTL}
                    \<loop>
                    \\ \quad (\ti{t_1}{a_1}^{*};\ti{t}{a_3}^{*};(\circ,\ti{t_1}{a_1}^{*},\ti{t}{a_3}^{*})
                    \\ \quad\; \rightarrow \ti{t_2}{a_2}^{*};\ti{t}{a_4}^{*};(\circ,\ti{t_2}{a_2}^{*},\ti{t}{a_4}^{*}))
                    \\ \quad \embed[e^{*}]{e^{*}}^{t^{*}}
            \end{stackTL}} \\
            && \<nsend>\\

            embed_{e^{*}}({\begin{stackTL}
                \<wif>
                {\begin{stackTL}
                    (t_1^{*}\rightarrow t_2^{*})\;
                    \\e^{*}
                \end{stackTL}}\\
            \<wend>)^{t^{*}}
            \end{stackTL}}
            &=& {\begin{stackTL}
                    \<if>
                    \\ \quad (\ti{t_1}{a_1}^{*};\ti{t}{a_3}^{*};(\circ,\ti{t_1}{a_1}^{*},\ti{t}{a_3}^{*})
                    \\ \quad\; \rightarrow \ti{t_2}{a_2}^{*};\ti{t}{a_4}^{*};(\circ,\ti{t_2}{a_2}^{*},\ti{t}{a_4}^{*}))
                    \\ \quad \embed[e]{e_1^{*}}^{t^{*}}\; \embed[e]{e_2^{*}}^{t^{*}}
                \end{stackTL}} \\
            && \<nsend>\\

            embed_{e^{*}}(
                {\begin{stackTL}
                    \<wcallindirect>
                    \\\quad (t_1^{*}\rightarrow t_2^{*}))^{t^{*}}
                \end{stackTL}}
            &=& {\begin{stackTL}
                \<callindirect>
                \\ \quad (\ti{t_1}{a_1}^{*};\ti{t}{a_3}^{*};(\circ,\ti{t_1}{a_1}^{*},\ti{t}{a_3}^{*})
                \\ \quad\; \rightarrow \ti{t_2}{a_2}^{*};\ti{t}{a_4}^{*};(\circ,\ti{t_2}{a_2}^{*},\ti{t}{a_4}^{*}))
            \end{stackTL}} \\

            \embed[e^{*}]{e}^{t^{*}} &=& e \text{, otherwise} \\
            \embed[e^{*}]{e^{*}}^{t^{*}} &=& (\embed[e^{*}]{e}^{t^{*}})^{*} \\
        \end{array}
    \end{mathpar}
\end{definition}

These are not the only differences in the surface syntax between \wasm and \name: we also introduced four new instructions (the \prechk-tagged instructions).
The definition of embedding we have introduced has been entirely syntactic, but that will not work for replacing non-\prechk-tagged instructions with \prechk-tagged versions during embedding since we must be able to ensure that stronger guarantees are met.
Instead, one could, for example, check at every $\<div>$, $\<callindirect>$, $\<load>$, and $\<store>$ whether the \prechk-tagged version of the instruction is welltyped, and only if it is well typed replace the instruction with the \prechk-tagged version.
However, a more sophisticated static analysis could be able to provide more precise type annotations and therefore potentially allow even check eliminations.

Want to prove that the embedding produces well-typed \name, embedding works on a module level, so the proof should be done on a module level.


\begin{theorem}{Sound Module Typing Embedding}
    \label{thm:moduleembedding}

    If $\vdash \<wmodule> f^{*}\;glob^{*}\;tab^{?}\;mem^{?}$ and
    $tab = (ex_t^{*}\; \<wtable> n\; i^{*})$,
    \\ then $\vdash \embed[module]{\<wmodule> f^{*}\;glob^{*}\;tab^{?}\;mem^{?}}$
\end{theorem}
\begin{proof}

    Note that the globals $glob^{*}$ and memory $mem^{?}$ are not affected by embedding, and have the same module typing rules in \wasm as in \name.
    Thus, we only need to reason about the functions $f^{*}$ and table $tab^{?}$.

    $C_w = \{ \text{func } tf^{*},\; \text{global } tg^{*},\; \text{table } n^{?},\; \text{memory } n^{?} \}$ since it is a premise of WASM module.

    If $(C_w \vdash f : ex_f^{*}\; tf)^{*}$,
    $(annotation(\embed[f]{f}) = \tfi)^{*}$, and
    $C_p = \{ \text{func } \tfi^{*}, \text{global } tg^{*}, \text{table } n^{?},\; \text{memory } n^{?} \}$,
    then $(C_p \vdash \embed[f]{f}^{*} : ex^{*}_f\; \tfi)^{*}$ by \todo{TODO}.

    $\embed[tab]{tab^{?}}^{\tfi^{*}} = (ex_t^{*}\; \<table> n\; \tfi_t^n)^{?}$,
    where $\forall i. \tfi^{*}(i) = \tfi_t$ by \autoref{def:embed-t}.

    Then, $(C_p \vdash ex_t^{*}\; \<table> n\; i^n : ex_t^{*}\; (n,\tfi_t^n))^{?}$, by \refrule{Table}.

    Therefore, $\vdash \embed[module]{\<wmodule> f^{*}\;glob^{*}\;tab^{?}\;mem^{*}}$.
\end{proof}

\subsection{Erasing \name to \wasm}
\label{subsec:erasure}
We provide an erasure function for \name that transforms \name programs into \wasm programs by discarding the extra information from the \name type system and replacing \prechk-tagged instructions with their non-tagged counterparts.
Erasure is useful for demonstrating backwards compatibility.
Additionally, erasure is useful in the type safety proof in Section \ref{subsec:progress}.
We show that erasing a well-typing \name program produces a well-typed \wasm program.

We typeset \name instructions in a \tbbf{bold blue font} and \wasm instruction in a \textcolor{red}{\textsf{red, sans serif font}} to set them apart.

\begin{definition}{$\erase{tfi} = \trsf{tf}$}

    $\erase{\ti{t_1}{a_1}^{*};l_1;\phi_1 \rightarrow \ti{t_2}{a_2}^{*};l_2;\phi_2} = \trsf{t_1^{*} \rightarrow t_2^{*}}$
\end{definition}

\begin{definition}{$\erase{e} = \trsf{e}$}
    \begin{mathpar}
        \erase{\<block> tfi\; e^{*} \<end>} = \<wblock> \erase{tfi}\; \erase{e^{*}} \<wend>

        \erase{\<loop> tfi\; e^{*} \<end>} = \<wloop> \erase{tfi}\; \erase{e^{*}} \<wend>

        \erase{\<if> tfi\; e_1^{*}\; e_2^{*} \<end>} = \<wif> \erase{tfi}\; \erase{e_1^{*}}\; \erase{e_2^{*}} \<wend>

        \erase{\<callindirect> tfi} = \<wcallindirect> \erase{tfi}

        \thought{Does it really make sense to have prechk instructions here for the progress proof? I guess it's fine cause we list them explicitly.}

        \erase{t.\<divpc>} = t.\<wdiv>

        \erase{t.\<callindirectpc>} = t.\<wcallindirect>

        \erase{t.\<storepc> tp^{?}\; align\; o} = t.\<wstore> tp^{?}\; align\; o

        \erase{t.\<loadpc> (tp\_sx)^{?}\; align\; o} = t.\<wload> (tp\_sx)^{?}\; align\; o

        \erase{e} = e \text{, otherwise}
    \end{mathpar}

\end{definition}

\begin{definition}{$\erase{C} = \trsf{C}$}
    \begin{mathpar}
        {\begin{stackTL} erase(
            {\begin{stackTL}
                \text{func } tfi^{*}, \text{ global } tg^{*}, \text{ table } (n,tfi^{*})^{?}, \text{ memory } m^{?},
                \\ \text{local } t^{*}, \text{ label } (\ti{t_1}{a_1}^{*};l_1;\phi_1)^{*}, \text{ return } (\ti{t_2}{a_2}^{*};l_2;\phi_2)^{?})
            \end{stackTL}}
        \\=
        {\begin{stackTL}\{\text{func } \erase{tfi^{*}}, \text{ global } tg^{*}, \text{ table } n^{?}, \text{ memory }\; m^{?},
            \\ \text{local } t^{*}, \text{ label } (t_1^{*})^{*}, \text{ return } (t_2^{*})^{?}\}
        \end{stackTL}}
        \end{stackTL}}
    \end{mathpar}
\end{definition}

\begin{definition}{$\erase{S} = \trsf{S}$}

    $\erase{\{\text{inst } C^{*}, \text{tab } n^{*}, \text{mem } m^{*}\}}=
    \{\text{inst } \erase{C}^{*}, \text{tab } n^{*}, \text{mem } m^{*}\}$
\end{definition}

\begin{lemma}{(Erased-Well-Typed-Admin)}

    If $S;C \vdash e^{*} : ti_1^{*};l_1;\phi_1 \rightarrow ti_2^{*};l_2;\phi_2$,
    \\ then $\erase{S};\erase{C} \vdash \erase{e^{*}} : \erase{\epsilon;l_1;\phi_1 \rightarrow \ti{t}{a}^{*};l_2;\phi_2}$
\end{lemma}
\begin{proof}
    \thought{These proofs are so simple that I really don't think it's worth it to enumerate them.}

    We proceed by induction over typing rules. Most proof cases are omitted as they are quite simple but we provide a few to give an idea of what the proofs look like.

    \begin{itemize}
        \item $S;C \vdash t.binop : \ti{t}{a_1}\;\ti{t}{a_2};l_1;\phi_1 \rightarrow \ti{t}{a_3};l_1;\phi_1,\ti{t}{a_3},(= a_3\; (binop\;a_1\;a_2))$

        $\erase{S};\erase{C} \vdash \erase{t.binop} : \erase{\ti{t}{a_1}\;\ti{t}{a_2};l_1;\phi_1 \rightarrow \ti{t}{a_3};l_1;\phi_1,\ti{t}{a_3},(= a_3\; (binop\;a_1\;a_2))}$ = $\erase{S};\erase{C} \vdash t.binop : t\;t \rightarrow t$, which holds under \wasm's type system.
        \item $S;C \vdash \<unreachable> : ti_1^{*};l_1;\phi_1 \rightarrow ti_2^{*};l_2;\phi_2$

        $\erase{S};\erase{C} \vdash \erase{\<unreachable>} : \erase{ti_1^{*};l_1;\phi_1 \rightarrow ti_2^{*};l_2;\phi_2}$ = $\erase{S};\erase{C} \vdash \<wunreachable> : t_1^{*} \rightarrow t_2$, which holds under \wasm's type system.

        \item $S;C \vdash \<drop> : \epsilon;l_1;\phi_1 \rightarrow \epsilon;l_1;\phi_1$

        $\erase{S};\erase{C} \vdash \erase{\<nop>} : \erase{\epsilon;l_1;\phi_1 \rightarrow \epsilon;l_1;\phi_1}$ = $\erase{S};\erase{C} \vdash \<wnop> : \epsilon \rightarrow \epsilon$, which holds under \wasm's type system.
    \end{itemize}
\end{proof}


\section{Type Safety}
\label{sec:typesafety}
\emph{Type safety} is the property that a well-typed program either reduces to another well-typed program, is an intentionally irreducible expression (in the case of \name, a sequence of values), or throws an error (trap, in the case of \name).
Thus, type safety assures us that the behavior of a well-typed program is always well defined.
The type safety of \wasm guarantees a number of important properties, including memory safety.
Proving the type safety of \name gives us a high degree of assurance that it has the same level of safety as \wasm.

\subsection{Subject Reduction}
\label{subsec:subject-reduction}
\emph{Subject reduction}, also sometimes referred to as ``type preservation'', ensures that if a program has a specific type, then the program will have the same type after a reduction step.
We use a number of lemmas in the subject reduction proof.
\todo{Worst transition sentence ever}

\subsection{Lemmas Used in Subject-Reduction Proof}
\begin{lemma}{(Inversion)}
    \todo{So many cases to write out}

    %% const
    If $C \vdash t.\<const> c : ti_1^{*};l_1;\phi_1 \rightarrow ti_2^{*};l_2;\phi_2$,
    then $ti_1^{*};l_1;\phi_1 \rightarrow ti_1^{*}\;\ti{t}{a};l_1;\phi_1,\ti{t}{a},(= a \; \ti{t}{c}) <: ti_1^{*};l_1;\phi_1 \rightarrow ti_2^{*};l_2;\phi_2$.

    %% binop
    If $C \vdash t.binop : ti_1^{*};l_1;\phi_1 \rightarrow ti_2^{*};l_2;\phi_2$,
    then $ti_1^{*} = ti^{*} \; \ti{t}{a_1} \; \ti{t}{a_2}$,
    and $ti_1^{*};l_1;\phi_1 \rightarrow ti^{*} \; \ti{t}{a_3};l_1;\phi_1,\ti{t}{a_3},(= a_3\;(binop\;a_1\;a_2)) <: ti_1^{*};l_1;\phi_1 \rightarrow ti_2^{*};l_2;\phi_2$.

    %% testop
    If $C \vdash t.testop : ti_1^{*};l_1;\phi_1 \rightarrow ti_2^{*};l_2;\phi_2$,
    then $ti_1^{*} = ti^{*} \; \ti{t}{a_1}$,
    and $ti_1^{*};l_1;\phi_1 \rightarrow ti^{*} \; \ti{\<ithreetwo>}{a_2};l_1;\phi_1,\ti{\<ithreetwo>}{a_2},(= a_2\;(testop\;a_1)) <: ti_1^{*};l_1;\phi_1 \rightarrow ti_2^{*};l_2;\phi_2$.

    %% relop
    If $C \vdash t.relop : ti_1^{*};l_1;\phi_1 \rightarrow ti_2^{*};l_2;\phi_2$,
    then $ti_1^{*} = ti^{*} \; \ti{t}{a_1} \; \ti{t}{a_2}$,
    and $ti_1^{*};l_1;\phi_1 \rightarrow ti^{*} \; \ti{\<ithreetwo>}{a_3};l_1;\phi_1,\ti{\<ithreetwo>}{a_3},(= a_3\;(relop\;a_1\;a_2)) <: ti_1^{*};l_1;\phi_1 \rightarrow ti_2^{*};l_2;\phi_2$.

    %% set-local
    If $C \vdash \<setlocal> i : ti_1^{*};l_1;\phi_1 \rightarrow ti_2^{*};l_2;\phi_2$,
    then $ti_1^{*} = ti_2^{*} \; \ti{t}{a}$, $l_2 = l_1 \text{ except } l_2(i) = \ti{t}{a_2}$
    and $ti_1^{*};l_1;\phi_1 \rightarrow ti_2^{*};l_2;\phi_1,\ti{t}{a_2},(= a_2\;a) <: ti_1^{*};l_1;\phi_1 \rightarrow ti_2^{*};l_2;\phi_2$,
    where $t = C_\text{local}(i)$.

    %% composition
    If $C \vdash e_1^{*} \; e_2 : ti_1^{*};l_1;\phi_1 \rightarrow ti_2^{*};l_2;\phi_2$,
    then $C \vdash e_1^{*} : ti_3^{*};l_3;\phi_3 \rightarrow ti_4^{*};l_4;\phi_4$,
    $C \vdash e_2 : ti_4^{*};l_4;\phi_4 \rightarrow ti_5^{*};l_5;\phi_5$,
    and $ti_1^{*};l_1;\phi_1 \rightarrow ti_2^{*};l_2;\phi_2 <: ti_3^{*};l_3;\phi_3 \rightarrow ti_5^{*};l_5;\phi_5$.

    %% admin-program
    If $\vdash_i s;v^{*};e^{*} : ti^{*};l;\phi$,
    then $\vdash s : S$,
    and $S;\epsilon \vdash_i v^{*};e^{*} : ti^{*};l;\phi$.

    %% admin-code
    If $S;(ti^n;l;\phi)^{?} \vdash_i s;v^{*};e^{*} : ti^n;l;\phi$,
    then $(\vdash v : \ti{t}{a};\phi_v)^{*}$,
    and $S;C \vdash e^{*} : \epsilon:\ti{t}{a}^{*};\phi_v^{*} \rightarrow ti^n;l;\phi$,
    where $C = S_{\text{inst}}(i),\text{local} \; t^{*}, \text{return} \; (ti^n;l;\phi)^{?}$.

    %% admin-const
    If $\vdash v : ti;\phi_v$,
    then $ti = \ti{t}{a}$,
    and $\phi_v = \circ,\ti{t}{a},(= a \; \ti{t}{c})$,
    where $t.\<const> c = v$.

    %% local
    If $S;C \vdash \<local> \{ i;v_l^{*} \} \; e^{*} \<end> : ti_1^{*};l_1;\phi_1 \rightarrow ti_2^{*};l_2;\phi_2$,
    then $S;(ti^n;l_3;\phi_3) \vdash_i v_l^{*};e^{*} : ti^n;l_3;\phi_3$,
    and $ti_1^{*};l_1;\phi_1 \rightarrow ti_1^{*} \; ti^n;l_1;\phi_1,\phi_3 <: ti_1^{*};l_1;\phi_1 \rightarrow ti_2^{*};l_2;\phi_2$.

\end{lemma}
\begin{proof}
    Proof omitted, but follows from induction over typing derivations.
\end{proof}

\begin{lemma}{(Coherence)}
    \todo{Nested composition can be expressed with only one subtyping relation.}
\end{lemma}
\begin{proof}
\end{proof}

\begin{lemma}{(Nested-Type-Preserved)}

    If $C \vdash v^n : \epsilon;l_3;\phi_3 \rightarrow ti^n;l_3;\phi_4$ is a subderivation of $C \vdash  L^j [v^n] : s_1;l_1;\phi_1 \rightarrow s_2;l_2;\phi_2$,
    \\then $C \vdash v^n : \epsilon;l_1;\phi_1 \rightarrow ti^n;l_3;\phi_4$ after reduction
\end{lemma}
\begin{proof}
    By induction on $j$.
    \begin{itemize}
        \item Base case: $j=0$

            $C \vdash v_0^{*} \; v^n \; e^{*} \<end> : s_1;l_1;\phi_1 \rightarrow s_2;l_2;\phi_2$ for some $v_0^{*}$ and $e^{*}$ by expanding $L^0$.

            $C \vdash (t.\<const> c)^{*} : \epsilon;l_1;\phi_0 \rightarrow \ti{t}{a}^{*};l_1;\phi_0,\ti{t}{a}^{*},(\<eq> a \; \ti{t}{c})$ where $v_0^{*}=(t.\<const> c)^{*}$ and $\phi_1 \implies \phi_0$ by $inversion$.

            Then, $\phi_0,\ti{t}{a}^{*},(\<eq> a \; \ti{t}{c}) \implies \phi_3$.

            $C \vdash v^n : \ti{t}{a}^{*};l_3;\phi_0,\ti{t}{a}^{*},(\<eq> a \; \ti{t}{c}) \rightarrow \ti{t}{a}^{*},\ti{t}{a}^{*}\;ti^n;l_3;\phi_4$, by $subtyping$.

            $C \vdash v^n : \epsilon;l_3;\phi_0,\ti{t}{a}^{*},(\<eq> a \; \ti{t}{c}) \rightarrow \ti{t}{a}^{*}\;ti^n;l_3;\phi_4$, by $inversion$.

            If $v_0^{*}$ are not executed (\ie they are not part of the reduced expression), then $a^{*}$ are fresh, so $\phi_0 \implies \phi_0,\ti{t}{a}^{*},(\<eq> a \; \ti{t}{c})$, and therefore $C \vdash v^n : \epsilon;l_1;\phi_0 \rightarrow ti^n;l_1;\phi_4$ by $subtyping$ and since $l_1=l_3$.

            Then, $C \vdash v^n : \epsilon;l_1;\phi_1 \rightarrow ti^n;l_1;\phi_4$ by $subtyping$.

        \item Induction case: $j=k+1$

            $C \vdash \<label>_n \{ e_0^{*} \} \; v_0^{*} \; L^k[v^n] \; e_1^{*} \<end> : s_1;l_1;\phi_1 \rightarrow s_2;l_2;\phi_2$ for some $v_0^{*}$, $e_0^{*}$, and $e_1^{*}$ by expanding $L^j$.

            $C \vdash (t.\<const> c)^{*} : \epsilon;l_1;\phi_0 \rightarrow \ti{t}{a}^{*};l_1;\phi_0,\ti{t}{a}^{*},(\<eq> a \; \ti{t}{c})$ where $v_0^{*}=(t.\<const> c)^{*}$ and $\phi_1 \implies \phi_0$ by $inversion$.

            Then, $C \vdash L^k[v^n] : \ti{t}{a}^{*};l_1;\phi_0,\ti{t}{a}^{*},(\<eq> a \; \ti{t}{c}) \rightarrow s_5;l_5;\phi_5$ for some $s_5;l_5;\phi_5$ by $inversion$.

            $C \vdash v^n : \epsilon;l_1;\phi_0,\ti{t}{a}^{*},(\<eq> a \; \ti{t}{c}) \rightarrow ti^n;l_1;\phi_4$ by the inductive hypothesis.

            If $v_0^{*}$ are not executed (\ie after one reduction step), $a^{*}$ are fresh, so $\phi_0 \implies \ti{t}{a}^{*},(\<eq> a \; \ti{t}{c})$, and therefore $C \vdash v^n : \epsilon;l_1;\phi_0 \rightarrow \ti{t}{a}^{*}\;ti^n;l_3;\phi_5$ by $subtyping$ and since $l_1=l_3$.

            Then, $C \vdash v^n : \epsilon;l_1;\phi_1 \rightarrow \ti{t}{a}^{*}\;ti^n;l_3;\phi_3$ by $subtyping$.

    \end{itemize}
\end{proof}


In many reduction cases, there are values on the stack that get consumed by reducing an instruction.
This creates a bit of a problem because those values represent intermediate state, and as such will introduce new index variables to the index type context in their postcondition.
After reduction, the intermediate state is no longer present, so we lose those index variables from the postconditions.

For example, $(t.\<const> c)\; \<drop>$ could be typed as $\epsilon;l;\phi \rightarrow \epsilon;l;\phi,\ti{t}{a},(= a\; \ti{t}{c})$ where $a$ represent the value on the stack $t.\<const> c$.
This would reduce to $\epsilon$, and then we lose the information about $a$ in the postcondition index type system.
However, this can be solved using implication, as we know $a$ is fresh from the const rule, and therefore $\satisfies{\phi}{l}{\phi,\ti{t}{a},(= a\; \ti{t}{c})}$ after reduction.
This type of pattern will appear in any case of the proof that consumes values.

\begin{theorem}{Subject Reduction}
  If $\vdash_i s;v^{*};e^{*} : ti^{*};l;\phi$ and $s;v^{*};e^{*} \hookrightarrow_i s';v'^{*};e'^{*}$ then $\vdash_i s';v'^{*};e'^{*} : ti^{*};l;\phi$.
\end{theorem}
\begin{proof}
By case analysis on the reduction rules.

\begin{itemize}
    %% Binop -> const
    \item $C\vdash (t.\<const> c_1)\; (t.\<const> c_2)\; t.binop : ti_1^{*};l_1;\phi_1 \rightarrow ti_2^{*};l_2;\phi_2$
    \\ $\land$ $(t.\<const> c_1)\; (t.\<const> c_2)\; t.binop \hookrightarrow t.\<const> c$ where $c=binop(c_1,c_2)$

        By $inversion$ on $const$ and $binop$, we know that $ti_2^{*} = ti_1^{*} \ti{t}{a_3}$, $l_2=l_1$, and that
        \begin{align*}
            \phi_1&,
            \begin{stackTL}
                \ti{t}{a_1}, (= a_1\; \ti{t}{c_1}), \\
                \ti{t}{a_2}, (= a_2\; \ti{t}{c_2}), \\
                \ti{t}{a_3}, (= a_3\; (binop\; a_1\; a_2))
            \end{stackTL} \\
            &\implies \phi_2
        \end{align*}

        By $const$, $C \vdash t.\<const> c :
            \begin{stackTL}
                \epsilon;l_1;\phi_1 \\
                \rightarrow \ti{t}{a_3};l_1;g_1;\phi_1,\ti{t}{a_3},(\<eq> a_3\;\ti{t}{c})
            \end{stackTL}$.

        Because $c=binop_t(c_1,c_2)$, then by $\implies$,
        \begin{align*}
            \phi_1,\ti{t}{a},(= a\; \ti{t}{c}) &\implies \phi_1,
            \begin{stackTL}
                \ti{t}{a_1}, (= a_1\; \ti{t}{c_1}), \\
                \ti{t}{a_2}, (= a_2\; \ti{t}{c_2}), \\
                \ti{t}{a_3}, (= a_3\; (binop\; a_1 a_2))
            \end{stackTL}
        \end{align*}

        Therefore, $C \vdash (t.\<const> c) : ti_1^{*};l_1;\phi_1 \rightarrow ti_1^{*}\; \ti{t}{a_3};l_1;\phi_2$, by $stack-poly$ and $sub-typing$

    %% Binop -> trap
    \item  $C\vdash (t.\<const> c_1)\; (t.\<const> c_2)\; t.binop : ti_1^{*};l_1;\phi_1 \rightarrow ti_2^{*};l_2;\phi_2$
    \\ $\land$ $(t.\<const> c_1)\; (t.\<const> c_2)\; t.binop \hookrightarrow \<trap>$

        Trivially, $C\vdash \<trap> : ti_1^{*};l_1;\phi_1 \rightarrow ti_2^{*};l_2;\phi_2$ by $trap$.

    %% Relop
    \item $C\vdash (t.\<const> c_1)\; (t.\<const> c_2)\; t.relop : ti_1^{*};l_1;\phi_1 \rightarrow ti_2^{*};l_2;\phi_2$
    \\$\land$ $(t.\<const> c_1)\; (t.\<const> c_2)\; t.relop \hookrightarrow t.\<const> c$ where $c=relop(c_1,c_2)$

        Similar to $binop$.

    %% Testop
    \item $C\vdash (t.\<const> c)\; t.testop : ti_1^{*};l_1;\phi_1 \rightarrow ti_2^{*};l_2;\phi_2$
    \\ $\land$ $(t.\<const> c)\; (t.\<const> c)\; t.testop \hookrightarrow t.\<const> c_2$ where $c_2=testop(c)$

        By $inversion$ on $const$ and $testop$, we know that $ti_2^{*}=ti_1^{*}\; \ti{t}{a_2}$, $l_2=l_1$, and that
        \begin{align*}
            \phi_1&,
            \begin{stackTL}
                \ti{t}{a_1}, (= a_1\;\ti{t}{c}), \\
                \ti{t}{a_2}, (= a_2\;(testop\;a_1))
            \end{stackTL} \\
            &\implies \phi_2
        \end{align*}

        By $const$, $C \vdash t.\<const> c :
            \begin{stackTL}
                \epsilon;l_1;g_1;\phi_1 \\
                \rightarrow \ti{t}{a_2};l_1;\phi_1,\ti{t}{a_2},(= a_2\;\ti{t}{c_2})
            \end{stackTL}$.

        Because $c_2=testop_t(c)$, then by $\implies$,
        \begin{align*}
            \phi_1,\ti{t}{a},(= a\;\ti{t}{c_2}) &\implies \phi_1,
            \begin{stackTL}
                \ti{t}{a_1}, (= a_1\;\ti{t}{c}), \\
                \ti{t}{a_2}, (= a_2\;(testop\;a_1))
            \end{stackTL}
        \end{align*}

    %% Unreachable
    \item $C\vdash \<unreachable> : ti_1^{*};l_1;\phi_1 \rightarrow ti_2^{*};l_2;\phi_2$
    \\ $\land$ $\<unreachable> \hookrightarrow \<trap>$

        Trivially, $C\vdash \<trap> : ti_1^{*};l_1;\phi_1 \rightarrow ti_2^{*};l_2;\phi_2$ by $trap$.

    %% Nop
    \item $C\vdash \<nop> : ti_1^{*};l_1;\phi_1 \rightarrow ti_2^{*};l_2;\phi_2$
    \\ $\land$ $\<nop> \hookrightarrow \epsilon$

        By $inversion$ on $nop$, we know that $ti_2^{*} = ti_1^{*}$, $l_2 = l_1$, and $\phi_1 \implies \phi_0$ and $\phi_0 \implies \phi_2$ for some $\phi_0$.

        $C\vdash \epsilon : \epsilon;l;g;\phi_0 \rightarrow \epsilon;l;g;\phi_0$ by $empty$.

        Then, $C \vdash \epsilon ti_1^{*};l;g;\phi_1 \rightarrow ti_1^{*};l;g;\phi_2$ by $stack-poly$ and $sub-typing$.

    %% Drop
    \item $C\vdash (t.\<const> c)\; \<drop> : ti_1^{*};l_1;\phi_1 \rightarrow ti_2^{*};l_2;\phi_2$
    \\ $\land$ $(t.\<const> c)\; \<drop> \hookrightarrow \epsilon$

        By $inversion$ on $compostion$, $const$, and $drop$, we know that $ti_2^{*} = ti_1^{*}$, $l_2 = l_1$, and $\phi_1 \implies \phi_0$ and $\phi_0 \implies \phi_2$ for some $\phi_0$.

        By $empty$, $C\vdash \epsilon : \epsilon;l_1;\phi_0 \rightarrow \epsilon;l_1;\phi_0$.

        Then, $C\vdash \epsilon : ti_1^{*};l_1;\phi_1 \rightarrow ti_1^{*};l_1;\phi_2$ by $stack-poly$ and $sub-typing$.

    %% Select
    \item Case: $C\; {\begin{stackTL}
        \vdash (t.\<const> c_1)\;(t.\<const> c_2)\;(\<ithreetwo>.\<const> 0)\;\<select>
        \\ : ti_1^{*};l_1;\phi_1 \rightarrow ti_2^{*};l_2;\phi_2
    \end{stackTL}}$
    \\ $\land$ $(t.\<const> c_1)\;(t.\<const> c_2)\;(\<ithreetwo>.\<const> 0)\;\<select> \hookrightarrow (t.\<const> c_2)$

        By $const$ and $select$, we know that $ti_2^{*} = ti_1^{*}\;\ti{a_3}$, $l_2 = l_1$, and
        $
        {\begin{stackTL}
            \phi_1, {\begin{stackTL}
                \ti{t}{a_1}, (= a_1\;\ti{t}{c_1}), \\
                \ti{t}{a_2}, (= a_2\;\ti{t}{c_2}), \\
                \ti{\<ithreetwo>}{a}, (= a\;\ti{\<ithreetwo>}{0}), \\
                \ti{t}{a_3},(if\; (= a\; \ti{\<ithreetwo>}{0})\; (= a_3\; a_2)\; (= a_3\; a_1))
            \end{stackTL}} \\
            \implies \phi_2
        \end{stackTL}}
        $

        By $const$, \\
        $ C \vdash (t.\<const> c_2) :
            {\begin{stackTL}
                \epsilon;l_1;\phi_1 \\
                \rightarrow \ti{t}{a_3};l_1;\phi_1,\ti{t}{a_3},(= a_3\; \ti{t}{c_2}) \\
            \end{stackTL}} $

        $C \vdash (t.\<const> c_2) : ti_1^{*};l_1;\phi_1 \rightarrow ti_1^{*}\;\ti{t}{a_3};l_1;\phi_1,\ti{t}{a_3},(= a_3 \; \ti{t}{c_2})$ by $stack-poly$.

        By $\implies$, we have \\
        $\phi_1,\ti{t}{a_3},(\<eq> a_3\; \ti{t}{c_2}) \implies \phi_1, {\begin{stackTL}
            \ti{t}{a_1}, (\<eq> a_1\; \ti{t}{c_1}), \\
            \ti{t}{a_2}, (\<eq> a_2\; \ti{t}{c_2}), \\
            \ti{\<ithreetwo>}{a}, (= a\;\ti{\<ithreetwo>}{0}), \\
            \ti{t}{a_3},(if\; (= a\; \ti{\<ithreetwo>}{0})\; (= a_3\; a_2)\; (= a_3\; a_1))
        \end{stackTL}} \\$

        Therefore,
        $ C \vdash (t.\<const> c_2) :
        ti_1^{*};l_1;\phi_1
            \rightarrow ti_2^{*}\;\ti{t}{a_3};l_1;\phi_2$ by $sub-typing$

    %% Block
    \item Case: $C \vdash v^n \; \<block> tfi \; e^{*} \<end> : ti_1^{*};l_1;\phi_1 \rightarrow ti_2^{*};l_2;\phi_2$
    \\ $\land$ $v^n \; \<block> tfi \; e^{*} \<end> \hookrightarrow \<label>_m \{ \epsilon \} \; v^n \; e^{*} \<end>$

        Let $ti_3^n;l_3;\phi_3 \rightarrow ti_4^m;l_4;\phi_4=tfi$, $(t.\<const> c)^n=v^n$.

        $C \vdash \<block> tfi \; e^{*} \<end> : ti_1^{*}\; \ti{t}{a}^n;l_1;\phi_1,\ti{t}{a}^n,(\<eq> a \; \ti{t}{c})^n \rightarrow ti_2^{*};l_2;\phi_2$ by $inversion$ on $composition$ and $const$.

        Therefore, by $inversion$ on $block$, $l_1=l_3$ and $l_2=l_4$. We will use $l_1,l_2$ in place of $l_3,l_4$, respectively, for the remainder of the proof case.

        Further, $\ti{t}{a}^n=ti_3^n$, $ti_2^{*}=ti_1^{*}\; ti_4^m$, $\phi_1,\ti{t}{a}^n,(\<eq> a \; \ti{t}{c})^n \implies \phi_3$, and $\phi_4 \implies \phi_2$ by $inversion$ on $block$.

        $C,\text{label}(t_4^{m};l_2;\phi_4) \vdash (t.\<const> c)^n : \epsilon;l_1;\phi_1 \rightarrow \\ \ti{t}{a}^n;l_1;\phi_1,\ti{t}{a}^n,(\<eq> a \; \ti{t}{c})^n$ by $const$.

        $C,\text{label}(t_4^{m};l_2;g_2;\phi_4) \vdash (t.\<const> c)^n : \epsilon;l_1;\phi_1 \rightarrow \\ \ti{t}{a}^n;l_1;\phi_3$ by $sub-typing$.

        $C,\text{label}(t_4^{m};l_2;\phi_4) \vdash e^{*} : \ti{t}{a}^n;l_1;\phi_3 \rightarrow ti_4^m;l_2;\phi_4$ because it is a sub-derivation of $block$ which we have already assumed to hold.

        Then $C,\text{label}(t_4^{m};l_2;\phi_4) \vdash (t.\<const> c)^n\; e^{*} : \epsilon;l_1;\phi_1 \rightarrow \\ ti_4^m;l_2;\phi_4$ by $composition$.

        By $empty$ and $stack-poly$, $C \vdash \epsilon : ti_2^m;l_2;\phi_4 \rightarrow ti_2^m;l_2;\phi_4$.

        Therefore, $C \vdash \<label>_m \{ \epsilon \} \; v^n \; e^{*} \<end> : \epsilon;l_1;\phi_1 \rightarrow ti_2^m;l_2;\phi_4$ by $label$.

        $C \vdash \<label>_m \{ \epsilon \} \; v^n \; e^{*} \<end> : ti_1^{*};l_1;\phi_1 \rightarrow ti_1^{*}\; ti_4^m;l_2;\phi_2$ by $stack-poly$ and $sub-typing$.

    \item Case: $C \vdash v^n \; \<loop> tfi \; e^{*} \<end> : ti_1^{*};l_1;\phi_1 \rightarrow ti_2^{*};l_2;\phi_2$
    \\ $\land$ $v^n \; \<loop> tfi \; e^{*} \<end> \hookrightarrow \<label>_n \{ \<loop> tfi \; e^{*} \<end> \} \; v^n \; e^{*} \<end>$

        Let $ti_3^n;l_3;\phi_3 \rightarrow ti_4^m;l_4;\phi_4=tfi$, $(t.\<const> c)^n=v^n$.

        $C \vdash \<loop> tfi \; e^{*} \<end> : ti_1^{*}\; \ti{t}{a}^n;l_1;\phi_1,\ti{t}{a}^n,(\<eq> a \; \ti{t}{c})^n \rightarrow ti_2^{*};l_2;g_2;\phi_2$ by $inversion$ on $composition$ and $const$.

        Therefore, by $inversion$ on $loop$, $l_1=l_3$ and $l_2=l_4$. We will use $l_1,l_2$ in place of $l_3,l_4$, respectively, for the remainder of the proof case.

        Further, $\ti{t}{a}^n=ti_3^n$, $ti_2^{*}=ti_1^{*}\; ti_4^m$, $\phi_1,\ti{t}{a}^n,(\<eq> a \; \ti{t}{c})^n \implies \phi_3$, and $\phi_4 \implies \phi_2$ by $inversion$ on $loop$.

        $C,\text{label}(t_3^{n};l_1;\phi_3) \vdash (t.\<const> c)^n : \epsilon;l_1;\phi_1 \rightarrow \\ \ti{t}{a}^n;l_1;\phi_1,\ti{t}{a}^n,(\<eq> a \; \ti{t}{c})^n$ by $const$.

        $C,\text{label}(t_3^{n};l_1;\phi_3) \vdash (t.\<const> c)^n : \epsilon;l_1;\phi_1 \rightarrow \\ \ti{t}{a}^n;l_1;\phi_3$ by $sub-typing$.

        $C,\text{label}(t_3^{n};l_1;\phi_3) \vdash e^{*} : ti_1^n;l_1;\phi_3 \rightarrow ti_2^m;l_1;\phi_4$ because it is a sub-derivation of $loop$ which we have already assumed to hold.

        Then $C,\text{label}(t_3^{n};l_1;\phi_3) \vdash (t.\<const> c)^n\; e^{*} : \epsilon;l_1;\phi_1 \rightarrow \\ ti_4^m;l_2;\phi_4$ by $composition$.

        $C \vdash \<loop> tfi \; e^{*} \<end> : \ti{t}{a}^n;l_1;\phi_1,\ti{t}{a}^n,(\<eq> a \; \ti{t}{c})^n \rightarrow ti_4^{m};l_2;g_2;\phi_4$ by $loop$.

        Therefore, $C \vdash \<label>_m \{ \<loop> tfi \; e^{*} \<end> \} \; v^n \; e^{*} \<end> : \epsilon;l_1;\phi_1 \rightarrow ti_4^m;l_2;\phi_4$ by $label$.

        $C \vdash \<label>_m \{ \epsilon \} \; v^n \; e^{*} \<end> : ti_1^{*};l_1;\phi_1 \rightarrow ti_2^{*};l_2;\phi_2$ by $stack-poly$ and $sub-typing$.

    \item Case: $C \vdash (\<ithreetwo>.\<const> 0) \; \<if> tfi \; e_1^{*} \<else> e_2^{*} \<end> : ti_1^{*};l_1;\phi_1 \rightarrow ti_2^{*};l_2;\phi_2$
    \\ $\land$ $(\<ithreetwo>.\<const> 0) \; \<if> tfi \; e_1^{*} \<else> e_2^{*} \<end> \hookrightarrow \<block> tfi \; e_2^{*} \<end>$

        Let $tfi = ti_3^n \; \ti{<ithreetwo>}{a};l_3;\phi_3 \rightarrow ti_4^m;l_4;\phi_4$, \\ $tfi_1 = ti_3^n;l_3;\phi_3,\neg(= a\; \ti{\<ithreetwo>}{0}) \rightarrow ti_4^m;l_4;\phi_4$, \\
        and $tfi_2 = ti_3^n;l_3;\phi_3,(= a\; \ti{\<ithreetwo>}{0}) \rightarrow ti_4^m;l_4;\phi_4$.

        By $inversion$ on $composition$, $const$, and $if$, $ti_1^{*}=ti_0^{*}\; ti_3^{n}$ and $ti_2^{*}=ti_0^{*} \; ti_4^{m}$ for some $ti_0^{*}$, $l_1=l_3$, $l_2=l_4$, $\phi_1,\ti{\<ithreetwo>}{a},(\<eq> a\; 0) \implies \phi_3$, and $\phi_4 \implies \phi_2$.

        $C,\text{label}(ti_4^m;l_4;\phi_4) \vdash e_2^{*} : tfi_2$ because it is a sub-derivation of $if$ which we have assumed to hold.

        Then, $C \vdash \<block> tfi_2 \; e_2^{*} \<end>$ by $block$.

        Since $a$ is fresh after reduction, $\phi_1 \implies \phi_1,\ti{t}{a},(\<eqz> a)$ by $\implies$.

        Therefore, $C \vdash \<block> tfi_2\; e_2^{*} \<end> : \\ ti_0^{*}\; ti_3^n;l_1;\phi_1,\ti{t}{a},(\<eqz> a) \rightarrow s\; ti_0^{*}\;ti_4^m;l_2;\phi_2$ by $extension$ and $sub-typing$.

    \item Case: $C \vdash (\<ithreetwo>.\<const> k+1) \; \<if> tfi \; e_1^{*} \<else> e_2^{*} \<end> : ti_1^{*};l_1;\phi_1 \rightarrow ti_2^{*};l_2;\phi_2$
    \\ $\land$ $(\<ithreetwo>.\<const> k+1) \; \<if> tfi \; e_1^{*} \<else> e_2^{*} \<end> \hookrightarrow \<block> tfi \; e_1^{*} \<end>$

        Similar to above.

    \item Case: $C \vdash \<label>_n \{ e^{*} \} \; v^n \<end> : ti_1^{*};l_1;\phi_1 \rightarrow ti_2^{*};l_2;\phi_2$
    \\ $\land$ $\<label>_n \{ e^{*} \} \; v^n \<end> \hookrightarrow v^n$

        $C \vdash \<label>_n \{ e^{*} \} \; v^n \<end> : \epsilon;l_1;\phi_1 \rightarrow ti_4^{n};l_2;\phi_2$ by $inversion$ on $label$.

        By $inversion$, we know $ti_2^{*}=ti_1^{*}\;ti_4^{n}$.

        $C \vdash v^n : \epsilon;l_1;\phi_1 \rightarrow ti_4^{n};l_2;\phi_2$ because it is a premise of $label$ which we have assumed to hold.

        Therefore, $C \vdash v^n : ti_1^{*};l_1;\phi_1 \rightarrow ti_1^{*}\;ti_4^{n};l_1;\phi_2$ by $stack-poly$.

    \item Case: $C \vdash \<label>_n \{ e^{*} \} \; \<trap> \<end> : ti_1^{*};l_1;\phi_1 \rightarrow ti_2^{*};l_2;\phi_2$
    \\ $\land$ $\<label>_n \{ e^{*} \} \; \<trap> \<end> \hookrightarrow \<trap>$

        Trivially, $C\vdash \<trap> : ti_1^{*};l_1;\phi_1 \rightarrow ti_2^{*};l_2;\phi_2$ by $trap$.

    \item Case: $C \vdash \<label>_n \{ e^{*} \} \; L^j [v^n \; (\<br> j)] \<end> : ti_1^{*};l_1;\phi_1 \rightarrow ti_2^{*};l_2;\phi_2$
    \\ $\land$ $\<label>_n \{ e^{*} \} \; L^j [v^n \; (\<br> j)] \hookrightarrow v^n \; e^{*}$

        By $inversion$, $ti_2^{*}=ti_1^{*}\;ti_4^{*}$.

        Let $(t.\<const> c)^n = v^n$.

        $C,\text{label}(ti_1^n;l_3;\phi_5)^j \vdash v^n\; (\<br> j) : \epsilon;l_3;\phi_3 \rightarrow ti_\emptyset^{*};l_\emptyset;g_\emptyset;\phi_\emptyset$ for some $l_3$ and $\phi_3$, where $\phi_5=\phi_3,\ti{t}{a}^n,(= a\; \ti{t}{c})^n$, by $inversion$ on $label$ and $br$.

        $C,\text{label}(ti_1^n;l_3;\phi_5)^j \vdash (\<br> j) : ti_1^n;l_3;\phi_5 \rightarrow ti_\emptyset^{*};l_\emptyset;g_\emptyset;\phi_\emptyset$, by $inversion$ on $composition$ and $const$.

        Then, $C,\text{label}(ti_1^n;l_3;\phi_5)^j \vdash v^n : \epsilon;l_3;\phi_3 \rightarrow ti_1^n;l_3;\phi_5$ since it is a premise of $composition$ which we have assumed to hold.

        $C \vdash e^{*} : ti_1^n;l_3;\phi_5 \rightarrow ti_2^{*};l_2;\phi_4$ since it is a premise of $label$ which we have assumed to hold, and $\phi_4 \implies \phi_2$ by $inversion$.

        Then, $C \vdash v^n \; e^{*} : \epsilon;l_1;\phi_1 \rightarrow ti_2^{*};l_2;\phi_4$ by $nested-type-preserved$ and $composition$.

        Finally, $C \vdash v^n \; e^{*} : ti_1^{*};l_1;\phi_1 \rightarrow ti_1^{*}\;ti_4^{*};l_2;\phi_2$ by $stack-poly$ and $sub-typing$.

    \item Case: $C \vdash (\<ithreetwo>.\<const> 0)\;(\<brif> j) : ti_1^{*};l_1;\phi_1 \rightarrow ti_2^{*};l_2;\phi_2$
    \\ $\land$ $(\<ithreetwo>.\<const> 0)\;(\<brif> j) \hookrightarrow \epsilon$

        $ti_1^{*}=ti_2^{*}$, $l_1=l_2$, and $\phi_1,\ti{\<ithreetwo>}{a},(\<eq> a\; \ti{\<ithreetwo>}{0}),(\<eqz> a) \implies \phi_2$ by $inversion$ on $composition$, $const$, and $br \_ if$.

        $C \vdash \epsilon : \epsilon;l_1;\phi_1 \rightarrow \epsilon;l_1;\phi_1$ by $empty$.

        $C \vdash \epsilon : ti_1^{*};l_1;\phi_1 \rightarrow ti_1^{*};l_1;\phi_1$ by $stack-poly$.

        $\phi_1 \implies \phi_1,\ti{\<ithreetwo>}{a},(\<eq> a\; \ti{\<ithreetwo>}{0}),(\<eqz> a)$ because $a$ is fresh after reduction, and therefore $\phi_1 \implies \phi_2$.

        Then, $C \vdash \epsilon : ti_1^{*};l_1;\phi_1 \rightarrow ti_1^{*};l_1;\phi_2$ by $sub-typing$.

    \item Case: $C \vdash (\<ithreetwo>.\<const> k+1)\;(\<brif> j) : ti_1^{*};l_1;\phi_1 \rightarrow ti_2^{*};l_2;\phi_2$
    \\ $\land$ $(\<ithreetwo>.\<const> k+1)\;(\<brif> j) \hookrightarrow \<br> j$

        $C_label(j)=ti_1^{*};l_1;\phi_1,\ti{t}{a},\neg(\<eqz> a)$ because it is a side condition of $br\_if$ which we have assumed to hold.

        $C \vdash \<br> j : ti_1^{*};l_1;\phi_1,\ti{t}{a},\neg(\<eqz> a) \rightarrow ti_2^{*};l_2;\phi_2$ by $br$.

        Because $a$ is fresh after reduction, $\phi_1 \implies \phi_1,\ti{\<ithreetwo>}{a},\neg(\<eqz> a)$.

        Therefore, $C \vdash \<br> j : ti_1^{*};l_1;\phi_1 \rightarrow ti_2^{*};l_2;\phi_2$ by $sub-typing$.

    \item Case: $C \vdash (\<ithreetwo>.\<const> k)\;(\<brtable> j_1^k\; j\; j_2^{*}) : ti_1^{*};l_1;\phi_1 \rightarrow ti_2^{*};l_2;\phi_2$
    \\ $\land$ $(\<ithreetwo>.\<const> k)\;(\<brtable> j_1^k\; j\; j_2^{*}) \hookrightarrow \<br> j$

        By $inversion$, we know that $C_\text{label}(j) = ti^{*};l_1;\phi_3$, $ti_1^{*} = ti_0^{*} \; ti^{*}$ for some $ti_0^{*}$, and $phi_1 \implies \phi_3$.

        $C \vdash \<br> j : ti_1^{*};l_1;\phi_3 \rightarrow ti_2^{*};l_2;\phi_2$ by $br$.

        $C \vdash \<br> j : ti_1^{*};l_1;\phi_1 \rightarrow ti_2^{*};l_2;\phi_2$ by $sub-typing$.

    \item Case: $C \vdash (\<ithreetwo>.\<const> k+n)\;(\<brtable> j_1^k\; j) : ti_1^{*};l_1;\phi_1 \rightarrow ti_2^{*};l_2;\phi_2$
    \\ $\land$ $(\<ithreetwo>.\<const> k+n)\;(\<brtable> j_1^k\; j) \hookrightarrow \<br> j$

        By $inversion$, we know that $C_\text{label}(j) = ti^{*};l_1;\phi_3$, $ti_1^{*} = ti_0^{*} \; ti^{*}$ for some $ti_0^{*}$, and $phi_1 \implies \phi_3$.

        $C \vdash \<br> j : ti_1^{*};l_1;\phi_3 \rightarrow ti_2^{*};l_2;\phi_2$ by $br$.

        $C \vdash \<br> j : ti_1^{*};l_1;\phi_1 \rightarrow ti_2^{*};l_2;\phi_2$ by $sub-typing$.

    \item Case: $S;C \vdash \<call> j : ti_1^{*};l_1;\phi_1 \rightarrow ti_2^{*};l_2;\phi_2$
    \\ $\land$ $s;\<call> j \hookrightarrow_i \<call> s_\text{func}(i,j)$

        By $inversion$, we know that $l_2 = l_1$, $ti_1^{*} = ti^{*} \; ti_3^{*}$, $ti_1^{*} = ti^{*} \; ti_4^{*}$, $\phi_1 \implies \phi_3$, and $\phi_3,\phi_4 \implies \phi_2$, where $ti_3^{*};l_3;\phi_3 \rightarrow ti_4^{*};l_4;\phi_4 = C_\text{func}(j)$.

        We know $S \vdash s_\text{inst}(i) : C$ since it is a premise of $\vdash s : S$ which we have assumed to hold.

        Then we know $S \vdash s_\text{func}(i,j) : ti_3^{*};l_3;\phi_3 \rightarrow ti_4^{*};l_4;\phi_4$ because it is a premise of $S \vdash s_\text{inst}(i) : C$.

        Therefore, $S;C\vdash \<call> s_\text{func}(i,j) : ti_3^{*};l_3;\phi_3 \rightarrow ti_4^{*};l_4;\phi_4$ by $call-cl$.

        $S;C\vdash \<call> s_\text{func}(i,j) : ti_1^{*};l_1;\phi_1 \rightarrow ti_2^{*};l_2;\phi_2$ by $stack-poly$ and $sub-typing$.

    \item Case: $S;C \vdash_i (\<ithreetwo>.\<const> j)\; \<callindirect> tfi : ti_1^{*};l_1;\phi_1 \rightarrow ti_2^{*};l_2;\phi_2$
    \\ $\land$ $s;(\<ithreetwo>.\<const> j)\; \<callindirect> tfi \hookrightarrow_i \<call> s_\text{tab}(i,j)$ where $s_\text{tab}(i,j)_\text{code}=(\<func> tfi_0\; \<local>\; t^{*}\; e^{*})$ and $tfi_0 <: tfi$

        Let $ti_3^{*};l_3;\phi_3 \rightarrow ti_4^{*};l_4;\phi_4 = tfi$

        By $inversion$ on $composition$, $const$, and $call-indirect$, we know that $ti_1^{*}=ti_0^{*}\; ti_3^{*}$ and $ti_2^{*}=ti_0^{*}\; ti_4^{*}$ for some $ti_0^{*}$, $l_1=l_3$, $l_2=l_4$, $\phi_1 \implies \phi_3$, and $\phi_4 \implies \phi_2$.

        $S \vdash s_\text{tab}(i,j) : tfi_0$ since it is a premise of $\vdash s : S$ which we have assumed to hold.

        Then, $S;C \vdash_i \<call> s_\text{tab}(i,j) : tfi_0$ by $call-cl$.

        $S;C \vdash_i \<call> s_\text{tab}(i,j) : tfi$ by $sub-typing$.

        Therefore, $S;C \vdash_i \<call> s_\text{tab}(i,j) : ti_0^{*}\;ti_1^{*};l_1;\phi_1 \rightarrow ti_0^{*}\;ti_1^{*};l_2;\phi_2$ by $stack-poly$.

    \item Case: $S;C \vdash_i (\<ithreetwo>.\<const> j)\; \<callindirect> tfi : ti_1^{*};l_1;\phi_1 \rightarrow ti_2^{*};l_2;\phi_2$
    \\ $\land$ $s;(\<ithreetwo>.\<const> j)\; \<callindirect> tfi \hookrightarrow_i \<trap>$.

        Trivially, $S;C \vdash_i \<trap> : ti_1^{*};l_1;\phi_1 \rightarrow ti_2^{*};l_2;\phi_2$ by $trap$.

    \item Case: $S;C \vdash_i v^n\; \<call> cl : ti_1^{*};l_1;\phi_1 \rightarrow ti_2^{*};l_2;\phi_2$
    \\ $\land$ $s;v^n\; \<call> cl \hookrightarrow_i \<local>_m\{j;v^n \; (t.\<const> 0)^k\} \; \<block> tfi_1\; e^{*} \<end> \<end>$
    \\ where $cl_\text{code} = \<func> tfi_2\; \<local>\; t^k \; e^{*}$ and $cl_\text{inst} = j$

        \todo{This needs an overhaul}

        Let $tfi_0 = ti_1^{*};l_1;\phi_1 \rightarrow ti_2^{*};l_2;\phi_2$, $tfi_1 = \epsilon;l_3;\phi_3 \rightarrow ti_4^{m};l_4;\phi_4$, and $tfi_2 = ti_3^{n};\ti{t_2}{a_2}^{n}\; \ti{t}{a}^k;\phi_3,\ti{t}{a}^k,(= a \;\ti{t}{0})^k \rightarrow ti_4^{m};l_4;\phi_4$ by $inversion$.

        $S;C \vdash (t_2 \<const> c)^n : ti_1^{*};l_1;\phi_1 \rightarrow ti_1^{*}\;ti_5^n;l_1;\phi_1,\ti{t_2}{a_2},(= a_2 \; \ti{t_2}{c})$, where $v^n=(t_2 \<const> c)^n$,
        and $S;C\vdash \<call> cl : ti_1^{*}\;ti_5^n;l_1;\phi_1,\ti{t_2}{a_2}^{n},(= a_2 \; \ti{t_2}{c})^{n} \rightarrow ti^{*}\;ti_2^m;l_2;\phi_2$ because they are premises of $composition$ which we have assumed to hold.

        By inversion, $l_2=l_1$, $ti_2^{*}=ti_1^{*}\;ti_4^m$, $\phi_1,\ti{t_2}{a_2},(\<eq> a_2 \; \ti{t_2}{c}) \implies \phi_3$, $\phi_4 \implies \phi_2$, and $S\vdash cl : tfi_1$.

        Therefore, $C \vdash \<func> tfi_1\; \<local>\; t^k \; e^{*} : tfi_1$ because it is a premise of $S \vdash cl : tfi_1$.

        $S;C,\text{local } t_2^n\; t^k,\text{label }(ti_4^{m};l_4;g_4;\phi_4),\text{return }(ti_4^{m};l_4;g_4;\phi_4) \vdash e^{*}: tfi_2$ because it is a premise of the above derivation.

        $S;C,\text{local } t_2^n\; t^k,\text{return }(ti_4^{m};l_4;g_4;\phi_4) \vdash \<block> tfi_2\; e^{*} \<end> : tfi_2$ by $block$.

        There are now two cases, based on whether or not the called closure was in the current module being called:

        \begin{itemize}
            \item Case: $i=j$
                By $inversion$, $g_1=g_3$ and $g_2=g_4$, so we will use $g_1$ instead of $g_3$ and $g_2$ instead of $g_4$.

                $C \vdash_j v^n \; (t \<const> 0)^k : \epsilon;l_3;g_1;\phi_1 \rightarrow \ti{t_2}{a_2}^n\;\ti{t}{a}^k ;l_3;g_1;\phi_1,\ti{t_2}{a_2},(\<eq> a_2 \; \ti{t_2}{c})^n,\ti{t}{a},(\<eq> a \; \ti{t}{0})^k$ by $const$.

                $\phi_1,\ti{t_2}{a_2},(\<eq> a_2 \; \ti{t_2}{c})^n \implies \phi_3$, and therefore $\phi_1,\ti{t_2}{a_2},(\<eq> a_2 \; \ti{t_2}{c})^n,\ti{t}{a},(\<eq> a \; \ti{t}{0})^k \implies \phi_3,\ti{t}{a}^k,(\<eq> a \;\ti{t}{0})^k$.

                $S;C,\text{local } t_2^n\; t^k,\text{return }(ti_4^{m};l_4;g_2;\phi_4) \vdash \<block> tfi_2\; e^{*} \<end> :  ti_3^{n};\ti{t_2}{a_2}^{n}\; \ti{t}{a}^k;g_1;\phi_1,\ti{t_2}{a_2},(\<eq> a_2 \; \ti{t_2}{c})^n,\ti{t}{a},(\<eq> a \; \ti{t}{0})^k \rightarrow ti_4^{m};l_4;g_2;\phi_4$ by $sub-typing$.

                $S;(ti_4^{m};l_4;g_2;\phi_4) \vdash_j v^n \; (t \<const> 0)^k;\<block> tfi_2\; e^{*} \<end> : \epsilon;l_3;g_1;\phi_1 \rightarrow ti_4^{m};l_4;g_2;\phi_4$ by $with-return$.

                $S;C \vdash_i \<local>_m\{j;v^n \; (t.\<const> 0)^k\} \; \<block> tfi_2\; e^{*} \<end> \<end> : \epsilon;l_1;g_1;\phi_1 \rightarrow \epsilon\;ti_4^m;l_1;g_2;\phi_4$ by $local-same-inst$.

                $S;C \vdash_i \<local>_m\{j;v^n \; (t.\<const> 0)^k\} \; \<block> tfi_2\; e^{*} \<end> \<end> : tfi_0$ by $stack-poly$ and $sub-typing$.

            \item Case: $i \neq j$
                By $inversion$, $g_2=\ti{t_g}{a_3}^{*}$ where $C_\text{global}=(mut?\; t_g)^{*}$ and $a_3^{*}$ are fresh.

                $C \vdash_j v^n \; (t \<const> 0)^k : \epsilon;l_3;g_3;\phi_1 \rightarrow \ti{t_2}{a_2}^n\;\ti{t}{a}^k ;l_3;g_3;\phi_1,\ti{t_2}{a_2},(\<eq> a_2 \; \ti{t_2}{c})^n,\ti{t}{a},(\<eq> a \; \ti{t}{0})^k$ by $const$

                $\phi_1,\ti{t_2}{a_2},(\<eq> a_2 \; \ti{t_2}{c})^n \implies \phi_3$, and therefore $\phi_1,\ti{t_2}{a_2},(\<eq> a_2 \; \ti{t_2}{c})^n,\ti{t}{a},(\<eq> a \; \ti{t}{0})^k \implies \phi_3,\ti{t}{a}^k,(\<eq> a \;\ti{t}{0})^k$.

                $S;C,\text{local } t_2^n\; t^k,\text{return }(ti_4^{m};l_4;g_4;\phi_4) \vdash \<block> tfi_2\; e^{*} \<end> :  ti_3^{n};\ti{t_2}{a_2}^{n}\; \ti{t}{a}^k;g_3;\phi_1,\ti{t_2}{a_2},(\<eq> a_2 \; \ti{t_2}{c})^n,\ti{t}{a},(\<eq> a \; \ti{t}{0})^k \rightarrow ti_4^{m};l_4;g_4;\phi_4$ by $sub-typing$.

                $S;(ti_4^{m};l_4;g_4;\phi_4) \vdash_j v^n \; (t \<const> 0)^k;\<block> tfi_2\; e^{*} \<end> : \epsilon;l_3;g_3;\phi_1 \rightarrow ti_4^{m};l_4;g_4;\phi_4$ by $with-return$.

                $S;C \vdash_i \<local>_m\{j;v^n \; (t.\<const> 0)^k\} \; \<block> tfi_2\; e^{*} \<end> \<end> : \epsilon;l_1;g_1;\phi_1 \rightarrow \epsilon\;ti_4^m;l_1;\ti{t_g}{a_3}^{*};\phi_4$ by $local-diff-inst$.

                $S;C \vdash_i \<local>_m\{j;v^n \; (t.\<const> 0)^k\} \; \<block> tfi_2\; e^{*} \<end> \<end> : tfi_0$ by $stack-poly$ and $sub-typing$.
        \end{itemize}

    \item Case: $S;C \vdash \<local>_n \{ i;v_l^{*} \} \; v^n \<end> : ti_1^{*};l_1;\phi_1 \rightarrow ti_2^{*};l_2;\phi_2$
    \\ $\land$ $\<local>_n \{ i;v^{*}_l \} \; v^n \<end> \hookrightarrow_j \; v^n$

        By $inversion$ on $local$, $ti_2^{*} = ti_1^{*} \; ti^n$, $l_1 = l_2$,
        $S;(ti^n;l_3;\phi_3) \vdash_i v_l^{*};v^n : ti^n;l_3;\phi_3$,
        and $\satisfies{\phi_1,\phi_3}{ti_2^{*}\;l_2}{\phi_2}$.

        By $inversion$ on $admin-code$, $(\vdash v_l : ti_l;\phi_l)^{*}$,
        and $S;C_l \vdash v^n : \epsilon:ti_l^{*};\phi_l^{*} \rightarrow ti^n;l_3;\phi_3$.

        By $inversion$ on $admin-const$, $\phi_l^{*} = \circ,\ti{t}{a}^{*},(= a \ti{t}{c})^{*}$.

        By $inversion$ on $const$, $\satisfies{\phi_l^{*},\phi_v}{ti^n\;l_3}{\phi_3}$.

        Since $l_2 \not\in \phi_3$; $\satisfies{\phi_l^{*},\phi_v}{ti^n\;l_2}{\phi_3}$.

        Since $a^{*} \not\in \phi_v,ti^n,l_2$; $\satisfies{\phi_v}{ti^n\;l_2}{\phi_l^{*},\phi_v}$.

        $S;C \vdash v^n : \epsilon;l_1;\phi_1 \rightarrow ti^n;l_2;\phi_1,\phi_v$ by $const$.

        $S;C \vdash v^n : \epsilon;l_1;\phi_1 \rightarrow ti^n;l_2;\phi_1,\phi_3$ by $subtyping$.

        $S;C \vdash v^n : \epsilon;l_1;\phi_1 \rightarrow ti^n;l_2;\phi_2$ by $subtyping$.

        Therefore $S;C \vdash v^n : ti_1^{*};l_1;\phi_1 \rightarrow ti_2^{*};l_2;\phi_2$ by $stack-poly$.

    \item Case: $S;C \vdash \<local>_n \{ i;v_l^{*} \} \; \<trap> \<end> : ti_1^{*};l_1;\phi_1 \rightarrow ti_2^{*};l_2;\phi_2$
    \\ $\land$ $\<local>_n \{ i;v_l^{*} \} \; \<trap> \<end> \hookrightarrow \; \<trap>$

        Trivially, $S;C \vdash \<trap> : ti_1^{*};l_1;\phi_1 \rightarrow ti_2^{*};l_2;\phi_2$ by $trap$.

    \item Case: $S;C \vdash \<local>_n \{ i;v_l^{*} \} \; L^k[v^n \; \<return>] \<end> : ti_1^{*};l_1;\phi_1 \rightarrow ti_2^{*};l_2;\phi_2$
    \\ $\land$ $\<local>_n \{ i;v_l^{*} \} \; L^k[v^n \; \<return>] \<end> \hookrightarrow_j \; v^n$

        By $inversion$, $ti_2^{*} = ti_1^{*} \; ti^n$, $\phi_1,\phi_3 \implies \phi_2$, and $l_1 = l_2$.

        $S;(ti^n;l_3;\phi_3) \vdash_i L^k[v^n \; \<return>] : ti^n;l_3;\phi_3$ because it is a premise of $local$ which we have assumed to hold.

        $(\vdash v_l : ti_l;\phi_l)^{*}$, and\\
        $S;C_l \vdash L^k[v^n \; \<return>] : \epsilon;ti_l^{*};\phi_l^{*} \rightarrow ti^n;l_3;\phi_3$, where\\
        $C_l = S_{\text{inst}}(i),\text{local} \; t^{*}, \text{return} \; (ti^n;l_3;\phi_3)$ because they are premises of $admin-code$ which we have assumed to hold.

        $S;C_l \vdash \<return> : ti_3^{*} \; ti^n;l_3;\phi_3 \rightarrow ti_4^{*};l_4;\phi_4$ by $return$.

        $S;C_l \vdash v^n : \epsilon;l_3;\phi_5 \rightarrow ti^n;l_3;\phi_6$, where $\phi_6 \implies \phi_3$ by $inversion$ on $const$, $composition$, and $subtyping$.

        $S;C_l \vdash v^n : \epsilon;l_3;\phi_l^{*} \rightarrow ti^n;l_3;\phi_6$ by (Nested-Type-Preserved).

        $\phi_l^{*} = \circ,\ti{t}{a}^{*},(\<eq> a \ti{t}{c})^{*}$ by $admin-const$.

        By $inversion$ on $const$, $\phi_6 = \phi_l^{*},\phi_v$.

        \thought{Don't have to account for subtyping here because $\phi_6$ is an existential claim.}

        Because $a^{*}$ are fresh, $\phi_v \implies \phi_l^{*},\phi_v$.

        \thought{Uses the same idea from the values case.}

        $S;C \vdash v^n : \epsilon;l_1;\phi_1 \rightarrow ti^n;l_2;\phi_1,\phi_v$ by $const$.

        $S;C \vdash v^n : \epsilon;l_1;\phi_1 \rightarrow ti^n;l_2;\phi_1,\phi_3$ by $subtyping$.

        $S;C \vdash v^n : \epsilon;l_1;\phi_1 \rightarrow ti^n;l_2;\phi_2$ by $subtyping$.

        Therefore $S;C \vdash v^n : ti_1^{*};l_1;\phi_1 \rightarrow ti_2^{*};l_2;\phi_2$ by $stack-poly$.

    \item Case: $C \vdash \<getlocal> j : ti_1^{*};l_1;\phi_1 \rightarrow ti_2^{*};l_2;\phi_2$
    \\ $\land$ $v_1^j \; v \; v_2^k;\<getlocal> j \hookrightarrow v$

        By $inversion$ on $get-local$, $ti_2^{*} = \ti{t}{a_1} \; ti_1^{*}$, $l_2 = l_1$, and $\phi_1,\ti{t}{a_1},(= a_1 \; a) \implies \phi_2$, where $\ti{t}{a} = l_1(j)$.

        By $inversion$ on $admin-code$ and $admin-const$, $\phi_1 = \circ,\phi_{v_1}^j, \ti{t}{a}, (= a \; \ti{t}{c}), \phi_{v_2}^k$, where $v = (t.\<const> c)$.

        $C \vdash v : \epsilon;l_1;\phi_1 \rightarrow \ti{t}{a_1};l_2;\phi_1,\ti{t}{a_1},(= a_1 \; \ti{t}{c})$ by $const$.

        $\phi_1,\ti{t}{a_1},(= a_1 \; \ti{t}{c}) \iff \phi_1,\ti{t}{a_1},(= a_1 \; a)$ because $\phi_1$ contains the constraint $(= a \; \ti{t}{c})$.

        $C \vdash v : \epsilon;l_1;\phi_1 \rightarrow \ti{t}{a_1};l_2;\phi_2$ by $subtyping$.

        Therefore, $C \vdash v : ti_1^{*};l_1;\phi_1 \rightarrow ti_2^{*};l_2;\phi_2$ by $stack-poly$.

    \item Case: $S;\epsilon \vdash_i v_1^j \; v \; v_2^k; v' \; (\<setlocal> j) : ti^{*};l;\phi$
    \\ $\land$ $v_1^j \; v \; v_2^k;v' \; \<setlocal> j \hookrightarrow v_1^j \; v' \; v_2^k;\epsilon$

        By $inversion$ on $admin-code$, $\vdash v : \ti{t}{a};\phi_v$,\\
        $S;C \vdash v' \; (\<setlocal> j) : \epsilon:l_1;\phi_1^j,\phi_v,\phi_2^k \rightarrow ti^{*};l;\phi$,\\
        $l_1(j) = \ti{t}{a}$, and $C_\text{local}(j) = t$.

        By $inversion$ on $admin-const$, $t.\<const> c = v$, and\\
        $\phi_v = \circ,\ti{t}{a},(= a \; \ti{t}{c})$.

        By $inversion$ on $composition$,
        $C \vdash v' : \epsilon;l_1;\phi_1^j,\phi_v,\phi_2^k \rightarrow ti_3^{*};l_3;\phi_3$,
        $C \vdash \<setlocal> j : ti_3^{*};l_3;\phi_3 \rightarrow ti^{*};l;\phi_4$,
        and $\satisfies{\phi_4}{ti^{*}\;l}{\phi}$.

        Recalling that $t = C_\text{local}(j)$;
        by $inversion$ on $set-local$,
        $ti_3^{*} = ti^{*} \; \ti{t}{a'}$,
        $l = l_3 \text{ except } l(j) = \ti{t}{a_l}$,
        and $\satisfies{\phi_3,\ti{t}{a_l},(= a_l\;a')}{ti^{*}\;l}{\phi_4}$.

        By $inversion$ on $const$,
        $t.\<const> c' = v'$, $ti^{*} = \epsilon$, $l_1 = l_3$, and\\
        $\satisfies{\phi_1^j,\phi_v,\phi_2^k,\ti{t}{a'},(= a' \; \ti{t}{c'})}{\ti{t}{a'}\;l_1}{\phi_3}$.

        $C \vdash \epsilon : \epsilon;l;\phi \rightarrow \epsilon;l;\phi$ by $empty$.

        $C \vdash \epsilon : \epsilon;l;\phi_3,\ti{t}{a_l},(= a_l\;a') \rightarrow \epsilon;l;\phi$ by $subtyping$.

        Since $a_l \not\in \phi_3$; $\satisfies{\phi_1^j,\phi_v,\phi_2^k,\ti{t}{a'},(= a' \; \ti{t}{c'})}{l}{\phi_3}$.

        Since $a \not\in \phi_1^j,\phi_2^k,\ti{t}{a'},(= a' \; \ti{t}{c'}),l$;\\
        $\satisfies{\phi_1^j,\phi_2^k,\ti{t}{a'},(= a' \; \ti{t}{c'})}{l}{\phi_1^j,\phi_v,\phi_2^k,\ti{t}{a'},(= a' \; \ti{t}{c'})}$.

        $C \vdash \epsilon : \epsilon;l;\phi_1^j,\phi_2^k,\ti{t}{a'},(= a' \; \ti{t}{c'}),\ti{t}{a_l},(= a_l\;a') \rightarrow \epsilon;l;\phi$ by $subtyping$.

        Trivially, $\ti{t}{a'},(= a' \; \ti{t}{c'}),\ti{t}{a_l},(= a_l\;a') \iff \ti{t}{a'},(= a' \; \ti{t}{c'}),\ti{t}{a_l},(= a_l \; \ti{t}{c'})$.

        Since $a' \not\in \phi_1^j,\phi_2^k,\ti{t}{a_l},(= a_l \; \ti{t}{c'}),l$;\\
        $\satisfies{\phi_1^j,\phi_2^k,\ti{t}{a_l},(= a_l \; \ti{t}{c'})}{l}{\phi_1^j,\phi_2^k,\ti{t}{a'},(= a' \; \ti{t}{c'}),\ti{t}{a_l},(= a_l\;a')}$.

        $C \vdash \epsilon : \epsilon;l;\phi_1^j,\ti{t}{a_l},(= a_l \; \ti{t}{c'}),\phi_2^k \rightarrow \epsilon;l;\phi$ by $subtyping$.

        $\vdash v' : \ti{t}{a_l};\circ,\ti{t}{a_l},(= a_l \; \ti{t}{c'})$ by $admin-const$.

        Therefore, $S;\epsilon \vdash_i v_1^j\;v'\;v_2^k;\epsilon : ti^n;l;\phi$.

        \thought{I set out to write a very verbose proof case, but I didn't expect it to be this verbose.}

    \item Case: $C \vdash \<getglobal> j : ti_1^{*};l_1;\phi_1 \rightarrow ti_2^{*};l_2;\phi_2$
    $\land$ $\vdash s : S$
    \\ $\land$ $s;\<getglobal> j \hookrightarrow_i s_\text{glob}(i,j)$

        By $inversion$, $ti_2^{*} = ti_1^{*} \; \ti{t}{a}$, $l_2 = l_1$, and $\phi_1,\ti{t}{a} \implies \phi_2$, where $C_\text{global}(j) = \text{mut}^{?} \; t$.

        $\vdash s_\text{glob}(i,j) : \ti{t}{a_2};\phi_v$ because it is a premise of $S \vdash s_\text{inst}(i) : C$, which is a sub-derivation of $\vdash s : S$.

        $C \vdash s_\text{glob}(i,j) : \epsilon;l_1;\phi_1 \rightarrow \ti{t}{a};l_2;\phi_1,\ti{t}{a},(= a \; \ti{t}{c})$ by $const$.

        $C \vdash s_\text{glob}(i,j) : \epsilon;l_1;\phi_1 \rightarrow \ti{t}{a};l_2;\phi_2$ by $subtyping$.

        Therefore, $C \vdash s_\text{glob}(i,j) : ti_1^{*};l_1;\phi_1 \rightarrow ti_2^{*};l_2;\phi_2$ by $stack-poly$.

    \item Case: $C \vdash v \; (\<setglobal> j) : ti_1^{*};l_1;\phi_1 \rightarrow ti_2^{*};l_2;\phi_2$
    $\land$ $\vdash s : S$
    \\ $\land$ $s;v \; (\<setglobal> j) \hookrightarrow_i s';\epsilon$, where $s' = s$ with $\text{glob}(i,j) = v$

        By $inversion$, $ti_2^{*} = ti_1^{*}$, $l_2 = l_1$, and $\phi_1,\ti{t}{a},(= a \; \ti{t}{c}) \implies \phi_2$, where $(t.\<const> c) = v$.

        Because $a$ is fresh in $ti_2^{*}$, $l_2$, and $\phi_1$, $\phi_1 \implies \phi_1,\ti{t}{a},(= a \; \ti{t}{c})$.

        $C \vdash \epsilon : \epsilon;l_1;\phi_1 \rightarrow \epsilon;l_2;\phi_1$ by $empty$.

        $C \vdash \epsilon : \epsilon;l_1;\phi_1 \rightarrow \epsilon;l_2;\phi_2$ by $subtyping$.

        Therefore, $C \vdash \epsilon : ti_1^{*};l_1;\phi_1 \rightarrow ti_2^{*};l_2;\phi_2$ by $stack-poly$.

        $C_\text{global}(j) = \text{mut} \; t$ because it is a premise of $set-global$ which we have assumed to hold.

        $\vdash v_0 : \ti{t}{a_0};\phi_0$, where $v_0 = s_\text{glob}(i,j)$ because it is a premise of $S \vdash s_\text{inst}(i) : C$, which is a sub-derivation of $\vdash s : S$.

        $\vdash v : \ti{t}{a_1};\phi_v$ by $admin-const$.

        $S \vdash s'_\text{inst}(i) : C$ by $instance$, note that the other premises are the same as the premises from $S \vdash s_\text{inst}(i) : C$.

        \thought{Maybe I should explicitly invoke the other premises?}

        Therefore, $\vdash s' : S$ by $store$, note that the other premises are the same as the premises from $\vdash s : S$.

        \thought{Ditto here.}

\end{itemize}
\end{proof}

\subsection{Progress}
\label{subsec:progress}
\emph{Progress} ensures that if a program is well-typed then it either: entirely consists of values, traps, or is reducible (\ie there exists another program that it reduces to).
Proving progress for \name is the key metatheoretic property that ensures that our claim that \name is as safe as \wasm is valid.
This is because it connects the static guarantees of the type system to the dynamic assumptions of \prechk-tagged instructions.
By proving that well-typed \prechk-tagged instructions will always be reducible, we prove that the static guarantees are sufficient to ensure that they will not trap and therefore the dynamic checks are unnecessary.

Since most \name instructions have the same semantics as in \wasm, and every \name type includes all the information of a \wasm type, we can reuse the \wasm proof for those instructions by using the erasure function from Section \ref{subsec:erasure}.
The intuition for this is that the \name indexed type system provides strictly more information than the \wasm type system.
However, for \name instructions that do not have the same semantics as in \wasm, specifically \prechk-tagged instructions, we still must prove those cases.

\begin{theorem}{Progress}
    If $\vdash_i s;v^{*};e^{*} : ti^{*};l;\phi$ then either $e^{*} = v'^{*}$, $e^{*}= \<trap>$, or $s;v^{*};e^{*} \hookrightarrow_i s';v'^{*};e'^{*}$.
\end{theorem}
\begin{proof}
    We proceed by induction on $\vdash_i s;v^{*};e^{*} : ti^{*};l;\phi$.

    Because $\vdash_i s;v^{*};e^{*} : ti^{*};l;\phi$, we know that $\vdash s : S$ for some $S$, and that $S; \epsilon \vdash_i v^{*};e^{*}:ti^{*};l;\phi$ because they are premises of $admin-program$ which we have assumed to hold.

    Then we know that $(\vdash v: \ti{t_v}{a_v};\phi_v)^{*}$ and $S;S_\text{inst}(i),\text{local } t_v^{*} \vdash e^{*} : \epsilon;\ti{t_v}{a_v}^{*};\phi_v^{*} \rightarrow ti^{*};l;\phi$ because they are premises of $admin-code$ which we have assumed to hold.

    \begin{itemize}
        \item Case: $\vdash_i s;v^{*};(t.\<const> c_1)\;(t.\<const> c_2)\;t.\<divpc>$

        We must show that $(t.\<const> c_1)\;(t.\<const> c_2)\;t.\<divpc> \hookrightarrow e'^{*}$ for some $e'^{*}$.

        We have $S;() \vdash_i v^{*};(t.\<const> c_1)\;(t.\<const> c_2)\;t.\<divpc> : ti^{*};l;\phi$ for some $ti^{*}$, $l$, and $\phi$ because it is a premise of \refrule{Program} which we have assumed to hold.

        Then, $(\vdash v : \ti{t_v}{a_v};\phi_v)^{*}$ for some $\ti{t_v}{a_v}^{*}$ and $\phi_v^{*}$, since it is a premise of \refrule{Code} which we have assumed to hold.

        It is important to note that $\phi_v^{*}$ cannot contain a contradiction because it contains a single equality constraint per fresh index variable (see \refrule{Admin-Const}).

        Further,
        $$S;S_\text{inst}(i),\text{local } t_v^{*}\;
        {\begin{stackTL}
            \vdash (t.\<const> c_1)\;(t.\<const> c_2)\;t.\<divpc>
            \\: \epsilon;\ti{t_v}{a_v}^{*});\phi_v^{*} \rightarrow ti^{*};l;\phi
        \end{stackTL}}$$
        because it too is a premise of \refrule{Code}.

        Then,
        $$S_\text{inst}(i)
        {\begin{stackTL}
            \vdash (t.\<const> c_1)\;(t.\<const> c_2)
            \\ : \epsilon;\ti{t_v}{a_v}^{*});\phi_v^{*}
            \\ \;\; \rightarrow \ti{t}{a_1}\;\ti{t}{a_2};\ti{t_v}{a_v}^{*});\phi_v^{*},
            {\begin{stackTL}
                \ti{t}{a_1},(= a_1\; \ti{t}{c_1}),
                \\ \ti{t}{a_2},(= a_2\; \ti{t}{c_2})
            \end{stackTL}}
        \end{stackTL}}$$
        where $\\phi_v^{*},\ti{t}{a_1},(= a_1\; \ti{t}{c_1}),\ti{t}{a_2},(= a_2\; \ti{t}{c_2}) \implies \neg(= a_2\; \ti{t}{0})$ by \reflemma{Inversion} on \refrule{Composition} and \refrule{Div-Prechk}.

        If $c_2=0$, then
        $$\phi_v^{*},\ti{t}{a_1},(= a_1\; \ti{t}{c_1}),\ti{t}{a_2},(= a_2\; 0) \implies \neg(= a_2\; \ti{t}{0})$$
        which is a contradiction since $a_2$ is fresh and therefore cannot be otherwise be constrained in $\phi_v^{*}$

        Therefore, it must be the case that $c_2\neq 0$, and therefore there must exist some $c_3$ such that $c_3=div(c_1,c_2)$ since $div(c_1,c_2)$ is well-defined when $c_2$ is non-zero.
        Then, $s;(t.\<const> c_1)\;(t.\<const> c_2)\;t.\<divpc> \hookrightarrow_i (t.\<const> c_3)$.

        \item Case: $\vdash_i s;v^{*};(t.\<const> c)\;t.\<loadpc> (tp\_sx)\; align\;o$

        We must show that $s;(t.\<const> c)\;t.\<loadpc> (tp\_sx)\; align\;o \hookrightarrow e'^{*}$ for some $e'^{*}$.

        We have $S;() \vdash_i v^{*};(t.\<const> c)\;t.\<loadpc> (tp\_sx)\; align\;o : ti^{*};l;\phi$ for some $ti^{*}$, $l$, and $\phi$ because it is a premise of \refrule{Program} which we have assumed to hold.

        We also have that $\vdash s : S$, and therefore $(n \leq |b^{*}|)^{*}$ where $S_\text{tab}=n^{*}$ and $s_\text{mem}=(b^{*})^{*}$.

        Then, $(\vdash v : \ti{t_v}{a_v};\phi_v)^{*}$ for some $\ti{t_v}{a_v}^{*}$ and $\phi_v^{*}$, since it is a premise of \refrule{Code} which we have assumed to hold.

        It is important to note that $\phi_v^{*}$ cannot contain a contradiction because it contains a single equality constraint per fresh index variable (see \refrule{Admin-Const}).

        Further, we have that
        $$S;S_\text{inst}(i),\text{local } t_v^{*}\;
        {\begin{stackTL}
            \vdash (t.\<const> c)\;t.\<loadpc> (tp\_sx)\; align\;o
            \\ : \epsilon;\ti{t_v}{a_v}^{*});\phi_v^{*} \rightarrow ti^{*};l;\phi
        \end{stackTL}}$$
        because it too is a premise of \refrule{Code}.

        Then,
        $$S_\text{inst}(i) \vdash (t.\<const> c) :
        {\begin{stackTL}
            \epsilon;\ti{t_v}{a_v}^{*});\phi_v^{*}
            \\ \rightarrow \ti{t}{a}\;\ti{t_v}{a_v}^{*});\phi_v^{*}, \ti{t}{a},(= a\; \ti{t}{c})
        \end{stackTL}}$$
        where
        $$\phi_v^{*},\ti{t}{a},(= a\; \ti{t}{c}) \implies
        {\begin{stackTL}
            (\<ge> (\<add> a\; \ti{\<ithreetwo>}{o}) \ti{\<ithreetwo>}{0}),
            \\ (\<le>
            {\begin{stackTL}
                (\<add> a\; (\<add> \ti{\<ithreetwo>}{o+width}))
                \\ \ti{\<ithreetwo>}{n_2*64 \text{Ki}})
            \end{stackTL}}
        \end{stackTL}}$$ and $n_2*64 \text{Ki} = S_\text{mem}(i,j)$
        by \reflemma{Inversion} on \refrule{Composition} and \refrule{Load-Prechk}.

        If $c+o<0$, then
        $$\phi_v^{*},\ti{t}{a},(= a\; \ti{t}{c}) \implies
        {\begin{stackTL}
            (\<ge> (\<add> a\; \ti{\<ithreetwo>}{o}) \ti{\<ithreetwo>}{0}),
            \\ (\<le>
            {\begin{stackTL}
                (\<add> a\; (\<add> \ti{\<ithreetwo>}{o+width}))
                \\ \ti{\<ithreetwo>}{n_2*64 \text{Ki}})
            \end{stackTL}}
        \end{stackTL}}$$
        is a contradiction since $a+o<0$ and therefore $(\<ge> (\<add> a\; \ti{\<ithreetwo>}{o}) \ti{\<ithreetwo>}{0})$ is contradictory.

        If $c+o+|tp|>=n_2*64 \text{Ki}$, then
        $$\phi_v^{*},\ti{t}{a},(= a\; \ti{t}{c}) \implies
        {\begin{stackTL}
            (\<ge> (\<add> a\; \ti{\<ithreetwo>}{o}) \ti{\<ithreetwo>}{0}),
            \\ (\<le>
            {\begin{stackTL}
                (\<add> a\; (\<add> \ti{\<ithreetwo>}{o+width}))
                \\ \ti{\<ithreetwo>}{n_2*64 \text{Ki}})
            \end{stackTL}}
        \end{stackTL}}$$
        is a contradiction since $c+o+|tp|>n_2*64 \text{Ki}$, and therefore the proposition $(\<le> (\<add> a\; (\<add> \ti{\<ithreetwo>}{o+width}))\; \ti{\<ithreetwo>}{n_2*64 \text{Ki}})$ is contradictory.

        Recall $\vdash s : S$, then, since $n_2*64 \text{Ki} = S_\text{mem}(i,j)$, we have $s_\text{mem}(i,j)=b_2^{*}$ where $n_2*64 \text{Ki} \leq |b_2^{*}|$.

        Therefore, it must be the case that $c+o \geq 0$ and $c+o+|tp|<|b_2^{*}|$, and therefore $s_\text{mem}(i,k+0,|tp|)=b_3^{*}$ for some $b_3^{*}$ that is a subsequence of $b_2^{*}$.
        Then, $s;(t.\<const> c)\;t.\<loadpc> (tp\_sx)\; align\;o \hookrightarrow_i t.\<const> \text{const}_t^{sx}(b_3^{*})$.

        \item Case: $\vdash_i s;v^{*};(t.\<const> c)\;t.\<loadpc> align\;o$

        Same as above, except with $|t|$ replacing $|tp|$ and $\text{const}_t^(b_3^{*})$.
    \end{itemize}
\end{proof}


\chapter{Discussion}
\label{chp:discussion}
 
By creating \name, we have taken the first step towards creating a practical system in which an expressive type system is used with a low-level language for safety and performance.
This is a first step in the sense that it provides the scaffolding to build such a system: unlike prior work, \name provides \emph{concrete} ways to use type information for compiler optimization at the assembly language level.
However, there are still a number of unanswered questions.
We have a number of future ideas for this work some based on what we think is necessary to realize our eventual goal of making \name practical in the real-world, and others based on problems identified during the course of the project so far.

\paragraph{Empirical Evaluation}
The first step would be to implement a compiler for name so we are able to perform experiments and measure the real performance benefit provided by \name.
This would allow us to empirically test whether \name actually improves performance.
Our plan is to implement \name in Rust building on the CraneLift compilier.
This will also require constructing an algorithm to build typing derivations as well as checking them.

\paragraph{More Optimizations}
There is also the potential to find other optimizations we can perform with the additional type information.
For example, remember from \autoref{sec:typesys} that an $\<if>$ may have a contradiction in the index type context one one of its branches.
In this case, that branch will never be executed, and therefore the other branch must always be taken, so we can safely replace the $\<if>$ instruction with the other branch.
We can do similar optimizations with $\<brif>$ and $\<select>$.

\paragraph{Types Annotations}
Recall that the embedding of \wasm into \name from Section \ref{subsec:embedding} does not take advantage of the possibility of using type annotations on functions and blocks to check stronger guarantees about programs.
Type annotations can be added by the developer, who will then get stronger guarantees of correctness along with the potential for more optimizations.
However, we would prefer for the developer not to have to hand-annotate compiled \wasm.
Instead, we could use static analysis to try to find the weakest preconditions that guarantee the safety of \prechk-tagging instructions.
We could also attempt to have have a compiler from a higher-level language to \wasm add annotations as a form of type preserving compilation similar to Sytem F to Typed Assembly Language \cite{FtoTAL}.

\paragraph{Reasoning About Global Variables}
Reasoning about global variables is made difficult because static typechecking is restricted to within the module we are checking.
Thus, it is difficult to reason about global variables imported from another module.
Concretely, imagine, in the $j$th module calling a function $f$ that was imported from the $i$th module.
The call instruction will be reduced to $\<call> \{\text{inst } i, \text{ func } f\}$ where $i$ is the module index for the module instance where $f$ is defined.
Theoretically, $f$ should not change the global variables in the $j$th module.
However, it may call a function in the $j$th module which could change the globals in the $j$th module, and since we do not know what the behavior of $f$ is statically within $j$, we have to assume the worst and can make no assumptions about the global variables after $f$ returns.

\paragraph{Handlind the Dynamic Resizing of Memory}
While linear memory chunks are initialized with a static size, \wasm supports dynamically growing memory using the $\<growmemory>$ instruction.
Currently, \name only supports erasing memory bounds checks based on the static size.
However, it should be possible to reason about the size of memory being increased by inserting a dependency on the result of the $\<growmemory>$ instruction.
If the result is -1, we know that the memory will not have grown and remains the same size.
Otherwise, the result will be equal to the new memory size.
This would require more dependency in the type system then we currently have with indexed types, since static type values would depend on dynamic control flow.

\paragraph{Support Streaming Compilation}
The format of \wasm code allows compilation and execution to begin with only part of the program downloaded.
Similar streaming compilation is theoretically possible with \name, but there are unanswered questions about how to work in typechecking in such a compilation pipeline.
Here are two examples of problems that we expect to face implementing such a system.
First of all, we must make sure such a system is safe, which is complicated by the fact that we may begin executing code before we have finished type checking.
This should not be too much of an issue as long as we ensure code is type checked before we can execute it, so we only execute well-typed code and if we come across code that is not well typed we halt execution and throw a type error.
Second of all, this will require highly efficient type checking, preferably performed in parallel to type check many functions at one.
We could also try to be clever and prioritize type checking on functions that we expect to be executed sooner.

\chapter{Conclusion}
\label{chp:conclusion}

We have introduced \name, a low-level language that uses an expressive type system to potentially improve performance via the elimination of unnecessary run-time checks.
To ensure the safety of \name, we have proven the type safety of the \name language as well as showing a sound type erasure to \wasm, demonstrating that \name is at least as safe as \wasm.
Further, \name is based on \wasm, a real-world language commonly used in performance-critical and untrusted contexts, where both safety and performance are critical.
This demonstrates the usefulness of using expressive type systems as a practical tool to improve performance and ensure safety for low-level languages in real use cases.


\bibliographystyle{plainnat}
\bibliography{citations}

\end{document}
